%%%%%%%%%%%%%%%%%%%%%%%%% GENERAL INFORMATION %%%%%%%%%%%%%%%%%%%%%%%%%%%%
%
% File Explanation:  LaTeX Macros for Fitch-Style subproofs, courtesy
% of CSLI Publications.
% 
% Apparently, these were used in producing "Language, Proof and Logic",
% by Jon Barwise and John Etchemendy.
%
% These comments prepared by Rich Thomason, rich@thomason.org.
% Please send bug reports and/or improvements to me.
%
% EXERCISE: Figure out if you can how to do this with two macros:
%        \subproof         3 arguments: HYPOTHESIS  
%                                       HYPOTHESIS REASON, IF ANY
%                                       STEPS 
%        \e                2 arguments: STEP 
%                                       STEP REASON, IF ANY
%
%
%%%%%%%%%%%%%%%%%%%%%%%%% INSTRUCTIONS %%%%%%%%%%%%%%%%%%%%%%%%%%%%
%
% No general instructions included in this document.  Users need to
% figure out how to enter subproofs from the attached sample.
%

%
%%%%%%%%%%%%%%%%%%%%%%%%%%% PART I: MACRO PACKAGE %%%%%%%%%%%%%%%%%%%%%%%%
%

\newcommand{\slider}{\mbox {$\triangleright \;$}}
\newcommand{\mf}[1]{$\sf{#1}$}

\def\tpskip{\vskip .25ex}
\def\smllskip{\vskip .50ex}
\def\mdskip{\vskip .75ex}
\def\btskip{\vskip 1.00ex}

% NOTE: \m should be set to the width of the slider
\newdimen\m
\setbox0=\hbox{\begin{tabular}{r@{}|}\slider\end{tabular}}
\m=-\wd0

\newdimen\fitchargone   \fitchargone 3.5in
\newdimen\fitchone   \fitchone 2.5in
\newdimen\fitchtwo   \fitchtwo 0pt
\newdimen\fitchthree \fitchthree=\fitchtwo
\advance\fitchthree by 10pt

\newcommand{\fitchmain}[1]{\advance \fitchone by -\fitchthree%
\hspace{\m}\begin{tabular}[t]{r@{}|p{\fitchone}@{}l}
 \phantom{\slider} & & \\[-1.75ex]
 #1 \\[-1.75ex] & & 
 \end{tabular}}

 %\Fitch creates a barrier flagged with 1st arg
\newcommand{\Fitch}[2]{\advance \fitchone by -\fitchthree%
\begin{tabular}[t]{p{0ex}l}%
 #1& \hspace{\m}\begin{tabular}[t]{r@{}|p{\fitchone}@{}l}
  \phantom{\slider} & & \\[-1.75ex]
 #2 \\[-1.75ex] & & 
 \end{tabular}\end{tabular} }

 %Adjustable width \Fitch
\newcommand{\adjFitch}[3]{\advance \fitchone by -\fitchthree%
\begin{tabular}[t]{p{0ex}l}%
 #2& \hspace{\m}\begin{tabular}[t]{r@{}|p{#1ex}@{}l}
  \phantom{\slider} & & \\[-1.75ex]
 #3 \\[-1.75ex] & & 
 \end{tabular}\end{tabular} }

\newcommand{\fitch}[2]{\advance \fitchone by -\fitchthree%
\hspace*{.35em}\begin{tabular}[t]{|p{\fitchtwo}@{}p{\fitchone}@{}l}
 \multicolumn{3}{@{}l@{}}{\ }\\[-2.35ex]
  #1 \\ 
  \ \\[-2.5ex] \cline{1-1}\\[-2ex]
  #2 \\ \multicolumn{3}{@{}l@{}} \ \\[-2.35ex]
 \end{tabular}}


 %\FITCH creates flagged derivation with hypothesis
\newcommand{\FITCH}[3]{\advance \fitchone by -\fitchthree%
\hbox{\hspace*{.35em}
 \raisebox{.2ex}{#1} \hspace*{-.3ex} 
 \begin{tabular}[t]{|p{\fitchtwo}@{}p{\fitchone}@{}l}
 \multicolumn{3}{@{}l@{}}{\ }\\[-2.35ex]
 \raisebox{2.5ex}{\hspace*{0ex}}#2 \\ 
  \ \\[-2.5ex] \cline{1-1}\\[-2ex]
  #3 \\ \multicolumn{3}{@{}l@{}} \ \\[-2.35ex]
 \end{tabular}}}


 %\FITCH creates flagged derivation with hypothesis
% \newcommand{\adjFITCH}[4]{\advance \fitchone by -\fitchthree%
% \hbox{\hspace*{.35em}
%  \raisebox{.2ex}{#2} \hspace*{-.3ex} 
%  \begin{tabular}[t]{|p{\fitchone}@{}p{#1ex}@{}l}
%  \multicolumn{3}{@{}l@{}}{\ }\\[-2.35ex]
%  \raisebox{2.5ex}{\hspace*{0ex}}#3 \\ 
%   \ \\[-2.5ex] \cline{1-1}\\[-2ex]
%   #4 \\ \multicolumn{3}{@{}l@{}} \ \\[-2.35ex]
%  \end{tabular}}}

 \newcommand{\adjFITCH}[4]{\advance \fitchone by -\fitchthree%
 \hbox{\hspace*{.35em}%
 \raisebox{.2ex}{#2} \hspace*{-.3ex} 
\begin{tabular}[t]{|p{\fitchtwo}@{}p{#1ex}@{}l}
  \multicolumn{3}{@{}l@{}}{\ }\\[-2.35ex]
   #3 \\ 
   \ \\[-2.5ex] \cline{1-1}\\[-2ex]
   #4 \\ \multicolumn{3}{@{}l@{}} \ \\[-2.35ex]
  \end{tabular}}}
 

 %FOR BODYLESS DERIVATIONS 
\newcommand{\hyponly}[1]{
   \hbox{\hspace*{2ex}\raisebox{-2.3ex}{\rule{.07ex}{4.4ex}}
  \hspace*{-.8ex}\raisebox{-.7ex}{\rule{1.2ex}{.07ex}}\rule{0ex}{3ex}#1}
 }

\newcommand{\fitchsub}[1]
{\advance \fitchone by -\fitchthree%
\vspace*{-5pt}
\begin{tabular}[t]{|p{\fitchtwo}@{}p{\fitchone}l}
  #1 \\[-2pt] \end{tabular} }

\newcommand{\fitcharg}[2]{\advance \fitchargone by -\fitchthree%
\hspace*{.35em}\begin{tabular}[t]{|p{\fitchtwo}@{}p{\fitchargone}}
 \multicolumn{2}{@{}l@{}}{\ }\\[-2.35ex]
  #1 \\ 
  \ \\[-2.5ex] \cline{1-1}\\[-2ex]
  #2 \\ \multicolumn{2}{@{}l@{}} \ \\[-2.35ex]
 \end{tabular}}

% EXAMPLE:
% \fitchmain{ & \fitch{ & {\sf P}    &    }
%                     { & $\;\vdots$ & \\
%                       & $\bot$     &    }
%              & \\
%     \slider & ${\sf \neg P}$              & }
% 


\newcommand{\nfitch}[3]{\advance \fitchone by -\fitchthree%
\hspace*{.35em}\begin{tabular}[t]{|p{\fitchtwo}@{}p{\fitchone}@{}l}
 \multicolumn{3}{l@{}@{}}{\ }\\[-2.35ex]
  #1 \\ 
  \ \\[-2.5ex] \cline{1-1}\\[-2ex]
  #2 \\ \multicolumn{3}{@{}l@{}} \ \\[-2.35ex]
 \end{tabular}}

% TEMPLATE FOR EMBEDDED FITCH PROOFS
%
%  \fitchmain{ & 
%               & \\
%               & ${\sf \neg P}$              & 
%             }
%  

\newcommand{\ef}{ \raisebox{-1.5ex}{\hspace*{0ex}}\\}



\newcommand{\bigfitchmain}[1]{\advance \fitchone by -\fitchthree%
\hspace{\m}\begin{tabular}[t]{r@{}|p{5in}@{}l}
 \phantom{\slider} & & \\[-1.75ex]
 #1 \\[-1.75ex] & & 
 \end{tabular}}

\newcommand{\adjfitchmain}[2]{\advance \fitchone by -\fitchthree%
\hspace{\m}\begin{tabular}[t]{r@{}|p{#1ex}@{}l}
 \phantom{\slider} & & \\[-1.75ex]
 #2 \\[-1.75ex] & & 
 \end{tabular}}

\newcommand{\bigfitch}[2]{\advance \fitchone by -\fitchthree%
\hspace*{.35em}\begin{tabular}[t]{|p{\fitchtwo}@{}p{5in}@{}l}
 \multicolumn{3}{@{}l@{}}{\ }\\[-2.35ex]
  #1 \\ 
  \ \\[-2.5ex] \cline{1-1}\\[-2ex]
  #2 \\ \multicolumn{3}{@{}l@{}} \ \\[-2.35ex]
 \end{tabular}}

\newcommand{\adjfitch}[3]{\advance \fitchone by -\fitchthree%
\hspace*{.35em}\begin{tabular}[t]{|p{\fitchtwo}@{}p{#1ex}@{}l}
 \multicolumn{3}{@{}l@{}}{\ }\\[-2.35ex]
  #2 \\ 
  \ \\[-2.5ex] \cline{1-1}\\[-2ex]
  #3 \\ \multicolumn{3}{@{}l@{}} \ \\[-2.35ex]
 \end{tabular}}

\newcommand{\num}[1]{\begin{tabular}[t]{p{3.5ex}}
 #1
 \end{tabular}}

\newcommand{\just}[1]{\begin{tabular}[t]{p{15ex}}
 #1
 \end{tabular}}

