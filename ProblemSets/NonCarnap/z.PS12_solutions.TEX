\documentclass[12pt]{memoir}


 \usepackage{forallx-ubc-Hunt} 
 
 %use $\mathit{def}$ to eliminate whitespace between multivariable letters in math mode! 
 % e.g. $\mathit{SND}$
 
  \usepackage{geometry}
\geometry{verbose,tmargin=1in,bmargin=1in,lmargin=1in,rmargin=1in}

\begin{document}

\newcommand{\detritus}[1]{}

%\newcommand*{\emph}[1]{\textbf{#1}}

%\newcommand*{\define}[1]{\textsc{\lowercase{#1}}}

%\def\eor{\ensuremath{\vee}}


\thispagestyle{empty}


\begin{center}
\vspace{-2em}
\Large \vspace{-2em} Problem Set 12 \\[1ex] 
\large Keep the internet from becoming logically complete! Never share these! \\[1ex] 
\normalsize Apologies for typos! Let me know if you catch any! (Gotta catch-em all!)
% \textbf{Answer FOUR questions total: 1 and 2, 3 Xor 4, 5 Xor 6 (exclusive or's!)}
\end{center}

\begin{enumerate}[1.)]

\item Case 10 of Soundness proof for $\mathit{SND}$ (Negation Elimination). We start by drawing a schematic derivation, labeling the relevant lines with letters so that we can easily refer to them. Let ``$\Gamma_{k+1}$'' denote the set of open assumptions at line k+1.  Our goal is to show that line k+1 satisfies the inductive property (i.e. is ``righteous'', so that $\Gamma_{k+1} \entails \metav{P}$).\\ (Note: you could start your own diagram at line $j$ with just some vertical dots above it)

\begin{equation*}
\begin{nd}
\have{1}{\textrm{First premise}} \pr{}
\ellipsesline
\hypo[f]{f}{\textrm{Last premise}} \pr{}
\ellipsesline
\open
	\hypo[j]{na}{\enot\metav{P}} \as{for \enot E}
		\ellipsesline
	\have[\ell]{b}{\metav{R}}
	\have[m]{nb}{\enot\metav{R}}
\close
\have[k+1]{a}[\ ]{\metav{P}}\ne{na-nb}
\end{nd}
\end{equation*}

By assumption, we derive $\metav{R}$ in $l$-steps from $\Gamma_l$ and $\sim \metav{R}$ in $m$-steps from $\Gamma_m$. Hence, by the Induction Hypothesis, we have $\Gamma_l \vDash \metav{R} $ and $\Gamma_m \vDash \sim \metav{R} $ (in the IH, we assume that each line less than k+1 is ``righteous'', i.e. if  $h < k+1$ and $\Gamma_h \vdash \metav{S}$, then $\Gamma_h \vDash \metav{S}$). 

Next we note the relevant subset relations connecting these assumption-sets and the assumption set $\Gamma_{k+1}$ for line k+1: $\Gamma_l \subseteq \Gamma_{k+1} \cup \{\sim\metav{P}\}$ and $\Gamma_m \subseteq \Gamma_{k+1} \cup \{\sim\metav{P}\}$. To see these relations, note that every assumption that is open at line $\ell$ is open at line k+1 except for $\enot\metav{P}$ on line j. Likewise for $\Gamma_m$. 

	Hence, any truth-value assignment that makes every sentence in $\Gamma_{k+1} \cup \{\sim\metav{P}\}$ true must also make true every sentence in $\Gamma_l$ and every sentence in $\Gamma_m$. Since these latter sets semantically entail \metav{R} and \enot  \metav{R} respectively, we see that the superset must entail these sentences as well: $\Gamma_{k+1} \cup \{\sim \metav{P}\} \vDash \metav{R} $ and $\Gamma_{k+1} \cup \{\sim \metav{P}\} \vDash \sim \metav{R} $ (here, we have justified and applied the book's lemma 6.3.2). 

Therefore, any TVA that makes true every sentence in $\Gamma_{k+1} \cup \{\sim \metav{P}\}$ must make true both \metav{R} and \enot  \metav{R}. But this is impossible: there is no such TVA. Hence, the superset $\Gamma_{k+1} \cup \{\sim \metav{P}\}$ must be unsatisfiable (i.e. semantically inconsistent). Hence, any TVA that makes true every sentence in $\Gamma_{k+1}$ must make true \metav{P} (this applies the book's lemma 6.3.5, noting that \enot \enot \metav{P} is semantically equivalent to \metav{P}). Hence, line k+1 has the inductive property of righteousness. 

\iffalse %%JTapp argument from his 303 PSet 6 solutions:

Since there is a derivation of length $l$ of $\metav{R}$ from $\Gamma_l$ (with $\Gamma_l \subseteq \Gamma_{k+1} \cup \{\sim\!\!\metav{P}\}$), and a derivation of length $m$ of $\sim \metav{R}$ from $\Gamma_m$ (with $\Gamma_m \subseteq \Gamma_{k+1} \cup \{\sim\!\!\metav{P}\}$), 
with $l \leq k$ and  $m \leq k$, we can apply the induction hypothesis to obtain:

$\Gamma_l \vDash \metav{R} $ and $\Gamma_m \vDash \sim \metav{R} $, so by {\textbf{6.3.2}} of the textbook. 
%(aka  ``{\bf{Useful fact 1}}'' in lecture):\\

$\Gamma_{k+1} \cup \{\sim \metav{P}\} \vDash \metav{R} $ and $\Gamma_{k+1} \cup \{\sim \metav{P}\} \vDash \sim \metav{R} $

Since $\Gamma_{k+1} \cup \{\sim \metav{P}\} \vDash \metav{R}$ and $\Gamma_{k+1} \cup \{\sim \metav{P}\} \vDash \sim \metav{R} $ we know that any truth-assignment making all of $\Gamma_{k+1} \cup \{\sim \metav{P}\}$ true makes  $\metav{R}$ true {\textit{and}}  {$\sim \metav{R}$} true.

 This cannot happen, so there is {\textit{no}} TVA making all of $\Gamma_{k+1} \cup \{\sim \metav{P}\}$ true.

So if a truth assignment makes all of $\Gamma_{k+1} $ true, it must make ${\sim \metav{P}}$ false, and so it makes ${\metav{P}}$ true. That is, $\Gamma_{k+1} \vDash {\metav{P}}$\\

\fi


\item We prove case (c) of the membership lemma (book's 6.4.11), used in the completeness proof for  $\mathit{SND}$ (and, suitably modified, for  $\mathit{QND}$ as well): if $\Gamma^*$ is a maximally syntactically-consistent set of SL sentences, then: 
${\metav{P}} \lor {\metav{Q}} \in \Gamma^*$ if and only if either ${\metav{P}} \in \Gamma^*$ or ${\metav{Q}} \in\Gamma^*$.
%Problem from \textit{The Logic Book}, p. 261 6.4 E \# 5 a)

%\iffalse 

Remember that we need to prove both the forwards and backwards directions, possibly splitting each of these into further subcases. 

($\Rightarrow$-Direction): Assume that ${\metav{P}} \lor {\metav{Q}} \in \Gamma^*$. Need to show (NTS): either ${\metav{P}} \in \Gamma^*$ or ${\metav{Q}} \in\Gamma^*$. \\ We split this into two cases, since ${\metav{P}}$ is either in the club or it is not in the club. \\ Case (i): Note that if ${\metav{P}} \in \Gamma^*$ the relevant either-or claim is true. So move to case (ii), where we assume that ${\metav{P}} \not\in \Gamma^*$. Our goal is to show that ${\metav{Q}} \in\Gamma^*$. To show this, it suffices to derive ${\metav{Q}}$ from finitely-many premises assumed to be in $\Gamma^*$, since we can then apply The Door lemma to conclude that ${\metav{Q}} \in\Gamma^*$. 

At this point, there are two ways to proceed directly. We could either note that by Case (a) of the membership lemma, ${\metav{P}} \not\in \Gamma^*$ entails that ${\enot \metav{P}} \in \Gamma^*$. We could then derive \metav{Q} in SND from the finite premise set $\{\metav{P} \eor \metav{Q}, \enot \metav{P} \} \subset \Gamma^*$ by constructing a derivation similar to that below. Alternatively, we can recall the definition of a maximally SND-consistent set and note that ${\metav{P}} \not\in \Gamma^*$ entails that $\Gamma^*\cup \{{\metav{P}}\}$ is SND-INconsistent. So there exists an SL-sentence \metav{R} such that $\Gamma^*\cup \{{\metav{P}}\} \vdash {\metav{R}}$ and  $\Gamma^*\cup \{{\metav{P}}\} \vdash \sim{\metav{R}}$. You ought to memorize this criterion, so we'll use it below to reinforce that memory! 

Let ${\metav{A}}_1; {\metav{A}}_2; \ldots ; {\metav{A}}_n ; {\metav{P}}$ be finitely-many premises from $\Gamma^* \cup \{{\metav{P}}\}$ that derive ${\metav{R}}$ and $\sim {\metav{R}}$.\\
We show that there is a derivation of ${\metav{Q}}$ from ${\metav{A}}_1; {\metav{A}}_2; \ldots ; {\metav{A}}_n ; {\metav{P}} \lor {\metav{Q}}$ (given shamelessly without line numbers below). You should include schematic line numbers in the justifications!:
%with derivations of ${\metav{R}}$ and $\sim{\metav{R}}$ used in a subproof. 
 \begin{equation*}
  \begin{kfitch*}                           
 \fa  [\textrm{Assumptions from } \Gamma^*]               & :PRs        \\
\fj   {\metav{P}} \lor {\metav{Q}}    & :PR from $\Gamma^*$      \\
\fa\fh  {\metav{Q}} & A/ $\lor E$      \\
\fa\fa {\metav{Q}}   & R       \\
\fa\fh {\metav{P}}   & A/ $\lor E$        \\ 
\fa\fa\fh \sim{\metav{Q}}    & A/ $\sim E$      \\
\fa\fa\fa    \vdots   &      \\
\fa\fa\fa    {\metav{R}}  &      \\
\fa\fa\fa    \sim{\metav{R}}  &      \\
\fa\fa {\metav{Q}}  & $\sim E$     \\
\fa {\metav{Q}}  & $\lor E$     \\
   \end{kfitch*}
\end{equation*}

Since $\{{\metav{A}}_1; {\metav{A}}_2; \ldots ; {\metav{A}}_n ; {\metav{P}} \lor {\metav{Q}}\}   \subseteq \Gamma^*$, this proves that $\Gamma^* \vdash {\metav{Q}}$. So by The Door Lemma (book's 6.4.9), ${\metav{Q}} \in\Gamma^*$. We proceed to the next direction! 

($\Leftarrow$-direction): Now, assume that either ${\metav{P}}\in\Gamma^*$ or ${\metav{Q}}\in \Gamma^*$ (where our inclusive-or includes the possibility that both are in $\Gamma^*$). In subcase (i), we assume ${\metav{P}}\in\Gamma^*$. Note that we can derive ${\metav{P}} \lor {\metav{Q}}$ from ${\metav{P}}$ with one application of $\lor$ introduction: $\{ \metav{P} \} \subset \Gamma^* \vdash \metav{P} \eor \metav{Q}$. So by the Door Lemma, ${\metav{P} \lor \metav{Q}}\in \Gamma^*$. In subcase (ii), we assume ${\metav{Q}}\in \Gamma^*$. Then similarly, $\{ \metav{Q} \} \vdash \metav{P} \eor \metav{Q}$ by $\eor$I. So in either case, ${\metav{P} \lor \metav{Q}}\in \Gamma^*$, which is what we needed to show.  

This completes both directions. Welcome to the club haha! \\

%We'll treat the subcase of ${\metav{P}}\in\Gamma^*$, since the case of ${\metav{Q}}\in\Gamma^*$ is identical, {\textit{mutatis mutandis}}. \\

%\fi 

\item[3.)] We aim to leverage the membership lemma case we just proved in problem \#2, to complete missing case (3) in our induction over SL. Recall that in this induction, we construct a TVA \metav{I} such that a sentence \metav{P} is true on \metav{I} if and only if \metav{P} belongs to the club, i.e. $\metav{P} \in \Gamma^*$. We induct over the number of connectives in SL sentences, assuming that this property (they be ``clubbin' '') holds for each sentence with less than k+1 connectives. \\ Note that we again have two directions to prove, since this is an ``iff'' statement. 

In case 3, \metav{P} has the form $\metav{Q} \eor \metav{R}$, with k+1-many connectives. 

($\Rightarrow$-Direction): Assume that $\metav{Q} \eor \metav{R}$ is true on \metav{I}. Then by the truth-table for \eor, at least one of \metav{Q} or \metav{R} is true on \metav{I}. Since each of these sentences contains less than k+1 connectives, they are clubbin' by the Induction Hypothesis. So at least one of \metav{Q} or \metav{R} belongs to $\Gamma^*$. Hence, by case (c) of the membership lemma, $\metav{Q} \eor \metav{R} \in \Gamma^*$ as well. 

($\Leftarrow$-direction): Assume that  $\metav{Q} \eor \metav{R} \in \Gamma^*$. NTS: $\metav{Q} \eor \metav{R}$ is true on \metav{I}. We can immediately apply membership lemma case (3) to note that since $\metav{Q} \eor \metav{R} \in \Gamma^*$, at least one of \metav{Q} or \metav{R} belongs to $\Gamma^*$. Since \metav{Q} and \metav{R} each have less than k+1 connectives, by the IH they are true on the TVA \metav{I}. So by the truth-table for \eor, $\metav{Q} \eor \metav{R}$ is also true on \metav{I}. 

Note that we could streamline our proof by taking advantage of relevant iff-claims at each step, thereby handling both directions in one go (NB: ``$ \Leftrightarrow$'' means bi-directional entailment, NOT the biconditional connective \eiff): 

$\metav{Q} \eor \metav{R}$ is true on \metav{I} $ \Leftrightarrow\, {\metav{Q}}$ is true on  \metav{I} or ${\metav{R}}$ is true on  \metav{I} by the truth table for $\lor$.\\
$\Leftrightarrow\, {\metav{Q}} \in \Gamma^*$ or ${\metav{R}} \in \Gamma^*$ by the induction hypothesis
$\Leftrightarrow$ by 6.4.11 c) ${\metav{Q}} \lor {\metav{R}} \in \Gamma^*$.\\ But I don't recommend this shortcut since it is harder to parse the argument in the backwards direction (since we write from left to right). \\



\item[4.)] 

Recall that a natural deduction is always finite. So no matter how many sentences there might be in a set $\Gamma$, a derivation can draw on \textit{only finitely-many} sentences from $\Gamma$. \\ So if $\Gamma \vDash S$, then by the completeness theorem, $\Gamma \vdash_{\mathit{SND}} S$. Hence, there exists a finite subset $\Delta \subset \Gamma$ such that $\Delta \vdash_{\mathit{SND}} S$. By the soundness theorem, we can convert this single turnstile to a dolla dolla double: $\Delta \vDash S$ (i.e. the finite set $\Delta$ semantically entails $S$). 

\newpage

\item[5.)] 

Assume for \textit{reductio} that you could derive a contradictory sentence pair \metav{R} and \enot\metav{R}  from the atomic sentence letter $B$. Then we'd have $B \vdash_{\mathit{SND}} \metav{R}$ and $B \vdash_{\mathit{SND}} \enot \metav{R}$. 

Applying the soundness theorem, we could convert these singles to doubles, so we'd have $B \vDash \metav{R}$ and $B \vDash \enot \metav{R}$. Yet, recall that a sentence (or set of sentences) can entail a contradiction only if it is unsatisfiable! But $B$ is satisfiable: it is true on any TVA that assigns ``true'' to $B$. Hence, we have reached a contradiction: It is impossible for any TVA that makes $B$ true to make both \metav{R} and \enot\metav{R} true. Hence, the set $\{ B \}$ is syntactically-consistent in SND. \\

An alternative, very slick \textit{reductio}: if you could derive a contradictory sentence pair \metav{R} and \enot\metav{R}  from the atomic sentence letter $B$, then you could derive $\enot B$ from $B$ by negation introduction. So then we'd have $B \vdash_{\mathit{SND}} \enot B$, and by soundness $B \vDash \enot B$. This would mean that on any TVA where $B$ is true, $\enot B$ is also true, which is clearly absurd. \\
%This is clearly impossible. 
%proof from Nathan Sheffield(?)


\item [6.)]
By the completeness theorem for $\mathit{SND}$, $\Gamma \vDash S \Rightarrow \Gamma \vdash_{\mathit{SND}} S$. Hence, we can appeal to the contrapositive of completeness: if $\Gamma \nvdash_{\mathit{SND}} S$, then $\Gamma \nvDash S$. 

We are told that for some SL-set $\Gamma$, $\Gamma \nvdash_{\mathit{SND}} S$. Hence, by completeness$_{contra}$, $\Gamma \nvDash S$. This is a counterexample to the soundness of modified system SND$^*$, i.e. a case where $\Gamma \vdash_{\mathit{SND}^*} S$ but it's \textit{not the case} that $\Gamma \vDash S$. \\

Spelled out in more detail: since $\Gamma \nvDash S$, the set $\Gamma \cup \{ \enot S \}$ is satisfiable: i.e. there is a TVA that makes true every sentence in $\Gamma$ while making $\enot S$ true and hence $S$ false. \\ So since $\Gamma \vdash_{\mathit{SND}^*} S$, our modified system allows a case where we go from true premises to derive a false conclusion, which just means that $\mathit{SND}^*$ is unsound. 

%%we can also give a short/clean reductio proof as well! see AK solution. 


\iffalse

\item  %JTapp PS10, number 1. Winter 2019 
\begin{equation*}
\begin{nd}
	\hypo{1}{Gb \eif Fb} \pr{}
	\hypo{2}{Gb} \pr{}
	\have{3}{Fb} \ce{1, 2}
	\have{4}{Fb \eor Hb} \oi{3}
	\have{5}{\qt{\exists}{x}(Fx \eor Hx)} \Ei{4}
\end{nd}
\end{equation*}

\fi 











\end{enumerate}







\end{document}