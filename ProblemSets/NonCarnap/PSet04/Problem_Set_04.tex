\documentclass[12pt]{article}

\usepackage{amssymb}
\usepackage{geometry}
  \geometry{verbose,tmargin=1in,bmargin=1in,lmargin=1in,rmargin=1.25in}
 
\newcommand{\set}[1]{\lbrace#1\rbrace} %%Set brackets
\newcommand{\I}{\mathcal{I}} %%Corner quotes
\newcommand{\J}{\mathcal{J}} %%Corner quotes
\newcommand{\V}[1]{\mathcal{V}_{#1}} %%Corner quotes


\begin{document}

\pagestyle{empty}

%%% FOR OTHER PROBLEMS EXCLUDED BELOW
  % \usepackage{xref}
  %  %    \textheight     10.0truein
  %  \usepackage{graphics}
  %  \usepackage{pstricks}
  % % \usepackage{pst-tree}
  % % \usepackage{pst-node,pst-tree}
  %  \usepackage{makeidx}
  % \input macs
  %\input fitch

\newcommand{\detritus}[1]{}

\thispagestyle{empty}
\parindent = 0pt
\vspace{30pt}

% \iffalse
\hspace*{0.0in}\parbox[t]{2.5in}{
Philosophy 24.241\\[3pt]
Symbolic Logic \\[3pt]
Fall, 2023
}
% \fi 


\bigskip %\bigskip


\begin{center}
\Large Problem Set 4\\[1ex] 
 Due Fri. October 6 by 5pm Eastern\\ 
  \vspace{.1in}
  \normalsize{(Please scan and upload to Canvas as a pdf)} \\[3ex] 
\end{center}


%Some of these problems draw from the posted Induction and Recursion notes.\\
Question 0: If you worked with up to two classmates, please list their names! 

Use mathematical induction to prove the following claims.

For EACH problem, label the base case(s) AND label the induction step/hypothesis. 

\begin{enumerate}
\item	Call a string over $\{a, b\}$
an ``a-palindrome" if it is a palindrome that has ``$a$" as a middle letter. (An a-palindrome therefore must have an odd number of letters.) 
% You can assume the language has an empty string, or that it doesn't, but state which option you are going with, as the answer %will be slightly different in each case.\\

(i) Give a recursive definition of the set of ``a-palindromes", and \\
(ii) Prove by induction that every a-palindrome has an even number of ``$b$"'s. \\
%\\ (This one is meant to be quite straightforward, and just to give you practice in setting up recursive definitions and giving arguments in inductive form. Don't make it more complicated than it has to be.) \\
%For this question, you don't need to prove that the recursive definition you give defines exactly the set of a-palindromes.\\

% \item Here is the recursive definition of n! (read ``n factorial"):\\
% \noindent $1! = 1$\\
% \noindent $(n+1)! = (n + 1) \times n!$\\
%
% \noindent That is, $n! = \underbrace{(n \times (n-1) \times \ldots 3 \times 2 \times 1)}_{n \ times}$\\
%
% \noindent {\bf{Prove by induction:}} For every $n$ greater than or equal to 5, $3^{n-1} < n!$ \\%< n^n$. 
%
% Hint: A simple bit of algebra will be useful here: if $a, b, c, d$ are all natural numbers, with $a < b$ and $c < d$,
% then $a\cdot c < b\cdot d$.
  
   
% \item Prove by induction that if you just have 4 and 11 cent stamps, you can get a combination of stamps for 30 cents, and {\it{any}} amount greater than 30. \\ (Hint: The base case(s) you need in this one needs to be crafted carefully. You will need to prove more than one case.)

  % \item Prove by induction that the product of $n$ odd numbers (with $2 \leq n$) is odd.
  %
  % (Hint: Any odd natural number $m$ can be written as $2k+1$, for some other natural number $k$.)\\

  \item No sentence of SL ever contains consecutive atomic formulas (e.g., `$(PP\&Q)$').\\

  \item Let $\I$ and $\J$ be interpretations of SL where $\I(\alpha)=\J(\alpha)$ for every sentence letter $\alpha$ of SL.
    Prove that $\V{\I}(\varphi)=\V{\J}(\varphi)$ for every sentence $\varphi$ of SL.\\

  \item Let $\I$ be an SL interpretation.
    Prove that $\V{\I}(\varphi)\in\set{0,1}$ for ever SL sentence $\varphi$.

  \item Show that for any SL sentence $\varphi$, if $\varphi$ contains at most one occurrence of any sentence letter, then $\nvDash \varphi$. 

\end{enumerate}


\detritus{ BEGIN DETRITUS - question about CHESSBOARD TILING
\item This question would be, without hints and guidance, quite challenging. But it's a very nice example of an inductive argument that doesn't use numbers, so I wanted you to work through it, with guide rails along the way. Don't be spooked - almost all the work is done below, and I have a meaty hint for the step you have to fill in.

 Prove that any square chessboard that is $2^n$ squares wide and $2^n$ squares high, with one square removed,
can be completely covered (with no overlap) by a collection of tiles, where each tile is three squares in an
L-shape. In pictures:\\
Say you have a $2^n$ x $2^n$ chessboard, with one square removed (removed square marked in red). Example:\\

\begin{pspicture}(2, 0)(7,5) 

%Tiled chessboard with one colored square 
%vertical:
 \psline[linewidth=0.3mm]{-}(3,0)(3,4)
  \psline[linewidth=0.3mm]{-}(3.5,0)(3.5,4)
 \psline[linewidth=0.3mm]{-}(4,0)(4,4)
 \psline[linewidth=0.3mm]{-}(4.5,0)(4.5,4)
 \psline[linewidth=0.3mm]{-}(5,0)(5,4)
 \psline[linewidth=0.3mm]{-}(5.5,0)(5.5,4)
 \psline[linewidth=0.3mm]{-}(6,0)(6,4)
 \psline[linewidth=0.3mm]{-}(6.5,0)(6.5,4)
 \psline[linewidth=0.3mm]{-}(7,0)(7,4)


%horizontal:

 \psline[linewidth=0.3mm]{-}(3,0)(7,0)
 \psline[linewidth=0.3mm]{-}(3,.5)(7,.5)
 \psline[linewidth=0.3mm]{-}(3,1)(7,1)
 \psline[linewidth=0.3mm]{-}(3,1.5)(7,1.5)
 \psline[linewidth=0.3mm]{-}(3,2)(7,2)
 \psline[linewidth=0.3mm]{-}(3,2.5)(7,2.5)
 \psline[linewidth=0.3mm]{-}(3,3)(7,3)
 \psline[linewidth=0.3mm]{-}(3,3.5)(7,3.5)
 \psline[linewidth=0.3mm]{-}(3,4)(7,4)

%horizontal measuring standard

\put(5,4.7){$2^n$}
 \psline[linewidth=0.3mm]{-}(3,4.5)(7,4.5)
 \psline[linewidth=0.3mm]{-}(3,4.4)(3,4.5)
 \psline[linewidth=0.3mm]{-}(7,4.4)(7,4.5)

%vertical measuring standard

\put(2,2){$2^n$}
 \psline[linewidth=0.3mm]{-}(2.5,0)(2.5,4)
 \psline[linewidth=0.3mm]{-}(2.6,0)(2.5,0)
 \psline[linewidth=0.3mm]{-}(2.6,4)(2.5,4)

%fill in square
\pspolygon[fillstyle=solid, fillcolor=red](6,2.5)(6,3)(6.5,3)(6.5,2.5)
%\pspolygon[fillstyle=solid, fillcolor=yellow](.7,.2)(.7,.7)(1.2,.7)(1.2,.2)


\end{pspicture}
(Note: the chessboard in the diagram is 8 x 8, but it is meant to represent an arbitrary $2^n$ x $2^n$ 
board.)\\
The white squares can all be covered (with no overlap) by tiles in this shape:
 %\fbox{  
\begin{pspicture}(0,0)(2,2) 

. 

 \psline[linewidth=0.3mm]{-}(.2,.2)(.7,.2)
  \psline[linewidth=0.3mm]{-}(.2,.7)(1.2,.7)
 \psline[linewidth=0.3mm]{-}(.2,1.2)(1.2,1.2)
 \psline[linewidth=0.3mm]{-}(.2,.2)(.2,1.2)
 \psline[linewidth=0.3mm]{-}(.7,.2)(.7,1.2)
 \psline[linewidth=0.3mm]{-}(1.2,.7)(1.2,1.2)

\pspolygon[fillstyle=solid, fillcolor=red](.2,.2)(.7,.2)(.7,.7)(1.2,.7)(1.2,1.2)(.2,1.2)
%\pspolygon[fillstyle=solid, fillcolor=yellow](.7,.2)(.7,.7)(1.2,.7)(1.2,.2)

\end{pspicture}


Remarks about this problem:\\


a) To give you an idea of what we are trying to prove, here is an example of a 4 x 4 chessboard
(i.e. $2^2$ x $2^2$) with one square removed (in red). 

\begin{pspicture}(2, 0)(5.5,2.5) 

%Tiled chessboard with one colored square 
%vertical:
 \psline[linewidth=0.3mm]{-}(3,0)(3,2)
  \psline[linewidth=0.3mm]{-}(3.5,0)(3.5,2)
 \psline[linewidth=0.3mm]{-}(4,0)(4,2)
 \psline[linewidth=0.3mm]{-}(4.5,0)(4.5,2)
 \psline[linewidth=0.3mm]{-}(5,0)(5,2)



%horizontal:

 \psline[linewidth=0.3mm]{-}(3,0)(5,0)
 \psline[linewidth=0.3mm]{-}(3,.5)(5,.5)
 \psline[linewidth=0.3mm]{-}(3,1)(5,1)
 \psline[linewidth=0.3mm]{-}(3,1.5)(5,1.5)
 \psline[linewidth=0.3mm]{-}(3,2)(5,2)


%fill in square
\pspolygon[fillstyle=solid, fillcolor=red](4,1.5)(4.5,1.5)(4.5,1)(4,1)

\end{pspicture}

The above chessboard can be L-tiled this way. (The L-shaped tiles are displayed in 
different colors, for easier viewing.) You want to show that this kind of a tiling
will be possible for any  $2^n$ x $2^n$ board with one square removed.



\begin{pspicture}(2, 0)(5.5,2.5) 

%Tiled chessboard with one colored square 
%vertical:
 \psline[linewidth=0.3mm]{-}(3,0)(3,2)
  \psline[linewidth=0.3mm]{-}(3.5,0)(3.5,2)
 \psline[linewidth=0.3mm]{-}(4,0)(4,2)
 \psline[linewidth=0.3mm]{-}(4.5,0)(4.5,2)
 \psline[linewidth=0.3mm]{-}(5,0)(5,2)



%horizontal:

 \psline[linewidth=0.3mm]{-}(3,0)(5,0)
 \psline[linewidth=0.3mm]{-}(3,.5)(5,.5)
 \psline[linewidth=0.3mm]{-}(3,1)(5,1)
 \psline[linewidth=0.3mm]{-}(3,1.5)(5,1.5)
 \psline[linewidth=0.3mm]{-}(3,2)(5,2)


%fill in square
\pspolygon[fillstyle=solid, fillcolor=red](4,1.5)(4.5,1.5)(4.5,1)(4,1)
\pspolygon[fillstyle=solid, fillcolor=yellow](3,0)(4,0)(4,.5)(3.5,.5)(3.5,1)(3,1)
\pspolygon[fillstyle=solid, fillcolor=blue](3,1)(3,2)(4,2)(4,1.5)(3.5,1.5)(3.5,1)
\pspolygon[fillstyle=solid, fillcolor=green](3.5,.5)(3.5,1.5)(4,1.5)(4,1)(4.5,1)(4.5,.5)
\pspolygon[fillstyle=solid, fillcolor=yellow](4,2)(5,2)(5,1)(4.5,1)(4.5,1.5)(4,1.5)
\pspolygon[fillstyle=solid, fillcolor=blue](4,0)(4,.5)(4.5,.5)(4.5,1)(5,1)(5,0)
\end{pspicture}


b) You will have to proceed by induction. The base case, for a 2 x 2 board, is automatic. Whatever
square is chosen, there will be an L-shape left over, as in this case (chosen square in red, remainder
in blue): 
%Tiled 4x4 board -- base case
\begin{pspicture}(0,0)(1.5,1.5) 
\pspolygon[fillstyle=solid, fillcolor=blue](.2,.2)(.7,.2)(.7,.7)(1.2,.7)(1.2,1.2)(.2,1.2)
\pspolygon[fillstyle=solid, fillcolor=red](.7,.2)(.7,.7)(1.2,.7)(1.2,.2)
\end{pspicture}



c) The point of the inductive proof is to reduce the problem of tiling a $2^{n+1}$ x $2^{n+1}$ board
to some question or family of questions about tiling $2^n$ x $2^n$ boards, since $2^n$ x $2^n$
boards are covered by the induction hypothesis. In a picture:\\


\begin{pspicture}(1, -.5)(7.5,5.5) 

%Tiled chessboard with one colored square 
%vertical:
 \psline[linewidth=0.3mm]{-}(3,0)(3,4)
  \psline[linewidth=0.3mm]{-}(3.5,0)(3.5,4)
 \psline[linewidth=0.3mm]{-}(4,0)(4,4)
 \psline[linewidth=0.3mm]{-}(4.5,0)(4.5,4)
 \psline[linewidth=0.7mm]{-}(5,0)(5,4)
 \psline[linewidth=0.3mm]{-}(5.5,0)(5.5,4)
 \psline[linewidth=0.3mm]{-}(6,0)(6,4)
 \psline[linewidth=0.3mm]{-}(6.5,0)(6.5,4)
 \psline[linewidth=0.3mm]{-}(7,0)(7,4)


%horizontal:

 \psline[linewidth=0.3mm]{-}(3,0)(7,0)
 \psline[linewidth=0.3mm]{-}(3,.5)(7,.5)
 \psline[linewidth=0.3mm]{-}(3,1)(7,1)
 \psline[linewidth=0.3mm]{-}(3,1.5)(7,1.5)
 \psline[linewidth=0.7mm]{-}(3,2)(7,2)
 \psline[linewidth=0.3mm]{-}(3,2.5)(7,2.5)
 \psline[linewidth=0.3mm]{-}(3,3)(7,3)
 \psline[linewidth=0.3mm]{-}(3,3.5)(7,3.5)
 \psline[linewidth=0.3mm]{-}(3,4)(7,4)


%big horizontal measuring standard

\put(5,5.2){$2^{n+1}$}
 \psline[linewidth=0.3mm]{-}(3,5)(7,5)
 \psline[linewidth=0.3mm]{-}(3,4.9)(3,5)
 \psline[linewidth=0.3mm]{-}(7,4.9)(7,5)

%big vertical measuring standard

 \put(1,2){$2^{n+1}$}
 \psline[linewidth=0.3mm]{-}(1.8,0)(1.8,4)
 \psline[linewidth=0.3mm]{-}(1.9,0)(1.8,0)
 \psline[linewidth=0.3mm]{-}(1.9,4)(1.8,4)

%small horizontal measuring standards

\put(4,4.6){$2^n$}
\put(6,4.6){$2^n$}



 \psline[linewidth=0.3mm]{-}(3,4.5)(7,4.5)
 \psline[linewidth=0.3mm]{-}(3,4.4)(3,4.5)
 \psline[linewidth=0.3mm]{-}(7,4.4)(7,4.5)
 \psline[linewidth=0.3mm]{-}(5,4.4)(5,4.5)


%small vertical measuring standards

 \put(2,1){$2^n$}
 \put(2,3){$2^n$}
\put(2.7,-.4){{\bf{III}}}
\put(2.7,4){{\bf{I}}}
\put(7.2,0){{\bf{IV}}}
\put(7.2,4){{\bf{II}}}

 \psline[linewidth=0.3mm]{-}(2.5,0)(2.5,4)
 \psline[linewidth=0.3mm]{-}(2.6,0)(2.5,0)
 \psline[linewidth=0.3mm]{-}(2.5,4)(2.6,4)
 \psline[linewidth=0.3mm]{-}(2.5,2)(2.6,2)

%fill in square
\pspolygon[fillstyle=solid, fillcolor=red](6,2.5)(6,3)(6.5,3)(6.5,2.5)
%\pspolygon[fillstyle=solid, fillcolor=yellow](.7,.2)(.7,.7)(1.2,.7)(1.2,.2)


\end{pspicture}

By dividing the $2^{n+1}$ x $2^{n+1}$ board into 4 quadrants, each of size $2^n$ x $2^n$ 
we get a situation where the inductive hypothesis can help us. The red square will be in 
one of the quadrants - in the diagram above, it is in the upper right hand quadrant, labelled II. 
By the induction hypothesis we know that that quadrant II can be covered in L-tiles, leaving the 
red square uncovered. {\bf{But}} this leaves us with one more problem to solve before we can conclude that
every $2^{n+1}$ x $2^{n+1}$ board with one square missing can be covered in L-tiles:
How can we show that the rest of the board (quadrants I, III, and IV in 
the above diagram) can be tiled with L-tiles? The induction hypothesis just addresses chessboards
with one square missing, not chessboards with all the squares in place. Here we need a bit of inspiration
to figure out how to produce a situation where we can use the induction hypothesis. Once that is done, 
the problem falls into place.\\

\noindent Hint: Place one L-tile somewhere, to get a situation where the inductive hypothesis can apply to the quadrants without the red square.\\
% We can get a situation to which the inductive hypothesis applies to each of the quadrants without the 
%red square by placing one L-tile on the part of the board consisting of the quadrants without the red square.
%(In our diagram: by placing one L-tile on the part of the board consisting of quadrants I, III, and IV, we can get
%a situation where the induction hypothesis applies to quadrants I, III, and IV.) 

END DETRITUS }


\end{document}


