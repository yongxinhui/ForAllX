% !TeX root = ./slides-02.tex

\setcounter{section}{1}

% Replaced `basic sentence' with `atomic sentence'

\section{Symbolization in SL}
\subsection{Symbolization keys and paraphrase}

\begin{frame}
  \frametitle{Symbolizing arguments}

  \begin{block}<1->{Argument 2}
    Mandy enjoys skiing or Mandy enjoys hiking.\\
    Not: Mandy enjoy hiking.\\
    $\therefore$ Mandy enjoys skiing.
  \end{block}

  \begin{block}<2->{Form of argument 2}
    $S$ or $H$.\\
    Not $H$.\\
    $\therefore$ $S$.
  \end{block}

  \begin{block}<3->{Symbolization of argument 2 in SL}
    $(S \lor H)$\\
    $\lnot H$\\
    $\therefore$ $S$
  \end{block}

\end{frame}

\begin{frame}
  \frametitle{Symbolization keys}

  \begin{definition}
    A symbolization key is a list that pairs \emph{sentence letters} with
    the basic English sentences they represent.
  \end{definition}

  For instance:

  \begin{block}{Symbolization key}
    $S$: Mandy enjoys skiing\\
    $H$: Mandy enjoys hiking
  \end{block}

\end{frame}

\begin{frame}
  \frametitle{Symbolization keys}

  \begin{itemize}[<+->]
  \item Sentence letters are uppercase letters, possibly with subscripts
  (e.g., $H_1$, $H_2$).
  \item Usually the symbolization key is given to you \footnotesize{(to facilitate grading)}
  \item It should not be possible to further decompose the ``atomic
  sentences'' represented by sentence letters.

  \item[] For instance:
  \centerline{$A$: Mandy enjoys skiing or hiking}
  is a bad choice of symbolization.
  \end{itemize}
\end{frame}

\begin{frame}
  \frametitle{Paraphrase}

  \begin{itemize}[<+->]
  \item Successful symbolization sometimes requires \emph{paraphrase} to
  ensure atomic sentences appear explicitly.
  \item Two things to watch for: pronouns and coordination.
  \item Pronouns stand in for singular terms (e.g., names): \\ explicitly replace
  pronouns by those names.
  \item ``or'', ``both \dots and'', ``neither \dots nor'' can connect sentences but
  also noun phrases and verb phrases: \\ paraphrase them so that they connect
  sentences.
  \end{itemize}
\end{frame}

\begin{frame}
  \frametitle{Pronouns}

  \begin{block}{Example}
    If Mandy enjoys hiking, \emph{she} also enjoys skiing.

    Replace ``she'' by ``Mandy'':\\
    If [Mandy enjoys hiking] then [Mandy enjoys skiing].
  \end{block}
\end{frame}

\begin{frame}
  \frametitle{Coordination of noun phrases}

  \begin{block}{Example}
    \emph{Mandy and Sanjeev} enjoy hiking.

    Both [Mandy enjoys hiking] and [Sanjeev enjoys hiking].
  \end{block}

  \begin{block}{Example}
    Sanjeev lives in \emph{Erie or Chicago}.

    Either [Sanjeev lives in Erie] or [Sanjeev lives in Chicago].
  \end{block}

\end{frame}

\begin{frame}
  \frametitle{Exercise caution!}

  \begin{block}{Good}
    \emph{Mandy and Sanjeev} ate pizza.

    Both [Mandy ate pizza] and [Sanjeev ate pizza].
  \end{block}

  \begin{block}{\textcolor{red}{Bad}}
    \emph{Mandy and Sanjeev} ate the whole pizza.

    Both [Mandy ate the whole pizza] and [Sanjeev ate the whole pizza].
  \end{block}

\end{frame}

\begin{frame}
  \frametitle{Coordination of verb phrases}

  \begin{block}{Example}
  Mandy enjoys \emph{skiing or hiking}.

  Either [Mandy enjoys skiing] or [Mandy enjoys hiking].
  \end{block}

  \begin{block}{Example}
  If Sanjeev enjoys \emph{skiing and hiking}, he lives in Chicago.

  If [Sanjeev enjoys skiing] and [Sanjeev enjoys hiking], then [Sanjeev lives in Chicago].
  \end{block}

\end{frame}

\subsection{Basic symbolization}

\begin{frame}
  \frametitle{Negation}

  \begin{itemize}[<+->]
  \item \emph{Paraphrase} grammatical negation (``is not'', ``does not'') using the
  corresponding atomic sentence prefixed by ``\emph{it is not the case that}.''

  \item \emph{Symbolize} ``it is not the case that $A$'' as
  \emph{$\lnot A$}.

  \begin{block}{Example}
    \begin{itemize}[<+->]
    \item[] Mandy \emph{doesn't} enjoy skiing.
    \item[] It is not the case that [\emph{Mandy enjoys skiing}].
    \item[] \emph{It is not the case that} $S$.
    \item[] $\lnot S$
    \end{itemize}
  \end{block}
  \end{itemize}
\end{frame}

\begin{frame}
  \frametitle{Conjunction}

  \begin{itemize}[<+->]
  \item \emph{Paraphrase} sentences connected by ``and'', ``but'', ``even
  though'', ``yet'', and ``although'' using
  ``\emph{both $A$ and $B$}''

  \item \emph{Symbolize} ``both $A$ and $B$'' as
  \emph{$(A \land B)$}.

  \begin{block}{Example}
  \begin{itemize}[<+->]
  \item[] \emph{Even though} Mandy lives in Erie\emph{,} she enjoys hiking.

  \item[] Both [\emph{Mandy lives in Erie}] and [\emph{Mandy enjoys hiking}].

  \item[] \emph{Both} $E$ \emph{and} $H$.

  \item[] $(E \land H)$
  \end{itemize} 
  \end{block}
\end{itemize}

\end{frame}

\begin{frame}
  \frametitle{Disjunction}

  \begin{itemize}[<+->]
  \item \emph{Paraphrase} sentences connected by ``or'' using\\
  ``\emph{either $A$ or $B$}''

  \item \emph{Symbolize} ``either $A$ or $B$'' as
  \emph{$(A \lor B)$}.

  \begin{block}{Example}
  \begin{itemize}[<+->]
  \item[] Sanjeev lives in Chicago \emph{or} Erie.

  \item[] Either [\emph{Sanjeev lives in Chicago}] or [\emph{Sanjeev lives in Erie}].

  \item[] \emph{Either} $C$ \emph{or} $E$.

  \item[] $(C \lor E)$
  \end{itemize}
  \end{block}
  
  \item[] Ignore the suggestion that ``either \dots or \dots'' is exclusive.
  We'll always treat it as inclusive unless explicitly stated.
\end{itemize}

\end{frame}

\begin{frame}
  \frametitle{Conditional}

  \begin{itemize}[<+->]
  \item \emph{Paraphrase} using ``\emph{if $A$ then $B$}'' any sentence of the form:
  \begin{itemize}[<+->]
  \item ``if $A$, $B$''
  \item ``$B$ if $A$'' (note order is reversed!)
  \item ``$B$ provided $A$''
  \end{itemize}
  \item \emph{Symbolize} ``if $A$ then $B$'' as
  \emph{$(A \eif B)$}.

  \begin{block}{Example}
  \begin{itemize}[<+->]
  \item Mandy enjoys hiking \emph{if} Sanjeev lives in Chicago.

  \item If [\emph{Sanjeev lives in Chicago}] then [\emph{Mandy enjoys hiking}].

  \item \emph{If} $C$ \emph{then} $H$.

  \item $(C \eif H)$
  \end{itemize}
  \end{block}
  \end{itemize}
\end{frame}

\begin{frame}{The parts of a conditional}

\begin{itemize}
  \item \emph{$(A \eif B)$} symbolizes:
  \begin{itemize}
    \item ``if $A$, $B$''
    \item ``$B$ if $A$'' (note order is reversed!)
    \item ``$B$ provided $A$''
  \end{itemize}
  \item $A$ is the \emph{antecedent}: it symbolizes the condition that has to
  be met for the ``then'' part to apply.
  \item $B$ is the \emph{consequent}: it symbolizes what must be true
  if the antecedent condition is true.
\end{itemize}
\end{frame}

\begin{frame}
  \frametitle{Mix \& match}

  \begin{block}{Example}
    \begin{itemize}[<+->]
      \item[] Mandy doesn't enjoy hiking, \emph{provided} Sanjeev lives in Chicago or Erie.

  \item[] If [Sanjeev lives in Chicago \emph{or} Erie] then [Mandy \emph{doesn't} enjoy hiking].

  \item[] If [either [\emph{Sanjeev lives in Chicago}] or [\emph{Sanjeev lives in Erie}]]
  then [it is not the case that [\emph{Mandy enjoys hiking}]].

  \item[] If [either $C$ or $E$]
  then [it is not the case that $H$].

  \item[] $((C \lor E) \eif \lnot H)$
  \end{itemize}
  \end{block}

\end{frame}

\subsection{Conditionals}

\begin{frame}
  \frametitle{A logic puzzle}

\begin{itemize}
\item Every card has a letter on one side and a number on the other side.
\item You're a card inspector tasked with making sure that cards satisfy this quality standard:

\bigskip
\begin{quote}
If a card has an even number on one side, then it has a vowel on the other.
\end{quote}
\end{itemize}

\end{frame}

\frame<1>[label=wason]{
  \frametitle{A logic puzzle}

Which card(s) do you have to turn over to make sure that:
\bigskip

\begin{quote}
If a card has an even number on one side, then it has a vowel on the other.
\end{quote}

\begin{tabular}{cccc}
\begin{beamerboxesrounded}[width=5em]{}
\vskip 2ex
\hfil \Large E\hfil\\
\end{beamerboxesrounded} &
\begin{beamerboxesrounded}[width=5em]{}
\vskip 2ex
\hfil \Large \alert<2>{K}\hfil\\
\end{beamerboxesrounded} &
\begin{beamerboxesrounded}[width=5em]{}
\vskip 2ex
\hfil \Large 3\hfil\\
\end{beamerboxesrounded} &
\begin{beamerboxesrounded}[width=5em]{}
\vskip 2ex
\hfil \Large \alert<2>{4}\hfil\\
\end{beamerboxesrounded} \\
(1) & \alert<2>{(2)} & (3) & \alert<2>{(4)}
\end{tabular}

}

\begin{frame}
  \frametitle{Another logic puzzle}

\begin{itemize}
\item At an all-ages event where everyone has a drink
\item You know how old some of the people are, and you can tell what some of them are drinking
\item You're tasked with making sure that the following rule is followed:
\bigskip 

\begin{quote}
If a person is drinking alcohol, then they are at least 18 years old.
\end{quote}
\end{itemize}

\end{frame}

\begin{frame}
  \frametitle{Another logic puzzle}

Which of these people do you have to check (age or drink) to ensure
that:
\bigskip

\begin{quote}
If a person is drinking alcohol, then they must be at least 18 years old.
\end{quote}

\begin{tabular}{cccc}
\begin{beamerboxesrounded}[width=5em]{}
\vskip 2ex
\Large 22 years\\
\end{beamerboxesrounded} &
\begin{beamerboxesrounded}[width=5em]{}
\vskip 2ex
\Large \alert<2>{16 years}\\
\end{beamerboxesrounded} &
\begin{beamerboxesrounded}[width=5em]{}
\vskip 2ex
\Large drinks pop\\
\end{beamerboxesrounded} &
\begin{beamerboxesrounded}[width=5em]{}
\vskip 2ex
\Large \alert<2>{drinks beer}\\
\end{beamerboxesrounded} \\
(1) & \alert<2>{(2)} & (3) & \alert<2>{(4)}
\end{tabular}

\end{frame}

\againframe<2|handout:0>{wason}

\begin{frame}
  \frametitle{Truth conditions of conditionals}

\[\text{If\ } \underbrace{\text{X is drinking alcohol}}_{A},
\text{then\ }\underbrace{\text{X is over 18}}_{B}\]

\begin{itemize}[<+->]
\item ``If $A$, then $B$'' can only be \emph{false} if:
\begin{itemize}
\item $A$ is \emph{true}: we check age if X is drinking beer ($A$~true), not if
 drinking pop; \emph{and}
\item $B$ is \emph{false}: we check drink if X underage ($B$ false),\\ not
if over 18 ($B$ true)
\end{itemize}
\item ``If $A$, then $B$'' is true if:
\begin{itemize}
\item $A$ is \emph{false} (we don't check people drinking pop); \emph{or}
\item $B$ is \emph{true} (we don't card if X is over 18);
\item (or both)
\end{itemize}
\end{itemize}
\end{frame}


\subsection{``Only if'' and ``unless''}

\begin{frame}
  \frametitle{`If' and `only if'}

\begin{itemize}[<+->]
  \item Sue drinks beer ($A$) \emph{only if} she is over 18 ($B$)
  \[
  A \eif B
  \]
  \item False if Sue is underage, but drinks beer.
  \item Sue drinks beer ($A$) if she is over 18 ($B$).
  \[
  B \eif A
  \]
  \item False if she's 25 but drinks pop.
  \item Not false if she's 16 and drinking beer.
\end{itemize}
\end{frame}

\begin{frame}
  \frametitle{Conditional}

  \begin{itemize}[<+->]
  \item \emph{Paraphrase} ``$A$ only if $B$'' as
  ``\emph{if $A$ then $B$}''.
  \item \emph{Symbolize} ``$A$ only if $B$'' as
  \emph{$(A \eif B)$}.
  \item Note: 
    \begin{itemize} \item ``$A$ if $B$'' is $(B \eif A$)\\
  \item ``$A$ only if $B$'' is $(A \eif B$)
    \end{itemize}
  \item \emph{Symbolize} ``$A$ if and only if $B$'' as
  \emph{$(A \eiff B)$}.
\end{itemize}
\end{frame}

\begin{frame}
\frametitle{Unless}

Which of these people do you have to check (age or drink) to ensure that:
\begin{quote}
People are drinking pop unless they are over 18.
\end{quote}

\begin{tabular}{cccc}
\begin{beamerboxesrounded}[width=5em]{}
\vskip 2ex
\Large 22 years\\
\end{beamerboxesrounded} &
\begin{beamerboxesrounded}[width=5em]{}
\vskip 2ex
\Large 16 years\\
\end{beamerboxesrounded} &
\begin{beamerboxesrounded}[width=5em]{}
\vskip 2ex
\Large drinks pop\\
\end{beamerboxesrounded} &
\begin{beamerboxesrounded}[width=5em]{}
\vskip 2ex
\Large drinks beer\\
\end{beamerboxesrounded} \\
(1) & (2) & (3) & (4)
\end{tabular}

\end{frame}


\begin{frame}
\frametitle{Unless}

\[\underbrace{\text{X is drinking pop}}_{A},
\text{unless\ }\underbrace{\text{X is over 18}}_{B}\]

\begin{itemize}[<+->]
\item ``$A$ unless $B$'' can only be \emph{false} if:
\begin{itemize}
\item $A$ is \emph{false}\\
(we check age if person is drinking beer), \emph{and}
\item $B$ is \emph{false}\\
(we check drink if person not at least 18)
\end{itemize}
\item ``$A$ unless $B$'' is true (test OK) if
$A$ or $B$ or both are true.
\item ``$A$ unless $B$'' can be paraphrased and symbolized by:
\begin{itemize}
\item ``$A$ if not $B$'' ($\enot B \eif A$)
\item ``either $A$ or $B$'' ($A \eor B$)
\end{itemize}
\end{itemize}

\end{frame}

\begin{frame}
  \frametitle{Unless}

  Treat ``\emph{unless}'' the same way you would treat ``or''

  \begin{block}{Example}
  Mandy enjoys hiking unless Sanjeev lives in Chicago.

  $(H \eor C)$
  \end{block}
\end{frame}


\subsection{More connectives}

\begin{frame}
  \frametitle{If and only if}

  \begin{block}{Example}
    \begin{itemize}[<+->]
    \item[] Mandy enjoys hiking if and only if she enjoys skiing.
    \item[] Both [if $S$ then $H$] and [if $H$ then $S$].
    \item[] $((S \eif H)\eand(H \eif S))$
    \item[] $(H \eiff S)$
    \end{itemize}
  \end{block}

\end{frame}

\begin{frame}
  \frametitle{Exclusive or}

  \uncover<7->{\emph{Paraphrase} sentences containing ``either $A$ or $B$ but not both'' using\\
  ``\emph{both [either $A$ or $B$] and\\ \qquad [it is not the case
  that [both $A$ and $B$]]}''}

  \begin{block}{Example}
  Mandy enjoys \alert<1>{hiking or skiing} \alert<5>{but} \alert<3>{not both}.

  \uncover<6->{Both} \uncover<2->{[either $H$ or $S$]} \uncover<6->{and}\\ \phantom{Both} \uncover<4->{[it is not the case that [both
  $H$
  and $S$]].}

  \uncover<8>{$((H \eor S) \land \lnot(H \eand S))$}
  \end{block}

\end{frame}

\begin{frame}
  \frametitle{Neither \dots nor \dots}

  \uncover<2->{\emph{Paraphrase} sentences containing ``neither $A$ nor $B$'' using\\
  ``\emph{both [it is not the case that $A$] and\\ \qquad [it is not the case
  that $B$]}''}

  \begin{block}{Example}
  Mandy enjoys neither hiking nor skiing.

  \uncover<2->{Both [it is not the case that $H$] and\\ \qquad [it is not the case that $S$].}

  \uncover<3>{$(\lnot H \land \lnot S)$}
  \end{block}

\end{frame}

\begin{frame}
  \frametitle{Mix \& match}
  \begin{block}{Example}
  Sarah lives in Chicago or Erie.\\
  \uncover<2->{\emph{Either [Sarah lives in Chicago] or [Sarah lives in Erie].}}\\
  Amir lives in Chicago unless he enjoys hiking.\\
  \uncover<3->{\emph{Either [Amir lives in Chicago] or [Amir
    enjoys hiking].}}\\
  If Amir lives in Chicago, Sarah doesn't.\\
  \uncover<4->{\emph{If [Amir lives in Chicago] then [it is not
    the case that [Sarah lives in Chicago]].}}\\
  Neither Sarah nor Amir enjoy hiking.\\
  \uncover<5->{\emph{Both [it is not the case that [Sarah enjoys
      hiking]] and [it is not the case that [Amir enjoys hiking]].}}\\
  $\therefore$ Sarah lives in Erie.
  \end{block}
\end{frame}

\begin{frame}
  \frametitle{Mix \& match}

  \begin{block}{Example}
  Sarah lives in Chicago or Erie.\\
  \alert{$(C \eor E)$}\\
  Amir lives in Chicago unless he enjoys hiking.\\
  \alert{$(A \eor M)$}\\
  If Amir lives in Chicago, Sarah doesn't.\\
  \alert{$(A \eif \enot C)$}\\
  Neither Sarah nor Amir enjoy hiking.\\
  \alert{$( \enot S \eand\enot M)$}\\
  $\therefore$ Sarah lives in Erie.\\
  \alert{$\therefore\ E$}.
  \end{block}
\end{frame}

\subsection{Ambiguity}

\begin{frame}
    \frametitle{Types of ambiguity}

\begin{itemize}
  \item Lexical ambiguity: one word---many meanings \\
  e.g., ``bank'', ``crane''
  \item Syntactic ambiguity: one sentence---many readings\\
  e.g.,
  \begin{itemize}
  \item ``Flying planes can be dangerous'' (Chomsky)
  \item ``One morning I shot an elephant in my pajamas.\\ How he got in my pajamas, I don't know.'' (Groucho Marx)
  \end{itemize}
\end{itemize}

\end{frame}

\begin{frame}
  \frametitle{The man who was hanged by a comma}

\begin{columns}
\begin{column}{3cm}
\pgfimage[width=3cm]{../assets/casement}
\end{column}
\begin{column}{7cm}
\begin{itemize}
\item Sir Roger Casement (1864--1916)
\item British consul to Congo and Peru
\item Tried to recruit Irish revolutionaries in Germany during WWI
\item Tried for treason
\end{itemize}
\end{column}
\end{columns}

\end{frame}

\begin{frame}
  \frametitle{Treason Act of 1351}

\small
ITEM, Whereas divers Opinions have been before this Time in what Case
Treason shall be said, and in what not; the King, at the Request of
the Lords and of the Commons, hath made a Declaration in the Manner as
hereafter followeth, that is to say; When a Man doth compass or
imagine the Death of our Lord the King, or of our Lady his Queen or of
their eldest Son and Heir; or if a Man do violate the King's
Companion, or the King's eldest Daughter unmarried, or the Wife of the
King's eldest Son and Heir; or \emph{if a Man} do levy War against our Lord
the King in his Realm, or \emph{be adherent to the King's Enemies in his
Realm, giving to them Aid and Comfort in the Realm}\onslide<2>{\colorbox{highlightbg}{\textbf{,}}} \emph{or elsewhere}, and
thereof be probably attainted of open Deed by the People of their
Condition: \dots And it is to be
understood, that in the Cases above rehearsed, that ought to be judged
Treason which extends to our Lord the King, and his Royal Majesty:
\dots

\end{frame}

\begin{frame}
  \frametitle{R v. Casement in QL}
  \begin{itemize}[<+->]
\item Symbolization key:
  \begin{ekey}
    \item[A] Casement was adherent to the King's enemies in the realm.
    \item[G] Casement gave aid and comfort to the King's enemies in the realm.
    \item[B] Casement was adherent to the King's enemies abroad.
    \item[H] Casement gave aid and comfort to the King's enemies abroad.
  \end{ekey}
  \item Not treason:
  \begin{earg}
    \item[] $A \lor (G \lor H)$
  \end{earg} \item Treason:
  \begin{earg}
    \item[] $(A \lor B) \lor (G \lor H)$
  \end{earg}
\end{itemize}
\end{frame}

\begin{frame}
  \frametitle{Ambiguity of \eand{} and \eor}

  \begin{itemize}[<+->]
    \item English sentences don't have parentheses.
    \item This can lead to ambiguity, e.g.,
    \begin{earg}
      \item[] Ahmed admires Brit and Cara or Dina.
    \end{earg}
    \item It might mean one of:
    \begin{earg}
      \item[] Ahmed admires either [both Brit and Cara] or Dina.
      \item[] Ahmed admires both Brit and [either Cara or Dina].
    \end{earg}
    \item In SL, symbolizations are unambiguous:
    \begin{earg}
      \item[] $((B \eand C) \eor D)$
      \item[] $(B \eand (C \eor D))$
    \end{earg}
  \end{itemize}
\end{frame}
