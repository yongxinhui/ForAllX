% !TeX root = ./3-handout.tex

\setcounter{section}{2}
\section{Mathematical Induction \& Recursive Definitions}

\subsection{Intro to Math. Induction}

\frame{\frametitle{A Super-natural example!}
%\large
\begin{itemize}[<+->]

\item You're doing time in purgatory, and are automatically enrolled in philosophy (lucky you!)

\item Good news: To pass your class, you just need to produce one satisfactory piece of work

\item Bad news bears: your teacher is Gordon Ramsay, and he is never satisfied with any assignment

\item Whenever you turn something in, Ramsay returns it with suggestions for improvement, with revisions due tomorrow

\item How long should you expect to spend in this class?

\end{itemize}
}



\frame{\frametitle{Mathematical Induction for SL Sentences}
%\large

If you want to prove that \emph{ALL} sentences (wffs) of SL have a particular property, it suffices to show the following:

\begin{enumerate}[<+->]

\item All \emph{atomic} sentences have that property (`\textbf{base case}')

\item If \metav{A} and \metav{B} are two \emph{arbitrary wffs} with that property, then so are the sentences built out of them by adding a connective \\ \hspace{4em} `\textbf{Induction step}' (There are five cases to consider:) 

\begin{enumerate}[(i)]
  \item $\enot \metav{A}$
  \item $(\metav{A}\eand\metav{B})$ 
  \item $(\metav{A}\eor\metav{B})$ 
  \item $(\metav{A}\eif\metav{B})$
  \item $(\metav{A}\eiff\metav{B})$

\end{enumerate}

%\item \textit{Induction Step}: assume antecedent of point 2

\end{enumerate}
}


\frame{\frametitle{A Simple Example}
\large
\begin{itemize}

\item Prove by induction that every well-formed formulae (wff) of SL has exactly as many left parentheses as it has \\ right parentheses

\item Don't forget to explicitly state the \textbf{base case}

\item Don't forget to explicitly state the \textbf{Induction Step}!

\end{itemize}
}

\frame{\frametitle{Three More Examples!}
\large

Prove the following by induction. Don't forget to explicitly state the base case and the induction step! 

\begin{enumerate}[<+->]

\item No wff begins or ends with a binary connective

\item No wff contains two consecutive binary connectives (i.e. with no symbols between them)

\item If a wff doesn't contain any binary connectives, then it is contingent. \normalsize{(hint: say that a wff is \textit{baller} if it either contains a binary connective or is contingent. Use induction to show that every wff is baller.)}

\end{enumerate}
}




\frame{\frametitle{Two DISTINCT notions of `Induction' }
%\large
\begin{itemize}[<+->]

\item \emph{Mathematical Induction} is completely different from \emph{scientific induction} or what we earlier called `inductive arguments' (which we contrasted with `deductive arguments')

\item \emph{Mathematical Induction}: Rigorous method of proving that a property holds for an infinite number of cases. Follow the steps we've outlined!

\item \emph{Scientific Induction}: fallible method of supporting a claim, based on a finite number of previous observations.

  \begin{itemize}[<+->]

\item All of the ravens I have ever seen are black

\item Therefore, the next raven I see will be black 

\item Notice how your conclusion could be wrong! Albino ravens!

\end{itemize} 

\end{itemize}
}







\iffalse
\begin{frame}
\frametitle{Sentence letters and connectives}

  \begin{itemize}[<+->]
    \item d
    \emph{d} ($\enot$, $\eor$, $\eand$, $\eif$, $\eiff$)
  
  \begin{block}{blah}
    \begin{itemize}[<+->]
      \item[] d

  \item[] d

  \item[] d
\end{itemize} 
\end{block}

  \begin{definition}
  d
  \end{definition}


\end{itemize}
\end{frame}

\fi