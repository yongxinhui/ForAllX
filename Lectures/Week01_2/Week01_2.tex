\documentclass[a4paper, 11pt]{article} % Font size (can be 10pt, 11pt or 12pt) and paper size (remove a4paper for US letter paper)
\usepackage[protrusion=true,expansion=true]{microtype} % Better typography
\usepackage{../lecture} %calls local modified style file
\usepackage{graphicx} % Required for including pictures
\usepackage{wrapfig} % Allows in-line images
\usepackage{enumitem} %%Enables control over enumerate and itemize environments
\usepackage{setspace}
\usepackage{amssymb, amsmath, mathrsfs} %%Math packages
\usepackage{stmaryrd}
\usepackage{mathtools}
\usepackage{mathpazo} % Use the Palatino font
\usepackage[T1]{fontenc} % Required for accented characters
\usepackage{array}
\usepackage{bibentry}
\usepackage[round]{natbib} %%Or change 'round' to 'square' for square backers
\setcitestyle{aysep={}}

\makeatletter
\renewcommand{\maketitle}{
\begin{flushright}
{\LARGE\@title}

\vspace{10pt}

{\@author}
\\ \@date
\end{flushright}

\vspace{0pt}

}
\makeatother


%----------------------------------------------------------------------------------------
%	TITLE
%----------------------------------------------------------------------------------------

\title{\textbf{Regimentation}} % Subtitle

\author{\textsc{Logic I}\\ \em Benjamin Brast-McKie} % Institution

\date{\today} % Date

%----------------------------------------------------------------------------------------

\begin{document}

\maketitle % Print the title section

\thispagestyle{empty}

%----------------------------------------------------------------------------------------

\section*{From Last Time\ldots}

\begin{itemize}[leftmargin=1.5in,labelsep=.15in] %,label=(\arabic*)]%,label=\roman*]
  \item[\it Definitions:] Here is slightly different take on the same definitions:
  \item[\it Well-Formed Sentences:] The set \textsc{wfss} of $\PL$ is the smallest set to satisfy:
    \item $\metaA$ is a wfs of $\PL$ if $\metaA$ is a sentence letter of $\PL$;
    \item $\neg\metaA$ is a wfs of $\PL$ if $\metaA$ is a wfs of $\PL$;
    \item $(\metaA\eand\metaB)$ is a wff of $\PL$ if $\metaA$ and $\metaB$ are wfss of $\PL$;
    \item $(\metaA\eor\metaB)$ is a wff of $\PL$ if $\metaA$ and $\metaB$ are wfss of $\PL$;
    \item $(\metaA\eif\metaB)$ is a wff of $\PL$ if $\metaA$ and $\metaB$ are wfss of $\PL$; 
    \item $(\metaA\eiff\metaB)$ is a wff of $\PL$ if $\metaA$ and $\metaB$ are wfss of $\PL$.
  \item[\it Semantics:] For an interpretation $\I$, a \textsc{valuation} function $\V{\I}$ is the smallest function to assign truth-values to every sentence of SL that satisfies the semantic clauses:
    \item $\V{\I}(\metaA)=\I(\metaA)$ if $\metaA$ is a sentence letter of $\PL$.
    \item $\V{\I}(\neg\metaA)=1$ iff $\V{\I}(\metaA)=0$~~ (i.e., $\V{\I}(\metaA)\neq 1$).
    \item $\V{\I}(\metaA \eand \metaB)=1$ iff $\V{\I}(\metaA)=1$ and $\V{\I}(\metaB)=1$.
    \item $\V{\I}(\metaA \eor \metaB)=1$ iff $\V{\I}(\metaA)=1$ or $\V{\I}(\metaB)=1$ (or both).
    \item $\V{\I}(\metaA \eif \metaB)=1$ iff $\V{\I}(\metaA)=0$ or $\V{\I}(\metaB)=1$ (or both).
    \item $\V{\I}(\metaA \eiff \metaB)=1$ iff $\V{\I}(\metaA)=\V{\I}(\metaB)$.
  \item[\bf Observe:] Observe the symmetry between the above.
  \item[\it Recall:] The hierarchy of sentences from before\ldots
\end{itemize}




\section*{Complexity}

\begin{itemize}[leftmargin=1.5in,labelsep=.15in] %,label=(\arabic*)]%,label=\roman*]
  \item[\it Complexity:] $\comp{\metaA}$ is the smallest function to satisfy all of the following conditions for all wfss $\metaA$ and $\metaB$ of $\PL$: 
    \item $\comp{\metaA} = 0$ if $\metaA$ is a sentence letter; 
    \item $\comp{\enot\metaA} = \comp{\metaA} + 1$; 
    \item $\comp{\metaA \eand \metaB} = \comp{\metaA} + \comp{\metaB} + 1$; 
    \item $\ldots$
  \item[\bf Question:] Do we need to include corner quotes?
\end{itemize}




\section*{Validity}

\begin{itemize}[leftmargin=1.2in,labelsep=.15in] %,label=(\arabic*)]%,label=\roman*]
  \item[\it $\PL$ Validity:] An argument in SL is \textit{valid} iff its conclusion is a logical consequence of its premises.
  \item[\it English Validity:] An argument in English is \textit{valid} iff it has a (faithful) regimentation (in some language) that is valid.
    \item Note the imprecision here; there is no avoiding this.
  \item[\it Soundness:] An argument is \textit{sound} iff it is valid and has true premises (on an interpretation we care about, probably the intended interpretation).
\end{itemize}





\section*{Examples}


\subsection*{\it \textbf{Rain}}

\begin{enumerate}
  \eitem{If it is raining on a week day, Sam took his car.}
  \eitem{Kate borrowed Sam's car only if Sam did not take it.}
  \eitem{Kate borrowed Sam's car just in case she visited her parents.}
  \uitem{It is raining and Kate visited her parents.}
  \eitem{Either it is not a week day or it is not raining.}
\end{enumerate}

\noindent
\textbf{Task 2:} Regiment this argument and construct its truth table.
\vspace{.05in}

\noindent
\textit{Observe:} This argument can be adequately regimented and evaluate in SL.






\section*{Negation}

\subsection*{\it \textbf{Uninitiated}}

\begin{earg}
  \eitem{If Sam attended the gathering, then he has been initiated.}
  \uitem{Sam is uninitiated.}
  \eitem{Sam did not attend the gathering.}
\end{earg}

\begin{itemize}[leftmargin=1in,labelsep=.15in] %,label=(\arabic*)]%,label=\roman*]
  % \item[\bf Task:] Regiment this argument and construct its truth table.
  \item[\bf Observe:] Being uninitiated is the same as not being initiated.
\end{itemize}





\subsection*{\it \textbf{Uninvited}}

\begin{earg}
  \uitem{Arden is not invited.}
  \eitem{Arden is uninvited.}
\end{earg}

\begin{itemize}[leftmargin=1in,labelsep=.15in] %,label=(\arabic*)]%,label=\roman*]
  \item[\bf Observe:] Arden can fail to be invited without being uninvited.
  \item[\bf Question:] What about the converse?
\end{itemize}



\section*{Disjunction}

\subsection*{\it \textbf{Party}}

\begin{earg}
  \uitem{If Adi or James make it to the party, Isa will be happy. }
  \eitem{If Adi and James make it to the party, Isa will be happy. }
\end{earg}

\begin{itemize}[leftmargin=1in,labelsep=.15in] %,label=(\arabic*)]%,label=\roman*]
  \item[\bf Observe:] This argument suggests an inclusive reading of `or'.
\end{itemize}





\subsection*{\it \textbf{Race}}

\begin{earg}
  \eitem{Sasha won the 100 meter dash. }
  \uitem{Josh won the high jump. }
  \eitem{Either Sasha won the 100 meter dash or Josh won the high jump}
\end{earg}

\begin{itemize}[leftmargin=1in,labelsep=.15in] %,label=(\arabic*)]%,label=\roman*]
  \item[\bf Observe:] We could strengthen the conclusion.
\end{itemize}





\subsection*{\it \textbf{Vault}}

\begin{earg}
  \eitem{If Kin uses the remote, the trunk will open. }
  \eitem{If Yu tries the handle, the trunk will open.}
  \uitem{If Kin uses the remote and Yu tries the handle, the trunk won't open.}
  \eitem{If Kin uses the remote or Yu tries the handle, the trunk will open. }
\end{earg}

\begin{itemize}[leftmargin=1in,labelsep=.15in] %,label=(\arabic*)]%,label=\roman*]
  \item[\bf Observe:] We cannot regiment the conclusion with inclusive-`or'.
  \item[\bf Question:] Can we salvage the validity of this argument?
\end{itemize}





\section*{Conjunction}

\subsubsection*{\it \textbf{Exam}}

\begin{earg}
  \uitem{Henry failed and Megan passed.}
  \eitem{Megan passed and Henry failed.}
\end{earg}

\begin{itemize}[leftmargin=1in,labelsep=.15in] %,label=(\arabic*)]%,label=\roman*]
  \item[\bf Observe:] Perfectly adequate and valid regimentation.
\end{itemize}




\subsubsection*{\it \textbf{Gym}}

\begin{earg}
  \uitem{Kate took a shower and went to the gym.}
  \eitem{Kate went to the gym and took a shower.}
\end{earg}

\begin{itemize}[leftmargin=1in,labelsep=.15in] %,label=(\arabic*)]%,label=\roman*]
  \item[\bf Observe:] Conjunction in English can track temporal order.
  \item[\bf Question:] How can we capture the invalidity of this argument in $\PL$?
\end{itemize}







% \vfill
%
% \bibliographystyle{Phil_Review} %%bib style found in bst folder, in bibtex folder, in texmf folder.
% \bibliography{Zotero} %%bib database found in bib folder, in bibtex folder


\end{document}

