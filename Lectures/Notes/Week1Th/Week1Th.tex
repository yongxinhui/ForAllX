\documentclass[a4paper, 11pt]{article} % Font size (can be 10pt, 11pt or 12pt) and paper size (remove a4paper for US letter paper)
\usepackage[protrusion=true,expansion=true]{microtype} % Better typography
\usepackage{graphicx} % Required for including pictures
\usepackage{wrapfig} % Allows in-line images
\usepackage{enumitem} %%Enables control over enumerate and itemize environments
\usepackage{setspace}
\usepackage{amssymb, amsmath, mathrsfs} %%Math packages
\usepackage{stmaryrd}
\usepackage{mathtools}
\usepackage{mathpazo} % Use the Palatino font
\usepackage[T1]{fontenc} % Required for accented characters
\usepackage{array}
\usepackage{bibentry}
\usepackage[round]{natbib} %%Or change 'round' to 'square' for square backers
\setcitestyle{aysep={}}

% \linespread{1} % Change line spacing here, Palatino benefits from a slight increase by default

\newcommand{\corner}[1]{\ulcorner#1\urcorner} %%Corner quotes
\newcommand{\tuple}[1]{\langle#1\rangle} %%Angle brackets
\newcommand{\set}[1]{\lbrace#1\rbrace} %%Set brackets
\newcommand{\interpret}[1]{\llbracket#1\rrbracket} %%Double brackets
%\DeclarePairedDelimiter\ceil{\lceil}{\rceil}    
\def\therefore{\ensuremath{\ldotp\dot{}\,\ldotp}}
\newcommand{\I}{\mathcal{I}}
\newcommand{\V}[1]{\mathcal{V}_{#1}} %%Corner quotes

\makeatletter
\renewcommand\@biblabel[1]{\textbf{#1.}} % Change the square brackets for each bibliography item from '[1]' to '1.'
\renewcommand{\@listI}{\itemsep=0pt} % Reduce the space between items in the itemize and enumerate environments and the bibliography

\renewcommand{\maketitle}{ % Customize the title - do not edit title and author name here, see the TITLE block below
\begin{flushright} % Right align
{\LARGE\@title} % Increase the font size of the title

\vspace{10pt} % Some vertical space between the title and author name

{\@author} % Author name
\\\@date % Date

\vspace{-10pt} % Some vertical space between the author block and abstract
\end{flushright}
}

%----------------------------------------------------------------------------------------
%	TITLE
%----------------------------------------------------------------------------------------

\title{\textbf{The Connectives}} % Subtitle

\author{\textsc{Logic I}\\ \em Benjamin Brast-McKie} % Institution

\date{\today} % Date

%----------------------------------------------------------------------------------------

\begin{document}

\maketitle % Print the title section

\thispagestyle{empty}

%----------------------------------------------------------------------------------------

\section*{Definitions}

\begin{enumerate}[leftmargin=1.5in,labelsep=.15in] %,label=(\arabic*)]%,label=\roman*]
  \item[\it Previously:] We considered the sentences that could be constructed from the sentence letters with the connectives. We will now seek to specify this construction precisely.
  \item[\it Object Language:] We will be concerned to define the sentences of SL, where the language of SL $=\tuple{\mathbb{L},\neg,\wedge,\vee,\supset,\equiv,(,)}$ will be referred to as the \textsc{object language}.
  \item[\it Strings:] An \textsc{expression} of SL is any finite string of symbols from the language of SL.
  \item[\it Quotation:] To talk about strings we will need to name them, where a quoted string is the \textsc{canonical name} for the string quoted.
  \item[\it Use/Mention:] We mention expressions by putting them in quotes, where otherwise they are used.
  \item[\bf Example 1:] `Sue' is a three letter name but Sue is not.
  \item[\bf Example 2:] The complex sentence `$A \supset B$' includes the sentence letters `$A$' and `$B$'.
  \item[\it Metalanguage:] We talk about the expressions of SL with the resources of our \textsc{metalangauge} mathematical English.
  \item[\it Metalinguistic Variables:] Letting $\varphi,\psi,\chi,\ldots$ be variables whose values are expressions, we may quantify over the expressions of SL in order to define the wffs of SL.
\end{enumerate}




\section*{The Sentences of SL}

\begin{enumerate}[leftmargin=1.5in,labelsep=.15in] %,label=(\arabic*)]%,label=\roman*]
  \item Every atomic sentence in $\mathbb{L}$ is a wff of SL.
  \item If $\varphi$ and $\psi$ are wffs of SL, then:
    \begin{enumerate}
      \item $\neg\varphi$ is a wff of SL;
      \item $(\varphi\wedge\psi)$ is a wff of SL;
      \item $(\varphi\vee\psi)$ is a wff of SL;
      \item $(\varphi\supset\psi)$ is a wff of SL; and
      \item $(\varphi\equiv\psi)$ is a wff of SL.
    \end{enumerate}
  \item Nothing else is a wff of SL.
\end{enumerate}




\section*{Observations and Conventions}

\begin{enumerate}[leftmargin=1.5in,labelsep=.15in] %,label=(\arabic*)]%,label=\roman*]
  \item[\it Corner Quotes:] Strictly speaking, this definition is non-sense and we need to use corner quotes to fix it.
  \item[\it Well-formed Formulas:] The wffs are the grammatical expressions of SL of type $t$, and so candidates for interpretation.
  \item[\it Sentences:] Since all wffs in SL are \textit{good} candidates for interpretation (it makes sense to assign them truth-values), we may identify the wffs with the sentences of SL. By contrast, not all the wffs of QL are sentences of QL.
  \item[\it Sentential Variables:] Restrict $\varphi,\psi,\chi,\ldots$ to sentences of SL.
  \item[\bf Task 1:] Build increasingly complex sentences from just $A$.
  \item[\it Conventions:] We will often drop quotes and parentheses for ease: $A\vee B\vee C$ vs $A\vee B\wedge C$.
  \item[\it Therefore:] $\therefore$ is not part of SL.
\end{enumerate}






\section*{Truth Functionality}

\begin{enumerate}[leftmargin=1.5in,labelsep=.15in] %,label=(\arabic*)]%,label=\roman*]
  \item[\it Sentential Operators:] The connectives are \textsc{sentential operators} which map sentences to sentences.
  \item[\it Interpretations:] Last time we said that an \textsc{interpretation} $\I$ assigns truth-values to sentence letters.
  \item[\it Valuation:] We may then define a \textsc{valuation} function $\V{\I}$ which assigns truth-values to every sentence of SL by way of the following semantic clauses:
    \item[($A$)] $\V{\I}(\varphi)=\I(\varphi)$ iff $\varphi$ is a sentence letter of SL.
    \item[($\neg$)] $\V{\I}(\neg\varphi)=1$ iff $\V{\I}(\varphi)=0$ (i.e., $\V{\I}(\varphi)\neq 1$).
    \item[($\wedge$)] $\V{\I}(\varphi \wedge \psi)=1$ iff $\V{\I}(\varphi)=1$ and $\V{\I}(\psi)=1$.
    \item[($\vee$)] $\V{\I}(\varphi \vee \psi)=1$ iff $\V{\I}(\varphi)=1$ or $\V{\I}(\psi)=1$ (or both).
    \item[($\supset$)] $\V{\I}(\varphi \supset \psi)=1$ iff $\V{\I}(\varphi)=0$ or $\V{\I}(\psi)=1$ (or both).
    \item[($\equiv$)] $\V{\I}(\varphi \equiv \psi)=1$ iff $\V{\I}(\varphi)=\V{\I}(\psi)$.
  \item[\it Homophonic Semantics:] The semantics for $\neg$, $\wedge$, and $\vee$ use analogous operators in the metalanguage, but not so for $\supset$ and $\equiv$. 
  \item[\it Truth Functional:] $\V{\I}(\neg\varphi)=1-\V{\I}(\varphi)$;\\
    $\V{\I}(\varphi\wedge\psi)=\V{\I}(\varphi)\times\V{\I}(\psi)$.
  \item[\sc Homework:] Given an interpretation $\I$, specify the truth-values of $\varphi\vee\psi$, $\varphi\supset\psi$, and $\varphi\equiv\psi$ as a function of the truth-values of $\varphi$ and $\psi$ in a similar fashion as above.
    % $\V{\I}(\varphi\vee\psi)=1-([1-\V{\I}(\varphi)]\times[1-\V{\I}(\psi)])$;\\
    % $\V{\I}(\varphi\supset\psi)=1-(\V{\I}(\varphi)\times[1-\V{\I}(\psi)])$;\\
    % \mbox{$\V{\I}(\varphi\equiv\psi)=[1-(\V{\I}(\varphi)\times[1-\V{\I}(\psi)])]\times[1-(\V{\I}(\psi)\times[1-\V{\I}(\varphi)])]$.}
  \item[\bf Task 2:] How many unary/binary truth-functions are there?
\end{enumerate}



\section*{Negation}

\subsection*{\it \textbf{Uninitiated}}

\begin{enumerate}
  \item[(1)] If Sam attended the gathering, then he has been initiated.
  \item[(2)] Sam is uninitiated.
  \item[\therefore] Sam did not attend the gathering.
\end{enumerate}

\noindent
\textit{Observe:} Being uninitiated is the same as not being initiated.





\subsection*{\it \textbf{Uninvited}}

\begin{enumerate}
  \item[(1)] Arden is not invited.
  \item[\therefore] Arden is uninvited.
\end{enumerate}

\noindent
\textit{Observe:} Arden can fail to be invited without being uninvited.
\vspace{.05in}

\noindent
\textit{Question:} What about the converse?





\section*{Conjunction}

\subsection*{\it \textbf{Gym}}

\begin{enumerate}
  \item[(1)] Kate took a shower and went to the gym.
  \item[\therefore] Kate went to the gym and took a shower.
\end{enumerate}

\noindent
\textit{Observe:} Conjunction in English can track temporal order.
\vspace{.05in}

\noindent
\textit{Question:} How can we capture the invalidity of this argument in SL?




\section*{Disjunction}

\subsection*{\it \textbf{Vault}}

\begin{enumerate}
  \item[(1)] If button $A$ is pressed, the vault will open. 
  \item[(2)] If button $B$ is pressed, the vault will open.
  \item[(3)] If buttons $A$ and $B$ are pressed, the vault will not open.
  \item[\therefore] If either button $A$ or $B$ are pressed, the vault will open. 
\end{enumerate}

\noindent
\textit{Observe:} We cannot regiment the conclusion with inclusive-`or'.
\vspace{.05in}

\noindent
\textit{Question:} Can we salvage the validity of this argument?



\section*{The Material Conditional}

\subsection*{\it \textbf{Square}}

\begin{enumerate}
  \item[(1)] The rug is square.
  \item[\therefore] If the rug is triangular, then it is square.
\end{enumerate}

\noindent
\textit{Regiment:} Use $S$ and $T$ to regiment this argument. 
\vspace{.05in}

\noindent
\textit{Observe:} It is natural to provide a modal reading for the conclusion.
\vspace{.05in}

\noindent
\textit{Conclude:} The material conditional does not track possibilities.




\subsection*{\it \textbf{Roses}}

\begin{enumerate}
  \item[(1)] Sugar is sweet.
  \item[\therefore] Roses are only red if sugar is sweet.
\end{enumerate}




\subsection*{\it \textbf{Vacation}}

\begin{enumerate}
  \item[(1)] Casey is not on vacation.
  \item[\therefore] If Casey is on vacation, then he is in Paris.
\end{enumerate}





\subsection*{\it \textbf{Crimson}}

\begin{enumerate}
  \item[(1)] Mary only likes the ball if it is crimson.
  \item[(2)] Mary likes the ball.
  \item[\therefore] If the ball is blue, then Mary likes it.
\end{enumerate}






\section*{Biconditional}

\subsection*{\it \textbf{Rectangle}}

\begin{enumerate}
  \item[(1)] The room is a square.
  \item[(2)] The room is a rectangle.
  \item[\therefore] The room is a square if and only if it is a rectangle.
\end{enumerate}





\subsection*{\it \textbf{Roses}}

\begin{enumerate}
  \item[(1)] Kin isn't a professor.
  \item[(2)] Sue isn't a chef.
  \item[\therefore] Kin is a professor just in case Sue is a chef.
\end{enumerate}



% \vfill
%
% \bibliographystyle{Phil_Review} %%bib style found in bst folder, in bibtex folder, in texmf folder.
% \bibliography{Zotero} %%bib database found in bib folder, in bibtex folder


\end{document}
