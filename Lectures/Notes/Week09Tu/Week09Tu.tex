\documentclass[a4paper, 11pt]{article} % Font size (can be 10pt, 11pt or 12pt) and paper size (remove a4paper for US letter paper)
\usepackage[protrusion=true,expansion=true]{microtype} % Better typography
\usepackage{graphicx} % Required for including pictures
\usepackage{wrapfig} % Allows in-line images
\usepackage{enumitem} %%Enables control over enumerate and itemize environments
\usepackage{setspace}
\usepackage{amssymb, amsmath, mathrsfs} %%Math packages
\usepackage{stmaryrd}
\usepackage{mathtools}
\usepackage{multicol} 
\usepackage{mathpazo} % Use the Palatino font
\usepackage[T1]{fontenc} % Required for accented characters
\usepackage{array}
\usepackage{bibentry}
\usepackage{prooftrees} 
\usepackage[round]{natbib} %%Or change 'round' to 'square' for square backers
\setcitestyle{aysep={}}
% \usepackage{fitchproof} 

% \linespread{1} % Change line spacing here, Palatino benefits from a slight increase by default

\newcommand{\corner}[1]{\ulcorner#1\urcorner} %%Corner quotes
\newcommand{\tuple}[1]{\langle#1\rangle} %%Angle brackets
\newcommand{\set}[1]{\lbrace#1\rbrace} %%Set brackets
\newcommand{\interpret}[1]{\llbracket#1\rrbracket} %%Double brackets
%\DeclarePairedDelimiter\ceil{\lceil}{\rceil}    
\def\therefore{\ensuremath{\ldotp\dot{}\,\ldotp}}
\newcommand{\I}{\mathcal{I}}
\newcommand{\J}{\mathcal{J}}
\newcommand{\B}{\mathcal{B}}
\newcommand{\D}{\mathbb{D}}
\renewcommand{\v}[1]{\mathbf{#1}}
\newcommand{\even}{\texttt{Even}}
\newcommand{\comp}{\texttt{Comp}}
\newcommand{\res}{\texttt{Res}}
\newcommand{\simp}{\texttt{Simple}}
\newcommand{\leng}{\texttt{Length}}
\newcommand{\V}[1]{\mathcal{V}_{#1}} %%Corner quotes

\makeatletter
\renewcommand\@biblabel[1]{\textbf{#1.}} % Change the square brackets for each bibliography item from '[1]' to '1.'
\renewcommand{\@listI}{\itemsep=0pt} % Reduce the space between items in the itemize and enumerate environments and the bibliography

\renewcommand{\maketitle}{ % Customize the title - do not edit title and author name here, see the TITLE block below
\begin{flushright} % Right align
{\LARGE\@title} % Increase the font size of the title

\vspace{10pt} % Some vertical space between the title and author name

{\@author} % Author name
\\\@date % Date

\vspace{30pt} % Some vertical space between the author block and abstract
\end{flushright}
}

%----------------------------------------------------------------------------------------
%	TITLE
%----------------------------------------------------------------------------------------

\title{\textbf{The Semantics for QL}} % Subtitle

\author{\textsc{Logic I}\\ \em Benjamin Brast-McKie} % Institution

\date{\today} % Date

%----------------------------------------------------------------------------------------

\begin{document}

\maketitle % Print the title section

\thispagestyle{empty}

%----------------------------------------------------------------------------------------

\section*{Examples}%
  \label{sec:Examples}
  
\begin{enumerate}
  \item[\it Monadic:] Casey is dancing.
  \item[\it Dyadic:] Al loves Max.
  \item[\it Triadic:] Kim is between Boston and New York.
\end{enumerate}

\section*{Extensions and Referents}%
  \label{sec:Extensions and Referents}
  
\begin{enumerate}
  \item[\it Constants:] Constants are interpreted as referring to individuals.
  \item[\it Existence:] Thus we need to know what things there are.
  \item[\it Domain:] A \textit{domain} is any nonempty set $\D$.
  \item[\it Referents:] Interpretations assign constants to elements of $\D$.
  \item[\bf Question 1:] How are we going to interpret predicates?
\end{enumerate}



\section*{Extensions}%
  \label{sec:Extensions}
  
\begin{enumerate}
  \item[\it Example:] `Al loves Max' is true \textit{iff} Al bears the loves-relation to Max.
  \item[\it Ordered Pairs:] A dyadic predicate is interpreted by a set of \textit{ordered pairs}.
  \item[\it Cartesian Product:] Dyadic predicates are interpreted by subsets of $\D^2$.
  \item[\bf Question 2:] How are we to interpret $n$-place predicates?
  \item[\it Extensions:] $n$-place predicates are interpreted by subsets of $\D^n$.
  \item[\bf Question 3:] How are we to interpret $0$-place and $1$-place predicates?
  \item[\it Singletons:] $1$-place predicates are interpreted by subsets of $\D^1=\set{\tuple{\v{x}}:\v{x}\in\D}$.
  \item[\it ]
\end{enumerate}


\section*{Restricting Quantifiers}

\begin{enumerate}
  \item[\it Regimentation:] Regiment the following sentences:
  \begin{itemize}
    \item Jim took every chance he got.
    \item At least the guests that remained were pleased with the party.
    \item I haven't met a cat that likes Merra.
    \item Kate found a job that she loved.
    \item Everybody everybody loves loves somebody.
    \item No set is a member of itself.
    \item There is a set with no members.
  \end{itemize}
\end{enumerate}



\section*{Arguments}

\begin{enumerate}
  \item[\it Love:] Regiment the following argument:
    \begin{itemize}
      \item Cam doesn't love anyone who loves him back.
      \item May loves everyone who loves themselves.
      \item[\therefore] If Cam loves himself, he doesn't love May.
    \end{itemize}
  \item[\it Bigger:] Regiment the following argument:
    \begin{itemize}
      \item Whenever something is bigger than another, the latter is not bigger than the former.
      \item[\therefore] Nothing is bigger than itself.
    \end{itemize}
  % \item[\it Gunk] Regiment the following argument:
  %   \item Nothing is a part of itself.
  %   \item Whenever one thing is bigger than a second thing, and the second thing is bigger than a third thing, then the first thing is bigger than the third thing. 
  %   \item[\therefore] Whenever something is bigger than a second thing, the second thing is not bigger than the first.
  %   % \item[\therefore] Nothing is bigger 
\end{enumerate}


\section*{Relations}

\begin{enumerate}
  \item[\it Domain:] Let the \textit{domain} $D$ be any set.
  \item[\it Relation:] A \textit{relation} $R$ on $D$ is any subset of $D^2$.
  \item[\it Reflexive:] A relation $R$ is \textit{reflexive} on $D$ \textit{iff} $\tuple{x,x}\in R$ for all $x\in D$.
  \item[\it Non-Reflexive:] A relation $R$ is \textit{non-reflexive} on $D$ \textit{iff} $R$ is not reflexive on $D$.
  \item[\bf Question 1:] What is it to be \textit{irreflexive}?
  \item[\it Irreflexive:] A relation $R$ is \textit{irreflexive} on $D$ \textit{iff} $\tuple{x,x}\notin R$ for all $x\in D$.
  \item[\it Symmetric:] A relation $R$ is \textit{symmetric iff} $\tuple{y,x}\in R$ whenever ${x,y}\in R$.
  \item[\bf Question 2:] Why don't we need to specify a domain?
  \item[\bf Question 3:] Why is a relation reflexive or irreflexive with respect to a domain?
  \item[\it Asymmetric:] A relation $R$ is \textit{asymmetric iff} $\tuple{y,x}\notin R$ whenever $\tuple{x,y}\in R$.
  \item[\bf Question 4:] What is it to be non-symmetric? How about non-asymmetric?
  \item[\bf Task 1:] Show that every asymmetric relation is irreflexive.
  \item[\it Transitive:] A relation $R$ is \textit{transitive iff} $\tuple{x,z}\in R$ whenever $\tuple{x,y},\tuple{y,z}\in R$.
  \item[\it Intransitive:] A relation $R$ is \textit{intransitive iff} $\tuple{x,z}\notin R$ whenever $\tuple{x,y},\tuple{y,z}\in R$.
  \item[\bf Question 5:] Is every symmetric transitive relation reflexive? (No: $R=\varnothing$)
  \item[\bf Task 2:] Show that every transitive irreflexive relation asymmetric?
  \item[\it Euclidean:] A relation $R$ is \textit{euclidean iff} $\tuple{y,z}\in R$ whenever $\tuple{x,y},\tuple{x,z}\in R$.
  \item[\bf Task 3:] Show that every transitive symmetric relation is euclidean.
\end{enumerate}


  % \item Diamonds last forever.




\end{document}

