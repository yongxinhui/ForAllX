\documentclass[a4paper, 11pt]{article} % Font size (can be 10pt, 11pt or 12pt) and paper size (remove a4paper for US letter paper)
\usepackage[protrusion=true,expansion=true]{microtype} % Better typography
\usepackage{graphicx} % Required for including pictures
\usepackage{wrapfig} % Allows in-line images
\usepackage{enumitem} %%Enables control over enumerate and itemize environments
\usepackage{setspace}
\usepackage{amssymb, amsmath, mathrsfs} %%Math packages
\usepackage{stmaryrd}
\usepackage{mathtools}
\usepackage{multicol} 
\usepackage{mathpazo} % Use the Palatino font
\usepackage[T1]{fontenc} % Required for accented characters
\usepackage{array}
\usepackage{bibentry}
\usepackage{prooftrees} 
\usepackage[round]{natbib} %%Or change 'round' to 'square' for square backers
\setcitestyle{aysep=}
% \usepackage{fitchproof} 

% \linespread{1} % Change line spacing here, Palatino benefits from a slight increase by default

\newcommand{\corner}[1]{\ulcorner#1\urcorner} %%Corner quotes
\newcommand{\tuple}[1]{\langle#1\rangle} %%Angle brackets
\newcommand{\set}[1]{\lbrace#1\rbrace} %%Set brackets
\newcommand{\interpret}[1]{\llbracket#1\rrbracket} %%Double brackets
%\DeclarePairedDelimiter\ceil{\lceil}{\rceil}    
\def\therefore{\ensuremath{\ldotp\dot\,\ldotp}}
\newcommand{\I}{\mathcal{I}}
\newcommand{\J}{\mathcal{J}}
\newcommand{\B}{\mathcal{B}}
\newcommand{\F}{\mathcal{F}}
\newcommand{\M}{\mathcal{M}}
\newcommand{\D}{\mathbb{D}}
\renewcommand{\v}[1]{\mathbf{#1}}
\newcommand{\even}{\texttt{Even}}
\newcommand{\comp}{\texttt{Comp}}
\newcommand{\res}{\texttt{Res}}
\newcommand{\simp}{\texttt{Simple}}
\newcommand{\leng}{\texttt{Length}}
\newcommand{\V}[1]{\mathcal{V}_{#1}} %%Corner quotes
\newcommand{\VV}[2]{\mathcal{V}_{#1}^{#2}} %%

\makeatletter
\renewcommand\@biblabel[1]{\textbf{#1.}} % Change the square brackets for each bibliography item from '[1]' to '1.'
\renewcommand{\@listI}{\itemsep=0pt} % Reduce the space between items in the itemize and enumerate environments and the bibliography

\renewcommand{\maketitle}{ % Customize the title - do not edit title and author name here, see the TITLE block below
\begin{flushright} % Right align
{\LARGE\@title} % Increase the font size of the title

\vspace{10pt} % Some vertical space between the title and author name

{\@author} % Author name
\\\@date % Date

\vspace{30pt} % Some vertical space between the author block and abstract
\end{flushright}
}

%----------------------------------------------------------------------------------------
%	TITLE
%----------------------------------------------------------------------------------------

\title{\textbf{The Semantics for QL}} % Subtitle

\author{\textsc{Logic I}\\ \em Benjamin Brast-McKie} % Institution

\date{\today} % Date

%----------------------------------------------------------------------------------------

\begin{document}

\maketitle % Print the title section

\thispagestyle{empty}

%----------------------------------------------------------------------------------------

\section*{Examples}%
  \label{sec:Examples}
  
\begin{enumerate}
  \item[\it Monadic:] Casey is dancing.
  \item[\it Dyadic:] Al loves Max.
  \item[\it Triadic:] Kim is between Boston and New York.
\end{enumerate}

\section*{Constants and Referents}%
  \label{sec:Extensions and Referents}
  
\begin{enumerate}
  \item[\it Constants:] Constants are interpreted as referring to individuals.
  \item[\it Existence:] Thus we need to know what things there are.
  \item[\it Domain:] A \textit{domain} is any nonempty set $\D$.
  \item[\it Referents:] Interpretations assign constants to elements of $\D$.
  \item[\bf Question 1:] How are we going to interpret predicates?
\end{enumerate}



\section*{Predicates and Extensions}%
  \label{sec:Extensions}
  
\begin{enumerate}
  \item[\it Example:] `Al loves Max' is true \textit{iff} Al bears the loves-relation to Max.
  \item[\it Dyadic Predicates:] Dyadic predicates are interpreted by sets of \textit{ordered pairs} in $\D^2$.
  \item[\bf Question 2:] How are we to interpret $n$-place predicates?
  \item[\it Cartesian Power:] $\D^n=\set{\tuple{\v{x}_1,\ldots,\v{x}_n}:\v{x}_1,\ldots,\v{x}_n\in\D}$.
  \item[\it Extensions:] $n$-place predicates are interpreted by subsets of $\D^n$.
  \item[\it Singletons:] $1$-place predicates are interpreted by subsets of $\D^1=\set{\tuple{\v{x}}:\v{x}\in\D}$.
  \item[\bf Question 3:] How are we to interpret $0$-place predicates? What is $\D^0$?
    \begin{itemize}
      % \item[\it Ordered Pairs:] An ordered pair $\tuple{\v{x}_1,\v{x}_2}=\set{\set{\v{x}_1},\set{\v{x}_1,\v{x}_2}}$.
      \item[\it $n$-Tuples:] Let $\tuple{\v{x}_1,\ldots,\v{x}_n}=\set{\tuple{1,\v{x}_1},\ldots,\tuple{n,\v{x}_n}}$.
      \item[\it $0$-Tuple:] $\tuple{}=\varnothing$.
        % \item\tuple{\tuple{\v{x}_1,\ldots,\v{x}_{n-1}},\v{x}_n}$.
        % \item\tuple{\v{x}_1,\tuple{\v{x}_2,\ldots,\tuple{\v{x}_n,\varnothing}}}$.
      % \item[\it $0$-Tuple:] $\tuple=\varnothing$ is the $0$-tuple.
        % \item $\tuple{\v{x}_1}=\set{1,\v{x}_1}$ is a $1$-tuple.
        % \item $\tuple{\v{x}_1,\ldots,\v{x}_n}=\tuple{\vec{\v{X}}_{n-1},\v{x}_n}$ is an $n+1$-tuple if $X_n$ is a $n$-tuple. 
      \end{itemize}
  \item[\it Truth-Values:] $0$-place predicates are interpreted by subsets of $\D^0=\set{\varnothing}$.
  \item[\it Ordinals:] Let $1=\set{\varnothing}$ and $0=\varnothing$ be the first two von Neumann ordinals.
\end{enumerate}



\section*{QL Models}

\begin{enumerate}
  \item[\it Interpretations:] $\I$ is an QL interpretation over $\D$ \textit{iff} both: 
    \begin{itemize}
      \item $\I(\alpha)\in\D$ for every constant $\alpha$ in QL. 
      \item $\I(\F^n)\subseteq\D^n$ for every $n$-place predicate $\F^n$.
    \end{itemize}
  \item[\bf Question 4:] What happens if $\D=\varnothing$?
  \item[\it Model:] $\M=\tuple{\D,\I}$ is a model of QL \textit{iff} $\I$ is a QL interpretation over $\D\neq\varnothing$.
  \item[\bf Task 1:] Regiment and interpret the sentences above.
    \begin{itemize}
      \item $Dc$, $Lam$, $Bkbn$.
      \item $\D=\set{c,a,m,k,b,n}$.
      \item $\I(D)=\set{\tuple{c}}$.
      \item $\I(L)=\set{\tuple{a,m}}$.
      \item $\I(B)=\set{\tuple{k,b,n}}$.
      \item $\I(c)=c,\ \I(a)=a,\ \ldots$
    \end{itemize}
  \item[\it Lagadonian:] We often take constants to name themselves.
  \item[\bf Question 5:] Do models give us truth-values?
\end{enumerate}



\section*{Variable Assignments}

\begin{enumerate}
  \item[\it Assignments:] A variable assignment $\hat{a}(\alpha)\in\D$ for every variable $\alpha$ in QL.
  \item[\it Singular Terms:] We may define the referent of $\alpha$ in $\M=\tuple{\D,\I}$ as follows:\\
    \begin{align*}
      \VV{\I}{\hat{a}}{(\alpha)}=
        \begin{cases}
          \I(\alpha) & \text{if } \alpha \text{ is a constant} \\
          \hat{a}(\alpha) & \text{if } \alpha \text{ is a variable.}
        \end{cases}
    \end{align*}
  \item[\it Variants:] A $\hat{c}$ is an $\alpha$-variant of $\hat{a}$ \textit{iff} $\hat{c}(\beta)=\hat{a}(\beta)$ for all $\beta\neq\alpha$.
  \item[\it Example:] Let $\D=\set{1,2,3,4,5}$ where $\hat{a}(x)=1$, $\hat{a}(y)=2$, and $\hat{a}(z)=3$.
    \begin{itemize}
      \item[\bf Task 2:] If $\hat{c}$ is a $y$-variant of $\hat{a}$, what is $\hat{c}(1)$, $\hat{c}(2)$, and $\hat{c}(3)$?
    \end{itemize}
\end{enumerate}



\section*{Example}

\begin{enumerate}
  \item[\it Universal:] Al loves everything, i.e., $\forall xLax$.
  \item[\it Existential:] Someone is dancing, i.e., $\exists x(Px \wedge Dx)$.
  \item[\it Mixed:] Everyone loves someone, i.e., $\forall x(Px \supset \exists yLxy)$. 
  \item[\it Complex:] Everything everything loves loves something, i.e., $\forall x(\forall yLyx \supset \exists zLxz)$.
\end{enumerate}



\section*{Semantics for QL}

\begin{enumerate}
  \item[($A$)] $\VV{\I}{\hat{a}}(\F^n\alpha_1,\ldots,\alpha_n)=1$ ~\textit{iff}~ $\tuple{\VV{\I}{\hat{a}}{(\alpha_1)},\ldots,\VV{\I}{\hat{a}}{(\alpha_n)}}\in\I(\F^n)$.
  \item[(\hspace{1pt}$\forall$\hspace{1pt})] $\VV{\I}{\hat{a}}(\forall\alpha\varphi)=1$ ~\textit{iff}~ $\VV{\I}{\hat{c}}(\varphi)=1$ for every $\alpha$-variant $\hat{c}$ of $\hat{a}$.
  \item[(\hspace{1pt}$\exists$\hspace{1pt})] $\VV{\I}{\hat{a}}(\exists\alpha\varphi)=1$ ~\textit{iff}~ $\VV{\I}{\hat{c}}(\varphi)=1$ for some $\alpha$-variant $\hat{c}$ of $\hat{a}$.
  \item[($\neg$)] $\VV{\I}{\hat{a}}(\neg\varphi)=1$ ~\textit{iff}~ $\VV{\I}{\hat{a}}(\varphi)=0$.
  \item[($\vee$)] $\VV{\I}{\hat{a}}(\varphi \vee \psi)=1$ ~\textit{iff}~ $\VV{\I}{\hat{a}}(\varphi)=1$ or $\VV{\I}{\hat{a}}(\psi)=1$ (or both).
  \item[($\wedge$)] $\VV{\I}{\hat{a}}(\varphi \wedge \psi)=1$ ~\textit{iff}~ $\VV{\I}{\hat{a}}(\varphi)=1$ and $\VV{\I}{\hat{a}}(\psi)=1$.
  \item[($\supset$)] $\VV{\I}{\hat{a}}(\varphi \supset \psi)=1$ ~\textit{iff}~ $\VV{\I}{\hat{a}}(\varphi)=0$ or $\VV{\I}{\hat{a}}(\psi)=1$ (or both).
  \item[($\equiv$)] $\VV{\I}{\hat{a}}(\varphi \equiv \psi)=1$ ~\textit{iff}~ $\VV{\I}{\hat{a}}(\varphi)=\VV{\I}{\hat{a}}(\psi)$.
\end{enumerate}



\section*{Truth and Entailment}

\begin{enumerate}
  \item[\it Truth:] $\VV{\I}{}(\varphi)=1$ ~\textit{iff}~ $\VV{\I}{\hat{a}}(\varphi)=1$ for some $\hat{a}$ where $\varphi$ is a sentence of QL. 
  \item[\it Satisfaction:] $\M=\tuple{\D,\I}$ satisfies $\Gamma$ \textit{iff} $\VV{\I}{}(\varphi)=1$ for every $\varphi\in\Gamma$.
  \item[\it Singletons:] As before $\M$ satisfies $\varphi$ \textit{iff} $\M$ satisfies $\set{\varphi}$.
  \item[\it Entailment:] $\Gamma\vDash\varphi$ just in case every model $\M$ that satisfies $\Gamma$ also satisfies $\varphi$.
  \item[\it Tautology:] $\varphi$ is a tautology \textit{iff} $\vDash \varphi$.
  \item[\it Contradiction:] $\varphi$ is a contradiction \textit{iff} $\vDash \neg\varphi$.
  \item[\it Contingent:] $\varphi$ is contingent \textit{iff} $\vDash \nvarphi$ and $\nvdash \neg\varphi$. 
  \item[\it Consistent:] $\Gamma$ is consistent \textit{iff} $\Gamma$ is satisfiable. 
\end{enumerate}



\section*{Minimal Models}

\begin{enumerate}
  \item[\bf Task 3:] Provide minimal models in which the examples above are true/false.
\end{enumerate}


\section*{Regimentation}

  \begin{itemize}
    \item Every rose has its thorn.
    \item At least the guests that remained were pleased with the party.
    \item I haven't met a cat that likes Merra.
    \item Kate found a job that she loved.
    \item Everybody everybody loves loves somebody.
    \item No set is a member of itself.
    \item There is a set with no members.
  \end{itemize}



\section*{Arguments}

\begin{enumerate}
  \item[\it Love:] Regiment the following argument:
    \begin{itemize}
      \item Cam doesn't love anyone who loves him back.
      \item May loves everyone who loves themselves.
      \item[\therefore] If Cam loves himself, he doesn't love May.
    \end{itemize}
  \item[\it Bigger:] Regiment the following argument:
    \begin{itemize}
      \item Whenever something is bigger than another, the latter is not bigger than the former.
      \item[\therefore] Nothing is bigger than itself.
    \end{itemize}
  % \item[\it Gunk] Regiment the following argument:
  %   \item Nothing is a part of itself.
  %   \item Whenever one thing is bigger than a second thing, and the second thing is bigger than a third thing, then the first thing is bigger than the third thing. 
  %   \item[\therefore] Whenever something is bigger than a second thing, the second thing is not bigger than the first.
  %   % \item[\therefore] Nothing is bigger 
\end{enumerate}


\section*{Relations}

\begin{enumerate}
  \item[\it Domain:] Let the \textit{domain} $D$ be any set.
  \item[\it Relation:] A \textit{relation} $R$ on $D$ is any subset of $D^2$.
  \item[\it Reflexive:] A relation $R$ is \textit{reflexive} on $D$ \textit{iff} $\tuple{x,x}\in R$ for all $x\in D$.
  \item[\it Non-Reflexive:] A relation $R$ is \textit{non-reflexive} on $D$ \textit{iff} $R$ is not reflexive on $D$.
  \item[\bf Question 1:] What is it to be \textit{irreflexive}?
  \item[\it Irreflexive:] A relation $R$ is \textit{irreflexive} on $D$ \textit{iff} $\tuple{x,x}\notin R$ for all $x\in D$.
  \item[\it Symmetric:] A relation $R$ is \textit{symmetric iff} $\tuple{y,x}\in R$ whenever ${x,y}\in R$.
  \item[\bf Question 2:] Why don't we need to specify a domain?
  \item[\bf Question 3:] Why is a relation reflexive or irreflexive with respect to a domain?
  \item[\it Asymmetric:] A relation $R$ is \textit{asymmetric iff} $\tuple{y,x}\notin R$ whenever $\tuple{x,y}\in R$.
  \item[\bf Question 4:] What is it to be non-symmetric? How about non-asymmetric?
  \item[\bf Task 1:] Show that every asymmetric relation is irreflexive.
  \item[\it Transitive:] A relation $R$ is \textit{transitive iff} $\tuple{x,z}\in R$ whenever $\tuple{x,y},\tuple{y,z}\in R$.
  \item[\it Intransitive:] A relation $R$ is \textit{intransitive iff} $\tuple{x,z}\notin R$ whenever $\tuple{x,y},\tuple{y,z}\in R$.
  \item[\bf Question 5:] Is every symmetric transitive relation reflexive? (No: $R=\varnothing$)
  \item[\bf Task 2:] Show that every transitive irreflexive relation asymmetric?
  \item[\it Euclidean:] A relation $R$ is \textit{euclidean iff} $\tuple{y,z}\in R$ whenever $\tuple{x,y},\tuple{x,z}\in R$.
  \item[\bf Task 3:] Show that every transitive symmetric relation is euclidean.
\end{enumerate}


  % \item Diamonds last forever.




\end{document}

