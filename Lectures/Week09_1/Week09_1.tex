\documentclass[a4paper, 11pt]{article} % Font size (can be 10pt, 11pt or 12pt) and paper size (remove a4paper for US letter paper)
\usepackage[protrusion=true,expansion=true]{microtype} % Better typography
\usepackage{graphicx} % Required for including pictures
\usepackage{wrapfig} % Allows in-line images
\usepackage{enumitem} %%Enables control over enumerate and itemize environments
\usepackage{setspace}
\usepackage{amssymb, amsmath, mathrsfs, mathabx} %%Math packages
\usepackage{../lecture} %calls local modified style file
\usepackage{stmaryrd}
\usepackage{mathtools}
\usepackage{multicol} 
\usepackage{mathpazo} % Use the Palatino font
\usepackage[T1]{fontenc} % Required for accented characters
\usepackage{array}
\usepackage{bibentry}
\usepackage{prooftrees} 
\usepackage[round]{natbib} %%Or change 'round' to 'square' for square backers
\setcitestyle{aysep=}

\makeatletter
\renewcommand{\maketitle}{
\begin{flushright}
{\LARGE\@title}

\vspace{10pt}

{\@author}
\\ \@date
\end{flushright}

\vspace{60pt}

}
\makeatother

%----------------------------------------------------------------------------------------
%	TITLE
%----------------------------------------------------------------------------------------

\title{\textbf{Identity}} % Subtitle

\author{\textsc{Logic I}\\ \em Benjamin Brast-McKie} % Institution

\date{\today} % Date

%----------------------------------------------------------------------------------------

\begin{document}

\maketitle % Print the title section

\thispagestyle{empty}

%----------------------------------------------------------------------------------------

\section*{Examples}

\begin{enumerate}
  \item[\it Transitive:] Is the following argument valid when regimented in $\FOL$?
    \item Hesperus is Venus.
    \item \underline{Venus is Phosphorus.\quad\quad}
    \item Hesperus is Phosphorus.
    \item[\bf Question:] What can do to make this argument valid?
    % \item[\bf Task:] Compare transitivity of `is taller than'. 
  \item[\it Rising Star:] Is the following argument valid when regimented in $\FOL$?
    \setcounter{enumi}{0}
    \item Hesperus is rising.
    \item \underline{Hesperus is Phosphorus.\quad\quad}
    \item Phosphorus is rising.
    \item[\bf Question:] What can do to make this argument valid?
    \item[\bf LL:] $(\metaA \eand \alpha = \beta) \eif \metaA[\beta/\alpha]$.
  \item[\it Modus Ponens:] Compare the following argument:
    \setcounter{enumi}{0}
    \item Lucy is lucky.
    \item \underline{If Lucy is lucky, then the letter will arrive in time.\quad\quad}
    \item The letter will arrive in time.
    \item[\bf Question:] Why don't we need to add anything to make this argument valid?
    \item[\it Answer:] `If\ldots, then\ldots' is regimented by $\eif$ which has a semantics. 
    \item[\it Logic:] The conditional $\eif$ is a \textit{logical term} of both $\PL$ and $\FOL$. 
  \item[\it Extensions:] FOL extends PL, but we needn't stop there.
  \item[\bf Question:] What can we add to $\FOL$ to make \textit{Transitive} and \textit{Rising Star} valid? 
  % \item[\bf Question:] How far could we go? What terms could we include?
\end{enumerate}





\section*{Logical Terms}

\begin{itemize}
  \item[\it Logicality:] The primitive symbols of PL and FOL can be divided in three:
    \begin{itemize}
      \item[\tt Logical Terms:] $\enot,\eand,\eor,\eif,\eiff,\forall\alpha,\exists\alpha,x_n,y_n,z_n\ldots$ for $n\geq 0$.
      \item[\tt Non-Logical Terms:] $a_n,b_n,c_n,\ldots$ and $A^n,B^n,\ldots$ for $n\geq 0$.
      \item[\tt Punctuation:] $(, )$
    \end{itemize}
  \item[\it Extensions:] The ``meanings'' of the non-logical terms are fixed by a model.
  \item[\it Semantics:] The ``meanings'' of the logical terms are fixed by the semantics.
  \item[\bf Question:] How many logical terms are there?
  \item[\it Identity:] At least one more, namely identity which we symbolize by `$=$'.
    \item We mean \textit{identity}, not \textit{duplication}.
  \item[\it Taller:] Can we make `is taller than' a logical term to validate the following?
    \setcounter{enumi}{0}
    \item Lu is taller than Kin.
    \item \underline{Kin is taller than Sara.\quad\quad}
    \item Lu is taller than Sara.
    \item[\bf Question:] Why is `=' a logical term but `is taller than' isn't?
\end{itemize}





\section*{Syntax for $\FI$}

\begin{enumerate}
  \item[\it Identity:] We include `$=$' in the primitive symbols of the language.
  \item[\it Well-Formed Formulas:] We may define the well-formed formulas (wffs) of $\FI$ as follows:
  \item $\mathcal{F}^n\alpha_1,\ldots,\alpha_n$ is a wff of $\FI$ if $\mathcal{F}^n$ is an $n$-place predicate and $\alpha_1,\ldots,\alpha_n$ are singular terms.
  \item $\alpha=\beta$ is a wff of $\FI$ if $\alpha$ and $\beta$ are singular terms.
  \item If $\varphi$ and $\psi$ are wffs and $\alpha$ is a variable, then:
    \begin{itemize}
      \begin{multicols}{2}
        \item $\exists\alpha\varphi$ is a wff of $\FI$;
        \item $\forall\alpha\varphi$ is a wff of $\FI$;
        \item $\enot\varphi$ is a wff of $\FI$;
        \item[] ~
        \item $(\varphi\eand\psi)$ is a wff of $\FI$;
        \item $(\varphi\eor\psi)$ is a wff of $\FI$;
        \item $(\varphi\eif\psi)$ is a wff of $\FI$; and
        \item $(\varphi\eiff\psi)$ is a wff of $\FI$.
      \end{multicols}
    \end{itemize}
  \vspace{-.2in}
  \item Nothing else is a wff of $\FI$.
  \vspace{.1in}
  \item[\it Atomic Formulas:] The wffs of $\FI$ defined by (1) and (2) are \textit{atomic}.
  \item[\it Complexity:] $\comp{\F^n\alpha_1,\ldots,\alpha_n}=\comp{\alpha=\beta}=0$.\\
    $\comp{\exists\alpha\varphi}=\comp{\forall\alpha\varphi}=\comp{\enot\varphi}=\comp{\varphi}+1$.\\
    $\comp{\varphi\eand\psi}=\comp{\varphi\eor\psi}=\ldots=\comp{\varphi}+\comp{\psi}+1$.\\
\end{enumerate}





\section*{Free Variables}

\begin{enumerate}
  \item[\it Free Variables:] We define the \textit{free variables} recursively:
  \item $\alpha$ is free in $\mathcal{F}^n\alpha_1,\ldots,\alpha_n$ if $\alpha=\alpha_i$ for some $1\leq i\leq n$ where $\alpha$ is a variable, $\mathcal{F}^n$ is an $n$-place predicate, and $\alpha_1,\ldots,\alpha_n$ are singular terms.
  \item $\alpha$ is free in $\beta=\gamma$ if $\alpha=\beta$ or $\alpha=\gamma$ where $\alpha$ is a variable.
  \item If $\varphi$ and $\psi$ are wffs and $\alpha$ and $\beta$ are variables, then:
    \begin{enumerate}
        \item $\alpha$ is free in $\exists\beta\varphi$ if $\alpha$ is free in $\varphi$ and $\alpha\neq\beta$;
        \item $\alpha$ is free in $\forall\beta\varphi$ if $\alpha$ is free in $\varphi$ and $\alpha\neq\beta$;
        \item $\alpha$ is free in $\enot\varphi$ if $\alpha$ is free in $\varphi$;
        % \item $\alpha$ is free in $(\varphi\eand\psi)$ if $\alpha$ is free in $\varphi$ or $\alpha$ is free in $\psi$;
        % \item $\alpha$ is free in $(\varphi\eor\psi)$ if $\alpha$ is free in $\varphi$ or $\alpha$ is free in $\psi$;
        % \item $\alpha$ is free in $(\varphi\eif\psi)$ if $\alpha$ is free in $\varphi$ or $\alpha$ is free in $\psi$;
        % \item $\alpha$ is free in $(\varphi\eiff\psi)$ if $\alpha$ is free in $\varphi$ or $\alpha$ is free in $\psi$;
        \item[\vdots] ~
    \end{enumerate}
  \item Nothing else is a free variable. 
\end{enumerate}





\section*{Sentences of $\FI$}

\begin{enumerate}
  \item[\it Sentences:] A \textit{wfs} of $\FI$ is any wff of $\FI$ without free variables.
  \item[\it Interpretation:] The truth-values of the wfss of $\FI$ are determined by the models of $\FI$ independent of a variable assignment.
\end{enumerate}





\section*{$\FI$ Models}

\begin{itemize}
  \item[\bf Question:] What in the semantics will have to change?
  \item[\it Model:] $\M=\tuple{\D,\I}$ is a model of $\FI$ \textit{iff} $\D \neq \varnothing$ and $\I$ satisfies both:
    \item $\I(\alpha)\in\D$ for every constant $\alpha$ in $\FI$. 
    \item $\I(\F^n)\subseteq\D^n$ for every $n$-place predicate $\F^n$.
  \item[\it Answer:] Nothing changes in the definition of a model.
\end{itemize}





\section*{Variable Assignments}

\begin{enumerate}
  \item[\it Assignments:] A variable assignment $\hat{a}(\alpha)\in\D$ for every variable $\alpha$ in $\FI$.
  \item[\it Referents:] We may define the referent of $\alpha$ in $\M=\tuple{\D,\I}$ as follows:
    \begin{align*}
      \VV{\I}{\hat{a}}{(\alpha)}=
        \begin{cases}
          \I(\alpha) & \text{if } \alpha \text{ is a constant} \\
          \hat{a}(\alpha) & \text{if } \alpha \text{ is a variable.}
        \end{cases}
    \end{align*}
  \item[\it Variants:] A $\hat{c}$ is an $\alpha$-variant of $\hat{a}$ \textit{iff} $\hat{c}(\beta)=\hat{a}(\beta)$ for all $\beta\neq\alpha$.
  \item[\bf Note:] Nothing changes in these definitions.
\end{enumerate}





\section*{Semantics for $\FI$}

\begin{itemize}
  \item[\it Semantics:] Given a model $\M = \tuple{\D, \I}$ and v.a., $\va{a}$ defined over $\D$, we recursively define $\VV{\I}{\va{a}}$ to be a function from wffs of $\FI$ to $\set{0,1}$ as follows: 
  \item $\VV{\I}{\hat{a}}(\F^n\alpha_1,\ldots,\alpha_n)=1$ ~\textit{iff}~ $\tuple{\val{\I}{\hat{a}}{(\alpha_1)},\ldots,\val{\I}{\hat{a}}{(\alpha_n)}}\in\I(\F^n)$.
  \item $\VV{\I}{\hat{a}}(\alpha=\beta)=1$ ~\textit{iff}~ $\val{\I}{\hat{a}}(\alpha)=\val{\I}{\hat{a}}(\beta)$.
  \item $\VV{\I}{\hat{a}}(\forall\alpha\varphi)=1$ ~\textit{iff}~ $\VV{\I}{\hat{c}}(\varphi)=1$ for every $\alpha$-variant $\hat{c}$ of $\hat{a}$.
  \item $\VV{\I}{\hat{a}}(\exists\alpha\varphi)=1$ ~\textit{iff}~ $\VV{\I}{\hat{c}}(\varphi)=1$ for some $\alpha$-variant $\hat{c}$ of $\hat{a}$.
  \item $\VV{\I}{\hat{a}}(\enot\varphi)=1$ ~\textit{iff}~ $\VV{\I}{\hat{a}}(\varphi)\neq 1$.
  % \item[($\eor$)] $\VV{\I}{\hat{a}}(\varphi \eor \psi)=1$ ~\textit{iff}~ $\VV{\I}{\hat{a}}(\varphi)=1$ or $\VV{\I}{\hat{a}}(\psi)=1$ (or both).
  % \item[($\eand$)] $\VV{\I}{\hat{a}}(\varphi \eand \psi)=1$ ~\textit{iff}~ $\VV{\I}{\hat{a}}(\varphi)=1$ and $\VV{\I}{\hat{a}}(\psi)=1$.
  % \item[($\eif$)] $\VV{\I}{\hat{a}}(\varphi \eif \psi)=1$ ~\textit{iff}~ $\VV{\I}{\hat{a}}(\varphi)=0$ or $\VV{\I}{\hat{a}}(\psi)=1$ (or both).
  % \item[($\eiff$)] $\VV{\I}{\hat{a}}(\varphi \eiff \psi)=1$ ~\textit{iff}~ $\VV{\I}{\hat{a}}(\varphi)=\VV{\I}{\hat{a}}(\psi)$.
  \item[\vdots] 
    \vspace{.1in}
  \item[\it Truth:] $\VV{\I}{}(\varphi)=1$ ~\textit{iff}~ $\VV{\I}{\hat{a}}(\varphi)=1$ for all $\hat{a}$ where $\varphi$ is a sentence of $\FI$. 
\end{itemize}





\section*{Logical Predicates}

\begin{itemize}
  \item[\it Taller-Than:] Suppose we were to take `taller than' ($T$) to be logical.
  \item[\bf Question:] Could we provide its semantics?
    \item $\VV{\I}{\hat{a}}(T\alpha\beta)=1$ ~\textit{iff}~ $\val{\I}{\hat{a}}(\alpha)$ is taller than $\val{\I}{\hat{a}}(\beta)$.
  \item[\it Theory:] The semantics would have to rely on a theory of being taller than.
    \item Providing such a theory lies outside the subject-matter of logic.
    \item By contrast, identity is something we already grasp.
    \item Compare our pre-theoretic grasp of negation, conjunction, and the quantifiers.
  \item[\bf Question:] Could we take set-membership $\in$ to be a logical term? 
  \item[\bf Question:] What is it to be a logical term?
  \item[\bf Question:] Could we take a term in sentence position to be logical?
    \begin{itemize}
      % \begin{multicols}{2}
        \item[($\bot$)] $\VV{\I}{\hat{a}}(\bot)=1$.
        \item[($\top$)] $\VV{\I}{\hat{a}}(\top)=0$.
      % \end{multicols}
    \end{itemize}
\end{itemize}




\section*{Existence}%
  \label{sec:Existence}
  
\begin{itemize}
  \item[\it Existence:] Observe that $\exists x(x=x)$ is a tautology. 
  \item[\bf Question:] What could we do to prevent this?
  \item Empty domain? But how would we interpret the constants?
  \item Free logics are non-classical and weak.
  \item[\bf Question:] Should logic be a neural arbiter?
\end{itemize}






\end{document}

