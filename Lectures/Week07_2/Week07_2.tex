\documentclass[a4paper, 11pt]{article} % Font size (can be 10pt, 11pt or 12pt) and paper size (remove a4paper for US letter paper)
\usepackage[protrusion=true,expansion=true]{microtype} % Better typography
\usepackage{../lecture} %calls local modified style file
\usepackage{graphicx} % Required for including pictures
\usepackage{wrapfig} % Allows in-line images
\usepackage{enumitem} %%Enables control over enumerate and itemize environments
\usepackage{setspace}
\usepackage{amssymb, amsmath, mathrsfs} %%Math packages
\usepackage{stmaryrd}
\usepackage{mathtools}
\usepackage{multicol} 
\usepackage{mathpazo} % Use the Palatino font
\usepackage[T1]{fontenc} % Required for accented characters
\usepackage{array}
\usepackage{bibentry}
\usepackage{prooftrees} 
\usepackage[round]{natbib} %%Or change 'round' to 'square' for square backers
\setcitestyle{aysep={}}
% \usepackage{fitchproof} 

\makeatletter
\renewcommand{\maketitle}{
\begin{flushright}
{\LARGE\@title}

\vspace{10pt}

{\@author}
\\ \@date
\end{flushright}

\vspace{-20pt}

}
\makeatother

%----------------------------------------------------------------------------------------
%	TITLE
%----------------------------------------------------------------------------------------

\title{\textbf{Regimentation and Relations}} % Subtitle

\author{\textsc{Logic I}\\ \em Benjamin Brast-McKie} % Institution

\date{\today} % Date

%----------------------------------------------------------------------------------------

\begin{document}

\maketitle % Print the title section

\thispagestyle{empty}

%----------------------------------------------------------------------------------------

\section*{Polyadic Predicates}

\begin{enumerate}
  \item[\it Triadic:] 
    `$x$ is between $y$ and $z$',\\
    `$x$ is more similar to $y$ than to $z$', \ldots
    % `$x$ is closer to $y$ than to $z$', \ldots 
  \item[\it Polyadic:] We may refer to predicates as $n$-place or $n$-adic.
  \item[\it Properties:] $n$-place predicates express $n$-place properties.
  \item[\bf Note:] We will typically consider at most binary predicates.
\end{enumerate}




\section*{Binary Predicates}

\begin{enumerate}
  \item[\it Height:]
    \underline{Kin is taller than Prema.\quad}\\ 
    Prema is shorter than Kin.
  \item[\bf Task:] Regiment the argument above.
  \item[\it Predicates:] `is taller than', `is shorter than', and `is the same height as'.
  \item[\it Relations:] Binary predicates express $2$-place properties, i.e., \textit{relations}. 
  \item[\bf Question:] Is the argument above valid? How about the following arguments?
    \begin{itemize}
      \item $Tkp\ \therefore\ Spk$.
      \item $Tkp\ \therefore\ \enot Tpk$.
      \item $Tkp\ \therefore\ \enot Tpk \eand \enot Epk$.
    \end{itemize}
  \item[\bf Question:] What can we add to make the arguments valid?
    \begin{itemize}
      \item $Tkp,\ Tkp \eif Spk\ \therefore\ Spk$.
      \item $Tkp,\ \forall x \forall y(Txy \eif Syx)\ \therefore\ Spk$.
    \end{itemize}
  \item[\bf Task:] Regiment the following argument.
  \item[\it Age:] 
    Jon is older than Sara.\\ 
    \underline{Sara is older than Ethan.\quad}\\
    Jon is older than Ethan.
  \item[\it Predicates:] `is older than'.
    \begin{itemize}
      \item $Ojs,\ Ose\ \therefore\ Oje$.
    \end{itemize}
  \item[\bf Question:] Is this argument valid, and if not how can we make it valid?
    \begin{itemize}
      \item $Ojs,\ Ose,\ (Ojs \eand Ose) \eif Oje\ \therefore\ Oje$.
      \item $Ojs,\ Ose,\ \forall x \forall y \forall z((Oxy \eand Oyz) \eif Oxz)\ \therefore\ Oje$.
    \end{itemize}
\end{enumerate}






\section*{Restricting Quantifiers}

\begin{enumerate}
  \item[\it Universals Quantifiers:] Regiment the following sentences:
    \begin{itemize}
      \item All dogs go to heaven.
      \item Jim took every chance he got.
      % \item All the monkeys that Amar loves love him back.
      \item Everyone who trained hard or got lucky made it to the top or else didn't compete.
    \end{itemize}
  \item[\it Hidden Quantifiers:] Regiment the following sentences:
    \begin{itemize}
      \item At least the guests that remained were pleased with the party.
      \item I haven't met a cat that likes Merra.
      \item Kiko's only friends are animals.
    \end{itemize}
  \item[\it Existential Quantifiers:] Regiment the following sentences: 
    \begin{itemize}
      \item Something great is around the corner.
      \item One of Ken's statues is very old.
      \item Kate found a job that she loved.
    \end{itemize}
\end{enumerate}





\section*{Mixed Quantifiers}

\begin{enumerate}
  \item Nothing is without imperfections.
  \item Every dog has its day.
  \item Everyone loves someone.
  \item Nobody knows everybody.
  \item Everybody everybody loves loves somebody.
  \item No set is a member of itself.
  \item There is a set with no members.
\end{enumerate}


\section*{Arguments}

\begin{enumerate}
  \item[\it Love:] Regiment the following argument:
    \item[1.] Cam doesn't love anyone who loves him back.
    \item[2.] \underline{May loves everyone who loves themselves.\quad}
    \item[3.] If Cam loves himself, he doesn't love May.
  \item[\it Bigger:] Regiment the following argument:
    \item[1.] \underline{Whenever something is bigger than another, the latter is not bigger than the former.\quad}
    \item[2.] Nothing is bigger than itself.
  % \item[\it Gunk] Regiment the following argument:
  %   \item Nothing is a part of itself.
  %   \item Whenever one thing is bigger than a second thing, and the second thing is bigger than a third thing, then the first thing is bigger than the third thing. 
  %   \item[\therefore] Whenever something is bigger than a second thing, the second thing is not bigger than the first.
  %   % \item[\therefore] Nothing is bigger 
\end{enumerate}


\section*{Relations}

\begin{enumerate}
  \item[\it Domain:] Let the \textit{domain} $D$ be any set.
  \item[\it Relation:] A \textit{relation} $R$ on $D$ is any subset of $D^2$.
  \item[\it Reflexive:] A relation $R$ is \textit{reflexive} on $D$ \textit{iff} $\tuple{x,x}\in R$ for all $x\in D$.
  \item[\it Non-Reflexive:] A relation $R$ is \textit{non-reflexive} on $D$ \textit{iff} $R$ is not reflexive on $D$.
  \item[\bf Question 1:] What is it to be \textit{irreflexive}?
  \item[\it Irreflexive:] A relation $R$ is \textit{irreflexive} on $D$ \textit{iff} $\tuple{x,x}\notin R$ for all $x\in D$.
  \item[\it Symmetric:] A relation $R$ is \textit{symmetric iff} $\tuple{y,x}\in R$ whenever ${x,y}\in R$.
  \item[\bf Question 2:] Why don't we need to specify a domain?
  \item[\bf Question 3:] Why is a relation reflexive or irreflexive with respect to a domain?
  \item[\it Asymmetric:] A relation $R$ is \textit{asymmetric iff} $\tuple{y,x}\notin R$ whenever $\tuple{x,y}\in R$.
  \item[\bf Question 4:] What is it to be non-symmetric? How about non-asymmetric?
  \item[\bf Task 1:] Show that every asymmetric relation is irreflexive.
  \item[\it Transitive:] A relation $R$ is \textit{transitive iff} $\tuple{x,z}\in R$ whenever $\tuple{x,y},\tuple{y,z}\in R$.
  \item[\it Intransitive:] A relation $R$ is \textit{intransitive iff} $\tuple{x,z}\notin R$ whenever $\tuple{x,y},\tuple{y,z}\in R$.
  \item[\bf Question 5:] Is every symmetric transitive relation reflexive? (No: $R=\varnothing$)
  \item[\bf Task 2:] Show that every transitive irreflexive relation asymmetric?
  \item[\it Euclidean:] A relation $R$ is \textit{euclidean iff} $\tuple{y,z}\in R$ whenever $\tuple{x,y},\tuple{x,z}\in R$.
  \item[\bf Task 3:] Show that every transitive symmetric relation is euclidean.
\end{enumerate}


\end{document}

