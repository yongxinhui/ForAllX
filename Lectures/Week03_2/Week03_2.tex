\documentclass[a4paper, 11pt]{article} % Font size (can be 10pt, 11pt or 12pt) and paper size (remove a4paper for US letter paper)
\usepackage[protrusion=true,expansion=true]{microtype} % Better typography
\usepackage{../lecture} %calls local modified style file
\usepackage{graphicx} % Required for including pictures
\usepackage{wrapfig} % Allows in-line images
\usepackage{enumitem} %%Enables control over enumerate and itemize environments
\usepackage{setspace}
\usepackage{amssymb, amsmath, mathrsfs} %%Math packages
\usepackage{stmaryrd}
\usepackage{mathtools}
\usepackage{multicol} 
\usepackage{mathpazo} % Use the Palatino font
\usepackage[T1]{fontenc} % Required for accented characters
\usepackage{array}
\usepackage{bibentry}
\usepackage{prooftrees} 
\usepackage[round]{natbib} %%Or change 'round' to 'square' for square backers
\setcitestyle{aysep={}}
% \usepackage{fitchproof} 

\makeatletter
\renewcommand{\maketitle}{
\begin{flushright}
{\LARGE\@title}

\vspace{10pt}

{\@author}
\\ \@date
\end{flushright}

\vspace{20pt}

}
\makeatother

%----------------------------------------------------------------------------------------
%	TITLE
%----------------------------------------------------------------------------------------

\title{\textbf{Natural Deduction in PL: Part II}} % Subtitle

\author{\textsc{Logic I}\\ \em Benjamin Brast-McKie} % Institution

\date{\today} % Date

%----------------------------------------------------------------------------------------

\begin{document}

\maketitle % Print the title section

\thispagestyle{empty}

%----------------------------------------------------------------------------------------

\section*{Biconditional}

\begin{enumerate}
  \item[\it Elimination:] $A \eiff (B \eif [(A \eand C)\eiff D])\ \vdash (A\eand B) \eif (D \eif C)$. 
  \item[\it Introduction:] $A \eif (B \eand C),\ C \eif (B \eand A) \vdash A \eiff C$.
\end{enumerate}






\section*{Negation and Reiteration}

\begin{enumerate}
  \item[\it Elimination Rule:] $\enot\enot A\ \vdash A$. \quad(\textit{Double Negation Elimination})
  \item $A \eor \enot A$. \quad(\textit{Law of Excluded Middle})
  \item $A,\ \enot A \vdash B$. \quad(\textit{Ex Falso Quodlibet})
  \item[\it Introduction Rule:] $\enot(A \eand \enot A)$. \quad(\textit{Law of Non-Contradiction})
  \item $A \vdash \enot\enot A$. \quad(\textit{Double Negation Introduction}) 
\end{enumerate}






\section*{Proof}

\begin{enumerate}
  \item[\it Proof:] A natural deduction \textsc{derivation} (or \textsc{proof}) of a conclusion $\metaA$ from a set of premises $\MetaG$ in PL is any finite sequence of lines ending with $\metaA$ on a live line where every line in the sequence is either:
      \begin{itemize}
        \item[(1)] a premise in $\MetaG$; 
        \item[(2)] a discharged assumption; or
        \item[(3)] follows from previous lines by the rules for PL.
      \end{itemize}
  \item[\it Provable:] An wfs $\metaA$ of $\PL$ is \textsc{derivable} (or \textsc{provable}) from $\MetaG$ in PL (i.e., $\MetaG \vdash \metaA$) \textit{iff} there is a natural deduction derivation (proof ) of $\metaA$ from $\MetaG$ in PL. 
  \item[\it Theorem:] A wfs $\metaA$ is a \textit{theorem} of PL (often written $\metaA \in$ PL) \textit{iff} $\vdash \metaA$. 
  \item[\it Interderivable:] Two wfss $\metaA$ and $\metaB$ of $\PL$ are \textsc{interderivable} (i.e., $\metaA \dashv\vdash \metaB$) \textit{iff} both $\metaA\vdash\metaB$ and $\metaB\vdash\metaA$.
  \item[\it Bottom:] We take $\bot \coloneq A\eand\enot A$ to abbreviate an arbitrarily chosen contradiction.
  \item[\it Inconsistent:] A set of sentences $\MetaG$ is \textsc{inconsistent} if and only if $\MetaG\vdash\bot$.
\end{enumerate}



% \section*{Further Problems}
%
%
% \begin{enumerate}
%   \begin{multicols}{2}
%   \item[\it Law of Excluded Middle:] $A \eor \enot A$. 
%     \begin{itemize}
%       \item $\enot(A \eor \enot A)$ \quad:AS
%       \begin{itemize}
%         \item $A$ \quad:AS 
%         \item $A\eor \enot A$ \quad:$\eor$I 
%         \item $\enot (A\eor \enot A)$ \quad:R 
%       \end{itemize}
%       \item $\enot A$ \quad:$\enot$I 
%       \item $A \eor \enot A$ \quad:$\eor$I 
%       \item $A\eor \enot A$ \quad:$\enot$E 
%     \end{itemize}
%   \item[\it LNC:] $\enot(A \eand \enot A)$. 
%     \item[\it EXQ:] $A,\ \enot A \vdash B$. (\textit{Ex Falso Quodlibet})
%     \begin{itemize}
%       \item $\enot B$ \quad:AS 
%       \begin{itemize}
%         \item $A$ \quad:R 
%         \item $\enot A$ \quad:R 
%       \end{itemize}
%       \item $B$ \quad:$\enot$E 
%       \item[] ~
%       \item[] ~
%     \end{itemize}
%   \end{multicols}
%   \item $L\eiff \enot O,\ L\eor \enot O\ \vdash L$.
%   \item $A\eiff B,\ \vdash \enot A\eiff\enot B$.
%   \item $Z \eif (C \eand \enot N),\ \enot Z \eif (N \eand \enot C)\ \vdash N \eor C$.
% \end{enumerate}

  % \begin{multicols}{2}
  % \end{multicols}



\iffalse

\begin{multicols}{2}


\textit{Conjunction Introduction} (\eand I) \vspace{-1em}
\begin{proof}
	\have[m]{a}{\metaA{}}
	\have[n]{b}{\metaB{}}
	\have[\ ]{c}{\metaA{}\eand\metaB{}} \ai{a, b}
	\have[\ ]{d}{\metaB{}\eand\metaA{}} \ai{a, b}
\end{proof}

\vspace{1em}

\textit{Conditional Introduction} (\eif I) \vspace{-1em}
%\nopagebreak
\begin{proof}
	\open
		\hypo[m]{a}{\metaA{}} \as{for \eif I}{}%\by{want \metaB{}}{}
		\have[n]{b}{\metaB{}}
	\close
	\have[\ ]{ab}{\metaA{}\eif\metaB{}}\ci{a-b}
\end{proof}

\vspace{0.6em}

\textit{enotatation Introduction} (\enot I) \vspace{-1em}
\begin{proof}
\open
	\hypo[m]{na}\metaA{} \as{for \enot I}   %\by{:AS for \enot I}{}
	\have[n]{b}\metaB{}
	\have[o]{nb}{\enot\metaB{}}
\close
\have[\ ]{a}[\ ]{\enot\metaA{}}\ni{na-nb}
\end{proof}

\vspace{0.6em}

\textit{Disjunction Introduction} (\eor I) \vspace{-1em}

\begin{proof}
	\have[m]{a}{\metaA{}}
	\have[\ ]{ab}{\metaA{}\eor\metaB{}}\oi{a}
	\have[\ ]{ba}{\metaB{}\eor\metaA{}}\oi{a}
\end{proof}

%\vspace{1.9em} %3.9 for no extra vspaces
\vspace{0.6em}

\textit{Biconditional Introduction} (\eiff I) \vspace{-1em}

\begin{fitchproof}
	\open
		\hypo[i]{a1}{\metaA{}} \as{for \eiff I}
		\have[j]{b1}{\metaB{}}
	\close
\breakline
	\open
		\hypo[k]{b2}{\metaB{}} \as{for \eiff I}
		\have[l]{a2}{\metaA{}}
	\close
	\have[\ ]{ab}{\metaA{}\eiff\metaB{}}\bi{a1-b1,b2-a2}
\end{fitchproof}


\vfill\null
\columnbreak

%\newpage

\textit{Conjunction Elimination} (\eand E) \vspace{-1em}

\begin{proof}
	\have[m]{ab}{\metaA{}\eand\metaB{}}
	\have[\ ]{a}{\metaA{}} \ae{ab}
	\have[\ ]{b}{\metaB{}} \ae{ab}
\end{proof}

%\vspace{1.9em}
%\vspace{2.9em}
\vspace{0.75em}

\textit{Conditional Elimination} (\eif E)  \vspace{-1em}

\begin{proof}
	\have[m]{ab}{\metaA{}\eif\metaB{}}
	\have[n]{a}{\metaA{}}
	\have[\ ]{b}{\metaB{}} \ce{ab,a}
\end{proof}

%\vspace{1em}
\vspace{0.45em}

\textit{Negation Elimination} (\enot E)  \vspace{-1em}
%%note that I think I'm missing some brackets around the sentences on various liens below! works in proof environment but less robust in nd environment. so i ought to fix these here and in Negation intro 

\begin{proof}
\open
	\hypo[m]{na}{\enot\metaA{}} \as{for \enot E}
	\ellipsesline
	\have[n]{b}\metaB{}
	\have[o]{nb}{\enot\metaB{}}
\close
\have[\ ]{a}[\ ]\metaA{}\ne{na-nb}
\end{proof}

\vspace{1em}

\textit{Disjunction Elimination} (\eor E)  \vspace{-1em}

\begin{proof}
\have[m]{ab}{\metaA{}\eor\metaB{}}
	\open
		\hypo[i]{a}{\metaA{}} \as{for \eor E}
		\have[j]{c1}{\metaC{}}
	\close
\breakline
	\open
		\hypo[k]{b}{\metaB{}} \as{for \eor E}
		\have[l]{c2}{\metaC{}}
	\close
	\have[\ ]{c}{\metaC{}} \oe{ab,a-c1, b-c2}
\end{proof}

\fi



%%% OLD








% \section*{Contradictions}
%
% \begin{enumerate}
%   \item[\it Arbitrary:] Does it really not matter which contradiction we choose?
%   \item[\bf Task 1:] Show that $\metaA\dashv\vdash\metaB$ \textit{iff} $\metaA\vdash\bot$ and $\metaB\vdash\bot$. 
%     \begin{itemize}
%       \item Assume $\bot = A \eand \enot A$.
%       \item Show that if $\metaA\vdash\bot$, then $\metaA\vdash\metaB$. 
%     \end{itemize}
%   \item[\bf Task 2:] Show that $\metaA\dashv\vdash\metaB$ \textit{iff} $\metaA\vDash\bot$ and $\metaB\vDash\bot$. 
% \end{enumerate}
%





% \section*{Soundness and Completeness}
%
% \begin{enumerate}
%   \item[\it Tautologies:] Coextensive with the theorems.
%   \item[\it Validity:] The valid SL arguments are derivable in SD, and \textit{vice versa}.
%   \item[\bf Task 1:] Can we ever use SD to determine that an argument is invalid?
%   \item[\it Uncertainty:] If we haven't found a proof, that doesn't mean one doesn't exist. 
%   % \item[\bf Task 2:] What if we can derive the Negation of the conclusion from the premises?
%   %   \begin{itemize}
%   %     \item Does $\MetaG \vdash \enot \metaA$ entail $\MetaG \nvdash \metaA$?
%   %   \end{itemize}
%   % \item[\bf Task 3:] What can we conclude if both $\MetaG \vdash \enot \metaA$ and $\MetaG \vdash \metaA$?
% \end{enumerate}



\section*{Logical Analysis}

\begin{itemize}
  \item[\it Sound and Complete:] $\MetaG \vdash \metaA$ \textit{iff} $\MetaG \vDash \metaA$.
    \item $\vdash \metaA$ \textit{iff} $\vDash \metaA$.
    \item $\MetaG \vdash \bot$ \textit{iff} $\MetaG \vDash \bot$. 
    % \item $\MetaG \vdash \metaA$ \textit{iff} $\MetaG \cup \set{\enot \metaA} \vdash \bot$. 
    % \item $\MetaG \vDash \metaA$ \textit{iff} $\MetaG \cup \set{\enot \metaA} \vDash \bot$. 
  \item[\bf Question:] How can we tell if an argument is valid? 
    \item Construct a truth table.
    \item Write a semantic proof.
    \item Derive the conclusion from the premises.
  \item[\bf Question:] What if we can mange to find a derivation?
    \item Natural deduction won't tell you if there is no proof.
    \item A semantic proof will yield a counterexample.
  \item[\bf Question:] How can we tell what the logical properties are for a wfs of $\PL$?
  \begin{multicols}{2}
    \item[\it Tautology?] \quad If YES, prove $\vdash\metaA$.\hfill
    \item[\it Contradiction?] \quad If YES, prove $\vdash\enot\metaA$.\hfill
    \item[\it Contingent?] \quad If YES, provide two models.\hfill
    \item[\it Equivalent?] \quad If YES, prove $\metaA\dashv\vdash\metaB$.\hfill
    \item[] If NO, provide a countermodel.
    \item[] If NO, provide a model.
    \item[] If NO, prove $\vdash\metaA$ or $\vdash\enot\metaA$. 
    \item[] If NO, provide a countermodel.
  \end{multicols}
\end{itemize}



\section*{Rule Schemata}

\begin{itemize}
  \item[\bf Task:] Compare the rules of inference for PL to their instances.
    \item Whereas the rules are general, PL proofs are particular.
    \item But nothing in our PL proofs depend on the particulars.
  \item[\bf Question:] How might we generalize our proofs beyond any instance?
  \item[\it Rule Schemata:] Replace sentence letters in PL proofs with schematic variables.
    \item Premises are replaced with the lines cited by that rule.
    \item New rules require new names if we are to use them.
  \item[\bf Question:] Can we also generalize proofs of theorems?
  \item These amount to lines that can be added without citing lines. 
  \item[\it Derived Schemata:] To speed up proofs, we want to derive rule schemata.
    \item These can then be employed just like our basic rules.
    \item This avoids having to rewrite the same types of proofs over and over.
\end{itemize}


\section*{Derivable Schemata}

\begin{enumerate}[leftmargin=1.5in]
  \item[\it Law of Excluded Middle:] $\vdash \metaA\eor\enot\metaA$.
  \item[\it Law of Non-Contradiction:] $\vdash \enot(\metaA\eand\enot\metaA)$.
  \item[\it Ex Falso Quodlibet:] $\metaA,\ \enot\metaA\ \vdash \metaB$.
  \item[\it Hypothetical Syllogism:] $\metaA \eif \metaB,\ \metaB \eif \metaC\ \vdash \metaA \eif \metaC$.
  \item[\it Modus Tollens:] $\metaA \eif \metaB,\ \enot\metaB\ \vdash \enot\metaA$.
  \item[\it Contraposition:] $\metaA \eif \metaB\ \vdash \enot\metaB \eif \enot\metaA$.
  \item[\it Dilemma:] $\metaA \eor \metaB,\ \metaA \eif \metaC,\ \metaB \eif \metaC\ \vdash \metaC$.
  \item[\it Disjunctive Syllogism:] $\metaA \eor \metaB,\ \enot \metaA \vdash \metaB$.
  \item[\it $\eor$-Commutativity:] $\metaA \eor \metaB\ \vdash \metaB \eor \metaA$.
  \item[\it $\eand$-Commutativity:] $\metaA \eand \metaB\ \vdash \metaB \eand \metaA$.
  \item[\it Biconditional MP:] $\metaA \eiff \metaB,\ \enot\metaA\ \vdash \enot\metaB$.
  \item[\it $\eiff$-Commutativity:] $\metaA \eiff \metaB\ \vdash \metaB \eiff \metaA$.
  \item[\it Double Negation:] $\enot\enot\metaA\ \dashv\vdash \metaA$.
  \item[\it $\eand$-De Morgan's:] $\enot(\metaA\eand\metaB)\dashv\vdash\enot\metaA\eor\enot\metaB$.
  \item[\it $\eor$-De Morgan's:] $\enot(\metaA\eor\metaB)\dashv\vdash\enot\metaA\eand\enot\metaB$.
  \item[\it ${\eor}{\eand}$-Distribution:] $\metaA\eor(\metaB\eand\metaC) \dashv\vdash (\metaA\eor\metaB)\eand(\metaA\eor\metaC)$.
  \item[\it ${\eand}{\eor}$-Distribution:] $\metaA\eand(\metaB\eor\metaC) \dashv\vdash (\metaA\eand\metaB)\eor(\metaA\eand\metaC)$.
  \item[\it ${\eor}{\eand}$-Absorption:] $\metaA\eor(\metaA\eand\metaB) \dashv\vdash \metaA$.
  \item[\it ${\eand}{\eor}$-Absorption:] $\metaA\eand(\metaA\eor\metaB) \dashv\vdash \metaA$.
  \item[\it $\eand$-Associativity:] $\metaA\eand(\metaB\eand\metaC) \dashv\vdash (\metaA\eand\metaB)\eand\metaC$.
  \item[\it $\eor$-Associativity:] $\metaA\eor(\metaB\eor\metaC) \dashv\vdash (\metaA\eor\metaB)\eor\metaC$.
\end{enumerate}




% \section*{Axiom System for SL}
%
% \begin{enumerate}
%   \item[\it Axiom System:] Consider the axiom and rule schemata, writing `$/$' for deduction.
%     \begin{itemize}
%       \item $\metaA \eif (\metaB \eif \metaA)$.
%       \item $(\metaA \eif (\metaB \eif \metaC)) \eif ((\metaA \eif \metaB) \eif (\metaA \eif \metaC))$.
%       \item $(\enot\metaA \eif \enot\metaB) \eif ((\enot\metaA \eif \metaB) \eif \metaA)$.
%       \item $\metaA \eif \metaB,\ \metaA\ /\ \metaB$.
%     \end{itemize}
%   \item[\it PL-Proof:] $\MetaG \vdash_{PL} \metaA$ \textit{iff} there is a finite sequence of SL sentences where every sentence in the sequence is either: (1) a member of $\MetaG$; (2) an axiom schemata; or (3) follows from previous sentences in the sequence by the single rule schemata given above.
%   \item[\it eiffalence:] Amazingly, it is possible to show that $\MetaG \vdash_{PL} \metaA$ \textit{iff} $\MetaG \vdash_{SD} \metaA$. 
%   \item[\it Definitions:] Given that the axioms and rule schemata only include $\enot$ and $\eif$, we may take these to be the \textit{only} primitive logical connectives, defining all other connectives in their terms. 
%     \begin{itemize}
%       \item This makes for a very compact description of the same logic.
%       \item This logic is much less natural to use, requiring that a lot of derived rules be added to system.
%       \item We don't have this problem, though our system is more complex.
%     \end{itemize}   
% \end{enumerate}

% \section*{Further Problems}
%
% \begin{enumerate}
%   \item $L\eiff \enot O,\ L\eor \enot O\ \vdash L$.
%   \item $A\eiff B\ \vdash \enot A\eiff\enot B$.
%   \item $Z \eif (C \eand \enot N),\ \enot Z \eif (N \eand \enot C)\ \vdash N \eor C$.
% \end{enumerate}











\end{document}

