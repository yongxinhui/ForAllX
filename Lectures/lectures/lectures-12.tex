% !TeX root = ./slides-12.tex

\setcounter{section}{11}

\section{Functional completeness and normal forms}

\subsection{Functional completeness}

\begin{frame}
    \frametitle{Truth functions}

\begin{block}{Definition}

An ($n$-place) \emph{truth function} $t$ is a mapping of $n$-tuples of \True{} and \False\ to either \True\ or \False.
\medskip

$n$-place truth functions correspond to truth tables of sentence $S$ with $n$ sentence letters $A_1$, \dots, $A_n$.
\end{block}

\begin{columns}
  \begin{column}{.5\textwidth}
$\begin{array}{cc|cc}
\uncover<2->{A_1} & \uncover<2->{A_2} & t_\& & 
\only<presentation:2|handout:0>{S}\only<3-4>{A_1 \land A_2}\\ \hline
\True & \True & \True & \only<presentation:2|handout:0>{?}\only<3->{\True} \\
\True & \False & \False & \only<presentation:2|handout:0>{?}\only<3->{\False}\\
\False & \True & \False & \only<presentation:2|handout:0>{?}\only<3->{\False}\\
\False & \False & \False & \only<presentation:2|handout:0>{?}\only<3->{\False}
\end{array}$
\end{column}
\begin{column}{.5\textwidth}
$\begin{array}{cc|cc}
  \uncover<2->{A_1} & \uncover<2->{A_2} & t_\lor & 
  \only<presentation:2-3|handout:0>{S}\only<4>{A_1 \lor A_2}\\ \hline
\True & \True & \True & \only<presentation:2-3|handout:0>{?}\only<4>{\True}\\
\True & \False & \True & \only<presentation:2-3|handout:0>{?}\only<4>{\True}\\
\False & \True & \True & \only<presentation:2-3|handout:0>{?}\only<4>{\True}\\
\False & \False & \False & \only<presentation:2-3|handout:0>{?}\only<4>{\False}
\end{array}$
\end{column}
\end{columns}

\end{frame}

\begin{frame}
    \frametitle{Truth functions}

\begin{block}{Definition}
A sentence $S$ containing the sentence letters $A_1$, \dots, $A_n$
\emph{expresses} the truth function $t$ iff the truth value of $S$ on
the valuation which assigns $v_i$ to $A_i$ is $t(v_1,
\dots, v_n)$.
\medskip

An $n$-place truth function is \emph{expressible} if there is a
sentence containing sentence letters $A_1$, \dots, $A_n$ that
expresses it.
\end{block}
\end{frame}


\begin{frame}
    \frametitle{Examples}

\begin{tabular}{cc}
$\begin{array}{cc|cl}
  \uncover<2->{A_1} & \uncover<2->{A_2} & t_1 & \uncover<2->{S?}\\ \hline
\True & \True & \True \\
\True & \False & \True \\
\False & \True & \False \\
\False & \False & \False
\end{array}$
& \uncover<3->{$A_1 \quad\text{or:}\quad A_1 \land (A_2 \lor \lnot A_2)$} \\ \ \\
$\begin{array}{cc|cl}
  \uncover<2->{A_1} & \uncover<2->{A_2} & t_\mathit{xor} &\uncover<2->{S?}\\ \hline
\True & \True & \False \\
\True & \False & \True \\
\False & \True & \True \\
\False & \False & \False
\end{array}$
&
\uncover<4->{$(A_1 \lor A_2) \land \lnot(A_1 \land A_2)$ or: $\enot (A_1 \eiff A_2)$}
\end{tabular}
\end{frame}

\begin{frame}
  \frametitle{Functional completeness}

  \begin{block}{Definition}
  A collection of connectives is \emph{functionally complete} if every
  truth function is expressible by a sentence containing only those
  connectives.
  \end{block}

\end{frame}

\begin{frame}
  \frametitle{Functional completeness: results}

\begin{itemize}[<+->]
  \item Functionally complete are:
  \begin{itemize}[<+->]
  \item Connectives we know:
  \[
    {\alert{\&}} + {\alert{\enot}} \qquad 
  {\alert{\lor}} +
  {\alert{\enot}} \qquad 
  {\alert{\eif}} +
  {\alert{\enot}} \qquad {\alert{\eif}} +
  {\alert{\bot}}\] 
  \item Any other set of connectives containing one of those.
  \item Two two-place connectives by themselves: neither--not (\textsc{nor}) and not--both (\textsc{nand}).
  \end{itemize}
  \item No other (sets of) one and two-place connectives are functionally complete.
  \item We'll prove this for $\eand + \eor$.
\end{itemize}
\end{frame}


\subsection{Proving connectives are functionally complete}

\begin{frame}
  \frametitle{$\land + \lor + \lnot$ are functionally complete}
\[
\begin{array}{lll|l@{\qquad}l}
A_1 & A_2 & A_3 & t_\mathit{odd} & S \\
\hline
\True & \True & \True & \True & \uncover<2->{(A_1 \land (A_2 \land A_3))} \uncover<6->{{}\lor{}}\\
\True & \True & \False & \False \\
\True & \False & \True & \False \\
\True & \False & \False & \True & \uncover<3->{(A_1 \land (\lnot A_2 \land \lnot A_3))} \uncover<6->{{}\lor{}}\\
\False & \True & \True & \False \\
\False & \True & \False & \True & \uncover<4->{(\lnot A_1 \land (A_2 \land \lnot A_3))} \uncover<6->{{}\lor{}}\\
\False & \False & \True & \True & \uncover<5->{(\lnot A_1 \land (\lnot A_2 \land A_3))}\\
\False & \False & \False & \False
\end{array}\]
\end{frame}

\begin{frame}
  \frametitle{$\land + \lor + \lnot$ are functionally complete}

\begin{itemize}[<+->]
\item Each line makes one, and only one, conjunction true, e.g.,
\item $\lnot A_1 \land A_2 \land \lnot A_3$ is true in, and only in, line \False\ \True\ \False.
\item Combine using $\lor$: make $S$ true in all (and only) the lines where it is supposed to be true.
\end{itemize}
\end{frame}

\begin{frame}
    \frametitle{The ``neither\dots nor \dots'' connective: $\downarrow$}

\[
\begin{array}{cc|c}
P & Q & (P \downarrow Q)\\
\hline
\True & \True & \False\\
\True & \False & \False\\
\False & \True & \False\\
\False & \False & \True
\end{array}
\]

\end{frame}

\begin{frame}
  \frametitle{$\downarrow$ is functionally complete}

\begin{itemize}[<+->]
\item We already know that ${\lnot} + {\land} + {\lor}$ is functionally complete, i.e.,
\item Every truth function can be expressed using only $\lor$, $\land$, $\lnot$.
\item To show $\downarrow$ is functionally complete, suffices to
  show that \emph{every sentence containing only $\lnot$, $\lor$, $\land$ is
   equivalent to one containing only $\downarrow$}.
\item For that, it suffices to show that any negated sentence,
  conjunction, disjunction, can be expressed using only $\downarrow$.
\end{itemize}

\end{frame}

\begin{frame}
    \frametitle{Expressing $\lnot$ using $\downarrow$}

\begin{columns}
\begin{column}{3cm}
\[\begin{array}{cc|c}
P & Q & (P \downarrow Q)\\
\hline
\True & \True & \False\\
\True & \False & \False\\
\False & \True & \False\\
\False & \False & \True
\end{array}\]
\end{column}
\begin{column}{7cm}
\begin{itemize}[<+->]
\item Note how $P \downarrow Q$ is \False{} in the first line and \True{} in the last (when $P$ and $Q$ have same truth value).
\item So $P \downarrow P$ is \False{} if $P$ is \True, and \True{} if $P$ is \False, i.e., \[
\lnot P \Leftrightarrow (P \downarrow P).
\]
\end{itemize}
\end{column}
\end{columns}

\end{frame}

\begin{frame}
    \frametitle{Expressing $\lor$ using $\downarrow$}

\begin{columns}
\begin{column}{3cm}
\[\begin{array}{cc|c}
P & Q & (P \downarrow Q)\\
\hline
\True & \True & \False\\
\True & \False & \False\\
\False & \True & \False\\
\False & \False & \True
\end{array}\]
\end{column}
\begin{column}{7cm}
\begin{itemize}[<+->]
\item $P \downarrow Q$ is the ``neither \dots nor'' connective, which can also
be expressed as $\lnot(P \lor Q)$, i.e.,
\[ \lnot(P \lor Q) \Leftrightarrow P \downarrow Q \]
\item Negate both sides:
\[
P \lor Q \Leftrightarrow \lnot(P \downarrow Q)
\]
\item Apply what we figured out in last slide:
\[
P \lor Q \Leftrightarrow (P \downarrow Q)\downarrow(P \downarrow Q)
\]
\end{itemize}
\end{column}
\end{columns}

\end{frame}

\begin{frame}
    \frametitle{Expressing $\land$ using $\downarrow$}

\begin{columns}
\begin{column}{3cm}
\[\begin{array}{cc|c}
P & Q & (P \downarrow Q)\\
\hline
\True & \True & \False\\
\True & \False & \False\\
\False & \True & \False\\
\False & \False & \True
\end{array}\]
\end{column}
\begin{column}{7cm}
\begin{itemize}[<+->]
\item $P \downarrow Q$ is the ``neither \dots nor'' connective, which can also
be expressed as $\lnot P \land \lnot Q$, i.e.,
\[ (\lnot P \land \lnot Q) \Leftrightarrow P \downarrow Q \]
\item Equivalence holds for \emph{all sentences} $P$, $Q$, so also if we replace $P$ by $\lnot R$ and $Q$ by $\lnot S$:
\[
\lnot\lnot R \land \lnot\lnot S \Leftrightarrow (\lnot R \downarrow \lnot S)
\]
\item Express $\lnot$ using $\downarrow$:
\[
R \land S \Leftrightarrow (R \downarrow R)\downarrow(S \downarrow S)
\]
\end{itemize}
\end{column}
\end{columns}

\end{frame}

\begin{frame}
    \frametitle{Functionally complete connectives}

\begin{itemize}[<+->]
\item De Morgan's Law: $\land$ can be expressed by $\lor$ and $\lnot$.
\item Similarly: $\lor$ can be expressed by $\land$, $\lnot$.
\item So $\lor, \lnot$ and $\land, \lnot$ are
  functionally complete.
\item $\to, \bot$ is functionally complete (HW).
\item $\to, \lnot$ is functionally complete.
\item No other sets of connectives that don't contain one of these
sets are functionally complete.
\item ``Neither \dots nor'' (\textsc{nor}) is functionally complete by itself.
\item ``Not both'' (\textsc{nand}) connective is functionally complete by itself.
\item No other 2-place connectives are functionally complete by themselves.
  \end{itemize}
\end{frame}

\subsection{Proving connectives aren't functionally complete}

\begin{frame}
  \frametitle{$\land + \lor$ not functionally complete}

\begin{itemize}[<+->]
\item $\eand + \lor$ is not functionally complete.
\item Remember: To be functionally complete, \emph{every} truth
  function would have to be expressible using only $\eand$ and $\eor$.
\item Which 2-place truth-functions can be expressed using $\eand$ and
  $\eor$?
\item Not this one:
\[
\begin{array}{cc|c}
  && t_{\mathit{xor}}\\ \hline
\True & \True & \False \\
\True & \False & \True \\
\False & \True & \True \\
\False & \False & \False
\end{array}
\]
\end{itemize}
\end{frame}

\begin{frame}
    \frametitle{Proof by induction}

\begin{itemize}[<+->]
\item Sometimes need to prove something for \emph{all} sentences.
\item E.g., ``every sentence containing only $\land$ and $\lor$ expresses a truth function \emph{other than} $t_\mathit{xor}$.''
\item Proof by \emph{induction}:
\begin{itemize}[<+->]
\item Show that it holds for \emph{sentence letters} (and $\bot$).
\item Suppose sentences \emph{$\metav{P}$}, \emph{$\metav{Q}$} have the property.
\item Now show that it then also holds for \emph{$(\metav{P} \land \metav{Q})$}, \emph{$(\metav{P} \lor \metav{Q})$}, etc.
\end{itemize}
\item Why does this work? 
\item This is how we form sentences (involving only $\land$, $\lor$).
\item Property ``$S$ is a sentence expressing a truth function other than~$t_\mathit{xor}$'' propagates from atomic sentences to all sentences.
\end{itemize}
\end{frame}

\begin{frame}
    \frametitle{Proof by induction: example}

\begin{theorem}
Every sentence contains an even number of parentheses.
\end{theorem}

\begin{itemize}[<+->]
\item Every atomic sentence contains an even number of parentheses:
\[
B \qquad \bot
\]
\item If $\metav{P}$ contains an even number of parentheses, so does $\lnot \metav{P}.$
\item If $\metav{P}$ and $\metav{Q}$ both contain an even number of parentheses, so do
\[ (\metav{P} \land \metav{Q}), (\metav{P} \lor \metav{Q}), (\metav{P} \eif \metav{Q}), (\metav{P} \eiff \metav{Q}). \]
\end{itemize}
\end{frame}

\begin{frame}
  \frametitle{$\land + \lor$ not functionally complete}

\begin{theorem}
Any sentence containing only $A_1$, $A_2$, $\land$, $\lor$
has a $\True$ in the first line of its truth table.
\end{theorem}

\begin{itemize}[<+->]
\item Sentence letters: truth table of $A_i$ is just copy of column under $A_i$, so has $\True$ in first line where valuation assigns $\True$ to $A$.  
\item Suppose $\metav{P}$, $\metav{Q}$ are sentences which
  contain only $A_1$, $A_2$, $\land$, $\lor$ and are true in first line.
\item $(\metav{P} \land \metav{Q})$ is true in first line, since $\True \land \True$ makes $\True$.
\item $(\metav{P} \lor \metav{Q})$ is true in first line, since $\True
\lor \True$ makes $\True$.
\end{itemize}
\end{frame}

\begin{frame}
  \frametitle{$\land + \lor$ not functionally complete}

\begin{theorem}
Any sentence containing only $A_1$, $A_2$, \&, $\lor$
  expresses a truth function $t$ with $t(\True, \True) = \True$.
\end{theorem}

\begin{itemize}[<+->]
\item Sentence letters: $A_1$, $A_2$: express $t_1$, $t_2$.
\item Suppose $\metav{P}$, $\metav{Q}$ are sentences which
  contain only $A_1$, $A_2$, $\&$, $\lor$ and express truth
  functions $t$, $t'$ with $t(\True, \True) = t'(\True, \True) = \True$
\item $(\metav{P} \land \metav{Q})$ expresses truth function $s$ with
\[s(\True, \True) = t_\&(t(\True, \True), t'(\True, \True)) = \True\]
\item $(\metav{P} \lor \metav{Q})$ expresses truth function $s$ with
\[s(\True, \True) = t_\lor(t(\True, \True), t'(\True, \True)) = \True\]
\end{itemize}
\end{frame}

\begin{frame}
  \frametitle{Non-functionally complete connectives}

  \begin{itemize}[<+->]
    \item We've shown that $\eand + \eor$ are not  functionally complete.
    \item Same idea shows that $\eif$ and $\eiff$ not functionally complete.
    \item When we add $\enot$ things get interesting:
    \begin{itemize}[<+->]
      \item Functionally complete:
      \[{\enot} + {\eor} \qquad {\enot} + {\eand} \qquad {\enot} + {\eif}\]
      \item Not functionally complete:
      \[{\enot} + {\eiff}\]
      (harder to prove).
    \end{itemize}
  \end{itemize}
\end{frame}

\subsection{Normal forms}

\begin{frame}
  \frametitle{Normal forms}

  \begin{itemize}[<+->]
    \item Sometimes interested in sentences that have specific \emph{form}, e.g.,
    \item Negations apply only to sentence letters.
    \item Alternation between $\land$ and $\lor$ is minimal.
    \item Useful for applications:
    \begin{itemize}
      \item Combinational circuits.
      \item \textsc{sat} solvers and theorem provers need inputs in \textsc{cnf}.
      \item Complexity theory talks about problems involving sentences in normal form.
    \end{itemize}
  \end{itemize}
\end{frame}

\begin{frame}
  \frametitle{Scope of a connective}

  \begin{definition}
    The \emph{scope} of an occurrence of a connective in a sentence is that sub-sentence of which the connective is the main connective.
  \end{definition}

  \[(\underbrace{\hilitebox{\enot(A \lor B)}}_{\text{scope of } \lnot} \lor \underbrace{\hilitebox{((A \eif B) \land (B \eif C))}}_{\text{scope of }\land})\]
\end{frame}

\begin{frame}
  \frametitle{Disjunctive normal form}

  \begin{block}{DNF}
    A sentence is in \emph{disjunctive normal form} (\textsc{dnf}) iff it:
    \begin{itemize}
      \item contains only $\land$, $\lor$, $\lnot$;
      \item only sentence letters are in scope of $\lnot$;
      \item only sentence letters, $\land$, and $\lnot$ are in scope of $\land$.
    \end{itemize}
  \end{block}

  In other words: \textsc{dnf} are disjunctions of conjunctions of
  sentence letters and negated sentence letters, e.g.:
  \begin{align*}
  & (A \land \lnot B) \lor ((\lnot A \land C) \lor (B \land C))\\
  & \lnot A \lor (B \land C)\\
  & A \land (B \land C)\\
  & A \lor (B \lor C)
\end{align*}

\end{frame}

\begin{frame}
\frametitle{DNF theorem}

\begin{theorem}
Every sentence is equivalent to one in disjunctive normal form.
\end{theorem}

\begin{proof}
  \begin{itemize}[<+->]
    \item Construct truth table.
    \item Apply method we used to show ${\land}
    + {\lor} + {\lnot}$ is functionally complete.
    \item This gives us a sentence
    involving only $\land$, $\lor$, $\lnot$ with same truth table, i.e.,
    is equivalent in SL.
    \item That sentence is always in DNF.
  \end{itemize}
\end{proof}
\end{frame}

\begin{frame}
  \frametitle{$\land + \lor + \lnot$ are functionally complete}
\[
\begin{array}{lll|l@{\qquad}l}
A_1 & A_2 & A_3 & t_\mathit{odd} & S \\
\hline
\True & \True & \True & \True & {(A_1 \land (A_2 \land A_3))} \lor\\
\True & \True & \False & \False \\
\True & \False & \True & \False \\
\True & \False & \False & \True & {(A_1 \land (\lnot A_2 \land \lnot A_3))} \lor\\
\False & \True & \True & \False \\
\False & \True & \False & \True & {(\lnot A_1 \land (A_2 \land \lnot A_3))} \lor\\
\False & \False & \True & \True & {(\lnot A_1 \land (\lnot A_2 \land A_3))}\\
\False & \False & \False & \False
\end{array}\]
\end{frame}

\begin{frame}
  \frametitle{Conjunctive normal form}

  \begin{block}{CNF}
    A sentence is in \emph{conjunctive normal form} (\textsc{cnf}) if it:
    \begin{itemize}
      \item contains only $\land$, $\lor$, $\lnot$;
      \item only sentence letters are in scope of $\lnot$;
      \item only sentence letters, $\lor$, and $\lnot$ are in scope of~$\lor$.
    \end{itemize}
  \end{block}

  In other words: \textsc{cnf} are conjunctions of disjunctions of
  sentence letters and negated sentence letters, e.g.:
  \begin{align*}
  & (A \lor \lnot B) \land ((\lnot A \lor C) \land (B \lor C))\\
  & \lnot A \land (B \lor C)\\
  & A \lor (B \lor C) \\
  & A \land (B \land C)
  \end{align*}

\end{frame}

\begin{frame}
\frametitle{CNF theorem}

\begin{theorem}
Every sentence is equivalent to one in conjunctive normal form.
\end{theorem}

\begin{proof}
\begin{itemize}[<+->]
  \item Construct truth table.
  \item For each line where sentence is $\False$, write a disjunction
  of sentence letters and negated sentence letters:
  \begin{itemize}[<+->]
    \item Write $A$ if $A$ is assigned \False.
    \item Write $\lnot A$ if $A$ is assigned \True.
  \end{itemize}
  \item Put $\land$'s between all of them.
  \item Resulting is true iff the original sentence is true, and is in CNF.
\end{itemize}
\end{proof}
\end{frame}

\begin{frame}
  \frametitle{CNF from truth table}
\[
\begin{array}{lll|l@{\qquad}l}
A_1 & A_2 & A_3 & S & CNF\\
\hline
\True & \True & \True & \True & \\
\True & \True & \False & \False & (\lnot A_1 \lor (\lnot A_2 \lor A_3)) \land {}\\
\True & \False & \True & \False & (\lnot A_1 \lor (A_2 \lor \lnot A_3)) \land {}\\
\True & \False & \False & \True \\
\False & \True & \True & \False & (A_1 \lor (\lnot A_2 \lor \lnot A_3)) \land {}\\
\False & \True & \False & \True & \\
\False & \False & \True & \True & \\
\False & \False & \False & \False & (A_1 \lor (A_2 \lor A_3))
\end{array}\]
\end{frame}

\subsection{Equivalent transformations}

\begin{frame}
\frametitle{Transformation equivalences}

Defining $\eif$, $\eiff$ (Cond, Bicond)
\begin{align*}
(\metav{P} \eif \metav{Q}) & \Leftrightarrow (\lnot\metav{P} \lor \metav{Q})\\
\lnot(\metav{P} \eif \metav{Q}) & \Leftrightarrow (\metav{P} \land \lnot\metav{Q})\\
(\metav{P} \eiff \metav{Q}) & \Leftrightarrow (\metav{P} \eif \metav{Q}) \land (\metav{Q} \eif \metav{P})\\
\intertext{Double negation (DN)}
\lnot\lnot\metav{P} & \Leftrightarrow \metav{P}
\end{align*}

\end{frame}

\begin{frame}
\frametitle{Transformation equivalences}

De Morgan's Laws (DeM):
\begin{align*}
\lnot(\metav{P} \lor \metav{Q}) & \Leftrightarrow (\lnot \metav{P} \land \lnot \metav{Q})\\
\lnot(\metav{P} \land \metav{Q}) & \Leftrightarrow (\lnot \metav{P} \lor \lnot \metav{Q})
\intertext{Commutativity (Comm):}
\metav{P} \lor \metav{Q} & \Leftrightarrow \metav{Q} \lor \metav{P}\\
\metav{P} \land \metav{Q} & \Leftrightarrow \metav{Q} \land \metav{P}
\intertext{Distributivity (Dist):}
\metav{P} \lor (\metav{Q} \land \metav{R}) & \Leftrightarrow (\metav{P} \lor \metav{Q}) \land (\metav{P} \lor \metav{R}) \\
\metav{P} \land (\metav{Q} \lor \metav{R}) & \Leftrightarrow (\metav{P} \land \metav{Q}) \lor (\metav{P} \land \metav{R})
\end{align*}

\end{frame}

\begin{frame}
\frametitle{Transforming sentences into DNF/CNF}

\begin{itemize}[<+->]
  \item Replace any subsentence of the form $(\metav{P} \eif \metav{Q})$, $(\metav{P} \eiff \metav{Q})$ by its equivalent.
  \item Use De Morgan's laws to place $\lnot$'s in front of sentence letters
  \item Remove double negations.
  \item Use distributivity and commutativity to ensure 
  \begin{itemize}[<+->]
    \item DNF: no $\lor$ is in the scope of~$\land$.
    \item CNF: no $\land$ is in the scope of~$\lor$.
\end{itemize}
\end{itemize}
\end{frame}

\begin{frame}
\frametitle{Transforming sentences into CNF/DNF}
\setlength{\leftmargini}{0cm}
\begin{itemize}[<+->]
  \item[] $\lnot[\only<presentation:2-3|handout:0>{\hilitebox}{(A \eiff B)} \lor \lnot(B \eif C)]$\pause
  \item[] $\lnot[\only<presentation:3-5|handout:0>{\hilitebox}{((A \eif B) \land (B \eif A))}\only<presentation:4-5|handout:0>{_\metav{P}} \lor \only<presentation:4-5|handout:0>{\hilitebox}{\lnot(B \eif C)}\only<presentation:4-5|handout:0>{_\metav{Q}}]$\hfill\alert<3>{Bicond}\pause
  \item[]
  $\lnot\only<presentation:5|handout:0>{\hilitebox}{(\only<presentation:6-7|handout:0>{\hilitebox}{(A
  \eif B)}\only<presentation:6-7|handout:0>{_\metav{P}} \land
  \only<presentation:6-7|handout:0>{\hilitebox}{(B \eif
  A)}\only<presentation:6-7|handout:0>{_\metav{Q}}
  \only<presentation:5|handout:0>{_\metav{Q}})}\only<presentation:5|handout:0>{_\metav{P}} \land \lnot\only<presentation:5|handout:0>{\hilitebox}{\lnot(B \eif C)}\only<presentation:5|handout:0>{_\metav{Q}}$\hfill\alert<5>{DeM}\pause 
  \item[] $(\only<presentation:8-9|handout:0>{\hilitebox}{\lnot\only<presentation:7|handout:0>{\hilitebox}{(A
  \eif B)}\only<presentation:7|handout:0>{_\metav{P}}} \lor
  \lnot\only<presentation:7|handout:0>{\hilitebox}{(B \eif
  A)}\only<presentation:7|handout:0>{_\metav{Q}}) \land \lnot\lnot(B
  \eif C)$\hfill\alert<7>{DeM}\pause  
  \item[] $(\only<presentation:9|handout:0>{\hilitebox}{(A
  \land \lnot B)} \lor
  \only<presentation:10-11|handout:0>{\hilitebox}{\lnot(B \eif
  A)}) \land \lnot\lnot(B
  \eif C)$\hfill\alert<9>{Cond}\pause  
  \item[] $((A \land \lnot B) \lor
  \only<presentation:11|handout:0>{\hilitebox}{(B \land \lnot
  A)}) \land \only<presentation:12-13|handout:0>{\hilitebox}{\lnot\lnot}(B
  \eif C)$\hfill\alert<11>{Cond}\pause  
  \item[] $((A \land \lnot B) \lor
  (B \land \lnot
  A)) \land \only<presentation:14-15|handout:0>{\hilitebox}{(B
  \eif C)}$\hfill\alert<13>{DN}\pause  
  \item[] $((A \land \lnot B) \lor
  (B \land \lnot
  A)) \land \only<presentation:15|handout:0>{\hilitebox}{(\lnot B
  \lor C)}$\hfill\alert<15>{Cond}\only<16>{}
\end{itemize}
\end{frame}

\begin{frame}
\frametitle{Transforming sentences into DNF}
\setlength{\leftmargini}{0cm}

\begin{itemize}[<+->]\small
  \item[] $\only<presentation:2-3|handout:0>{\hilitebox}{[(A \land \lnot B) \lor
  (B \land \lnot
  A)]}\only<presentation:2-3|handout:0>{_\metav{P}} \only<presentation:1|handout:0>{\hilitebox}\land (\only<presentation:2-3|handout:0>{\hilitebox}{\lnot B}\only<presentation:2-3|handout:0>{_\metav{Q}}
  \only<presentation:1|handout:0>{\hilitebox}\lor \only<presentation:2-3|handout:0>{\hilitebox}{C}\only<presentation:2-3|handout:0>{_\metav{R}})$\pause
  \item[] $(\only<presentation:3-5|handout:0>{\hilitebox}{[(A \land \lnot B) \lor
  (B \land \lnot
  A)]}\only<presentation:3-5|handout:0>{_\metav{P}} \land
  \only<presentation:3-5|handout:0>{\hilitebox}{\lnot
  B}\only<presentation:3-5|handout:0>{_\metav{Q}}) \lor
  (\only<presentation:3|handout:0>{\hilitebox}{[(A \land \lnot B) \lor
  (B \land \lnot
  A)]}\only<presentation:3|handout:0>{_\metav{P}} \land
  \only<presentation:3|handout:0>{\hilitebox}{C}\only<presentation:3|handout:0>{_\metav{R}})$\hfill\alert<3>{Dist}\pause
  \item[] $(\only<presentation:5-7|handout:0>{\hilitebox}{\lnot
  B}\only<presentation:5|handout:0>{_\metav{Q}}\only<presentation:6-7|handout:0>{_\metav{P}}
   \land
   \only<presentation:5|handout:0>{\hilitebox}{[
     \only<presentation:6-7|handout:0>{\hilitebox}{(A \land \lnot B)}
     \only<presentation:6-7|handout:0>{_\metav{Q}}
     \lor
     \only<presentation:6-7|handout:0>{\hilitebox}{(B \land \lnot A)}\only<presentation:6-7|handout:0>{_\metav{R}}]}
  \only<presentation:5|handout:0>{_\metav{P}}
  ) \land
  ([(A \land \lnot B) \lor (B \land \lnot A)]
  \land 
  C)$\hfill\alert<5>{Comm}\pause  
  \item[] $(
    [\only<presentation:7|handout:0>{\hilitebox}{\lnot B}\only<presentation:7|handout:0>{_\metav{P}}
   \land
     \only<presentation:7|handout:0>{\hilitebox}{(A \land \lnot B)}
     \only<presentation:7|handout:0>{_\metav{Q}}]
     \lor
     [\only<presentation:7|handout:0>{\hilitebox}{\lnot B}
     \only<presentation:7|handout:0>{_\metav{P}}
     \land 
     \only<presentation:7|handout:0>{\hilitebox}{(B \land \lnot A)}
     \only<presentation:7|handout:0>{_\metav{R}}])
   \lor
     ([\only<presentation:9-10|handout:0>{\hilitebox}{(A \land \lnot
     B)}\only<presentation:9-10|handout:0>{_\metav{Q}} 
    \lor \only<presentation:9-10|handout:0>{\hilitebox}{(B \land \lnot A)}\only<presentation:9-10|handout:0>{_\metav{R}}]
  \land 
  \only<presentation:9-10|handout:0>{\hilitebox}{C}
  \only<presentation:9-10|handout:0>{_\metav{P}})$\hfill\alert<8>{Dist}\pause
  \pause
  \item[] $([\lnot B \land (A \land \lnot B)]
     \lor
     [\lnot B \land (B \land \lnot A)])
     \lor
     ([\only<presentation:10|handout:0>{\hilitebox}{(A \land \lnot B)}
     \only<presentation:10|handout:0>{_\metav{Q}} \land
     \only<presentation:10|handout:0>{\hilitebox}{C}
     \only<presentation:10|handout:0>{_\metav{P}}]
     \lor
     [\only<presentation:10|handout:0>{\hilitebox}{(B \land \lnot A)}\only<presentation:10|handout:0>{_\metav{R}}
  \land 
  \only<presentation:10|handout:0>{\hilitebox}{C}
  \only<presentation:10|handout:0>{_\metav{P}}])$\hfill\alert<10>{Dist}\only<11>{}
\end{itemize}

\vfill
\end{frame}

\subsection{Simplification}

\begin{frame}
\frametitle{Simplification equivalences}

Associativity (Assoc):
\begin{align*}
\metav{P} \lor (\metav{Q} \lor \metav{R}) & \Leftrightarrow (\metav{P} \lor \metav{Q}) \lor \metav{R} &
\metav{P} \land (\metav{Q} \land \metav{R}) & \Leftrightarrow (\metav{P} \land \metav{Q}) \land \metav{R}
\end{align*}
Idempotence (Id):
\begin{align*}
  (\metav{P} \lor \metav{P}) & \Leftrightarrow \metav{P} &
  (\metav{P} \land \metav{P}) & \Leftrightarrow \metav{P}
\intertext{Absorption (Abs):}
\metav{P} \land (\metav{P} \lor \metav{Q}) & \Leftrightarrow \metav{P}&
\metav{P} \lor (\metav{P} \land \metav{Q}) & \Leftrightarrow \metav{P}
\intertext{Simplification (Simp):}
\metav{P} \land (\metav{Q} \lor \lnot\metav{Q}) & \Leftrightarrow \metav{P} &
\metav{P} \lor (\metav{Q} \land \lnot\metav{Q}) & \Leftrightarrow \metav{P}\\
\metav{P} \lor (\metav{Q} \lor \lnot\metav{Q}) & \Leftrightarrow (\metav{Q} \lor \lnot\metav{Q}) &
\metav{P} \land (\metav{Q} \land \lnot\metav{Q}) & \Leftrightarrow (\metav{Q} \land \lnot\metav{Q}) \\
\end{align*}

\end{frame}

\begin{frame}
\frametitle{Simplifying sentences}
\small
\setlength{\leftmargini}{0cm}
\begin{itemize}[<+->]
  \item[] $([\only<presentation:1|handout:0>{\hilitebox}{\lnot B} \land (\only<presentation:2-3|handout:0>{\hilitebox}{A \land \only<presentation:1|handout:0>{\hilitebox}{\lnot B}})]
     \lor
     [\lnot B \land (B \land \lnot A)])
     \lor
     ([(A \land \lnot B)
     \land
     C]
     \lor
     [(B \land \lnot A)
  \land 
  C])$\pause
  \item[] $([\only<presentation:4-5|handout:0>{\hilitebox}{\lnot B \land
  (\only<presentation:3|handout:0>{\hilitebox}{\lnot B \land A})}]
  \lor
  [\lnot B \land (B \land \lnot A)])
  \lor
  ([(A \land \lnot B)
  \land
  C]
  \lor
  [(B \land \lnot A)
\land 
C])$\hfill\alert<3>{Comm}\pause
\item[] $([\only<presentation:5|handout:0>{\hilitebox}{\only<presentation:6-7|handout:0>{\hilitebox}{(\lnot B \land
\lnot B)} \land A}]
\lor
[\lnot B \land (B \land \lnot A)])
\lor
([(A \land \lnot B)
\land
C]
\lor
[(B \land \lnot A)
\land 
C])$\hfill\alert<5>{Assoc}\pause
\item[] $([\only<presentation:7|handout:0>{\hilitebox}{\lnot B} \land A]
\lor
[\only<presentation:8-9|handout:0>{\hilitebox}{\lnot B \land (B \land \lnot A)}])
\lor
([(A \land \lnot B)
\land
C]
\lor
[(B \land \lnot A)
\land 
C])$\hfill\alert<7>{Id}\pause
\item[] $([\lnot B \land A]
\lor
[\only<presentation:9-10|handout:0>{\hilitebox}{(\lnot B \land B) \land \lnot A}])
\lor
([(A \land \lnot B)
\land
C]
\lor
[(B \land \lnot A)
\land 
C])$\hfill\alert<9>{Assoc}
\item[] $(\only<presentation:11-12|handout:0>{\hilitebox}{[\lnot B \land A]
\lor
[\only<presentation:10|handout:0>{\hilitebox}{\lnot B \land B}]})
\lor
([(A \land \lnot B)
\land
C]
\lor
[(B \land \lnot A)
\land 
C])$\hfill\alert<10>{Simp}\pause
\item[] $(\only<presentation:12,14-15|handout:0>{\hilitebox}{\lnot B \land A})
\lor
([(\only<presentation:14|handout:0>{\hilitebox}{A \land \lnot B})
\land
C]
\lor
[(B \land \lnot A)
\land 
C])$\hfill\alert<12>{Simp}\pause\pause
\item[] $(\only<presentation:15|handout:0>{\hilitebox}{A \land \lnot B})
\lor
([(A \land \lnot B)
\land
C]
\lor
[(B \land \lnot A)
\land 
C])$\hfill\alert<15>{Comm}\pause
\item[] $\only<presentation:18-19|handout:0>{\hilitebox}{((A \land \lnot B)
\lor
[(A \land \lnot B)
\land
C])}
\lor
[(B \land \lnot A)
\land 
C]$\hfill\alert<17>{Assoc}\pause
\item[] $\only<presentation:19|handout:0>{\hilitebox}{(A \land \lnot B)}
\lor
[(B \land \lnot A)
\land 
C]$\hfill\alert<19>{Abs}\only<20>{}
\end{itemize}
\end{frame}
