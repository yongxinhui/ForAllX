% !TeX root = ./1-handout.tex

\setcounter{section}{0}

% Replaced `basic sentence' with `atomic sentence'

\section{Symbolization in SL}
\subsection{Symbolization keys and paraphrase}

\begin{frame}
  \frametitle{Symbolizing arguments}

  \begin{block}<1->{Argument 2 (Paraphrased)}
    Mandy enjoys skiing or Mandy enjoys hiking.\\
    Not: Mandy enjoy hiking.\\
    $\therefore$ Mandy enjoys skiing.
  \end{block}

  \begin{block}<2->{Form of argument 2}
    $S$ or $H$.\\
    Not $H$.\\
    $\therefore$ $S$.
  \end{block}

  \begin{block}<3->{Symbolization of argument 2 in SL}
    $(S \lor H)$\\
    $\lnot H$\\
    $\therefore$ $S$
  \end{block}

\end{frame}

\begin{frame}
  \frametitle{Symbolization keys}

  \begin{definition}
    A symbolization key is a list that pairs \emph{sentence letters} with
    the basic English sentences they represent.
  \end{definition}

  For instance:

  \begin{block}{Symbolization key}
    $S$: Mandy enjoys skiing\\
    $H$: Mandy enjoys hiking
  \end{block}

\end{frame}

\begin{frame}
  \frametitle{Symbolization keys}

  \begin{itemize}[<+->]
  \item Sentence letters are UPPERCASE, possibly with subscripts
  (e.g., $H_1$, $H_2$).
  \item Usually the symbolization key is given to you \footnotesize{(to facilitate grading ;)}
  \item It should not be possible to further decompose the ``atomic
  sentences'' represented by sentence letters.

  \item[] For instance:
  \centerline{$B$: Mandy enjoys skiing or hiking}
  is a \textcolor{red}{bad} choice of symbolization.
  \end{itemize}
\end{frame}

\begin{frame}
  \frametitle{Paraphrase}

  \begin{itemize}[<+->]
  \item Successful symbolization sometimes requires \emph{paraphrase} to
  ensure atomic sentences appear explicitly.
  \item Two things to watch for: pronouns and coordination.
  \item Pronouns stand in for singular terms (e.g., names): \\ explicitly replace
  pronouns by those names.
  \item ``or'', ``both \dots and'', ``neither \dots nor'' can connect sentences but
  also noun phrases and verb phrases: \\ paraphrase them so that they connect
  sentences.
  \end{itemize}
\end{frame}

\begin{frame}
  \frametitle{Pronouns}

  \begin{block}{Example}
    If Mandy enjoys hiking, \emph{she} also enjoys skiing.

    Replace ``she'' by ``Mandy'':\\
    If [Mandy enjoys hiking] then [Mandy enjoys skiing].
  \end{block}
\end{frame}

\begin{frame}
  \frametitle{Coordination of noun phrases}

  \begin{block}{Example}
    \emph{Mandy and Sanjeev} enjoy hiking.

    Both [Mandy enjoys hiking] and [Sanjeev enjoys hiking].
  \end{block}

  \begin{block}{Example}
    Sanjeev lives in \emph{Erie or Chicago}.

    Either [Sanjeev lives in Erie] or [Sanjeev lives in Chicago].
  \end{block}

\end{frame}

\begin{frame}
  \frametitle{Exercise caution!}

  \begin{block}{Good}
    \emph{Mandy and Sanjeev} ate pizza.

    Both [Mandy ate pizza] and [Sanjeev ate pizza].
  \end{block}

  \begin{block}{\textcolor{red}{Bad}}
    \emph{Mandy and Sanjeev} ate the whole pizza.

    Both [Mandy ate the whole pizza] and [Sanjeev ate the whole pizza].
  \end{block}

\end{frame}

\begin{frame}
  \frametitle{Coordination of verb phrases}

  \begin{block}{Example}
  Mandy enjoys \emph{skiing or hiking}.

  Either [Mandy enjoys skiing] or [Mandy enjoys hiking].
  \end{block}

  \begin{block}{Example}
  If Sanjeev enjoys \emph{skiing and hiking}, he lives in Chicago.

  If [Sanjeev enjoys skiing] and [Sanjeev enjoys hiking], then [Sanjeev lives in Chicago].
  \end{block}

\end{frame}

\subsection{Basic symbolization}

\begin{frame}
  \frametitle{Negation}

  \begin{itemize}[<+->]
  \item \emph{Paraphrase} grammatical negation (``is not'', ``does not'') using the
  corresponding atomic sentence prefixed by ``\emph{it is not the case that}.''

  \item \emph{Symbolize} ``it is not the case that $A$'' as
  \emph{$\lnot A$}.

  \begin{block}{Example}
    \begin{itemize}[<+->]
    \item[] Mandy \emph{doesn't} enjoy skiing.
    \item[] It is not the case that [\emph{Mandy enjoys skiing}].
    \item[] \emph{It is not the case that} $S$.
    \item[] $\lnot S$
    \end{itemize}
  \end{block}
  \end{itemize}
\end{frame}

\begin{frame}
  \frametitle{Conjunction}

  \begin{itemize}[<+->]
  \item \emph{Paraphrase} sentences connected by ``and'', ``but'', ``even
  though'', ``yet'', and ``although'' using
  ``\emph{both $A$ and $B$}''

  \item \emph{Symbolize} ``both $A$ and $B$'' as
  \emph{$(A \land B)$}.

  \begin{block}{Example}
  \begin{itemize}[<+->]
  \item[] \emph{Even though} Mandy lives in Erie\emph{,} she enjoys hiking.

  \item[] Both [\emph{Mandy lives in Erie}] and [\emph{Mandy enjoys hiking}].

  \item[] \emph{Both} $E$ \emph{and} $H$.

  \item[] $(E \land H)$
  \end{itemize} 
  \end{block}
\end{itemize}

\end{frame}

\begin{frame}
  \frametitle{Disjunction (it's inclusive!)}

  \begin{itemize}[<+->]
  \item \emph{Paraphrase} sentences connected by ``or'' using\\
  ``\emph{either $A$ or $B$}''

  \item \emph{Symbolize} ``either $A$ or $B$'' as
  \emph{$(A \lor B)$}.

  \begin{block}{Example}
  \begin{itemize}[<+->]
  \item[] Sanjeev lives in Chicago \emph{or} Erie.

  \item[] Either [\emph{Sanjeev lives in Chicago}] or [\emph{Sanjeev lives in Erie}].

  \item[] \emph{Either} $C$ \emph{or} $E$.

  \item[] $(C \lor E)$
  \end{itemize}
  \end{block}
  
  \item[] Ignore the suggestion that ``either \dots or \dots'' is exclusive.
  We'll always treat it as inclusive unless explicitly stated.
\end{itemize}

\end{frame}

\begin{frame}
  \frametitle{Conditional (it's material!)}

  \begin{itemize}[<+->]
  \item \emph{Paraphrase} using ``\emph{if $A$ then $B$}'' any sentence of the form:
  \begin{itemize}[<+->]
  \item ``if $A$, $B$''
  \item ``$B$ if $A$'' (note order is reversed with lonely if!)
  \item ``$B$ provided $A$'';  and also  ``$B$ given $A$''
  \end{itemize}
  \item \emph{Symbolize} ``if $A$ then $B$'' as
  \emph{$(A \eif B)$}.
\item[] note our funny `horseshoe' symbol `\eif'.  \item[] \centerline{Round `em up! (On \textit{Carnap} use `$>$')}

  \begin{block}{Example}
  \begin{itemize}[<+->]
  \item Mandy enjoys hiking \emph{if} Sanjeev lives in Chicago.

  \item If [\emph{Sanjeev lives in Chicago}] then [\emph{Mandy enjoys hiking}].

  \item \emph{If} $C$ \emph{then} $H$.

  \item $(C \eif H)$
  \end{itemize}
  \end{block}
  \end{itemize}
\end{frame}

\begin{frame}{The parts of a conditional}

\begin{itemize}
  \item \emph{$(A \eif B)$} symbolizes:
  \begin{itemize}
    \item ``if $A$, $B$''
    \item ``$B$ if $A$'' (note order is reversed!)
    \item ``$B$ provided $A$''
  \end{itemize}
  \item $A$ is the \emph{antecedent}: it symbolizes the condition that has to
  be met for the ``then'' part to apply. (Like a promise!)
  \item $B$ is the \emph{consequent}: it symbolizes what must be true (e.g. to keep your promise!)
  if the antecedent condition is true.
\end{itemize}
\end{frame}

\begin{frame}
  \frametitle{Mix \& match}

  \begin{block}{Example}
    \begin{itemize}[<+->]
      \item[] Mandy doesn't enjoy hiking, \emph{provided} Sanjeev lives in Chicago or Erie.

  \item[] If [Sanjeev lives in Chicago \emph{or} Erie] then [Mandy \emph{doesn't} enjoy hiking].

  \item[] If \big(either [\emph{Sanjeev lives in Chicago}] or [\emph{Sanjeev lives in Erie}]\big)
  then \big(it is not the case that [\emph{Mandy enjoys hiking}]\big).

  \item[] If (either $C$ or $E$)
  then (it is not the case that $H$).

  \item[] $\big((C \lor E) \eif \lnot H \big)$
  \end{itemize}
  \end{block}

\end{frame}

\subsection{Conditionals}

\begin{frame}
  \frametitle{A logic puzzle}

\begin{itemize}
\item Every card has a letter on one side and a number on the other side.
\item You're a card inspector tasked with making sure that cards satisfy this quality standard:

\bigskip
\begin{quote}
If a card has an even number on one side, then it has a vowel on the other.
\end{quote}
\end{itemize}

\end{frame}

\frame<1>[label=wason]{
  \frametitle{A logic puzzle}

Which card(s) do you have to turn over to make sure that:
\bigskip

\begin{quote}
If a card has an even number on one side, then it has a vowel on the other.
\end{quote}

\begin{tabular}{cccc}
\begin{beamerboxesrounded}[width=5em]{}
\vskip 2ex
\hfil \Large E\hfil\\
\end{beamerboxesrounded} &
\begin{beamerboxesrounded}[width=5em]{}
\vskip 2ex
\hfil \Large \alert<2>{K}\hfil\\
\end{beamerboxesrounded} &
\begin{beamerboxesrounded}[width=5em]{}
\vskip 2ex
\hfil \Large 3\hfil\\
\end{beamerboxesrounded} &
\begin{beamerboxesrounded}[width=5em]{}
\vskip 2ex
\hfil \Large \alert<2>{4}\hfil\\
\end{beamerboxesrounded} \\
(1) & \alert<2>{(2)} & (3) & \alert<2>{(4)}
\end{tabular}

}

\begin{frame}
  \frametitle{Another logic puzzle}

\begin{itemize}
\item At an all-ages event where everyone has a drink
\item You know how old some of the people are, and you can tell what some of them are drinking
\item You're tasked with making sure that the following rule is followed:
\bigskip 

\begin{quote}
If a person is drinking alcohol, then they are at least 21 years old.
\end{quote}
\end{itemize}

\end{frame}

\begin{frame}
  \frametitle{Another logic puzzle}

Which of these people do you have to check (age or drink) to ensure
that:
\bigskip

\begin{quote}
If a person is drinking alcohol, then they must be at least 21 years old.
\end{quote}

\begin{tabular}{cccc}
\begin{beamerboxesrounded}[width=5em]{}
\vskip 2ex
\Large 22 years\\
\end{beamerboxesrounded} &
\begin{beamerboxesrounded}[width=5em]{}
\vskip 2ex
\Large \alert<2>{16 years}\\
\end{beamerboxesrounded} &
\begin{beamerboxesrounded}[width=5em]{}
\vskip 2ex
\Large drinks pop\\
\end{beamerboxesrounded} &
\begin{beamerboxesrounded}[width=5em]{}
\vskip 2ex
\Large \alert<2>{drinks beer}\\
\end{beamerboxesrounded} \\
(1) & \alert<2>{(2)} & (3) & \alert<2>{(4)}
\end{tabular}

\end{frame}

\againframe<2|handout:0>{wason}

\begin{frame}
  \frametitle{Truth conditions of Material conditionals}

\[\text{If\ } \underbrace{\text{X is drinking alcohol}}_{A},
\text{then\ }\underbrace{\text{X is over 21}}_{B}\]

\begin{itemize}[<+->]
\item ``If $A$, then $B$'' can only be \emph{false} if:
\begin{itemize}
\item $A$ is \emph{true}: we check age if X is drinking beer ($A$~true), not if
 drinking pop; \emph{and}
\item $B$ is \emph{false}: we check drink if X underage ($B$ false),\\ not
if over 21 ($B$ true)
\end{itemize}
\item ``If $A$, then $B$'' is true if:
\begin{itemize}
\item $A$ is \emph{false} (we don't check people drinking pop); \emph{or}
\item $B$ is \emph{true} (those 21+ can drink whatever they want!);
\item (or both)
\end{itemize}
\end{itemize}
\end{frame}

\frame{\frametitle{Understanding material conditionals as promises}
%\large
\begin{itemize}%[<+->]

\item A promise: \textit{if Sanjeev has to go to the airport, then I will drive him}

\item<2-> Notice that you \textit{vacuously} keep your promise if Sanjeev never goes to the airport

\item<2-> You can't break a promise whose conditions are not satisfied

\bigskip

\item<3-> The material conditional is just like this: it is \textit{vacuously} true whenever the antecedent is false

\item<3-> You'll have to get used to this!

\item<4-> Other conditionals (e.g. causal, subjunctive, counterfactual) are not  \textit{truth-functional} (so handling them is controversial!)

\end{itemize}
}


\subsection{``Only if'' and ``unless''}

\begin{frame}
  \frametitle{`\emph{only if}' vs. a lonely `\textcolor{blue}{if}'}

\begin{itemize}[<+->]
  \item Sue drinks beer ($A$) \emph{only if} she is over 21 ($B$)
  \[
  A \eif B
  \]
  \item False if Sue drinks beer ($A$), but is underage (\enot $B$)

%notice that order is preserved!!!! just like if P, then Q. 


  \item Sue drinks beer ($A$), \textcolor{blue}{if} she is over 21 ($B$).
  \[
  B \eif A
  \]
  \item False if she's 25 ($B$), but drinks pop (\enot $A$).
  \item Not false if she's 16 and drinking beer.
\end{itemize}
\end{frame}

\begin{frame}
  \frametitle{Conditional recap: \emph{only if} you can't remember!}

  \begin{itemize}[<+->]
  \item \emph{Paraphrase} ``$A$ only if $B$'' as
  ``\emph{if $A$ then $B$}''.
  \item \emph{Symbolize} ``$A$ only if $B$'' as
  \emph{$(A \eif B)$}.
  \item Note: 
    \begin{itemize}
 \item ``$A$ \textcolor{blue}{if} $B$'' is $(\textcolor{blue}{B \eif} A$) (lonely if) \\
  \item ``$A$ only if $B$'' is $(A \eif B$) (same order as `if P, then Q')
    \end{itemize}

\bigskip

\item We'll come back to the \textit{biconditional}:
  \item[] \emph{Symbolize} ``$A$ if and only if $B$'' as
  \emph{$(A \eiff B)$}.
\end{itemize}
\end{frame}


\frame{\frametitle{Conditional Practice: only if vs. lonely if}
\large
\begin{itemize}[<+->]

\item[] Schematize the following sentence:

\item[] \textit{Jack will go to the store only if Susie goes to the store, and Susie will go if Beatrice goes.}

\bigskip

\item[] Symbolization Key:\\$\bullet$ $J $: Jack will go to the store\\ \medskip $\bullet$ $S $: Susie goes to the store\\ \medskip $\bullet$ $B $: Beatrice goes to the store

\bigskip

\item[] Answer: $(J \eif S) \eand (B \eif S)$



\end{itemize}
}


\begin{frame}
\frametitle{Unless (confusing unless you use a trick!)}

Which of these people do you have to check (age or drink) to ensure that:
\begin{quote}
People are drinking pop unless they are over 21.
\end{quote}

\begin{tabular}{cccc}
\begin{beamerboxesrounded}[width=5em]{}
\vskip 2ex
\Large 22 years\\
\end{beamerboxesrounded} &
\begin{beamerboxesrounded}[width=5em]{}
\vskip 2ex
\Large 16 years\\
\end{beamerboxesrounded} &
\begin{beamerboxesrounded}[width=5em]{}
\vskip 2ex
\Large drinks pop\\
\end{beamerboxesrounded} &
\begin{beamerboxesrounded}[width=5em]{}
\vskip 2ex
\Large drinks beer\\
\end{beamerboxesrounded} \\
(1) & (2) & (3) & (4)
\end{tabular}

\end{frame}


\begin{frame}
\frametitle{Unless}

\[\underbrace{\text{X is drinking pop}}_{A},
\text{unless\ }\underbrace{\text{X is over 21}}_{B}\]

\begin{itemize}[<+->]
\item ``$A$ unless $B$'' can only be \emph{false} if:
\begin{itemize}
\item $A$ is \emph{false}\\
(we check age if person is drinking beer), \emph{and}
\item $B$ is \emph{false}\\
(we check drink if person not at least 21)
\end{itemize}
\item ``$A$ unless $B$'' is true (test OK) if
$A$ or $B$ or both are true.
\item ``$A$ unless $B$'' can be paraphrased and symbolized by:
\begin{itemize}
\item ``$A$ if not $B$'' ($\enot B \eif A$)
\item ``either $A$ or $B$'' ($A \eor B$) [\emph{Remember this one}!!!]
\end{itemize}
\end{itemize}

\end{frame}

\begin{frame}
  \frametitle{Trick for handling `Unless'}

  Treat ``\emph{unless}'' the same way you would treat ``or''

  \begin{block}{Example}
  Mandy enjoys hiking unless Sanjeev lives in Chicago.

  $(H \eor C)$
  \end{block}

\begin{itemize}[<2->]

\item Since disjunction is symmetric, you won't have to remember order of atomic sentences if you symbolize `unless' using `or'.

\item So you can't go wrong with this approach!
\end{itemize}

\end{frame}






\subsection{More connectives}

\begin{frame}
  \frametitle{\emph{Biconditional}: If and only if (`\eiff')}

  \begin{block}{Example}
    \begin{itemize}[<+->]
    \item[] Mandy enjoys hiking \emph{if and only if} she enjoys skiing.
    \item[] Both [if $S$ then $H$] and [if $H$ then $S$].
    \item[] $((S \eif H)\eand(H \eif S))$
    \item[] $(H  \emph{\eiff} S)$
    \end{itemize}
  \end{block}

Unlike the conditional, the biconditional is symmetric: the order of atomic sentences does not matter!

\end{frame}

\begin{frame}
  \frametitle{Exclusive or}

  \uncover<7->{\emph{Paraphrase} sentences containing ``either $A$ or $B$ but not both'' using\\
  ``\emph{both [either $A$ or $B$] and\\ \qquad [it is not the case
  that [both $A$ and $B$]]}''}

  \begin{block}{Example}
  Mandy enjoys \alert<1>{hiking or skiing} \alert<5>{but} \alert<3>{not both}.

  \uncover<6->{Both} \uncover<2->{[either $H$ or $S$]} \uncover<6->{and}\\ \phantom{Both} \uncover<4->{[it is not the case that [both
  $H$
  and $S$]].}

  \uncover<8->{$((H \eor S) \land \lnot(H \eand S))$}
  \end{block}

\uncover<9>{An alternative: $(H \eand \enot S) \eor (S \eand \enot H)$ \\ i.e. Either (H and not S) or (S and not H)}

\end{frame}

\begin{frame}
  \frametitle{Neither \dots nor \dots (`nand')}

  \uncover<3->{\emph{Paraphrase} sentences containing ``neither $A$ nor $B$'' using\\
  ``\emph{both [it is not the case that $A$] and\\ \qquad [it is not the case
  that $B$]}''}

  \begin{block}{Example}
  Mandy enjoys neither hiking nor skiing.

  \uncover<2->{Both [it is not the case that $H$] and\\ \qquad [it is not the case that $S$].}

  \uncover<4->{$(\lnot H \land \lnot S)$}
  \end{block}

 \uncover<5>{This connective has a special metalogical property that we might discuss at some point!}

\end{frame}

\begin{frame}
  \frametitle{Mix \& match}
  \begin{block}{Example}
  Sarah lives in Chicago or Erie.\\
  \uncover<2->{\emph{Either [Sarah lives in Chicago] or [Sarah lives in Erie].}}\\
  Amir lives in Chicago unless he enjoys hiking.\\
  \uncover<3->{\emph{Either [Amir lives in Chicago] or [Amir
    enjoys hiking].}}\\
  If Amir lives in Chicago, Sarah doesn't.\\
  \uncover<4->{\emph{If [Amir lives in Chicago] then [it is not
    the case that [Sarah lives in Chicago]].}}\\
  Neither Sarah nor Amir enjoy hiking.\\
  \uncover<5->{\emph{Both [it is not the case that [Sarah enjoys
      hiking]] and [it is not the case that [Amir enjoys hiking]].}}\\
  $\therefore$ Sarah lives in Erie.
  \end{block}
\end{frame}

\begin{frame}
  \frametitle{Mix \& match}

  \begin{block}{Example}
  Sarah lives in Chicago or Erie.\\
  \alert{$(C \eor E)$}\\
  Amir lives in Chicago unless he enjoys hiking.\\
  \alert{$(A \eor M)$}\\
  If Amir lives in Chicago, Sarah doesn't.\\
  \alert{$(A \eif \enot C)$}\\
  Neither Sarah nor Amir enjoy hiking.\\
  \alert{$( \enot S \eand\enot M)$}\\
  $\therefore$ Sarah lives in Erie.\\
  \alert{$\therefore\ E$}.
  \end{block}
\end{frame}

\subsection{Defining Formulae in SL}

\frame{\frametitle{Will the real SL please stand up?}
%\large

What we have said so far about Sentential Logic (SL): 
\begin{itemize}[<+->]

\item Uppercase letters, subscripts allowed: e.g. $J, H, B, N_3$

\item Punctuation: parentheses `(' and `)' ;(no brackets or braces!)

\item Connectives: $\enot, \eand, \eor, \eif, \eiff$

\item But so is any string of these symbols a legitimate formula of SL?

\item e.g. $((((A\eand \enot (B \eif C \eiff D((( $ 

\end{itemize}
}

\frame{\frametitle{Expressions vs. Well-formed Formula (WFFs)}
%\large
\begin{itemize}[<+->]

\item \emph{Expression of SL}: any finite string of symbols from language SL

\begin{itemize}[<+->]

\item Includes ridiculous-seeming strings, e.g. $((((())LOGIC_1FOREVER((($ 

\end{itemize}

\item \emph{Well-formed Formulae} (WFFs): often shortened to `formulae' or `\emph{sentences}' of SL

\begin{itemize}[<+->]

\item These are the expressions we are interested in

\item They end up being expressions that are true or false under an assignment of truth values to atomic sentence letters 

\end{itemize}

\item But how do we rigorously define the WFFs as a subset of the expressions? 

\end{itemize}
}

\frame{\frametitle{Look it's a bird! It's a plane! It's....RECURSION!}
%\large

We define the well-formed formulae (wffs) \textit{recursively}: 
\begin{enumerate}[<+->]

\item Each atomic sentence is a wff (base case)

\item If $\metav{P}$ is a wff, then so is $\enot \metav{P}$

\item If $\metav{P}$ and $\metav{Q}$ are both wffs, then so are:

\begin{itemize}[<+->]

\item   $(\metav{P} \eand  \metav{Q})$

\item   $(\metav{P} \eor  \metav{Q})$

\item   $(\metav{P} \eif  \metav{Q})$

\item   $(\metav{P} \eiff  \metav{Q})$

\end{itemize}

\item And that's all folks! (No other expressions of SL are wffs)


\end{enumerate}
}

\frame{\frametitle{Identifying the Main Connective}
%\large
\begin{itemize}[<+->]

\item Apply the recursive definition in reverse! 

\bigskip

\item Ask if the sentence is a negation of another wff

\item If not, locate the two wffs in the scope of the outermost parentheses, and figure out which connective combines them 

\item Rinse and repeat: stop when you reach the atomic sentences

%\item Work backwards from `largest' sentential components to the `smallest' ones

\bigskip

\item $\Big ( \enot \big (A \eif (\enot B \eor C) \big ) \eand \big ((A \eiff B) \eif C \big) \Big )$

\item $\Big ( \phantom{vvvv}   \metav{P}$ \phantom{vvvvvvv} $\eand$ \phantom{vvvv} $ \metav{Q} \phantom{vvvvv}        \Big )$

\end{itemize}
}

\frame{\frametitle{But wait? What are these `$\metav{P}$'s and `$\metav{Q}$'s}
%\large
\begin{itemize}[<+->]

\item $\metav{P}$ is a \emph{metavariable} we use to talk about (to mention!) an arbitrary expression from SL

\item The symbol `$\metav{P}$' is not itself an expression in SL

\item Rather, `$\metav{P}$' is part of our \emph{metalanguage}, which in this case is English, augmented by a bunch of symbols

\item So technically, `$\enot \metav{P}$' is not a sentence of SL! So we define a convention: `$\enot \metav{P}$' abbreviates the result of concatenating the sentence $\metav{P}$ with the negation symbol `$\enot$'. 

\item BOOM! 

\item Who knew that air-quotes (`scare-quotes'?) could play such an important role in the foundations of logic? 

\end{itemize}
}

\subsection{Use vs. `Mention'}

\frame{\frametitle{Use vs. `Mention'}
%\large
\begin{itemize}

\item When we \emph{use} a word or symbol, it does not have quotation marks:

\begin{itemize}[<+->]

\item Logic is a fascinating subject!

\item You are sitting through a logic lecture.

\item Logic is a 21st-century American rapper. 

\end{itemize}

\bigskip

\item<3-> When we \emph{mention} a word or symbol, we use quotation marks:

\begin{itemize}[<+->]

\item The name of this course is `Logic I' 

\item There is a 21st-century American rapper called `Logic'

\item The rapper Sir Robert Hall goes by the name `Logic' 

\item We symbolize conjunction as `\eand' and disjunction using `\eor'.  

\end{itemize}

%\item<4-> See the handout for additional examples!

\end{itemize}
}

\frame{\frametitle{Why this matters: Object vs. Metalanguage}
%\large
\begin{itemize}[<+->]

\item We talk \textit{about} an object language by \textit{using} a metalanguage

\item SL is an object language we talk about using English

\item When we talk about SL, we MENTION expressions in SL, and thereby should technically put quotation marks around these:

\begin{itemize}[<+->]

\item \textcolor{red}{BAD}: $P \eand Q$ is a sentence in SL

\item \emph{GOOD}: ``$P \eand Q$'' is a sentence in SL

\item We have to mention ``$P \eand Q$'' since it is not a sentence in English; it's a sentence in SL. So we can't \textit{use} it in English.

\end{itemize}

\item But psssst: we'll often be lazy and won't put quotes where we're technically supposed to...

\end{itemize}
}



\frame{\frametitle{Use vs. `Mention'}
%\large
\begin{itemize}[<+->]

\item Some more detailed glosses, in case one of these clicks:

\item When we use a word, it does not have quotation marks

\begin{itemize}[<+->]

\item The thing referenced by the word is being \textit{put to good use}
% by the sentence

\item Barack Obama was the 44th president of the USA

\end{itemize}

\item When we mention a word or string of symbols in a sentence, we put quotation marks around them to show that we are mentioning something:

\begin{itemize}[<+->]

\item  ``Obama'' names the 44th president of the USA

\item ``Obama'' names Obama; ``Obama'' is a name.

\item  `` ``Obama'' '' is a quotation-name (i.e. a name of a name)

\item ``$p \eand \enot q$'' is a conjunction, whose second conjunct is a negation.

%\item<4-> See the handout for additional examples!

\end{itemize}

\end{itemize}
}

\frame{\frametitle{A Mistake on the Wiki!}
\large
\begin{itemize}
 
 
 \item On a Wiki page discussing the geographical usage of `soda' vs. `pop' in the USA
 
 \item Sentence: \textit{To a lesser extent soda is also fairly common further down the east coast in eastern Virginia, eastern Carolinas and coastal Florida. Here, soda is not too dominant but competes with multiple other terms}
 
 \item \href{https://en.wikipedia.org/wiki/Names_for_soft_drinks_in_the_United_States}{Cite your sources!}


\end{itemize}
}

\frame{\frametitle{Use vs. `Mention' PRACTICE!}
%\large

Punctuate the following sentences with quotation-marks so as to make them true:

\begin{enumerate}[<+->]

\item Hunt is the last name of the instructor for this course. 

\item Hunt is a surname which begins with the letter H.   

\item I call my dog Ivy Ivy, but I never use Ivy's first name in addressing her.

\end{enumerate}

 \uncover<5->{Answers:}

\begin{enumerate}[<+->]

\item ``Hunt'' is the last name of the instructor for this course.

\item ``Hunt'' is a surname which begins with the letter ``H''. 

\item I call my dog Ivy ``Ivy'', but I never use ``Ivy's first name'' in addressing her.

\end{enumerate}

%\item[] \textit{``Hunt'' is the last name of the instructor of this course}

}

%%Some more examples, if you're into this sort of thing!:

% “p ⸧ p” is a valid schema.   

%Hunt is a quotation-name of the last name of the instructor of this course.   

%Answer: “ “Hunt””  is a quotation-name of the last name of the instructor the of this course

%“ “Hunt” ” is a quotation-name of a surname and begins with a pair of left quotes.   

%“ “ “Hunt” ” ” is the quotation-name of the quotation-name of a surname and begins with two pairs of left quotes.  

%Example of a Use-mention mistake on wiki discussion of ``soda'' vs. ``pop''
%``To a lesser extent soda is also fairly common further down the east coast in eastern Virginia, eastern Carolinas and coastal Florida. Here, soda is not too dominant but competes with multiple other terms''
%they mean to say the WORD soda! 
%https://en.wikipedia.org/wiki/Names_for_soft_drinks_in_the_United_States#:~:text=The%20more%20sharper%20%22soda%2Fpop,Syracuse)%20use%20%22soda.%22




\subsection{Ambiguity}

\begin{frame}
    \frametitle{Types of ambiguity}

\begin{itemize}
  \item Lexical ambiguity: one word---many meanings \\
  e.g., ``bank'', ``crane''
  \item Syntactic ambiguity: one sentence---many readings\\
  e.g.,
  \begin{itemize}
  \item ``Flying planes can be dangerous'' (Chomsky)
  \item ``One morning I shot an elephant in my pajamas.\\ How he got in my pajamas, I don't know.'' (Groucho Marx)
  \end{itemize}
\end{itemize}

\end{frame}

\begin{frame}
  \frametitle{Ambiguity of \eand{} and \eor}

  \begin{itemize}[<+->]
    \item English sentences don't have parentheses.
    \item This can lead to ambiguity, e.g.,
    \begin{earg}
      \item[] Ahmed admires Brit and Cara or Dina.
    \end{earg}
    \item It might mean one of:
    \begin{earg}
      \item[] Ahmed admires either [both Brit and Cara] or Dina.
      \item[] Ahmed admires both Brit and [either Cara or Dina].
    \end{earg}
    \item In SL, symbolizations are unambiguous:
    \begin{earg}
      \item[] $((B \eand C) \eor D)$
      \item[] $(B \eand (C \eor D))$
    \end{earg}
  \end{itemize}
\end{frame}















%JRH: initially pretty difficult to understand the following example. Wiki page helps:
%``Casement's crimes had been carried out in Germany and the Treason Act 1351 seemed to apply only to activities carried out on English (or arguably British) soil. A close reading of the Act allowed for a broader interpretation: the court decided that a comma should be read in the unpunctuated original Norman-French text, crucially altering the sense so that "in the realm or elsewhere" referred to where acts were done and not just to where the "King's enemies" might be."

\begin{frame}
  \frametitle{The man who was hanged by a comma}

\begin{columns}
\begin{column}{3cm}
\pgfimage[width=3cm]{../assets/casement}
\end{column}
\begin{column}{7cm}
\begin{itemize}
\item Sir Roger Casement (1864--1916)
\item British consul to Congo and Peru
\item Tried to recruit Irish revolutionaries in Germany during WWI
\item Tried for treason
\end{itemize}
\end{column}
\end{columns}

\end{frame}

\begin{frame}
  \frametitle{Treason Act of 1351 (20th century court added the comma!)}

\small
ITEM, Whereas divers Opinions have been before this Time in what Case
Treason shall be said, and in what not; the King, at the Request of
the Lords and of the Commons, hath made a Declaration in the Manner as
hereafter followeth, that is to say; When a Man doth compass or
imagine the Death of our Lord the King, or of our Lady his Queen or of
their eldest Son and Heir; or if a Man do violate the King's
Companion, or the King's eldest Daughter unmarried, or the Wife of the
King's eldest Son and Heir; or \emph{if a Man} do levy War against our Lord
the King in his Realm, or \emph{be adherent to the King's Enemies in his
Realm, giving to them Aid and Comfort in the Realm}\onslide<2>{\colorbox{highlightbg}{\textbf{,}}} \emph{or elsewhere}, and
thereof be probably attainted of open Deed by the People of their
Condition: \dots And it is to be
understood, that in the Cases above rehearsed, that ought to be judged
Treason which extends to our Lord the King, and his Royal Majesty:
\dots

\end{frame}

\begin{frame}
  \frametitle{R v. Casement in SL}
  \begin{itemize}[<+->]
\item Symbolization key:
  \begin{ekey}
    \item[A] Casement was adherent to the King's enemies in the realm. (false)
    \item[G] Casement gave aid and comfort to the King's enemies in the realm. (false)
    \item[B] Casement was adherent to the King's enemies abroad. (true)
    \item[H] Casement gave aid and comfort to the King's enemies abroad. (false)
  \end{ekey}

  \item Without the comma, `treason' defined as:
  \begin{earg}
    \item[] $A \lor (G \lor H)$
  \end{earg} \item With the comma, `treason' now defined as:
  \begin{earg}
    \item[] $(A \lor B) \lor (G \lor H)$ \hspace{3em} [and $B$ \emph{is} true]
  \end{earg}
\end{itemize}
\end{frame}

%still not sure about this schematization b/c A (adherent in realm) seems to be DEFINED as sentence G, and B (adherent elsewhere) seems to be defined as H. 
%so this example might be worth ditching!

\subsection{Practice Problems!}

%goal is to do some of these 50 minutes in on both days, although some risk of running out of time I suppose on second day....we'll see! could always run use/mention into week 2. since can't see truth tables taking two full days anyways! 

\frame{\frametitle{Validity, Paraphrasing, Symbolizing}
%\large

Classify the following arguments as sound, valid but unsound, or invalid. Next, paraphrase the argument in logical form. Finally, symbolize the argument in SL. 
% (assign sentence letters to the atomic sentences and express using logical connectives).
\begin{enumerate}[<+->]

\item Detroit will win the Super Bowl this year if they didn't make the playoffs. Detroit didn't make the playoffs. Therefore: Detroit will win the Super Bowl this year. 

\item If Aristotle was a student of Plato, then Aristotle lived in Athens. Aristotle lived in Athens. Hence, Aristotle was Plato's student.

\item Cleveland is a city in Ohio. Pittsburgh is a city in Ohio. Therefore: there are at least two cities in Ohio.

\item Either Toledo is in Michigan or the Upper Peninsula is in Michigan. Toledo is not in Michigan. Therefore: the Upper Peninsula is in Michigan.

\end{enumerate}
}

\frame{\frametitle{Solution to Question 1:}
%\large
\begin{itemize}[<+->]

\item Valid argument (e.g. of modus ponens---affirming the antecedent) \\ but unsound: the first premise (the conditional) is false

\item Paraphrase of this argument:

\begin{itemize}[<+->]

\item[] \underline{If} (\underline{it is not the case that} Detroit made the playoffs) \underline{then} (Detroit will win the Super Bowl). 
\item[] \underline{It is not the case that} Detroit made the playoffs. 
\item[] Therefore, Detroit will win the Super Bowl. 

\end{itemize}

\item Symbolization in SL: 

\begin{itemize}[<+->]

\item Let `P' stand for `Detroit made the playoffs' 
\item Let `D' stand for `Detroit will win the Super Bowl' 

\item[] $(\enot P) \eif D$

\item[] $\enot P$ 

\item[] $\therefore$ $D$

\end{itemize}

\end{itemize}
}


\frame{\frametitle{More Complex Symbolizations}
%\large

\small{Paraphrase the following English sentences using our logical symbolism. Any condensed clauses should be expanded into complete sentences. Identify as well the main connective and practice symbolizing with atomic sentences.} 
%(Make sure you provide a symbolization key)

\begin{enumerate}

\item If they either drain the swamp and reopen the road or dredge the harbor, they will provide the uplanders with a market and themselves with a bustling trade.

\item Unless his new novel does very well and gets him a large advance, Malone will take a position in a college writing program or he will take out a second mortgage and not sell his car.

\item If Germany annexes Austria, Czechoslovakia will be militarily defensible only if France honors her treaty obligations and arranges for the transit of Soviet troops across Poland or Romania.

\end{enumerate}
}

\frame{\frametitle{Drain the Swamp!}
%\large

Sentence: \textit{If they either drain the swamp and reopen the road or dredge the harbor, they will provide the uplanders with a market and themselves with a bustling trade.}

\begin{itemize}

\item<2-> Paraphrase: \underline{\textbf{IF}} [(they drain the swamp \underline{and} they reopen the road) \underline{or} (they dredge the harbor)] \underline{\textbf{THEN}} (they will provide the uplanders with a market \underline{and} they will provide themselves with a bustling trade). 

\item<3-> Alternative paraphrase using connectives:  [(they drain the swamp $\eand$ they reopen the road) \eor (they dredge the harbor)] \eif (they will provide the uplanders with a market $\eand$ they will provide themselves with a bustling trade)]. 

\end{itemize}
}

\frame{\frametitle{Drain the Swamp (still draining...)}
%\large

Paraphrase: \underline{\textbf{IF}} [(they drain the swamp \underline{and} they reopen the road) \underline{or} (they dredge the harbor)] \underline{\textbf{THEN}} (they will provide the uplanders with a market \underline{and} they will provide themselves with a bustling trade). 

Symbolization Key:

\begin{itemize}

\item[] S: they drain the swamp 

\item[] R: they reopen the road

\item[] H: they dredge the harbor

\item[] M: they will provide the uplanders with a market

\item[] T: they will provide themselves with a bustling trade

\end{itemize}

\uncover<2->{Symbolization-answer: $\big( (S \eand R) \eor H\big) \eif (M \eand T)$}

%seems like Carnap is fine w/ parentheses around atomic sentence letters, but technically this violates inductive definition haha, e.g. `(H)' 
 
}































