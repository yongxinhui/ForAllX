\documentclass[a4paper, 11pt]{article} % Font size (can be 10pt, 11pt or 12pt) and paper size (remove a4paper for US letter paper)
\usepackage[protrusion=true,expansion=true]{microtype} % Better typography
\usepackage{graphicx} % Required for including pictures
\usepackage{wrapfig} % Allows in-line images
\usepackage{enumitem} %%Enables control over enumerate and itemize environments
\usepackage{setspace}
\usepackage{amssymb, amsmath, mathrsfs} %%Math packages
\usepackage{stmaryrd}
\usepackage{mathtools}
\usepackage{mathpazo} % Use the Palatino font
\usepackage[T1]{fontenc} % Required for accented characters
\usepackage{array}
\usepackage{bibentry}
\usepackage[round]{natbib} %%Or change 'round' to 'square' for square backers
\setcitestyle{aysep={}}

% \linespread{1} % Change line spacing here, Palatino benefits from a slight increase by default

\newcommand{\corner}[1]{\ulcorner#1\urcorner} %%Corner quotes
\newcommand{\tuple}[1]{\langle#1\rangle} %%Angle brackets
\newcommand{\set}[1]{\lbrace#1\rbrace} %%Set brackets
\newcommand{\interpret}[1]{\llbracket#1\rrbracket} %%Double brackets
%\DeclarePairedDelimiter\ceil{\lceil}{\rceil}    
\def\therefore{\ensuremath{\ldotp\dot{}\,\ldotp}}
\newcommand{\I}{\mathcal{I}}
\newcommand{\V}[1]{\mathcal{V}_{#1}} %%Corner quotes

\makeatletter
\renewcommand\@biblabel[1]{\textbf{#1.}} % Change the square brackets for each bibliography item from '[1]' to '1.'
\renewcommand{\@listI}{\itemsep=0pt} % Reduce the space between items in the itemize and enumerate environments and the bibliography

\renewcommand{\maketitle}{ % Customize the title - do not edit title and author name here, see the TITLE block below
\begin{flushright} % Right align
{\LARGE\@title} % Increase the font size of the title

\vspace{10pt} % Some vertical space between the title and author name

{\@author} % Author name
\\\@date % Date

\vspace{-10pt} % Some vertical space between the author block and abstract
\end{flushright}
}

%----------------------------------------------------------------------------------------
%	TITLE
%----------------------------------------------------------------------------------------

\title{\textbf{Logical Entailment}} % Subtitle

\author{\textsc{Logic I}\\ \em Benjamin Brast-McKie} % Institution

\date{\today} % Date

%----------------------------------------------------------------------------------------

\begin{document}

\maketitle % Print the title section

\thispagestyle{empty}

%----------------------------------------------------------------------------------------

% NOT USED LAST TIME

% \section*{The Material Conditional}
%
% \subsubsection*{\it \textbf{Roses}}
%
% \begin{earg}
%   \uitem{Sugar is sweet.}
%   \eitem{The roses are only red if sugar is sweet.}
% \end{earg}
%
%
% \begin{itemize}[leftmargin=1in,labelsep=.15in] %,label=(\arabic*)]%,label=\roman*]
%   % \item[\bf Task:] Regiment this argument and construct its truth table.
%   \item[\bf Observe:] First paradox of the material conditional.
% \end{itemize}
%
%
%
%
% \subsubsection*{\it \textbf{Vacation}}
%
% \begin{earg}
%   \uitem{Casey is not on vacation.}
%   \eitem{If Casey is on vacation, then he is in Paris.}
% \end{earg}
%
%
%
%
%
% \subsubsection*{\it \textbf{Crimson}}
%
% \begin{earg}
%   \eitem{Mary doesn't like the ball unless it is crimson.}
%   \uitem{Mary likes the ball.}
%   \eitem{If the ball is blue, then Mary likes it.}
% \end{earg}
%
%
%
%
%
%
% \section*{The Biconditional}
%
% \subsubsection*{\it \textbf{Rectangle}}
%
% \begin{earg}
%   \eitem{The room is a square.}
%   \uitem{The room is a rectangle.}
%   \eitem{The room is a square if and only if it is a rectangle.}
% \end{earg}
%
%
%
%
%
% \subsubsection*{\it \textbf{Work}}
%
% \begin{earg}
%   \eitem{Kin isn't a professor.}
%   \uitem{Sue isn't a chef.}
%   \eitem{Kin is a professor just in case Sue is a chef.}
% \end{earg}


% \section*{Logical Consequence}
%
% \begin{itemize}[leftmargin=1.5in,labelsep=.15in] %,label=(\arabic*)]%,label=\roman*]
%   \item[\it Logical Truths:] $\metaA$ is a \textsc{tautology} of $\PL$ iff $\V{\I}(\metaA) = 1$ for all $\I$. 
%   \item[\it Logical Falsehoods:] $\metaA$ is a \textsc{contradiction} of $\PL$ iff $\V{\I}(\metaA) = 0$ for all $\I$. 
%   \item[\it Logical Entailment:] $\metaA$ \textsc{logically entails} $\metaB$ iff $\V{\I}(\metaA) \leq \V{\I}(\metaB)$ for all $\I$. 
%   \item[\it Logical Consequence:] \mbox{$\Gamma \vDash \metaA$ iff $\V{\I}(\metaA)=1$ for all $\I$ where $\V{\I}(\gamma)=1$ for all $\gamma \in \Gamma$.}
%   \item[\bf Question:] Which arguments might we hope to show to have valid regimentations in $\PL$?
% \end{itemize}


  % \item[\it Truth:] Soundness reaches beyond the scope of any logic course since truth on an interpretation requires subject-specific knowledge.
  % \item[\it Tautology:] We can talk about sentences of formal languages being \textit{tautologies} just in case they are true on all interpretations.
  % \item[\it Contradiction:] A sentence is a \textit{contradiction} just in case it is false on all interpretations.
  % \item[\it Logical Entailment:] One sentence \textit{logically entails} another just in case every interpretation in which the former is true also makes the latter true.
  % \item[\it Logical eiffalence:] One sentence is \textit{logically eiffalent} to another just in case they logically entail each other.
  % \item[\it Consistency:] A set of sentences is \textit{consistent} just in case there is an interpretation which makes every sentence in the set true, and \textit{inconsistent} otherwise.




\section*{Logical Entailment}

\begin{enumerate}[leftmargin=1.2in,labelsep=.15in] %,label=(\arabic*)]%,label=\roman*]
  % \item[\it Soundness:] An argument is \textit{sound} just in case it is both valid and has true premises.
  % \item[\it Truth:] Soundness reaches beyond the scope of any logic course since truth on an interpretation requires subject-specific knowledge.
  \item[\it Satisfaction:] An interpretation $\I$ of SL \textit{satisfies} a set of SL sentences $\Gamma$ \textit{iff} $\V{\I}(\varphi)=1$ for all $\varphi \in \Gamma$. Derivatively, an interpretation $\I$ of SL \textit{satisfies} a sentence $\varphi$ of SL \textit{iff} $\I$ satisfies $\set{\varphi}$.
  \item[\it Logical Entailment:] $\Gamma \vDash \varphi$ \textit{iff} every SL interpretation $\I$ that satisfies $\Gamma$ also satisfies $\varphi$.
  \item[\it Validity:] An argument in SL is \textit{valid} just in case its conclusion is true in any interpretation in which its premises are true.
  \item[\it Question:] How are we to describe the space of all valid arguments?
  \item[\it Answer:] In terms of entailment.
  \item[\bf Task 1:] Show that validity and entailment are distinct:
    \begin{itemize}
      \item $\Gamma \vDash \varphi$ does not determine a unique argument. 
      \item Entailment does not order the premises.
      \item Entailment admits of infinitely many premises.
      \item Entailment admits of no premises.
    \end{itemize}
  \item[\it Tautology:] An SL sentence $\varphi$ is a \textit{tautology} just in case $\vDash \varphi$.
  \item[\it Weakening:] If $\Gamma \vDash \varphi$, then $\Gamma \cup \Sigma \vDash \varphi$.
\end{enumerate}




\section*{Unsatisfiable}

\begin{enumerate}[leftmargin=1.2in,labelsep=.15in] %,label=(\arabic*)]%,label=\roman*]
  \item[\it Absurdity:] A contradiction entails everything: $A \wedge \neg A \vDash B$. 
  \item[\it Bottom:] Let `$\bot$' abbreviate any contradiction.
  \item[\it Unsatisfiable:] A sentence is \textit{unsatisfiable} just in case $\Gamma \vDash \bot$.
  \item[\bf Task 2:] Show that a set of SL sentences is unsatisfiable just in case no SL interpretation satisfies it.
  \item[\it Consistency:] Recall: a set of SL sentences is \textit{consistent} just in case there is a line on the complete truth table for those sentences which makes them all true, and \textit{inconsistent} otherwise.
  \item[\bf Task 3:] Show that consistency and satisfiability are co-extensional.
\end{enumerate}



\section*{Examples}

\noindent
Which sets of sentences are consistent? (e.g., is $\set{(1),(2)}$ consistent?)

\subsection*{\it \textbf{Taller}}

\begin{enumerate}
  \item[(1)] Liza is taller than Sue.
  \item[(2)] Sue is taller than Paul.
  \item[(3)] Paul is taller than Liza.
\end{enumerate}




\subsection*{\it \textbf{Lost}}

\begin{enumerate}
  \item[(4)] Kim is either in Somerville or Cambridge.
  \item[(5)] If Kim is in Somerville, then she is not far from home.
  \item[(6)] If Kim is not far from home, then she is in Cambridge.
  \item[(7)] Kim is not in Cambridge.
\end{enumerate}






\section*{Methods}

\begin{enumerate}[leftmargin=1.2in,labelsep=.15in] %,label=(\arabic*)]%,label=\roman*]
  \item[\it Truth Tables:] Mechanical but tedious.
  \begin{itemize}
    \item Bad if there are lots of sentence letters.
    \item Good for counterexamples.\\
      $A \equiv (B \supset C),\ A \wedge \neg B,\ D \vee \neg A\ \therefore C$.
  \end{itemize}
  \item[\it Semantic Arguments:] Good if there are lots of sentence letters.\\
        $(A \vee B) \supset (C \wedge D),\ \neg C \wedge \neg E\ \therefore \neg A$.
  \item[\bf Task 4:] Provide a semantic argument.
  \item[\it Inference Rules:] Suppose we were to schematize inferences.
    \begin{itemize}
      \item $\varphi \wedge \psi\ \vdash \varphi$.
      \item $\neg \varphi\ \vdash \neg(\varphi \wedge \psi)$.
      \item $\varphi \supset \psi,\ \neg \psi\ \vdash \neg \varphi$.
      \item $\neg(\varphi \vee \psi)\ \vdash \neg \varphi$.
    \end{itemize}
  \item[\it Observe:] Rules are valid.
  \item[\bf Task 5:] Use rules to derive above.
  \item[\it Proof Theory:] How many rules are there, and how should we describe the space of all of them?
\end{enumerate}









% \subsection*{\it \textbf{Roses}}
%
% \begin{earg}
%   \item[(1)] Sugar is sweet.
%   \item[] Roses are only red if sugar is sweet.
% \end{earg}
%
%
%
%
% \subsection*{\it \textbf{Vacation}}
%
% \begin{earg}
%   \item[(1)] Casey is not on vacation.
%   \item[] If Casey is on vacation, then he is in Paris.
% \end{earg}
%
%
%
%
%
% \subsection*{\it \textbf{Crimson}}
%
% \begin{earg}
%   \item[(1)] Mary only likes the ball if it is crimson.
%   \item[(2)] Mary likes the ball.
%   \item[] If the ball is blue, then Mary likes it.
% \end{earg}
%
%
%
%
%
%
% \section*{Biconditional}
%
% \subsection*{\it \textbf{Rectangle}}
%
% \begin{earg}
%   \item[(1)] The room is a square.
%   \item[(2)] The room is a rectangle.
%   \item[] The room is a square if and only if it is a rectangle.
% \end{earg}
%
%
%
%
%
% \subsection*{\it \textbf{Roses}}
%
% \begin{earg}
%   \item[(1)] Kin isn't a professor.
%   \item[(2)] Sue isn't a chef.
%   \item[] Kin is a professor just in case Sue is a chef.
% \end{earg}



% \section*{Examples}
%
% \noindent
% Which sets of sentences are consistent? (e.g., is $\set{(1),(2)}$ consistent?)
%
% \subsection*{\it \textbf{Taller}}
%
% \begin{enumerate}
%   \item[(1)] Liza is taller than Sue.
%   \item[(2)] Sue is taller than Paul.
%   \item[(3)] Paul is taller than Liza.
% \end{enumerate}
%
%
%
%
% \subsection*{\it \textbf{Lost}}
%
% \begin{enumerate}
%   \item[(4)] Kim is either in Somerville or Cambridge.
%   \item[(5)] If Kim is in Somerville, then she is not far from home.
%   \item[(6)] If Kim is not far from home, then she is in Cambridge.
%   \item[(7)] Kim is not in Cambridge.
% \end{enumerate}



\end{document}


