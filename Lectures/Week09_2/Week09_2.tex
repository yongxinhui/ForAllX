\documentclass[a4paper, 11pt]{article} % Font size (can be 10pt, 11pt or 12pt) and paper size (remove a4paper for US letter paper)
\usepackage[protrusion=true,expansion=true]{microtype} % Better typography
\usepackage{graphicx} % Required for including pictures
\usepackage{wrapfig} % Allows in-line images
\usepackage{enumitem} %%Enables control over enumerate and itemize environments
\usepackage{setspace}
\usepackage{amssymb, amsmath, mathrsfs, mathabx} %%Math packages
\usepackage{../lecture} %calls local modified style file
\usepackage{stmaryrd}
\usepackage{mathtools}
\usepackage{multicol} 
\usepackage{mathpazo} % Use the Palatino font
\usepackage[T1]{fontenc} % Required for accented characters
\usepackage{array}
\usepackage{bibentry}
\usepackage{prooftrees} 
\usepackage[round]{natbib} %%Or change 'round' to 'square' for square backers
\setcitestyle{aysep=}

\makeatletter
\renewcommand{\maketitle}{
\begin{flushright}
{\LARGE\@title}

\vspace{10pt}

{\@author}
\\ \@date
\end{flushright}

\vspace{-20pt}

}
\makeatother

%----------------------------------------------------------------------------------------
%	TITLE
%----------------------------------------------------------------------------------------

\title{\textbf{Uniqueness and Quantity}} % Subtitle

\author{\textsc{Logic I}\\ \em Benjamin Brast-McKie} % Institution

\date{\today} % Date

%----------------------------------------------------------------------------------------

\begin{document}

\maketitle % Print the title section

\thispagestyle{empty}

%----------------------------------------------------------------------------------------


\section*{Opacity}

\begin{enumerate}
  \item[\it Believes:] Regiment the following argument:
    \item Lois Lane believes that Superman can fly.
    \item \underline{Superman is Clark Kent.\quad\quad}
    \item Lois Lane believes that Clark Kent can fly.
  \item[\bf Question:] Are these arguments intuitively valid?
  \item[\it Opacity:] Whereas \textit{Rising} admits substitution, \textit{Believes} does not.
  \item[\it Transparency:] We may say that `is rising' is transparent and that `believes' is opaque.
  \item[\it Mathematics:] Mathematics does not include any opaque contexts.
\end{enumerate}






\section*{Substitution}

\begin{itemize}
  \item[\it Leibniz's Law:] $\alpha = \beta, \metaA \vDash \metaA[\beta/\alpha]$.
  \item[\it Free For:] $\beta$ is \textsc{free for} $\alpha$ in $\metaA$ just in case there is no free occurrence of $\alpha$ in $\metaA$ in the scope of a quantifier that binds $\beta$. 
    \item How would we define this recursively?
  \item[\it Constants:] If $\beta$ is a constant, then $\beta$ is free for any $\alpha$ and $\metaA$. 
  \item[\it Substitution:] If $\beta$ is free for $\alpha$ in $\metaA$, then the \textsc{substitution} $\metaA\unisub{\beta}{\alpha}$ is the result of replacing all free occurrences of $\alpha$ in $\metaA$ with $\beta$. 
  \item[\it Examples:] Consider the following cases:
    \item $z$ is free for $x$ in $\qt{\forall}{y}(Fxy \eif Fyx)$ 
    \item $y$ is not free for $x$ in $\qt{\forall}{y}(Fxy \eif Fyx)$
\end{itemize}





\section*{Uniqueness}

\begin{enumerate}
  \item[\it Uniqueness:] Ingmar trusts Albert, but no one else.
  \item[\it Only:] Regiment the following argument:
    \item Lois Lane only loves Clark Kent.
    \item \underline{Only Clark Kent is Superman.}
    \item Lois Lane loves Superman.
\end{enumerate}





\section*{Definite Descriptions}

\begin{itemize}
  % \item[\it Russell:] Compare the following sentences:
  \item[\bf Task:] Regiment the following sentences.
    \item The king of France is bald.
    \item The king of France is not bald.
    \item The king of France is bald or not.
    \item Everything is bald or not, so the king of France is bald or not.
    \item If the king of France is bald or not, then there is a king of France.
  \item[\bf Question:] What is the difference between these two cases?
  \item[\it Existence:] If the king of France is bald, then the king of France exists.
  \item[\it Definite Article:] `The king of France' can't be a name.
  \item[\it Regimentation:] Russell offered the following analysis:
    \item $\exists x(Kxf \eand \forall y(Kyf \eif x=y) \eand Bx)$.
    \item $\exists x(\forall y(Kyf \eiff x=y) \eand Bx)$.
  \item[\it Negation:] Negation applies to the predicate, not the sentence.
  \item[\bf Task:] Compare the following case:
    \item Everything can or cannot fly.
    \item Pegasus can or cannot fly.
    \item If Pegasus can or cannot fly, then Pegasus exists.
  \item[\bf Task:] Regiment the following:
  \item Superman is keeping something from his lover.
  % \item The Queen of spades is a black card.
  \item The man with the axe is not Jack.
  \item The Ace of diamonds is not the man with the axe.
  \item One-eyed jacks and the man with the axe are wild.
  \item No spy knows the combination to the safe.
  \item The one Ingmar trusts is lying.
  % \item Only Ingmar knows the combination to the safe.
  \item The person who knows the combination to the safe is not a spy.
\end{itemize}




\section*{At Least:}

\begin{enumerate}
  \item[\bf Task:] Regiment the following claims.
  % \item There are no wild cards.
  \item There is at least one wild card.
  \item There are at least two clubs.
  \item There are at least three hearts on the table.
  \item[\bf Question:] How can we define these quantifiers in general?
\end{enumerate}





\section*{Inequality Quantifiers Defined}

\begin{enumerate}
  \item[\it Definition:] We may define the following abbreviations recursively:
    \begin{itemize}
      \item[\it Base:] $\qt{\exists_{\geq 1}}{\alpha}\metaA \coloneq \qt{\exists}{\alpha}\metaA(\alpha)$.
      \item[\it Recursive:] $\qt{\exists_{\geq n+1}}{\alpha}\metaA \coloneq \qt{\exists}{\alpha}(\metaA(\alpha) \eand \qt{\exists_{\geq n}}{\beta}(\alpha \neq \beta \eand \metaA\unisub{\beta}{\alpha}))$ $\beta$ is free for $\alpha$ in $\metaA$. 
    \end{itemize}
  \item[\it Infinite:] $\Gamma_{\infty} \coloneq \set{\qt{\exists_{\geq n}}{x}(x=x): n\in\mathbb{N}}$.
  \item[\bf Question:] What is the smallest model to satisfy $\Gamma_\infty$?
  \item[\it At Most:] Regiment the following claims.
  \item There is at most one wild card.
  \item There are at most two one-eyed jacks.
  \item There are at most three black jacks.
  \item[\it At Most:] $\qt{\exists_{\leq n}}{\alpha}\metaA \coloneq \neg\qt{\exists_{\geq n+1}}{\alpha}\metaA$.
  \item[\it Between:] $\qt{\exists_{[n,m]}}{\alpha}\metaA \coloneq \qt{\exists_{\geq n}}{\alpha}\metaA \eand \qt{\exists_{\leq m}}{\alpha}\metaA$~ where $n\leq m$.
\end{enumerate}
 
 





\section*{Cardinality Quantifiers}
 
\begin{enumerate}
  \item[\bf Task:] Regiment the following.
  \item There is one wild card.
  \item There are two winning hands.
  \item There are three hearts on the table.
  \item[\bf Question:] How can we define the cardinality quantifiers in general?
  \item[\it Base:] $\qt{\exists_0}{\alpha}\metaA \coloneq \qt{\forall}{\alpha}\neg\metaA(\alpha)$.
  \item[\it Recursive:] $\qt{\exists_{n+1}}{\alpha}\metaA \coloneq \qt{\exists}{\alpha}(\metaA(\alpha) \eand \qt{\exists_n}{\beta}(\alpha \neq \beta \eand \metaA\unisub{\beta}{\alpha}))$ where $\beta$ is free for $\alpha$ in $\metaA$.
  \item[\bf Question:] How do the cardinality quantifiers relate to the inequality quantifiers?
  \item[\it Exact:] $\qt{\exists_{n}}{\alpha}\metaA \Dashv\vDash \qt{\exists_{[n,n]}}{\alpha}\metaA.$
    \item Proving this is too hard for a test or problem set, but maybe a fun challenge to think about.
\end{enumerate}





 % Buffy and Willow were born unto the same generation.
 % There is no more than one slayer born in each generation.
 % A slayer other than Buffy is one of the forces of darkness.
 % Willow will stand against any force of darkness other than a werewolf.
 % Faith will kick everyone except herself.
 % Buffy will kick anyone who stands against a slayer, unless they are also kicking vampires or demons.
 % In every generation a slayer is born.
 % In every generation a slayer is born. She will stand against the vampires, demons, and forces of darkness.
 % In every generation a slayer is born. She alone will stand against the vampires, demons, and forces of darkness.

% \item There are at least three horses in the world.
% \item There are at least three animals in the world.
% \item There is more than one horse in Farmer Brown's field.
% \item There are three horses in Farmer Brown's field.
% \item There is a single winged creature in Farmer Brown's field; any other creatures in the field must be wingless.
% \item The Pegasus is a winged horse.
% \item The animal in Farmer Brown's field is not a horse.
% \item The horse in Farmer Brown's field does not have wings.




\section*{Examples}

\begin{enumerate}
\item Show that $\set{{\neg}Raa, \qt{\forall}{x} (x{=}a \eor Rxa)}$ is satisfiable. 
  \item Show that $\set{{\neg}Raa, \qt{\forall}{x} (x{=}a \eor Rxa), \qt{\forall}{x}\qt{\exists}{y}Rxy}$ is satisfiable. 
% \item Show that $\set{\qt{\forall}{x}\qt{\forall}{y}\qt{\forall}{z}(x{=}y \eor y{=}z \eor x{=}z),\qt{\exists}{x}\qt{\exists}{y}\ x{\neq} y}$ is satisfiable.
\item Show that $\qt{\forall}{x}\qt{\forall}{y}\ x{=}y \vdash \neg \qt{\exists}{x}\ x \neq a$.
% \item Show that $\qt{\exists}{x} (x {=} h \eand x {=} i)$ is contingent.
% \item Show that \{$\qt{\exists}{x}\qt{\exists}{y}(Zx \eand Zy \eand x{=}y)$, $\neg Zd$, $d{=}s$\} is satisfiable.
% \item Show that $\qt{\forall}{x}(Dx \eif \qt{\exists}{y} Tyx)\nvdash\qt{\exists}{y} \qt{\exists}{z}\ y{\neq} z$.
\end{enumerate}









% \section*{Assignment Lemmas}
%
% \begin{enumerate}
%   \item[\it Lemma 1:] If $\hat{a}(\alpha)=\hat{c}(\alpha)$ for all free variables $\alpha$ in a wff $\metaA$, then $\VV{\I}{\hat{a}}(\metaA)=\VV{\I}{\hat{c}}(\metaA)$.
%     \begin{itemize}
%       \item[\it Base:] Assume $\comp(\metaA)=0$, so $\metaA=(\alpha=\beta)$ or $\metaA=\F^n\alpha_1,\ldots,\alpha_n$.
%       \item[($\alpha=\beta$):] \mbox{So $\VV{\I}{\hat{a}}(\metaA)=\VV{\I}{\hat{a}}(\alpha=\beta)=1$ \textit{iff} $\VV{\I}{\hat{a}}(\alpha)=\VV{\I}{\hat{a}}(\beta)$ \textit{iff} $\VV{\I}{\hat{c}}(\alpha)=\VV{\I}{\hat{c}}(\beta)\ldots$}
%       \item[($\F^n\alpha_1,\ldots,\alpha_n$):] \mbox{So $\VV{\I}{\hat{a}}(\metaA)=\VV{\I}{\hat{a}}(\F^n\alpha_1,\ldots,\alpha_n)=1$ \textit{iff} $\tuple{\VV{\I}{\hat{a}}(\alpha_1),\ldots,\VV{\I}{\hat{a}}(\alpha_n)}\in\I(F^n)\ldots$}
%     \end{itemize}
%   \item[\it Lemma 2:] For any sentence $\metaA$: $\VV{\I}{}(\metaA)= 1$ \textit{iff} $\VV{\I}{\hat{a}}(\metaA)= 1$ for every v.a. $\hat{a}$ over $\D$.
%     % \begin{itemize}
%     %   \item[\it LTR:] Assume $\VV{\I}{}(\metaA)= 1$, so $\VV{\I}{\hat{a}}(\metaA)= 1$ for some v.a. $\hat{c}$ over $\D$ .
%     %   \item Let $\hat{a}$ be any v.a. over $\D$.
%     %   \item Since $\metaA$ has no free variables, $\VV{\I}{\hat{a}}(\metaA)=\VV{\I}{\hat{c}}(\metaA)$ by \textit{Lemma 1}.
%     %   \item So $\VV{\I}{\hat{a}}(\metaA)=1$ for all v.a. $\hat{c}$ over $\D$.
%     %   \item[\it RTL:] Assume $\VV{\I}{\hat{a}}(\metaA)=1$ for all v.a. $\hat{a}$ over $\D$.
%     %   \item Since $\D$ is nonempty, there is some v.a. $\hat{a}$, and so $\VV{\I}{}(\metaA)= 1$. 
%     %   % \item So $\VV{\I}{}(\metaA)=1$.
%     % \end{itemize}
%   \item[\it Lemma 3:] For any sentence $\metaA$: $\VV{\I}{}(\metaA)\neq1$ \textit{iff} $\VV{\I}{\hat{a}}(\metaA)\neq 1$ for some v.a. $\hat{a}$ over $\D$.
% \end{enumerate}
%





\end{document}

