\documentclass[a4paper, 11pt]{article} % Font size (can be 10pt, 11pt or 12pt) and paper size (remove a4paper for US letter paper)
\usepackage[protrusion=true,expansion=true]{microtype} % Better typography
\usepackage{../lecture} %calls local modified style file
\usepackage{graphicx} % Required for including pictures
\usepackage{wrapfig} % Allows in-line images
\usepackage{enumitem} %%Enables control over enumerate and itemize environments
\usepackage{setspace}
\usepackage{amssymb, amsmath, mathrsfs} %%Math packages
\usepackage{stmaryrd}
\usepackage{mathtools}
\usepackage{multicol} 
\usepackage{mathpazo} % Use the Palatino font
\usepackage[T1]{fontenc} % Required for accented characters
\usepackage{array}
\usepackage{bibentry}
\usepackage{prooftrees} 
\usepackage[round]{natbib} %%Or change 'round' to 'square' for square backers
\setcitestyle{aysep={}}

\makeatletter
\renewcommand{\maketitle}{
\begin{flushright}
{\LARGE\@title}

\vspace{10pt}

{\@author}
\\ \@date
\end{flushright}

\vspace{-20pt}

}
\makeatother

%----------------------------------------------------------------------------------------
%	TITLE
%----------------------------------------------------------------------------------------

\title{\textbf{Mathematical Induction}} % Subtitle

\author{\textsc{Logic I}\\ \em Benjamin Brast-McKie} % Institution

\date{\today} % Date

%----------------------------------------------------------------------------------------

\begin{document}

\maketitle % Print the title section

\thispagestyle{empty}

%----------------------------------------------------------------------------------------

\section*{Examples}

\begin{enumerate}
  \item Show that $\enot R \eif \enot Q,\ P \eand Q \vDash P \eand R$.
  \item Show that $A \eor B,\ B \eif C,\ A \eiff C \vDash C$.
  \item Show that $P,\ P \eif Q,\ \enot Q \vDash A$.
  \item Show that $(P \eif Q) \eiff (\enot Q \eif \enot P)$ is a tautology.
  \item Show that $A \eiff \enot A$ is a contradiction.
  \item Show that $\{P,\ P \eif Q,\ Q \eif \enot P\}$ is unsatisfiable. 
  \item Show that $\set{P \eif Q,\ \enot P \eor \enot Q,\ Q \eif P}$ is satisfiable.
  \item[\bf Observe:] There seem to be patterns.
  \item[\bf Question:] How could we systematize these proofs?
\end{enumerate}


\section*{The Material Conditional}

\subsubsection*{\it \textbf{Roses}}

\begin{earg}
  \uitem{Sugar is sweet.}
  \eitem{The roses are only red if sugar is sweet.}
\end{earg}


\begin{itemize}[leftmargin=1in,labelsep=.15in] %,label=(\arabic*)]%,label=\roman*]
  % \item[\bf Task:] Regiment this argument and construct its truth table.
  \item[\bf Observe:] First paradox of the material conditional.
\end{itemize}




\subsubsection*{\it \textbf{Vacation}}

\begin{earg}
  \uitem{Casey is not on vacation.}
  \eitem{If Casey is on vacation, then he is in Paris.}
\end{earg}


\begin{itemize}[leftmargin=1in,labelsep=.15in] %,label=(\arabic*)]%,label=\roman*]
  % \item[\bf Task:] Regiment this argument and construct its truth table.
  \item[\bf Observe:] Second paradox of the material conditional.
\end{itemize}







\subsubsection*{\it \textbf{Crimson}}

\begin{earg}
  \eitem{Mary doesn't like the ball unless it is crimson.}
  \uitem{Mary likes the ball.}
  \eitem{If the ball is blue, then Mary likes it.}
\end{earg}






\section*{The Biconditional}

\subsubsection*{\it \textbf{Rectangle}}

\begin{earg}
  \eitem{The room is a square.}
  \uitem{The room is a rectangle.}
  \eitem{The room is a square if and only if it is a rectangle.}
\end{earg}





\subsubsection*{\it \textbf{Work}}

\begin{earg}
  \eitem{Kin isn't a professor.}
  \uitem{Sue isn't a chef.}
  \eitem{Kin is a professor just in case Sue is a chef.}
\end{earg}




% \begin{itemize}[leftmargin=1.5in,labelsep=.15in] %,label=(\arabic*)]%,label=\roman*]
  % \item[\it Complexity:] $\comp{\metaA}$ is the function from the wfss of $\PL$ to the natural number $\N$ which satisfies the following for any wfss $\metaA$ and $\metaB$ of $\PL$ and $\star \in \set{\eand, \eor, \eif, \eiff}$: 
  %   \item $\comp{\metaA} = 0$ if $\metaA$ is a sentence letter; 
  %   \item $\comp{\enot\metaA} = \comp{\metaA} + 1$; and
  %   \item $\comp{\metaA \star \metaB} = \comp{\metaA} + \comp{\metaB} + 1$.
  % \item[\bf Task 4:] Provide a semantic argument.
  % \item[\it Inference Rules:] Suppose we were to schematize inferences.
  %   \begin{itemize}
  %     \item $\metaA \wedge \metaB\ \vdash \metaA$.
  %     \item $\neg \metaA\ \vdash \neg(\metaA \wedge \metaB)$.
  %     \item $\metaA \eif \metaB,\ \neg \metaB\ \vdash \neg \metaA$.
  %     \item $\neg(\metaA \vee \metaB)\ \vdash \neg \metaA$.
  %   \end{itemize}
  % \item[\it Observe:] Rules are valid.
  % \item[\bf Task 5:] Use rules to derive above.
  % \item[\it Proof Theory:] How many rules are there, and how should we describe the space of all of them?
% \end{itemize}


\section*{Review from Last Time}

\begin{enumerate}
  \item Show that $A \vee B, B \supset C, A \equiv C \vDash C$.
  \item Show that $\{P, P \supset Q, Q \supset \neg P\}$ is unsatisfiable. 
  \item Show that $\set{P \supset Q, \neg P \vee \neg Q, Q \supset P}$ is satisfiable.
  \item Evaluate $P, P \supset Q, \neg Q \vDash A$.
  \item Evaluate $(A \wedge B) \supset C, C \equiv (D \wedge E), \neg D \wedge B \vdash \neg A$.
\end{enumerate}


\section*{Soundness and Completeness}

\begin{enumerate}
  \item[\it Soundness:] If $\Gamma \vdash \varphi$, then $\Gamma \vDash \varphi$.
  \item[\it Completeness:] If $\Gamma \vDash \varphi$, then $\Gamma \vdash \varphi$.
  \item[\it Induction:] These proofs will require mathematical induction. 
  \item[\it Definitions:] We will also need a few more recursive definitions.
\end{enumerate}



\section*{Recursive Definitions}

\begin{itemize}
  \item[\it Complexity:] We define $\comp(\varphi)$ to be the number of connectives in $\varphi$.
      \item If $\varphi$ is a sentence letter of SL, then $\comp(\varphi)=0$.
      \item For any SL sentences $\varphi$ and $\psi$: 
    \begin{itemize}
      \item[$(\neg)$] $\comp(\neg \varphi)=\comp(\varphi)+1$;
      \item[$(\wedge)$] $\comp(\varphi \wedge \psi)=\comp(\varphi)+\comp(\psi)+1$;
      \item[] $\vdots$
    \end{itemize}
  \item[\bf Note:] Could avoid redundancy by taking $\star$ to be any binary connective. 
  \item[\it Constituents:] We define $[\varphi]$ to be the set of sentence letters that occur in $\varphi$. % by taking $[\cdot]:\text{SL}\to\mathcal{P}(\text{SL})$ to be the smallest function to satisfy:
      \item If $\comp(\varphi)=0$, then $[\varphi]=\set{\varphi}$.
      \item For any SL sentences $\varphi$ and $\psi$, and binary connective $\star\in\set{\wedge,\vee,\supset,\equiv}$: 
    \begin{itemize}
      \item[$(\neg)$] $[\neg \varphi]=[\varphi]$;
      \item[$(\hspace{1pt}\star\hspace{1pt})$] $[\varphi \star \psi]=[\varphi] \cup [\psi]$;
    \end{itemize}
  \item[\it Simplicity:] We define $\simp(\varphi)$ to hold just in case the SL sentence $\varphi$ has at most one occurrence of each sentence letter in SL.
      \item If $\comp(\varphi)=0$, then $\simp(\varphi)$.
      \item For any SL sentences $\varphi$ and $\psi$, and binary connective $\star\in\set{\wedge,\vee,\supset,\equiv}$: 
    \begin{itemize}
      \item[$(\neg)$] $\simp(\neg \varphi)$ if $\simp(\varphi)$;
      \item[$(\hspace{1pt}\star\hspace{1pt})$] $\simp(\varphi \star \psi)$ if $\simp(\varphi)$, $\simp(\psi)$, and $[\varphi]\cap[\psi]=\varnothing$;
    \end{itemize}
  \item[\it Substitution:] We define $\varphi_{[\chi/\alpha]}$ to be the result of replacing every occurrence of the sentence letter $\alpha$ in $\varphi$ with $\chi$.
      \item If $\comp(\varphi)=0$, then $\varphi_{[\chi/\alpha]}=
        \begin{cases}
          \chi \quad\text{if } \varphi=\alpha,\\
          \varphi \quad\text{otherwise.}
        \end{cases}$
      \item For any SL sentences $\varphi$ and $\psi$, and binary connective $\star\in\set{\wedge,\vee,\supset,\equiv}$: 
    \begin{itemize}
      \item[$(\neg)$] $(\neg\varphi)_{[\chi/\alpha]}=\neg(\varphi_{[\chi/\alpha]})$;
      \item[$(\hspace{1pt}\star\hspace{1pt})$] $(\varphi\star\psi)_{[\chi/\alpha]}=\varphi_{[\chi/\alpha]}\star\psi_{[\chi/\alpha]}$;
    \end{itemize}
\end{itemize}





\section*{Strong Induction}

\begin{enumerate}
  \item[\it Step 1:] Identify the set of elements and the property in question.
  \item[\it Step 2:] Partition the set into a sequence of stages to run induction on.
  \item[\it Step 3:] Establish that every element in the base stage has the property.
  \item[\it Step 4:] Assume every element in stage $n$ and below have the property. 
  \item[\it Step 5:] Show that every element in stage $n+1$ have the property. 
\end{enumerate}





\section*{Examples}

\begin{enumerate}
  \item[\bf Task 1:] Show that every SL sentence has an even number of parentheses. 
  \item[\bf Task 2:] Show that for every SL sentence $\varphi$, if $\simp(\varphi)$, then there are SL interpretations $\I$ and $\J$ where $\V{\I}(\varphi)=1$ and $\V{\J}(\varphi)=0$. 
  \item[\bf Task 3:] For any SL sentences $\varphi,\psi,\chi$ and SL sentence letter $\alpha$, if $\vDash \varphi \equiv \psi$, then $\vDash \chi_{[\varphi/\alpha]}\equiv\chi_{[\psi/\alpha]}$.
  \item[\bf Task 4:] Let $\I^+(\alpha)=1$ for every sentence letter $\alpha$ in SL. Show that $\V{\I^+}(\varphi)=1$ for every SL sentence $\varphi$ that does not include negation. 
\end{enumerate}





% \section*{Weak Induction}

% \begin{enumerate}
%   \item[\it Even:] If $\varphi$ is a sentence letter of SL, then $\even(\varphi)$.\\
%     For any SL sentences $\varphi$ and $\psi$: 
%     \begin{itemize}
%       \item[$(\neg)$] $\even(\neg \varphi)$ if $\even(\varphi)$;
%       \item[$(\wedge)$] $\even(\varphi \wedge \psi)$ if $\even(\varphi)$ and $\even(\psi)$;
%       \item[] $\vdots$
%     \end{itemize}
% \end{enumerate}







\end{document}

