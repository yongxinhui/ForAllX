\documentclass[a4paper, 11pt]{article} % Font size (can be 10pt, 11pt or 12pt) and paper size (remove a4paper for US letter paper)
\usepackage[protrusion=true,expansion=true]{microtype} % Better typography
\usepackage{graphicx} % Required for including pictures
\usepackage{wrapfig} % Allows in-line images
\usepackage{enumitem} %%Enables control over enumerate and itemize environments
\usepackage{setspace}
\usepackage{amssymb, amsmath, mathrsfs} %%Math packages
\usepackage{stmaryrd}
\usepackage{mathtools}
\usepackage{mathpazo} % Use the Palatino font
\usepackage[T1]{fontenc} % Required for accented characters
\usepackage{array}
\usepackage{bibentry}
\usepackage[round]{natbib} %%Or change 'round' to 'square' for square backers
\setcitestyle{aysep={}}

\newcounter{eargletter} % Counter for the environment (resets per section)
\newcounter{eargnum}[eargletter]
\newenvironment{earg}[1][] % Optional argument to customize numbering
{%
  \refstepcounter{eargletter}% Step the environment counter
  \begin{enumerate}[label=\Alph{eargletter}\arabic{eargnum}., #1]%
  \usecounter{eargnum}
  \setlength{\itemsep}{-.4em}% Adjust item spacing
  % \newcommand{\eitem}{%
  %     \refstepcounter{eargnum}% Step the item counter
  %     \item[\Alph{eargletter}\arabic{eargnum}.]% Label format
  %     \label{earg:\theeargletter\theeargnum}}% Create a label for referencing
}
{%
  \end{enumerate}%
}

\newcommand{\uitem}[1]{% use to insert line before conclusion
  \item \underline{#1}\vspace{2pt}
}

\newcommand{\corner}[1]{\ulcorner#1\urcorner} %%Corner quotes
\newcommand{\tuple}[1]{\langle#1\rangle} %%Angle brackets
\newcommand{\set}[1]{\lbrace#1\rbrace} %%Set brackets
\newcommand{\interpret}[1]{\llbracket#1\rrbracket} %%Double brackets
%\DeclarePairedDelimiter\ceil{\lceil}{\rceil}    
\def\therefore{\ensuremath{\ldotp\dot{}\,\ldotp}}
\newcommand{\I}{\mathcal{I}}
\newcommand{\V}[1]{\mathcal{V}_{#1}} %%
\newcommand{\PL}{\mathcal{L}^{\textsc{pl}}}
\newcommand{\FOL}{\mathcal{L}^{\textsc{fol}}}

\makeatletter
\renewcommand\@biblabel[1]{\textbf{#1.}} % Change the square brackets for each bibliography item from '[1]' to '1.'
\renewcommand{\@listI}{\itemsep=0pt} % Reduce the space between items in the itemize and enumerate environments and the bibliography

\renewcommand{\maketitle}{ % Customize the title - do not edit title and author name here, see the TITLE block below
\begin{flushright} % Right align
{\LARGE\@title} % Increase the font size of the title

\vspace{10pt} % Some vertical space between the title and author name

{\@author} % Author name
\\\@date % Date

\vspace{-10pt} % Some vertical space between the author block and abstract
\end{flushright}
}

%----------------------------------------------------------------------------------------
%	TITLE
%----------------------------------------------------------------------------------------

\title{\textbf{What is Logic?}} % Subtitle

\author{\textsc{Logic I}\\ \em Benjamin Brast-McKie} % Institution

\date{\today} % Date

%----------------------------------------------------------------------------------------

\begin{document}

\maketitle % Print the title section

\thispagestyle{empty}

%----------------------------------------------------------------------------------------

\section*{Motivations}

\begin{itemize}[leftmargin=1in,labelsep=.15in]
  \item[\it Reasoning:] Logic is the study of formal reasoning.
    \item By `formal' we don't mean that it uses mathematical symbols.
    \item Rather, what follows from what \textit{in virtue of logical form}.
    \item Abstracting from specific subject-matters, logic describes general patterns of reasoning that apply across the disciplines. %, but is not exhaustive, describing all reasoning in every discipline.
  \item[\it Normativity:] Logic is not a \textit{descriptive} science studying how human beings in fact reason across the various disciplines.
    \item Logic is a \textit{normative} science, describing an especially strong form of reasoning that may serve as an ideal.
  \item[\it Artifical:] We will primarily work in artificial languages where we will stipulate how to reason in these languages.
    \item Regimenting English will expose and remove ambiguities.
    \item We will provide proof systems for our artificial languages by which to compute what follows from what in a manner that vastly extends our natural cognitive capacities.
\end{itemize}



\section*{Interpretations}

\begin{itemize}[leftmargin=1in,labelsep=.15in]
  \item[\it Proposition:] We will begin with propositional logic where a \textsc{proposition} is a way for things to be which either obtains or does not.
  \item[\it Declarative Sentence:] Given an interpretation of the language, an English sentence is \textsc{declarative} just in case it expresses a proposition.
    % \item A declarative sentence is true in an interpretation if the proposition it expresses obtains, and false otherwise.
    \item Interrogative, imperative, and exclamatory sentences are not declarative sentences and typically do not have truth-values. 
    \item We will restrict to declarative sentences throughout.
  \item[\it Truth-Values:] A declarative sentence is \textsc{true} in an interpretation if, given that interpretation, it expresses a proposition that obtains and \textsc{false} in that interpretation otherwise.
  \item[\it Interpretations:] We will only be concerned with the truth-values of sentences in this course, and so it is enough to take an \textsc{interpretation} to be an assignment of truth-values to sentences.
    \item This amounts to taking there to be just two propositions. % called `True' and `False', or later, `1' and `0'.
  % \item[\it Informal Interpretation:] An interpretation of a declarative English sentence assigns either the \textsc{truth-value} `True' or `False'.
  % \item[\it Deductive Argument:] A \textsc{deductive argument} in English is a nonempty sequence of declarative sentences where a single sentence is designated as the \textsc{conclusion} (typically the last line) and all of the other sentences (if any) are the \textsc{premises}.
\end{itemize}




\section*{Examples}

\begin{itemize}[leftmargin=1in,labelsep=.15in]
  \item[\it Deductive Argument:] A \textsc{deductive argument} in English is a nonempty sequence of declarative sentences where a single sentence is designated as the \textsc{conclusion} (typically the last line) and all of the other sentences (if any) are the \textsc{premises}.
\end{itemize}

\subsection*{\it \textbf{Snow}: This argument may be compelling, but it is not certain.}

\begin{earg}
  \uitem{It's snowing.}
  \item John drove to work.
\end{earg}

% \noindent
% \textit{This argument may be compelling, but it is not certain.}

\subsection*{\it \textbf{Red}: This argument provides certainty, but not on all interpretations.}

\begin{earg}
  \uitem{The ball is crimson.}
  \item The ball is red.
\end{earg}

% \noindent
% \textit{This argument provides certainty, but not on all interpretations.}

\subsection*{\it \textbf{Museum}: This argument's certainty is independent of the interpretation.}

\begin{earg}
  \item Kate is either at home or at the Museum.
  \uitem{Kate is not at home.}
  \item Kate is at the Museum.
\end{earg}

% \noindent
% \textit{This argument's certainty is independent of the interpretation.}




\section*{Informal Validity}

\begin{itemize}[leftmargin=1in,labelsep=.15in]
  \item[\bf Question 1:] What goes wrong if we assume the premises but deny the conclusion in \textit{Snow}, \textit{Red}, and \textit{Museum}?
  \item[\it Snow:] Improbable but possible.
  \item[\it Red:] Impossible on the intended interpretation.
  \item[\it Museum:] Impossible on all interpretations so long as we hold the meanings of logical terms `not' and `or' fixed. 
  % \item[\it Variance:] Allow the interpretation of non-logical terms to vary, holding the meaning of the logical terms constant.
  \item[\bf Task 1:] Clarify what it is to hold the logical terms fixed.
  \item[\it Informal Interpretation:] An \textsc{informal interpretation} assigns every declarative sentence of English to exactly one \textsc{truth-value} without offending the following informal semantic clauses:
    \item A \textit{negation} is true just in case the negand is false.
    \item A \textit{disjunction} is true just in case either disjunct is true.
    % \item Restrict to interpretations which respect the semantics.
  \item[\it Informal Validity:] An argument in English is \textsc{informally valid} just in case its conclusion is true in every informal interpretation in which all of its premises are true.
  % \item[\it Complex Sentences:] Observe that complex sentences in \textit{Museum} are composed from simpler sentences \textit{via} `not' and `or'.
  % \item[\it Atomic Sentences:] A declarative sentence is \textsc{atomic} just in case it is not composed of simpler declarative sentences.
  % \item[\bf Task 2:] Identify atomic sentences in \textit{Museum}.
\end{itemize}





\section*{Formal Languages}

\begin{itemize}[leftmargin=1in,labelsep=.15in]
  \item[\bf Problem 1:] There is no set of all declarative sentences of English, and so no clear notion of an informal interpretation of English.
  \item[\it Suggestion:] Could choose some large set of atomic English sentences, but this would be arbitrary and hard to specify precisely.
  \item[\bf Solution 1:] We will \textit{regiment} English arguments in artificial languages that are both general and easy to specify precisely.
  \item[\it Propositional Language:] The \textsc{sentences} of $\PL$ are composed of \textsc{sentence letters} $A, B, C, \ldots$ and sentential operators $\neg$ and $\vee$.
  \item[\bf Task 2:] Regiment \textit{Museum} in $\PL$: $H\vee M, \neg H \vDash M$.
    \item $H$ = `Kat is at home'. 
    \item $M$ = `Kat is at the Museum'. 
  \item[\bf Task 3:] Provide a way to interpret the sentences of $\PL$.
  % \item[\it Atomic:] If $\varphi$ is atomic, then the truth-value of $\varphi$ is given by the interpretation. 
  \item[\it Schematic Variables:] Let $\varphi, \psi, \ldots$ be variables with sentences of $\PL$ as values, and let $\Gamma, \Sigma, \ldots$ be variables for sets of sentences of $\PL$. 
  \item[\it Interpretation:] An \textsc{interpretation} $\V{}$ of $\PL$ assigns exactly one truth-value (1 or 0) to all sentences of $\PL$ where for any $\varphi$ and $\psi$: 
  % \item[\it Atomic:] $\V{\I}(\varphi)=\I(\varphi)$ if $\varphi$ is a sentence letter.
  \item $\V{}(\neg\varphi)=1$ just in case $\V{}(\varphi)=0$.
  \item $\V{}(\varphi\vee\psi)=1$ just in case $\V{}(\varphi)=1$ or $\V{}(\psi)=1$ (or both).
  \item[\it Logical Consequence:] $\Gamma \vDash \varphi$ just in case $\V{}(\varphi)=1$ for any interpretation $\V{}$ of $\PL$ where $\V{}(\gamma)=1$ for all $\gamma \in \Gamma$.
  \item[\it Logical Validity:] An argument is \textsc{logically valid} just in case its conclusion $\varphi$ is a logical consequence of its set of premises $\Gamma$, i.e. $\Gamma \vDash \varphi$.
  \item[\bf Task 4:] Show that \textit{Museum} is logically valid.
\end{itemize}




\section*{Logic}

\begin{itemize}[leftmargin=1in,labelsep=.15in]
  \item[\it Model Theory:] We have characterized logical reasoning as truth-preservation across a space of interpretations for an artificial language.
  % \item[\bf Task 6:] Can we make \textit{Snow} and \textit{Red} logically valid?
  \item[\it Proof Theory:] Another approach focuses entirely on syntactic rules that specify which inferences in a language are logically valid. % given the meanings of the logical constants.
    \item A system of basic rules for reasoning in an artificial language is referred to as a \textsc{logic} for that language.
    \item By composing basic rules, we will define what counts as a \textsc{proof} in each of the logics that we will study.
  \item[\it Metalogic:] Despite their differences, these two strategies will be shown to coincide for the languages that we will study in this book.
  % \item[\it Neutrality:] These methods accommodate reasoning about anything whatsoever, though not all logical constants are equally well understood.
\end{itemize}



\section*{Logical Form}

\subsection*{\it \textbf{Picasso}}

\begin{earg}
  \item The painting is either a Picasso or a counterfeit and illegally traded.
  \uitem{The painting is not a Picasso.}
  \item The painting is a counterfeit and illegally traded.
\end{earg}

% \noindent
% \textit{This argument is also logically valid.}

\begin{itemize}[leftmargin=1in,labelsep=.15in]
  \item[\bf Task 5:] Regiment \textit{Picasso} in $\PL$: $P\vee (Q\wedge R), \neg P \vDash Q\wedge R$.
    \item $P$ = `The painting is a Picasso'. 
    \item $Q$ = `The painting is a counterfeit'. 
    \item $R$ = `The painting is illegally traded'. 
  \item[\bf Question 2:] How does this argument relate to \textit{Museum}? 
  \item[\it Logical Form:] Both arguments are instances of $\varphi \vee \psi, \neg\varphi \vDash \psi$ which is a logically valid argument schema, i.e., all instances are valid. 
  \item[\bf Question 3:] How many logically valid argument schemata are there, and how could we hope to describe this space?
  \item[\it Suggestion:] The logical consequence relation $\vDash$ for $\PL$ describes the space of logically valid arguments, where the logically valid argument schemata are patterns in this space.
  \item[\bf Problem 2:] $\PL$ cannot regiment all logically valid arguments.
  \item[\it Socrates:] Every man is mortal, Socrates is a man $\vDash$ Socrates is mortal. 
    \item Our intuitive grasp on logical validity is not exhaustively captured by what we can regiment in $\PL$.
  \item[\bf Solution 2:] Rather, logical validity in $\PL$ provides a partial answer, where we may extend the language to provide a broader description of logical validity, e.g., $\FOL$.
    \item We will consider further extensions to $\FOL$ in later chapters. 
\end{itemize}





% \section*{Further Notions}
%
% \begin{itemize}[leftmargin=1.2in,labelsep=.15in] %,label=(\arabic*)]%,label=\roman*]
%   \item[\it Soundness:] An argument is \textit{sound} just in case it is both logically valid and has true premises.
%   \item[\it Truth:] Soundness reaches beyond the scope of any logic course since truth on a single interpretation requires subject-specific knowledge.
%   \item[\it Logical Truth:] Instead we can talk about sentences of formal languages being \textit{logically true} just in case they are true on all interpretations.
%   \item[\it Contradiction:] A sentence is a \textit{contradiction} (or \textit{logically false}) just in case it is false on all interpretations.
%   \item[\it Logical Entailment:] One sentence \textit{logically entails} another just in case every interpretation in which the former is true also makes the latter true.
%   \item[\it Logical Equivalence:] One sentence is \textit{logically equivalent} to another just in case they logically entail each other.
%   \item[\it Consistency:] A set of sentences is \textit{consistent} just in case there is an interpretation which makes every sentence in the set true, and \textit{inconsistent} otherwise.
% \end{itemize}
%
%
%
% \section*{Examples}
%
% \noindent
% Which sets of sentences are consistent? (e.g., is $\set{(1),(2)}$ consistent?)
%
% \subsection*{\it \textbf{Taller}}
%
% \begin{itemize}
%   \item[(1)] Liza is taller than Sue.
%   \item[(2)] Sue is taller than Paul.
%   \item[(3)] Paul is taller than Liza.
% \end{itemize}
%
%
%
%
% \subsection*{\it \textbf{Lost}}
%
% \begin{itemize}
%   \item[(4)] Kim is either in Somerville or Cambridge.
%   \item[(5)] If Kim is in Somerville, then she is not far from home.
%   \item[(6)] If Kim is not far from home, then she is in Cambridge.
%   \item[(7)] Kim is not in Cambridge.
% \end{itemize}
%






% \vfill
%
% \bibliographystyle{Phil_Review} %%bib style found in bst folder, in bibtex folder, in texmf folder.
% \bibliography{Zotero} %%bib database found in bib folder, in bibtex folder


\end{document}
