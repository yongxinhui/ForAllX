%!TEX root = ../../forallx-mit.tex

\chapter*{Preface to the Fall 2024 MIT Edition}
\label{ch.preface2}
\addcontentsline{toc}{chapter}{Preface to the MIT edition}

I inherited a version of Ichikawa and Mangus' version of \forallx\ from Josh Hunt who worked to bring the text closer to \textit{The Logic Book} by Bergmann, Moor, and Nelson for teaching Logic I at MIT.
Since then, I have dropped certain chapters, substantially rewritten the chapters that have been preserved, and included new chapters on soundness and completeness for a Fitch-style natural deduction system for first-order logic.
At this point, few traces of the version I started with remain.
Rather than attempting to record all of the changes that I have made, I will state the main changes that I still hope to make:

  \begin{enumerate}
    \item Include a primer on set theory, relations, and functions in an \textit{Appendix}.
    \item Include a glossary of terms and symbols.
    \item Include cheatsheets for the proof rules for PL and FOL$^=$ in an \textit{Appendix}.
    \item Include an index of lemmas, theorems, and corollaries.
    \item Include practice problems for each week along with solutions.
  \end{enumerate}

Whereas introductory logic courses often draw students who are new to writing mathematical proofs, the undergraduate students taking logic at MIT have strong formal backgrounds.
This textbook was rewritten on their behalf and aims to provide a philosophically and formally rigorous introduction to formal logic through the soundness and completeness of first-order logic with identity.
I have taken pains to provide rigorous proofs throughout, including commentary about how the proofs work in addition to discussing their motivations.

Although this text is written with greater formal rigor than is common in most introductory textbooks in logic, I have aimed to preserve the friendly and accessible character of \forallx\ in addition to preserving its title.
Additionally, I have included discussions of the philosophical foundations of logic which are often omitted in introductory books and simply alluded to in more advanced treatments without being discussed or defended.

\section*{Teaching}

The terms at MIT span fifteen weeks, two of which are devoted to soundness and completeness for propositional logic (PL) and another two weeks to extend these results to first-order logic with identity (FOL$^=$).
In order to introduce the material at a slower rate or over the course of a shorter term, the metalogical portions of the text can easily be omitted without compromising the integrity of the text.

In addition to the \href{https://github.com/benbrastmckie/ForAllX/tree/master/Book/Chapters}{textbook}, I have included source files for the \href{https://github.com/benbrastmckie/ForAllX/tree/master/Lectures}{lecture notes} (compiled here as a \href{https://github.com/benbrastmckie/ForAllX/blob/master/Handouts/All_Handouts.pdf}{PDF}) as well as the \href{https://github.com/benbrastmckie/ForAllX/blob/master/Syllabus/Syllabus.pdf}{syllabus} for the course at MIT on GitHub.
Feel free to contact me if you would like access to the exams, written problem sets, and assignments on \href{https://carnap.io/}{Carnap}.

\section*{Collaboration}

Although you are welcome to fork the repository on GitHub adapting these resources as you please, I would also be happy to accept pull requests and would greatly appreciate being notified of any errors.
If you have any suggestion or questions, I would encourage you to open an issue above.
At some point I hope to include documentation describing how to adapt these resources for those new to \LaTeX{} and GitHub.

If you are interested in using the text editor that I used to streamline work on this project, you can check out the \href{https://github.com/benbrastmckie/.config}{NeoVim config} that I maintain, as well as the \href{https://github.com/benbrastmckie/VSCodium}{VSCodium config} that I started for those who are looking for something a little more user friendly.

\begin{flushright}
\textsf{Benjamin Brast-McKie} \\
\textsf{MIT, August 2024} \\
\texttt{brastmck@mit.edu}
\end{flushright}

% Here is the list of changes that I have made. % throughout the course of revising the textbook.
%
% \begin{itemize}
%   \item Chapter 0 has been largely rewritten to provide a more compelling description of the subject matter of logic, and to remove distracting glosses and inaccurate examples of validity which are apt to mislead and confuse the introduction of the course.
%   \item Chapter 1 has also been substantially rewritten, clarifying the use of schematic variables in the definition of the wffs of PL, fixing conventions for dropping parentheses and quotes, reducing certain discussions, and polishing examples throughout.
%   \item Chapter 2 has been largely rewritten, explaining the relationship between validity in English and valid regimentations which does not rely on a circumstantial modal reading of validity. The connection between interpretations of the language and the lines on a truth table has been clarified and certain inaccurate examples have been modified or removed to avoid confusion. The validity of argument schemata has also been avoided, focusing on the validity of PL arguments themselves.
%   \item Chapter 3 has been substantially rewritten, including a definition of satisfaction which is used to define entailment and satisfiability. Consideration of the weakening principle has been added and the relationship between validity and entailment has been clarified. 
%   \item Chapter 4 has been revised for consistency, simplifying the rules and definitions in accordance with standard conventions, introducing rules in a systematic order.
%   \item Chapter 5 has been substantially rewritten, fixing errors and including proofs of soundness and completeness instead of rough sketches. The chapter begins with an extended discussion of mathematical induction.
%   \item Chapter 6 has been reordered, condensed, and substantially rewritten.
%   \item Chapter 8 has been largely rewritten, eliminating reference to a universe of discourse, including recursive definitions of free and bound variables, clearly distinguishing between sentences and wffs, and otherwise adapting the conventions for consistency and elegance.
%   \item Chapter 9 has been completely rewritten to include a standard semantics for first-order logic which does not take constants to be rigid designators and defines truth relative to both a model and variable assignment, before defining truth at a model for sentences. Minimal models have been carefully motivated and presented along with a range of techniques for establishing entailments as well as evaluating sentences on a given model.
%   \item Chapter 10 has been largely rewritten to reflect the same semantic conventions presented in Chapter 9. The motivations for adding identity to the language have been connected to ideas about the logical and non-logical terms of the language. Definitions of free-for and substitution has been included and discussions of inequality quantifiers, cardinality quantifiers, definite descriptions, and Leibniz's law have been added or adjusted.
%   \item Chapter 11 has been substantially rewritten. Substitution has been carefully defined and used to articulate the rules for the quantifiers and identity in a standard manner. The chapter closes with some discussion of undecidability, motivating proofs of soundness and completeness for FOL instead of the tree method.
%   \item Chapter 12 has been written from scratch, establishing the soundness of FOL.
%   \item Chapter 13 has been written from scratch, establishing the completeness of FOL.
%   % \item Chapters have been rearranged and re-numbered so as to match the week numbers at MIT, starting with a half-week `0', and leaving a space for the midterm on week 7.
% \end{itemize}


\iffalse

\chapter*{Preface to the UBC Edition}
\label{ch.preface}
\addcontentsline{toc}{chapter}{Preface to the UBC edition}

This preface outlines my approach to teaching logic, and explains the way this version of \emph{forall x} differs from Magnus's original. The preface is intended more for instructors than for students. 

I have been teaching logic at the University of British Columbia since 2011; starting in 2017, I decided to prepare this textbook, based on and incorporating much of P. D. Magnus's \emph{forall x}, which has been freely available for use and modification since 2005. Preparing this text had two main advantages for me: it allowed me to tailor the text precisely to my teaching preferences and emphasis, and, because it is available for free, it is saving money for students. (I encourage instructors to take this latter consideration pretty seriously. If you have a hundred students a year, requiring them each to buy a \$50 textbook takes \$5,000 out of students' pockets each year. If you teach with this or another free book instead, you'll save your students \$50,000 over ten years. It can be sort of annoying to switch textbooks if you're used to something already. But is staying the course worth \$50,000 of your students' money?)

This text was designed for a one-semester, thirteen-week course with no prerequisites. At UBC, the course has quite a mix of students with diverse academic backgrounds. For many it is their first philosophy course. As I teach Introduction to Formal Logic, the course has three central aims: (1) to help students think more clearly about arguments and argumentative structure, in a way applicable to informal arguments in philosophy and elsewhere; (2) to provide some familiarity and comfort with formal proof systems, including practice setting out formal proofs with each step justified by a syntactically-defined rule; and (3) to provide the conceptual groundwork for metatheoretical proofs, introducing the ideas of rigorous informal proofs about formal systems, preparing students for possible future courses focusing on metalogic and computability. I try to give those three elements roughly equal focus in my course, and in this book.

The book introduces two different kinds of formal proof systems --- analytic tableaux (`trees') and Fitch-style natural deduction. Unlike many logic texts, it puts its greater emphasis on trees. There are two reasons I have found this to be useful. One is that the algorithmic nature of tree proofs means that one can be assured to achieve successful proofs on the basis of patience and careful diligence, as opposed to requiring a difficult-to-quantify (and difficult-to-teach) `flash of insight'. The other is that the soundness and completeness theorems for tree methods are simpler and more intuitive than they are for natural deduction systems, and I find it valuable to expose students to proofs of significant metatheoretical results early in their logical studies. (I prove soundness and completeness for a sentential logic tree system in the fifth week of the semester.) As presented here, the soundness and completeness proofs emphasize contrasting the systems students learn with hypothetical alternative systems that modify the rules in various ways. A rule like this would undermine the soundness of the system, but not its completeness. If we changed the rules in this way, it would still be both sound and complete. Etc. This helps give intuitive substance to these theorems.

I also include a Fitch-style natural deduction system, both for sentential and quantified logic, both because its premise-conclusion form is particularly helpful for thinking about informal arguments, and because it is important to recognize and follow proofs laid out in that kind of format, for example in more advanced philosophical material. While students do learn to do Fitch-style proofs, I emphasize less of that puzzle-solving kind of skill here than in many textbooks.

The book begins with a systematic engagement with sentential logic in conventional ways: translations, sentential connectives, models, truth tables, and both proof systems, including soundness and completeness for the tree system. Students are thereby able to familiarize themselves with all the central metalogical ideas, incorporating relatively simple logical symbolism, before introducing predicates, quantifiers, and identity. Once we enrich the language, we go through those previous ideas again, using our more complex vocabulary.

The first book I used for teaching was Greg Restall's \emph{Logic} (McGill--Queen's University Press, 2006), which I used for several years. My approach to teaching logic is heavily informed by that book; its influence in this text is particularly clear in the discussion of trees. (The natural deduction system I use is rather different from Restall's.)

In preparing this text, I began with Magnus's original and edited freely. There are sections where Magnus's prose has been retained entirely, and many of the exercises I have taken unchanged from the original. But I have also restructured many things and added quite a bit of new material. Unlike my version, which focuses on sentential logic before introducing predicates and quantification, Magnus's version integrated the discussion of sentential and quantificational systems, e.g.\ covering translation for both before discussing models and proofs for either. The original also did not include trees or soundness and completeness proofs. The two chapters on trees (\ref{ch.PL.trees} and \ref{ch.FOL}) and soundness and completeness (\ref{ch.SLsoundcomplete} and \ref{ch.FOL}) were written from scratch; my chapter on identity (\ref{ch.identity}) is also original. The other material in this edition incorporates Magnus's original material, some parts more heavily edited than others. I have slightly modified Magnus's natural deduction rules.

After a couple of years working with `beta' versions of the text online, I released the 1.0 version, along with the source code, in December 2018. The 2.0 version is new in summer 2020. The biggest changes in the latest round of revisions are in Chapter \ref{ch.ND.proofs}, where the order of presentation of the natural deduction rules has changed, and more examples have been added within the text. The rationale of the change was to start illustrating proofs earlier in the presentation of the rules. I've also put a bit more emphasis on the importance of exact matching of rule forms, and written a bit more precisely about the difference between PL proofs and proof schemas, when discussing derived rules. The other slightly substantive change I've made is to attend more precisely to how I'm using the term `interpretation' in the formal semantics for PL and FOL. One of my aims is to emphasize the continuity between the two languages --- in my system, FOL is literally a generalization of PL, and definitions of truth, entailment, etc., can be preserved. Various other smaller changes have been made as well, mostly stylistic changes and typo corrections. In summer 2021, I standardized and slightly modified the notation for assumptions in natural deduction proofs. I frequently make quite small corrections; the latest version is always on Github.

Many thanks, first and foremost, to P.D.\ Magnus for providing this wonderful resource under a Creative Commons license, which made it freely available and gave me the right to modify and distribute it under the same licensing agreement. I hope other instructors will also feel free to either teach directly from this version, or to modify it to develop their own. The typesetting for trees is via Clea F.\ Rees's prooftrees package; thanks to her for making it available.

I'm grateful to the students in my 2017--20 PHIL 220 courses at UBC, who had an in-progress version of this book as their course textbook. They patiently and helpfully found and pointed out mistakes as I wrote them (incentivized, perhaps, by an offer of extra credit); this version has many fewer errors than it otherwise would have had. Thanks also to Cavell Chan and Joey Deeth, who did careful proofreading, and generated many solutions to exercises for the answer key, and to Laura Greenstreet for LaTeX and other technical help. These three assistants were supported by a UBC Library Open Access Grant in 2018--19.

I am maintaining a list of known issues and errors for this book, to be corrected in future editions, under `issues' at \url{https://github.com/jonathanichikawa/for-all-x}. If you see any mistakes, please feel free to add them there directly, or to email me with them. The most recent version of the book is also always available for download there too.

\begin{flushright}
Jonathan Ichikawa \\
University of British Columbia \\
October 2021 \\
ichikawa@gmail.com
\end{flushright}

\fi
