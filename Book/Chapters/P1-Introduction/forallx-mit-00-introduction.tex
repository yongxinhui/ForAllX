%!TEX root = ../../forallx-mit.tex

% TODO: add a bit more about proof theory to match the handout

\chapter{What is Logic?}
\label{ch.introduction}

Logic is the study of \textit{formal reasoning}, that is, what follows from what in virtue of logical form.
% By `formal' we don't just mean that we will be using mathematical symbols, although we will.
% After all, mathematics, physics, and many other fields use mathematical symbols, but that doesn't make all of those fields logic.
Although the reasoning that we will be concerned with will include mathematical symbols, the choice of symbols is arbitrary, and so this is not the sense of formality at issue.
After all, many fields engage in reasoning with mathematical looking symbols but we will not be concerned to describe all of what goes on in reasoning in those fields.

There are of course many other ways to reason besides the manner we will be concerned with in this book.
In order to characterize the subject-matter of logic, this chapter will contrast formal reasoning with a number of other common types of reasoning.
This will provide a theoretical target which subsequent chapters will describe first \textit{semantically} and then \textit{syntactically}, proving that these two accounts describe the same thing.

Reasoning moves the reasoner from some number of considerations to a further consideration.
In particular, we will be concerned with \define{declarative sentences} in English, that is, English sentences that are either true or false, and not both at once.\footnote{Whereas italics will be used for emphasis, key vocabulary will be capitalized throughout.}
% Since we will \textit{only} be concerned with declarative sentences throughout what follows, we may drop `declarative', referring simply to 
A \define{deductive argument} in English is a nonempty sequence of declarative sentences where a single sentence is designated the \define{conclusion} and all other sentences (if any) are referred to as the \define{premises}.
For instance, consider the following argument where the conclusion follows the horizontal line:

\begin{earg}
  \eitem{Kat is sitting down.}
  \uitem{If Kat is sitting down, then Kat is not standing.}
  \eitem{Kat is not standing.}
\end{earg}

The horizontal line is pronounced `therefore'.
The premises of an argument aim to provide \textit{reasons for taking the conclusion to be true} even if not all of the premises are true, known to be true, likely, or even possible.
For instance, even if we don't know whether A1 and A2 are true, we may admit that if A1 and A2 are both true, then the conclusion A3 is also true.
% Put otherwise, the conclusion (3) follows from the premises (1) and (2) by virtue of the logical form of this argument.

Since not all arguments in English take such explicit forms as the argument above, part of what we will be concerned with is regimenting arguments in English by translating them into an artificial formal language.
Although there is often more than one way to regiment an argument stated in English, once an argument has been regimented in a formal language, no further ambiguity remains as to how to evaluate that formalized argument.
An important first step in this process is to identify the premises and conclusion, writing them as a numbered list where the conclusion occurs on the last line with a line separating it from the premises.

In the course of this book, we will introduce the syntax for the formal languages $\PL$ and $\FOL$.
For the time being, this chapter will restrict consideration to arguments in English written in premise-conclusion form.
In the case of the argument above, we may observe that believing the premises provides a strong reason to believe the conclusion independent of whether we happen to believe the premises or not.
Of course, not all arguments succeed in providing such strong support for their conclusions.
% However, just because an argument takes premise-conclusion form doesn't mean that the argument is any good.
For instance, consider the following argument:

\label{argBunk}
\begin{earg}
  \eitem{Siya is out sailing.}
  \uitem{Nicky is hard at work.}
  \eitem{The house is empty.}
\end{earg}

Even though there might be ways to modify the argument in order to make it more compelling, as stated, the argument above is just a series of (logically) unrelated sentences.
Accordingly, one might be tempted to protest that this is no argument at all, but rather a random list of sentences.
Although this book will not take a stand on how the word `argument' ought to be used in general, we won't require arguments to be compelling in order to count as arguments.
% Rather, an argument is \textit{any} sequence of premises followed by a single conclusion.

% Logic is the business of evaluating arguments, sorting good ones from bad ones.
% In everyday language, we sometimes use the word `argument' to refer to belligerent shouting matches.
% If you and a friend have an argument in this sense, things are not going well between the two of you.
% This is not the kind of arguments that will concern us.
% Arguments in the sense with which we will be concerned aren't events that happen between people.
% Rather, we will take arguments to consist only of sentences. % which are abstract syntactic objects.

Whereas the first argument we provided was very compelling and the second argument was not compelling at all, many arguments fall somewhere between these two.
Here is an example:
% Here are two examples:

% \label{argRaining}
% \begin{earg}
%   \item It is raining heavily.
%   \uitem{When it rains, everyone outside without an umbrella gets wet.}
%   \item If Pavel doesn't want to get wet, he should take an umbrella.
% \end{earg}
% \vspace{-.2in}

% \label{argSnowing}
% \begin{earg}
%   \item It is either raining or snowing.
%   \item If it is colder than -10 degrees, it is not raining.
%   \uitem{It is -18 degrees.}
%   \item It is snowing.
% \end{earg}

\begin{earg}
  \eitem{San Francisco is West of Cambridge.}
  \uitem{Cambridge is West of Oxford.}
  \eitem{San Francisco is West of Oxford.}
\end{earg}
 
 
Although this argument might seem to be much more compelling than the second argument, its conclusion does not follows from its premises by virtue of logic alone. 
Put otherwise, the conclusion in this final argument is not a purely logical consequence of its premises, where something similar may be said for the second argument given above.
Whereas later chapters will carefully define what it is for a sentence to be a logical consequence of a set of sentences in formal languages, the remainder of the present chapter will focus on characterizing an informal analogue for arguments in English.
These considerations will provide an intuitive conception of what it is for a conclusion to follow as a matter of logic from some number of premises, motivating the study of the formal languages with which we will be concerned.

In addition to providing a means by which to resolve the ambiguities of natural language, regimenting English arguments in formal languages will provide an account of their logical form.
% Roughly speaking, you can think of the logical form of an argument as the mechanisms by which logical consequence is conveyed.
As we will see, much will turn on the expressive power of our formal languages since the conclusion of an argument in English may be a logical consequence of its premises when regimented in one formal language but not in another.
Whereas this book will introduce certain formal languages, this is not a stopping point but rather a place to begin.






\section{Arguments}

A crucial part of analyzing an argument is identifying its conclusion.
Every argument has a conclusion--- the conclusion is the claim the argument aims to establish.
Premises are starting-points, used to lend support to the conclusion.
In English, the conclusion is often signified by words like `so' or `therefore'.
Premises might be marked by words like `because'.
These words can provide some clue as to just what the argument is supposed to be.

\begin{description}
  \item[Premise indicators:] `since', `because', `given that'
  \item[Conclusion indicators:] `therefore', `hence', `thus', `then', `so'
\end{description}

In a natural language like English, arguments \textit{sometimes} begin with their premises and end with their conclusions, but not always.
Since arguments in which the conclusion does not occur at the end may be rearranged, we may restrict attention to arguments in which the conclusion occurs on the last line without loss of generality.
% For our purposes in this course, we will be working with \emph{idealizations} of natural language, where we work as if some generally applicable rules of thumb held without exception.
% Let's define (a slightly technical notion of) an \define{argument} as a sequence of sentences.
% The sentences at the beginning of the sequence are premises.
% The final sentence in the sequence is the conclusion.
For instance, in a single sentence, one might argue that, ``People often get wet in Cambridge because it often rains in Cambridge and people get wet when it rains.''
We may then We may then rearranging this argument as follows:

\begin{earg}
  \eitem{People get wet when it rains.}
  \uitem{It often rains in Cambridge.}
  \eitem{People often get wet in Cambridge.}
\end{earg}

% The aim of many arguments (their \textit{telos}) is for the premises to provide a reason to accept the conclusion (or more generally, to increase your degree of confidence in the conclusion).
% If I'm trying to convince you that people often get wet in Vancouver--- the conclusion of the argument above--- convincing you of the two premises might be a good way to get you there.
%
% Notice that our definition of an argument is quite general. Consider this example:
% \begin{earg}
%   \item[] Vancouver has more churches than any other Canadian city.
%   \item[] Two oboes are fighting a duel under the fireworks (watch out!).
%   \item[\therefore] J.\ Edgar Hoover was an honest man, allegedly.
% \end{earg}
%
% It may seem odd to call this an argument, but that is because it would be a terrible argument.
% The two premises have nothing at all to do with the conclusion.
% Moreover, they aren't very plausible.
% Nevertheless, given our definition, it still counts as an argument--- albeit a bad one.

% As brought out above, arguments aim to provide premises which support their conclusions.
One of our central aims in logic is to provide rigorous, formal tests for evaluating whether an argument's conclusion is a logical consequence of its premises.
% The argument above, however plausible it may seem, will turn out to fail the tests.
% \textit{Deductive arguments} draw on general rules of reasoning to derive a conclusion from some premises, where these rules of reasoning are given to us by the logic for the language in question.
We will have much more to say about all of this in what follows, but for now we may take this to describe the theoretical role that deductive arguments strive to play.
By contrast, \textit{inductive arguments} aim to increase our degree of confidence in a conclusion by appealing to some number of instances of a general inference which are taken to provide evidence in support of that general inference.
Additionally, \textit{abductive arguments} appeal to the best explanation for some number of claims in order to provide support for a further claim. 
By contrast with deductive arguments, even the strongest inductive and abductive arguments remain open to counterexamples.

For better or for worse, we will have nothing more to say about inductive and abductive arguments in this course.\footnote{We \textit{will} have occasion to employ an entirely different style of reasoning called `mathematical induction'.}
This is despite the fact that much of science and ordinary life operates using inductive and abductive rather than deductive reasoning.
The good news is that our systems will be versatile and have some elegant formal properties, making it a good candidate for a wide range of applications.
In particular, deductive logic is of vital importance for mathematics and computer science, significantly reshaping the world that we live in by making the rise of the information age possible.
% Inductive logic is still struggling to achieve comparable riches.






\section{Sentences and Propositions}
\label{intro.sentences}

The premises and conclusions of an argument in English are grammatical sentences, but not all grammatical English sentences are suitable for figuring in arguments.
For example, questions count as grammatical English sentences, but logical arguments never have questions as premises or conclusions.
As mentioned above, we are specifically interested in declarative sentences that can be true or false, and never both true and false at once.
In order to set them aside, here are some types of the sentences we will not be concerned with:

\paragraph{Questions} `Are you sleepy yet?' is a perfectly grammatical interrogative sentence.
Whether you are sleepy or not, the question itself is neither true nor false.
% Thus `Are you sleepy yet?' is not a declarative sentence--- it does not express a proposition that is either true or false.
Suppose you answer the question: `I am not sleepy.'
This answer is either true or false, and so is a declarative sentence.
Generally, questions will not be declarative sentences, but answers to questions often will. 
For instance, `What is this course about?' is not a declarative sentence, but `No one knows what this course is about' is a declarative sentence.

\paragraph{Imperatives} Commands such as `Wake up!', `Sit up straight!', and so on are grammatical imperative sentences.
Although it might be good for you to sit up straight or it might not, the command is neither true nor false.
Note, however, that commands are not always phrased with imperatives.
`You will respect my authority' \emph{is} either true or false--- either you will or you will not--- and so it counts as a declarative sentence.

\paragraph{Exclamations} Expressions like `Ouch!' or `Boo, Yankees!' are sometimes described as exclamatory sentences, but they are neither true nor false.
We will treat `Ouch, I hurt my toe!' as meaning the same thing as `I hurt my toe.'
The `Ouch' does not add anything that could be true or false, and so will simply be dropped if it occurs at all.

% TODO: add consideration of the interpretation when introducing 'expresses'
% TODO: since we will only be concerned with whether a proposition obtains or not, we may model the space of propositions by just two propositions, one which obtains and one which does not

In contrast to the types of sentences mentioned above, declarative sentences are used to describe \textit{ways for some particular things to be} which we will refer to as \define{propositions}.
Put otherwise, declarative sentences are used to \textit{express} propositions, where the very same proposition may be expressed in many different ways by different sentences in different languages.
For instance, the English sentence `snow is white' expresses the same proposition as the German sentence `Schnee ist wei\ss'.
Both express the proposition that snow is white.

For our purposes, the defining feature of a proposition is that it either \textit{obtains} or it does not.
For instance, suppose that Kat is sitting down.
The proposition that Kat is sitting down is a thing (a person)--- namely Kat--- being a certain way, i.e., sitting down.
Given our supposition, the proposition that Kat is sitting down obtains.
Similarly, we may suppose Tsovinar is not singing, and so the proposition that Tsovinar is signing does not obtain.
Despite failing to obtain, the proposition that Tsovinar is signing is a way for things to be all the same--- i.e., Tsovinar singing--- it is just that things are not in fact that way.
% Just because a proposition does not obtain does not mean that there is no such proposition.

% A sentence in English is said to be \define{declarative} just in case it expresses a proposition.
% For instance, the sentence `Kat is sitting down' expresses the proposition that Kat is sitting down.
Whereas a proposition is a way for some things to be which may either obtain or fail to obtain, a declarative sentence is a grammatical string of symbols which is either true or false.
For instance, supposing that that Kat is sitting down--- i.e., the proposition that Kat is sitting down obtains--- the sentence `Kat is sitting down' is true.
More generally, a declarative sentences in English is true on an interpretation if it expresses a proposition that obtains, and false otherwise.
The \define{truth-value} of a declarative sentence is either the value `True' or `False' depending on whether the proposition that sentence expresses happens to obtain.

Given that we will only be concerned with whether a proposition obtains or not, we may make the simplifying assumption that there are only two proposition: \define{truth} which obtains and \define{falsity} which does not.
Simpler still, we may ignore talk of propositions altogether by focusing on the corresponding truth-values of the sentences of our languages, using `$1$' in place of `True' and `$0$' in place of `False'.
More specifically, we will interpret the sentences of our various languages by assigning them truth-values, ignoring any further differences that may occur between the propositions that those sentences may express on an interpretation.
The present investigation belongs to classical logic insofar as we will restrict consideration to sentences which are either true or false and never both at once.

% TODO: equivalent to just two propositions

% Of course, there are many other types of sentences in English that do not have truth-values.
% For instance, commands like `Shut the door', 
% Throughout what follows, we will focus on declarative English sentences, 
% Although Kat cannot sit and stand at once, nothing stops us from considering these propositions independent of which (if either) is true.

% Since we are only interested in sentences that can figure as a premise or conclusion of an argument, we'll define a informal notion of a \define{sentence} as a grammatical string of symbols that expresses a proposition which is either true or false.
% Such strings are often referred to as \textit{declarative sentences}, though for brevity we will often drop `declarative'.
% When we say that `a sentence can be true or false,' all we really require is that we can intelligibly assign a truth-value to it.
% Although we will not theorize about them in this course, there are many other non-propositional sentences.

There are many sentences where it is unclear or controversial whether they have a truth-value.
Think of sentences such as `Almonds are yummy' or `The U.S.\ invasion of Iraq was unjustified'.
In an argument, we can assign a truth-value to such sentences, even if one might be skeptical that they have a truth-value independent of any subject.
We will handle these sentences just like sentences which have objective truth-values such as the sentence, `Tsovinar is signing'.
Indeed, often what is at stake in our more interesting arguments is whether various normative or evaluative claims are true or false.
So clearly, we have \textit{some} way of reasoning about such sentences using truth-values where this is all that we will aim to describe here.
What else there is to say about normative and evaluative sentences is beyond the scope of this course.

%Ichikawa: Don't confuse the idea that a sentence can be true or false with the difference between fact and opinion. Often, sentences in logic will express things that would count as facts --- such as `Kierkegaard was a hunchback' or `Kierkegaard liked almonds.' They can also express things that you might think of as matters of opinion --- such as, `Almonds are yummy' or `the U.S.\ invasion of Iraq was unjustified'. These are all examples of things that are either true or false.
% JH: But what does it mean to say that the opinion-sentences are true or false? Does this  na\"ively commit us to a kind of realism about their contents?

It is also important to keep clear the distinction between truth and knowledge.
Although a declarative sentence is the kind of thing that can be true or false, this does not mean that we will always be in a position to know whether it is true or false.
For example, assuming a clear account of what it is to be a human to be provided, the sentence, `There are an even number of humans on Earth right now' may be taken to have a truth-value even though it would be virtually impossible to determine.\footnote{There are other concerns to do with vagueness which we will also ignore.}
Similarly, there are many controversies where people disagree about what is true or false, and it very hard to settle the debate.
However, for the sake of an argument, we may treat controversial premises as true in order to see what follows as a result without committing ourselves to these assumptions.
% JH: not sure about the humans example. There are issues of vagueness regarding both conjoined twins and also when humans come into existence (e.g. at what point does a baby have to exit the womb? do late-stage fetuses count as babies and if so when)

% To recap: \emph{declarative sentences} are claims that can be true or false.
% One pretty good test you can run to see whether something is a sentence is to ask whether it makes sense to insert `It is true that' or `It is false that' in front of it.
% It's perfectly fine to say `It is true that Kierkegaard liked almonds' or `It is true that the U.S.\ invasion of Iraq was unjustified'.
% But it doesn't make sense to say `It is true that are you sleepy yet' or `It is true that sit up straight'.





\section{Logical Consequence}
  \label{sec:LogicalConsequence}

For our purposes, there are two important ways that arguments can go wrong.
To begin with, suppose that the following argument were presented in a court of law:

\begin{earg}
  \eitem{The victim was shot by a bullet from the gun that was found at the defendant's house.}
  \eitem{The fingerprints on the gun were shown to match the defendant.}
  \eitem{The gun was registered in the defendant's name.}
  \uitem{The defendant had recently been fired by the victim.}
  \eitem{The defendant shot the victim.}
\end{earg}

An argument is only compelling if its premises are true.
If the premises above can be shown to be false, the defendant may well be innocent.
More generally, we should not feel at all persuaded to believe a conclusion on the basis of an argument with false premises. 
This is the first way that an argument can go wrong: not all of the premises are true.

Assuming that the premises E1 -- E4 are true, the conclusion E5 would seem to follow.
What we mean by `follow' in this instance is that the truth of the premises makes the truth of the conclusion extremely likely, and perhaps so likely that it is beyond a reasonable doubt, compelling a jury to find the defendant guilty.

What may be good enough for the law is not good enough for logic.
This is not to disparage the criminal justice system but to observe that the truth of the conclusion in the argument above does not follow \textit{logically} from the truth of its premises.
However likely the conclusion may be given the truth of its premises, it is still \textit{possible} for the premises to be true and the conclusion false.
For instance, consider a possibility in which it was the defendant's partner who shot the victim.
Even though all of the premises are true, the conclusion is false in such a possibility.
We may not know if such a possibility took place or not, but it doesn't matter.
So long as it is possible for the premises to be true and the conclusion to be false we have reason to deny that the conclusion follows logically from its premises.

Is this what logical validity is about: ruling out \textit{possibilities} in which the premises are true and the conclusion is false where possibilities are ways for things to be?
Although logical validity is sometimes glossed this way, the answer is `No!'.
Let's take a look at some arguments that rule out possibilities in which the premises are true and the conclusion false:

\begin{earg}
  \uitem{The atom is gold.}
  \eitem{The atom has 79 protons.}
\end{earg}


\begin{earg}
  \eitem{Suela is a fox (the animal).}
  \uitem{Suela is female.}
  \eitem{Suela is a vixen.}
\end{earg}

In both of the arguments above, every possibility in which premises are true is one in which the conclusion is true.
Assuming the premises are true, the conclusion follows as a matter of necessity.
Put otherwise, the truth of the premises \emph{strictly imply} the truth of their conclusions.
Certainly these are stronger arguments than what we might find in a court of law.
Are there arguments that are even more powerful than this?
The answer is `Yes!', and this brings us to the topic of this course.
Consider the following argument:

\begin{earg}
  \eitem{Socrates is a man.}
  \uitem{Every man is mortal.}
  \eitem{Socrates is mortal.}
\end{earg}

Here too the premises strictly imply the conclusion since there is no possibility in which the premises are true and the conclusion is false.
However, unlike the previous arguments, we do not need to know what `Socrates', `man', or `mortal' each mean.
In order to get a sense of this, let's consider one more argument:

\begin{earg}
  \eitem{Gyre is a mome rath.}
  \uitem{All mome raths are slithy.}
  \eitem{Gyre is slithy.}
\end{earg}

Even without knowing who Gyre is, anything about mome raths, or what it is to be slithy, we may say with equal certainty that it is not possible for the premises to be true and the conclusion false.
In fact we can say more than this: there is no \textit{interpretation} of the premises and conclusion where the former are true and the latter is false.
% So what's an interpretation?

We will provide a precise definition of what an interpretation is when we set up semantic theories for the languages that we will study throughout this book (chapters $\ref{ch.PL-semantics}$, $\ref{ch.FOL-semantics}$, and $\ref{ch.Identity}$).
For the time being, it will help to get some sense of an informal analogue with which we are already familiar.
To do so, let's return to the gold argument from before.

Surly any possibility in which the atom is gold is also one in which the atom has 79 protons.
After all, having 79 protons is part of what it is to be gold, and so something couldn't be gold without having 79 protons.
Who could argue with that?

Although no one should balk at the gold argument given the normal interpretation of its sentences--- what is often called the \textit{intended interpretation}--- the same cannot be said if we entertain unintended interpretations.
For instance, suppose we were to take `gold' to mean what `carbon' means in the intended interpretation, that is: carbon.
What the premise mean on this unintended interpretation could equally be said by an intended interpretation of the sentence `The gold atom is carbon'.
Since carbon only has 6 protons and not 79, the conclusion is false when the premise is true on this unintended interpretation.
But why should we care about unintended interpretations?
Shouldn't we restrict attention to just the intended interpretations of our sentences that we are accustomed to using?

The reason for considering all interpretations and not just the intended one(s) we seem to most of the time is that it allows us to distinguish especially strong types of arguments that hold independent of how we choose to interpret our language.
Although we will improve on this characterization in later chapters, we may nevertheless draw on an intuitive understanding of the interpretations of our language to say that what it is for a sentence to be a \define{logical consequence} of a set of sentences is for the former to be true on any interpretation in which every sentence in the latter is true. 
An argument is \define{logically valid} just in case the conclusion of the argument is a logical consequence of its set of premises.

Functionally, it is helpful to think of logically valid arguments as arguments that can be relied on no matter how (or whether) you understand the non-logical terms like `is gold' or `Socrates' that occur in the sentences of the argument.
So even though the gold argument is extremely compelling when we maintain the intended interpretation of our language that we are all accustomed to, the gold argument is not a logically valid argument: there is an interpretation of the language in which its only premise is true and its conclusion is false.

We have identified the second way that an argument can go wrong: it can admit of an interpretation in which the premises are true but the conclusion is false.
Recall Gyre and those slithy mome raths from before.
We may not know much about these sorts of things, but we can be sure that Gyre is slithy if Gyre is a mome rath and all mome raths are slithy.
We may know considerably more about Socrates being male and mortal, but similar reasoning applies.
Indeed, these two arguments may be observed to have the same \textit{logical form}.
It is these logical forms of reasoning that make up the subject-matter of this course.

When an argument has neither of the defects considered above--- i.e., when it is both logically valid and has true premises on the intended interpretation--- we may say that it is \textit{sound}. % TODO: SAVE? \footnote{As we will see, there is an entirely distinct notion of \textit{soundness} pertaining to our proof systems.}
Sound arguments are a good thing, but fall outside the scope of this course.
Why is that?
Because securing the truth of the premises is often an empirical (or in general an extra-logical) matter and presumes that a single interpretation is to be privileged over the others.
Rather, we will only be concerned with identifying which arguments are logically valid, not which interpretation we should focus on or which premises are true on that interpretation.
We will also exclude consideration of a wider understanding of valid arguments which includes arguments that are really convincing--- e.g., in a court of law--- but not logically valid.
Accordingly, we will typically drop `logically' in talking about logically valid arguments, referring to arguments simply as \textit{valid} or, when they are not valid, as \textit{invalid}.

A parting question: how would you begin to describe the space of all valid forms of reasoning? 
It is a great intellectual achievement of the late 19th and early 20th centuries that we have devised systematic methods for answering this question (relative to a language).
In this course we will consider two types of answers, one belonging to proof theory, and the other belonging to model theory (also called semantics).
In the metalogical portions of this course, we will show how these two methods return the same answer, describing one and the same space of valid forms of reasoning despite doing so in radically different ways.
% In addition to its various applications, deductive logic sets an ideal for reasoning that we can aspire to in theorizing about a wide range of topics.

% Rather than providing a \textit{descriptive} theory about how we in fact reason, logic aims to provide a \textit{normative} theory which describes how we ought to reason, at least in certain applications.
% In this right, studying logic may literally change how you think.




\section{Logical Form}
\label{sec:LogicalForm}

We've seen that a valid argument does not need to have true premises nor a true conclusion.
Conversely, having true premises and a true conclusion on a particular interpretation is not enough to make an argument valid.
Consider this example:

\begin{earg}
  \eitem{Kamala Harris is a U.S.\ citizen.}
  \uitem{Justin Trudeau is a Canadian citizen.}
  \eitem{UBC is the largest employer in Vancouver.}
\end{earg}

The premises and conclusion of this argument are all true on the intended interpretation.
Nevertheless, this is quite a poor argument.
In particular, the definition of validity is not satisfied: there are interpretations in which the premises are true while the conclusion is false.
Although the conclusion is true on the intended interpretation, this may fail to hold on other interpretations.
For example, we may interpret `UBC' to mean what `Lululemon' does on the intended interpretation.
Accordingly, the premises are true, and yet the conclusion is false.

The important thing to remember is that validity is not about the truth or falsity of the sentences in the argument on any particular interpretation. 
Instead, it is about the \textit{logical form} of the argument.
But what is the logical form of an argument?

We have begun to see some valid arguments like the Socrates argument and the Gyre argument.
But was that one argument or two?
Recall that an argument in English is a sequence of declarative sentences in which the conclusion occurs on the last line.
Since the sentences in the Socrates and Gyre arguments differed, they are different arguments.
Nevertheless, these arguments share the same logical form.
Here is a valid argument with a different logical form:

\begin{earg}
  \eitem{Oranges are either fruits or musical instruments.}
  \uitem{Oranges are not fruits.}
  \eitem{Oranges are musical instruments.}
\end{earg}

This is a valid argument: there is no interpretation in which the premises are true and the conclusion is false.
Since, given the intended interpretation, it has a false premise--- premise (2)--- it does not establish its conclusion (it is not a sound argument), but it does have a valid \emph{logical form}.
Here is another example of an argument with a valid logical form:

\begin{earg}
  \eitem{If it is raining, then the streets are wet.}
  \uitem{The streets are not wet.}
  \eitem{It is not raining.}
\end{earg}

As we will see, there are many logical forms that arguments can have.
In order to characterize the abstract forms themselves, we will use variables.
To begin with, we will consider the variables `$\metaA$', `$\metaB$', `$\metaC$', \ldots\ for sentences, calling these \define{schematic variables}.
Schematic variables allow us to talk about the sentences of a language but they are not themselves sentences of the language.
Rather $\metaA$, $\metaB$, and $\metaC$ have sentences as \textit{values}.
It is helpful to compare the use of variables like `$x$' in mathematics.
The symbol `$x$' is used to stand for a number but `$x$' is not itself a name for a number the way that `2' is a name for the number two.
We'll return to this distinction in Chapter \ref{ch.PL-syntax}, and again later on when we discuss proving general facts the formal languages and proof systems introduced below.

Without introducing any notation beyond schematic variables for sentences, we may represent the logical form of the previous argument as follows:

\begin{earg}
  \eitem{If $\metaA$, then $\metaB$.}
  \uitem{It is not the case that $\metaB$.}
  \eitem{It is not the case that $\metaA$.}
\end{earg}

Instead of an argument itself, what we have above is a recipe where substituting declarative sentences for the variables `$\metaA$' and `$\metaB$' returns an argument that, like the raining argument, is valid.
By replacing sentences with variables, we were able to abstract away the non-logical parts of the raining argument, leaving behind the logical form of the original raining argument.
We may refer to the result as an \define{argument schema}.
Argument schemata are built up out of three elements: schematic variables, punctuation, and \define{logical constants}.
The logical constants included above were represented using `If\ldots, then\ldots' and `It is not the case that\ldots'.
We will introduce more elegant representations of these logical constants in the next chapter.
Until then, it is worth considering whether we can identify an argument schema for the Socrates argument in the very same way as in the raining argument.
% Yes, but we need more than schematic variables to do so.

Suppose that we were to maintain our restriction to schematic variables for sentences from before.
The plan is to replace sentences with variables and try to recover an argument schema just like we did previously.
Since `Socrates is a man' is a sentence and doesn't have any parts that are also sentences, it is an \define{atomic sentence}, and so all we can do is replace it with a variable.
Let's choose `$\metaA$'. 
We find something similar for the second premise `Every man is mortal'.
Suppose we choose the `$\metaB$'.
The conclusion is also as simple as sentences get, and so let's substitute `$\metaC$'.
This returns the following argument schema:

\begin{earg}
  \eitem{$\metaA$}
  \uitem{$\metaB$\quad}
  \eitem{$\metaC$}
\end{earg}

Although this does \textit{schematize} the Socrates argument, it does not leave behind anything which we might appeal to in explaining why the Socrates argument was valid.
For instance, if we replace the variables with any other sentences, we do not necessarily get a valid argument.
Have we made some mistake?
No.
Rather, the validity of the Socrates argument is not visible at the logical resolution that we have been working.
Instead of abstracting on sentences, we need to analyze the sub-sentential parts, identifying logical constants at this higher level of logical resolution.
In particular, we will need to split sentences up into predicates and singular terms, introducing logical constants for quantification like `for all' and `there is some'.
This ambition will be addressed in later chapters on First-Order Logic (FOL).
Until then, we will keep things simple to start, focusing on Propositional Logic (PL).





\section{Other Logical Notions}

We have begun to characterize the subject-matter of logic by schematizing arguments.
As brought out above, substituting schematic variables for the non-logical elements of an argument can be used to identify the logical form of the argument.
Given a sufficient degree of logical resolution power, we may appeal to the logical form of an argument in order to explain why the argument is valid.
We will make this process precise in the following chapter by restricting attention to a rigorously defined formal language.
For the time being, we may conclude the present chapter by introducing a few more terms to look out for in what follows.


\subsection{Logical Truth, Falsity, and Contingency}
\label{sec-tautologydef}

% In considering arguments, we care about whether the conclusion is true if the premises are true on any interpretation where interpretations will be carefully defined. 
Instead of being concerned with the truth value of sentences on any particular interpretation, we will be concerned with the truth-values of sentences across all interpretations of the language in question where interpretations will be carefully defined.
For instance, in a valid argument, the conclusion is true in any interpretation in which all of the premises are true even though neither the true premises nor conclusion are required to be true in any particular interpretation.
Nevertheless, there are certain sentences that are true on all interpretations.
To bring this out, compare the following sentences:

\begin{earg}
  \eitem{It is raining.}
  \eitem{Either it is hot outside, or it is not hot outside.}
  \eitem{John is sitting down and it is not the case that John is sitting down.}
\end{earg}
% TODO: fix `never' above which has a temporal reading and modal reading

Sentence O1 is true on some interpretations and false on others and so is said to be \define{logically contingent}.
Sentence O2 is different.
Even though we may not know what the weather is like, we know that it is either hot or it isn't.
Moreover, O2 is true no matter how you interpret `It is hot outside', and so O2 is true on all interpretations.
Accordingly, O2 is a \define{logical truth}, or what is also called a \define{tautology}.
By contrast, sentence O3 is false on all interpretations, and so is referred to as \define{logically false} or a \define{contradiction}.
In particular, you do not need to know what John is up to, or know how to interpret the sentence `John is sitting down' in order to know that O3 is false.
% TODO add contingent definition for English sentences here to mirror def for PL later

% I said above that a contingent sentence could be true on some interpretations and false on others.
% We can also define contingency in terms of tautologies and contradictions thus: a \define{contingent sentence} is a sentence that is neither a tautology nor a contradiction.


\subsection{Logical Entailment and Equivalence}

In addition to the logical properties that sentences may have on their own, we may also consider the logical relations that hold between two sentences.
For example:

\begin{earg}
  \eitem{Clara went to the store.}
  \eitem{Someone went to the store.}
\end{earg}

Regardless of how we interpret the sentences above, P2 is true if P1 is true.
Put otherwise, P1 \define{logically entails} P2.
Whereas logical consequence related zero or more premises two a single conclusion, logical entailment relates exactly two sentences.
Nevertheless, one sentence logically entails another just in case the latter is a logical consequence of the former.

When two sentences logically entail each other, those sentences are said to be \define{logically equivalent}.
For instance, consider the following sentences:

\begin{earg}
  \eitem{If Sunil went to the store, then he washed the dishes.}
  \eitem{If Sunil did not wash the dishes, then he did not go to the store.}
\end{earg}

Not only does Q1 logically entail Q2 but Q1 is also logically entailed by Q2.
It follows that Q1 and Q2 have the same truth-value in all interpretations.
For contrast, consider:

\begin{earg}
  \eitem{Edinburgh is North of London.}
  \eitem{London is South of Edinburgh.}
\end{earg}

Although these sentences have the same truth-value in the intended interpretation, they do not have the same truth-value in all interpretations.
For instance, if we took `is North of' to mean what `has a larger population than' means on the intended interpretation while maintaining the intended interpretation of the other terms, then R2 would be true and R1 would be false, and so R2 does not entail R1.
It follows that R1 and R2 are not logically equivalent: they do not have the same truth-value in all interpretations.

Not only does R2 fail to entail R1 we may show that R1 does not entail R2 since there is an unintended interpretation where R1 is true and R2 is false.
In a similar manner to above, we might take `is South of' to mean what `has a smaller population than' means on the intended interpretation while maintaining the intended interpretation of the other terms.
This also suffices to show that R1 and R2 are not logically equivalent.




\subsection{Satisfiability}

Consider these three sentences:

\begin{earg}
  \eitem{Sam is shorter than John.}
  \eitem{Sam is taller than John. }
  \eitem{If Sam is shorter than John, then Sam is not taller than John.}
\end{earg}

Independent of how we interpret the sentences above we may determine that there is no interpretation which makes all of these sentences true. 
For instance, we might reason as follows: suppose there were some interpretation which makes all three sentences true.
However, it follows from S1 and S3 that S2 is false.
This contradicts our supposition.
Since we have arrived at a contradiction on the supposition that there was an interpretation that makes all three sentences true, we may reject our supposition, concluding that there is no interpretation which makes all three sentences true.

A set of sentences is \define{satisfiable} just in case there is some interpretation which makes every sentence in the set true, and \define{unsatisfiable} otherwise.
In particular, the set $\set{\text{S1}, \text{S2}, \text{S3}}$ is unsatisfiable.
Given an unsatisfiable set of sentences $X$, if every sentence in $X$ is also a sentence in $Y$, then $Y$ is also unsatisfiable. 
The opposite, however, is not true: if every sentence in $Z$ is in $X$ and $X$ is unsatisfiable, it does not follow that $Z$ is unsatisfiable. 
For instance, every subset of $\set{\text{S1}, \text{S2}, \text{S3}}$ which has at most two members is satisfiable.
This includes the empty set $\varnothing$ since, trivially, every sentence in $\varnothing$ is true in all interpretations.

Here is another concrete example of a satisfiable set of sentences:

\begin{earg}
  \eitem{The Earth has more than one moon.}
  \eitem{Jupiter has exactly one moon.}
\end{earg}

Even though both of the sentences above are false on the intended interpretation, the set of sentences $\set{\text{T1}, \text{T2}}$ is satisfiable on account of the existence of an interpretation in which both sentences are true.
For instance, we might interpret `Jupiter' to mean what `Earth' means on the intended interpretation and interpret `Earth' to mean what `Jupiter' means on the intended interpretation while maintaining the intended interpretation of the other terms.

Despite the fact that satisfiability and unsatisfiability applies to sets of sentences rather than individual sentences, we may observe that any set of sentences which includes a contradiction is unsatisfiable.
Similarly, if $W'$ is a set of sentences that results from adding a tautology to a set of sentences $W$, then $W'$ is satisfiable just in case $W'$ is satisfiable. 
In particular, any set of sentences which only includes tautologies will be satisfiable.
We will have much more to say about the role that satisfiability will play in studying proof systems in later chapters.
But before all of that it will be important to introduce a formal language.

So far we have relied on an intuitive conception of an interpretation in order to introduce such notions as satisfiability and logical consequence, concepts which are of central importance in logic.
However familiar English may be, there is no formally precise definition of what counts as an English sentence, and so there is no definition to be had of what counts as an interpretation of the sentences of English.
In order to provide a mathematically precise definition of an interpretation, it will be important to provide an equally precise definition of the language we are interpreting.
We will attend to this task in the following chapter.


% \section{Formal languages}
%
% English is a natural language, not a formal one. Its rules are vague and messy, and constantly changing. We will spend some time translating between English and our formal languages, but the translations will not always be precise. There is a tension between wanting to capture as much of the structure of English as possible and wanting a simple formal language with tractable rules --- simpler formal languages will approximate natural languages less closely. There is no perfect formal language. Some will do a better job than others in translating particular English-language arguments.
%
% In this book, we make the assumption that \emph{true} and \emph{false} are the only possible truth-values. Logical languages that make this assumption are called \emph{bivalent}, which means \emph{two-valued}. SL and QL are both bivalent, but some philosophers have emphasized limits to the power of bivalent logic. Some logics, beyond the scope of this book, allow for sentences that are neither true nor false. Others allow for sentences that are both true \emph{and} false. Our logical system, which is often called \emph{classical logic}, will give every sentence exactly one truth value: every sentence will be either true or false, and not both.


% \section*{Summary of Terms}
%
%
% \begin{itemize}
%   \item A \define{declarative sentences} in English is a grammatical English sentences that is either true or false, and not both at once.
%   \item A \define{deductive argument} in English is a nonempty sequence of declarative sentences where a single sentence is designated the \define{conclusion} and all other sentences (if any) are referred to as the \define{premises}.
%   % \item A \define{proposition} is a way for things to be.
%   % \item The \define{truth-value} of a declarative sentence is either the value `True' or `False' depending on whether the proposition that sentence expresses obtains.
%   \item The conclusion of an argument is a \define{logical consequence} of its premises just in case the conclusion is true on any interpretation in which all of the premises are true. 
%   \item An argument is \define{logically valid} just in case the conclusion is a logical consequence of its premises.
%   \item A \define{schematic variable} has .
%   \item We may refer to the result as an \define{argument schema}.
%   \item Argument schemata are built up out of two elements: variables (in this case schematic variables) and \define{logical constants}.
%   \item Since `Socrates is a man' is a sentence and doesn't have any parts that are also sentences (we say it is an \define{atomic sentence}), all we can do is replace it with a variable.
%   \item Sentence O1 is true on some interpretations and false on others and so is said to be \define{logically contingent}.
%   \item Accordingly, O2 is a \define{logical truth}, or what is also called a \define{tautology}.
%   \item Accordingly, O2 is a \define{logical truth}, or what is also called a \define{tautology}.
%   \item By contrast, sentence O3 is false on all interpretations, and so is referred to as \define{logically false} or a \define{contradiction}.
%   \item By contrast, sentence O3 is false on all interpretations, and so is referred to as \define{logically false} or a \define{contradiction}.
%   \item Put otherwise, P1 \define{logically entails} P2.
%   \item A set of sentences is \define{satisfiable} just in case there is some interpretation which makes every sentence in the set true, and \define{unsatisfiable} otherwise.
%   \item A set of sentences is \define{satisfiable} just in case there is some interpretation which makes every sentence in the set true, and \define{unsatisfiable} otherwise.
%   \item Accordingly, O2 is a \define{logical truth}, or what is also called a \define{tautology}.
%   \item By contrast, sentence O3 is false on all interpretations, and so is referred to as \define{logically false} or a \define{contradiction}.
%   \item A set of sentences is \define{satisfiable} just in case there is some interpretation which makes every sentence in the set true, and \define{unsatisfiable} otherwise.

% \end{itemize}


\iffalse %moving these to separate file, so as to push to the end!

\practiceproblems
At the end of each chapter, you will find a series of practice problems that review and explore the material covered in the chapter. There is no substitute for actually working through some problems, because logic is more about a way of thinking than it is about memorizing facts. The answers to some of the problems are provided at the end of the book in appendix \ref{app.solutions}; the problems that are solved in the appendix are marked with a \solutions.

\solutions
\problempart
\label{pr.Sentences1}
Which of the following are `sentences' in the logical sense?
\begin{earg}
\item England is smaller than China.
\item Greenland is south of Jerusalem.
\item Is New Jersey east of Wisconsin?
\item The atomic number of helium is 2.
\item The atomic number of helium is $\pi$.
\item I hate overcooked noodles.
\item Blech! Overcooked noodles!
\item Overcooked noodles are disgusting.
\item Take your time.
\item This is the last question.
\end{earg}

\problempart
\label{hw1.B}
Which of the following are `sentences' in the logical sense?
	\begin{earg}
		\item I would like a double cheeseburger with no onions.
		\item Thank you very much for that gracious reception.
		\item If you strike me down, I shall become more powerful than you could possibly imagine.
		\item There are more trees at UBC than there are flowers in my office and my Uncle Jack really seems to like drinking apple juice, or if that's not apple juice, then he really seems to like whatever it is that he's drinking, but anyway, what I'm really trying to say is, I'm hungry and I could really go for a burger or a bag of scorpions right about now.
		\item I did it
		\item No invalid arguments have impossible premises.
	\end{earg}

\problempart
\label{pr.EnglishTautology}
For each of the following: Is it a tautology, a contradiction, or a contingent sentence?
\begin{earg}
\item Caesar crossed the Rubicon.
\item Someone once crossed the Rubicon.
\item No one has ever crossed the Rubicon.
\item If Caesar crossed the Rubicon, then someone has.
\item Even though Caesar crossed the Rubicon, no one has ever crossed the Rubicon.
\item If anyone has ever crossed the Rubicon, it was Caesar.
\end{earg}

% \solutions
% \problempart
% \label{pr.MartianGiraffes}
% Look back at the sentences G1--G4 on p.~\pageref{MartianGiraffes}, and consider each of the following sets of sentences. Which are consistent? Which are inconsistent?
% \begin{earg}
% \item G2, G3, and G4
% \item G1, G3, and G4
% \item G1, G2, and G4
% \item G1, G2, and G3
% \end{earg}

\solutions
\problempart
\label{pr.EnglishCombinations}
Which of the following is possible? If it is possible, give an example. If it is not possible, explain why.
\begin{earg}
\item A valid argument that has one false premise and one true premise.
\item A valid argument that has a false conclusion.
\item A valid argument, the conclusion of which is a contradiction.
\item An invalid argument, the conclusion of which is a tautology.
\item A tautology that is contingent.
\item Two logically equivalent sentences, both of which are tautologies.
\item Two logically equivalent sentences, one of which is a tautology and one of which is contingent.
\item Two logically equivalent sentences that together are an inconsistent set.
\item A consistent set of sentences that contains a contradiction.
\item An inconsistent set of sentences that contains a tautology.
\end{earg}


\problempart
\label{hw1.C}
For each, give an argument with the indicated features, or explain why it is impossible to do so:
	\begin{earg}
		\item Valid, but not sound.
		\item Valid, with a contradictory conclusion.
		\item Sound, with an contradictory premise.
		\item Sound, and an instance of this form:
			\begin{earg}
        \item[] If $\metaA$, then $\metaB$
        \item[] $\metaC$
        \item[\therefore] $\metaB$
			\end{earg}
	\end{earg}


\problempart
\label{pr.ImpossiblePremises}
Is this argument valid? Why or why not?
\begin{earg}
\item[(1)] PHIL 220 is a course with a final exam.
\item[(2)] No course has a final exams.
\item[\therefore] Everyone is going to get an A in PHIL 220.
\end{earg}

\problempart
\label{hw1.A}
For each, indicate whether it is true or false.
	\begin{earg}
		\item All arguments with true premises and true conclusions are sound.
		\item Only valid arguments are sound.
		\item If an argument with the conclusion $A$ is sound, then an argument with the conclusion not $A$ is not sound.
		\item All arguments with at least one contradictory premise are valid.
		\item No invalid arguments have contradictory premises.
	\end{earg}
	
	\fi 
