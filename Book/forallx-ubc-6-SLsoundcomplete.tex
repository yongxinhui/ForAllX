%!TEX root = forallx-ubc.tex
\chapter{Soundness and Completeness for SL Trees}
\label{ch.SLsoundcomplete}

% In Chapter \ref{ch.sl.trees} we introduced a proof system for SL.
Whereas the previous chapter focused on defining and using SL tree proofs, this chapter studies the proof system itself, and so belongs to the discipline \emph{metalogic}.
Although a matter of stipulation, the rules which constrain the construction of SL tree proofs were not selected at random.
Rather, they were designed to determine whether a set of sentences is satisfiable in an entirely syntactic, mechanical, and notationally elegant way.
Even so, a question remains as to whether the tree method succeeds in its ambitions.
% This chapter will show that it does.

Recall the distinction between our two turnstiles.
We use `$\Gamma \models \bot$' to say that $\Gamma$ is unsatisfiable.
This is because if $\Gamma \models \bot$ then every SL interpretation that satisfies $\Gamma$ also satisfies $\bot$.
However, no SL interpretation satisfies $\bot$, and so no SL interpretation satisfies $\Gamma$.
Conversely, if no interpretation satisfies $\Gamma$, then vacuously every interpretation that satisfies $\Gamma$ also satisfies $\bot$, and so $\Gamma \models \bot$.
Whereas satisfaction is a semantic concept, `$\Gamma \proves \bot$' means that a tree with root $\Gamma$ closes where this is an entirely syntactic matter.
In particular, we do not need to know what the symbols in SL mean in order to determine whether there is a closed SL tree with root $\Gamma$.
By itself, the existence of such a closed tree is not illuminating, at least not given what we have shown so far.
Rather, it is the ambition of the present chapter to establish:
  $$\Gamma \proves \bot \text{ just in case } \Gamma \models \bot.$$
Whereas the left-to-right direction is referred to as the \textit{Soundness} of the tree method, the right-to-left direction is referred to as the \textit{Completeness} of the tree method.
Accordingly, the existence of a closed tree proof--- however uninteresting on its own--- tells us something that we already have reason to care about, namely unsatisfiability.
Recall, the connection: % that unsatisfiability bears to entailment: 
    $$\Gamma \models \varphi \text{ just in case } \Gamma, \neg \varphi \models \bot.$$
Putting the pieces together, we may show that $\Gamma \models \varphi$ by showing that $\Gamma, \neg \varphi \proves \bot$, where the former entailment claim is something we have reason to care about.
After all, an argument is logically valid just in case its premises entail its conclusion.
Thus the SL tree proof system provides a method by which to determine which arguments are logically valid.





\section{Informal Proof}

In Chapter \ref{ch.sl.trees} we learned a formal proof system.
In this chapter, we will prove important results \emph{about} that system.
It is important to emphasize, however, that the formal proof system isn't the only way to prove things.
In particular, the tree method is \emph{not} an appropriate methodology for the task of this chapter.
When we say we wish to prove that the tree method provides a way to determine whether arguments are logically valid, we don't mean that we will put the negation of that claim in the root of a tree.
That would get us nowhere.
After all, we have no right to rely on the tree method to prove that the tree method is to be trusted, for such proof procedure has no bottom.
This raises a deep question about where reasoning begins: what can we rely on to get going?
The answer, perhaps surprisingly, is informal proof, where formal proof systems help clarify abstract forms of reasoning.

Thus we will be presenting an \emph{informal proof} about the formal tree proof system.
An informal proof needn't be any less compelling than a formal proof, but evaluating it relies on our ability to recognize what follows from what rather than using a list of syntactic rules.





\section{Soundness}

If a tree with root $\Gamma$ closes, we may conclude that $\Gamma$ is unsatisfiable.
If our system is \textit{sound}, then a tree will never mislead us in this respect.
Put flatly, tree closure should guarantee the unsatisfiability of the root.
The soundness of any logic is the first important metalogical result to establish since a logic that is not sound is not worth anything at all.
For instance, think of a calculator that sometimes gives you the wrong results.
Such a calculator cannot be trusted, and so shouldn't be considered a calculator at all.

\label{definesound}
\factoidbox{
\define{soundness}: If a tree closes, that guarantees that its root is unsatisfiable, i.e., if $\Gamma\proves\bot$, then $\Gamma\models\bot$.
It follows as consequence that: if $\Gamma \proves \metaA$, then $\Gamma \models \metaA$. 
}

Here is a way to illustrate that soundness is a substantive result, and to clarify what it is that we are trying to prove.
Recall the resolution rule for disjunction (see p.\ \pageref{subsec.DisjunctionTreeRule}):

\begin{center}
\begin{prooftree}
{not line numbering,
single branches}
[\metaA{}\eor\metaB{}
	[\metaA{}]
	[\metaB{}]
]
\end{prooftree}
\end{center}

Let's suppose for the sake of argument that we had a different disjunction rule instead of the presented above.
For example, suppose that our rule for disjunction had been this:

\label{unsoundrule}
\begin{center}
\begin{prooftree}
{not line numbering,
single branches}
[\metaA{}\eor\metaB{}
	[\metaA{}
	[\metaB{}, grouped]]
]
\end{prooftree}
\end{center}

If this had been our disjunction rule, our tree system would have been unsound even if all the other rules remained the same.
This is because there would have been satisfiable roots that have closed trees.
For example, consider the argument:

\begin{earg}
	\item[] $P$
	\item[] $\enot P \eor Q$
\end{earg}

This root is easy to satisfy: let $\I$ be an SL interpretation where $\I(P)=\I(Q)=1$.
However, if we used the tree system with the alternative disjunction rule above, the tree would close:

\begin{prooftree}
{
}
[P, name={P1}
	[\enot P \eor Q, grouped, checked, name={disj}
		[\enot P, just={alt.\eor:disj}
			[Q, grouped, close={:P1, !u}]
		]
	]
]		
\end{prooftree}

This would be a counterexample to the soundness of our tweaked SL tree system.
What's wrong with the alternative disjunction rule is that it allows trees to ``prove'' that a root is unsatisfiable, even if it really is satisfiable.
In considering the soundness of our system, we are investigating whether our actual proof system is defective in the same way as this hypothetical modification of the system would have been.
We will prove that it is not.




\section{Inductive Proofs}

Note that the claim that we are attempting to establish holds for all sets of SL sentences $\Gamma$.
Since we cannot check all such sets, one might attempt to proceed by \textit{reductio ad absurdum}, assuming that there is a counterexample and producing a contradiction.
This would be to assume that there is some $\Gamma$ as above where $\Gamma \proves \bot$ and yet $\Gamma \nvDash \bot$.
However, just because we are assuming that there is some closed tree with root $\Gamma$, we can't conclude much more than that, and so it is hard to see how we might produce a contradiction with $\Gamma \nvDash \bot$.

Our predicament is a common one: we are attempting to show something about an infinite number of cases and it is unclear how to do so by assuming there are any exceptions.
In order to overcome this challenge we will employ \emph{mathematical induction} which proceeds in stages.

\textit{Stage 1:} 
We must identify a domain of objects and the property which we are attempting to show is had by every object in that domain.
In the case of soundness, we want to show that every tree with a satisfiable root $\Gamma$ has a satisfiable branch.
Note that this property is a little different than what one might have expected it to be.
In particular, one might have thought that we were going to show that every tree that closes has an unsatisfiable root.
As we will see, showing that all trees have the former property is sufficient to show that all trees have the latter property, though the former makes for an easier induction proof.
This brings to light one of the trickiest part of an induction proof: not only is it important to accurately identify the domain of interest, the property in question must also be carefully chosen.

\textit{Stage 2:} 
We must now provide some way of organizing the domain into a sequence of stages.
For instance, if our domain was the set of natural numbers, we might consider their natural ordering where every number in the sequence is followed by its successor.
In the case of the trees of SL sentences, there is no similarly natural ordering already give to us.
The second stage is to come up with such an ordering, making sure that the ordering serves our needs without including any unnecessary complications.
Instead of directly ordering the trees themselves, we will partition the domain of all trees into stages that can be ordered into a sequence.
In particular, we will take two trees to belong to the same stage just in case they employ the same number of resolution rules.
This way of organizing the domain of trees (or proofs more generally) is called \textit{length of proof}, or in our case, \textit{length of trees}.
So we will have the stage of trees with length zero, the stage with length one, and so on.

\textit{Stage 3:} 
Next we will establish that the first stage (the set of trees with length zero) has the property in question.
That is, every tree of length zero with a satisfiable root has a satisfiable branch.
This step is often called the \textit{base case} of our induction proof.
Although the base case is often the easiest part of an induction proof--- sometimes so obvious it is hard to know what to write--- this is not always the case, and so should still be carefully considered.

\textit{Stage 4:} 
Now we will help ourselves to an important assumption called the \textit{induction hypothesis}.
This assumption can come in both weak and strong varieties.
In particular, we could assume for \textit{weak induction} that the property in question holds for the $n$-th stage, or we could assume for \textit{strong induction} that the property in question holds for the $n$-th stage and all previous stages.
Although weak induction is sometimes sufficient for a proof, there is typically no reason not to help ourselves to the stronger assumption.
In the case of soundness, weak induction will suffice, though it costs us nothing to make the stronger assumption that every tree with a satisfiable root that is length $n$ or shorter has a satisfiable branch. 

\textit{Stage 5:} 
We will complete the induction proof by showing that the property in question also holds for the $n+1$-th stage.
If we can establish this claim, then it follows that the property in question holds for every stage.
After all, we have shown that the property in question holds for the first stage, and that if property holds up through the $n$-th stage, then it holds for the $n+1$-th stage.
It is common for the property that we are concerned with to have a conditional form such as: if $x$ is $F$, then $x$ is $G$.
In the case of soundness, we want to show that if a tree has a satisfiable root, then that tree has a satisfiable branch.
Moreover, we may want to show that all objects belonging to a given stage have this property, e.g., all trees that are of length $n+1$ have the conditional property: if they have a satisfiable root, then they have a satisfiable branch. 
This is a common predicament.

In order to establish a general claim of the form `Everything in $D$ that is $F$ is also $G$', there is a tried a true method: choose an \textit{arbitrary} $F$ and show that it is $G$.
In the case of soundness, we will choose an arbitrary tree of length $n+1$ and assume that it has a satisfiable root.
By drawing on the induction hypothesis, we will then endeavour to show that this arbitrary tree has a satisfiable branch. 
This step typically constitutes the core of the induction proof.
Since our choice was arbitrary, we may conclude what we want: every tree of length $n+1$ with a satisfiable root has a satisfiable branch. 
Having shown that our conditional property applies to all trees of length $n+1$, we may conclude what we want to show by induction: every tree (of any length) with a satisfiable root has a satisfiable branch.
Accordingly, we may leave mention of the length of the tree off entirely.

Although the induction part of the proof is finished, it remains to establish soundness.
As we will see, soundness follows easily from what we have shown, i.e., that every tree with a satisfiable root has a satisfiable branch.
Drawing this final connection completes the proof.

This provides the rough outline of a proof by induction argument, with some reference to the proof for soundness.
In actual practice, the hardest part about induction proofs is staying organised and figuring out which properties to focus on, since sometimes you can make things a lot easier by proving something related to what you really want to show.





\section{Tree Length}

Before attempting to establish soundness, it will be important to provide a precise definition of the length of a proof.
This definition will take a recursive form.

Consider any root consisting of SL sentences.
Since no resolution rules have been applied, we will refer to such a tree as having length 0.
Even if the root includes many sentences, we are only measuring the number of applications of resolution rules, and a root on its own doesn't include any applications of resolution rules.
Now consider any tree of length $n$, i.e., any tree which results from $n$ applications of the resolution rules. 
We may refer to the result of applying any application rule on any one branch as a tree of length $n+1$.
A little more precisely, we will define $\length$ to be the smallest function satisfying:%
\begin{enumerate}[leftmargin=1in,labelsep=.15in]
  \item[\it Base:] $\length(X)=0$ for any root $X$.
  \item[\it Recursive:] For any tree $X$, if $\length(X)=n$ and $X'$ is the result of resolving a sentence in exactly one branch in $X$, then $\length(X')=n+1$.
\end{enumerate}
Even if the same sentence can be resolved on different branches, each time we resolve that sentence on a branch we get a tree with greater length.




% \subsection{Complexity}
%
% Whereas $\length$ applies to SL trees, it will also help to have a measure of the complexity of SL sentences.
% Intuitively, the complexity of an SL sentence is the number of connectives that it contains.
% However, we will often be quantifying over all SL sentences, and so will be using metavariables like $\metaA, \metaB,$ etc., to do so.
% Whereas we might be able to count the connectives in particular SL sentence, we cannot count the number of connectives in some arbitrary SL sentence $\metaA$. 
% Instead, we will define a function $\comp$ which provides the complexity of any SL sentence whatsoever.
% In order to conform to the recursive definition of SL sentences, we will define complexity recursively where $\comp$ is the smallest function to satisfy: 
% \begin{enumerate}[leftmargin=1in,labelsep=.15in]
%   \item[\it Base:] $\comp(\metaA)=0$ for any sentence letter $\metaA$ of SL. 
%   \item[\it Recursive:] For any sentences $\metaA$ and $\metaB$ of SL and connective $\star\in\set{\eand,\eor,\eif,\eiff}$:
%     \begin{itemize}
%       \item[($\enot$)] $\comp(\enot\metaA)=\comp(\metaA)+1$
%       \item[(\hspace{1.5pt}$\star$\hspace{1.5pt})] $\comp(\metaA \star \metaB)=\comp(\metaA)+\comp(\metaB)+1$.
%     \end{itemize}
% \end{enumerate}
% Instead of writing our clauses for each binary connective, we have used the metavariable $\star$ to range over the binary connectives included in SL. 
% Accordingly, we know that $\comp(\metaA \eand \metaB)=\comp(\metaA)+\comp(\metaB)+1$ for $\star=\;\eand$, and similarly for the other binary connectives.




\section{Supporting Lemmas}

Although we could dive into proving soundness, it will streamline our proof to first establish some supporting results which we can draw on in the course of our proof of soundness.
It is typical to call supporting results such as these \textit{lemmas} in order to flag their instrumental interest.
Were a lemma to have significant and far reaching consequences that are of interest in their own right, we would do better to call it a \textit{proposition} or even a \textit{theorem}.
For instance, it would be inappropriate to refer to soundness as a lemma given the significance of this result.
However, in the case of the claims established below, we are just looking to better organise our proof, presenting some minor results to help us to do so.
Knowing when to carve off a lemma to establish separately from the proof of primary interest is a skill in its own right, one that takes lots of practice writing proofs to cultivate.

Before presenting these results, let a branch $\B$ of a tree $X$ be \define{satisfied} by an SL interpretation $\I$ just in case $\V{\I}(\metaA)=1$ for every SL sentence $\metaA$ which occurs in $\B$.
It will also be convenient to refer to a tree whose root consists of the sentences in $\Gamma$ as a tree \define{with} root $\Gamma$.



\subsection{Open Branches}

In order to show that closed branches are unsatisfiable, we will establish the following lemma:

\label{OpenBranchLemma}
\factoidbox{
\textbf{Lemma 1:} Every satisfiable branch $\B$ in an SL tree $X$ is open. 
}

Let $X$ be an SL tree with a satisfiable branch $\B$.
It follows that $\V{\I}(\metaA)=1$ for every sentence $\metaA$ in $\B$.
Assume for contradiction that $\B$ includes both $\metaA$ and $\enot \metaA$ for some SL sentence $\metaA$.
Thus $\V{\I}(\metaA)=1$ and $\V{\I}(\enot\metaA)=1$, and so $\V{\I}(\metaA)=0$ by the semantic clause for negation.
Having arrived at a contradiction, we may conclude that no such pair of sentences $\metaA$ and $\enot \metaA$ occur in $\B$, and so $\B$ is open. 






\subsection{Satisfiable Branches}

Given an SL tree $X$ with a satisfiable branch $\B$, we may show that any tree $X'$ which is the result of resolving a sentence in $\B$ will also have a satisfiable branch $\B'$.
Put simply, the resolution rules preserve the existence of satisfiable branches.
% It is worth observing that in the case where $\B$ is complete, this claim holds vacuously since there are no trees $X'$ which result from resolving a sentence in $\B$. 

\label{SatisfiableBranchLemma}
\factoidbox{
\textbf{Lemma 2:} If $X$ is an SL tree with a satisfiable branch $\B$, then any tree $X'$ which is the result of resolving a sentence in $\B$ has a satisfiable branch $\B'$. 
}

Assume to start that $X$ is an SL tree with a satisfiable branch $\B$. 
It follows that there is some SL interpretation $\I$ where $\V{\I}(\metaA)=1$ for every SL sentence $\metaA$ that occurs in $\B$. 
By \textbf{Lemma 2}, $\B$ is open.
If $\B$ is complete, then it follows vacuously that every SL tree to result from resolving a sentence in $\B$ has a satisfiable branch $\B'$ since there are no such SL trees. 
Thus we need only consider the non-vacuous case where $\B$ is not complete on account of including an SL sentence that is resolvable in $\B$.

Assume that $\B$ is not complete and that $X'$ is an SL tree that results from resolving a sentence in $\B$. 
There are nine ways that this can happen corresponding to our nine resolution rules.
What we need to check for each case is that at least one of the branches that extend $\B$ is satisfiable given that $\B$ is satisfiable. 
We will attend to this now.


\subsubsection{Conjunction}

Suppose that $\B$ includes a resolvable SL sentence of the form $\metaA \eand \metaB$, and that $X'$ is the result of resolving that conjunction in $\B$.
We will represent this situation as follows:

\begin{center}
\begin{prooftree}
{not line numbering,
}
[\metaA{}\eand\metaB{}
	[\metaA{}
	[\metaB{}, grouped
	]
	]
]
\end{prooftree}
\end{center}

As assumed above, $\I$ satisfies $\B$, and so in particular, $\V{\I}(\metaA\eand\metaB)=1$.
By the semantic clause for conjunction, both $\V{\I}(\metaA)=1$ and $\V{\I}(\metaB)=1$.
It follows that $X'$ has a satisfiable branch $\B'$ which, in this case, is the result of appending $\metaA$ and $\metaB$ to the end of $\B$. 

Before moving on, it is worth reflecting on what we have just shown.
In particular, if $\I$ satisfies a branch above the resolution of a conjunction in that branch, then it also satisfies at least one branch (in this case the only branch) below that resolution.
Although we have not attempted to represent all of what may be above the resolution of the conjunction in question, or where the conjunction being resolved may occur in the tree, this much does not matter.
What counts is that the diagram above suggests that $\metaA \eand \metaB$ occurs somewhere in $\B$ and that its resolution rule has been applied at the bottom of $\B$. 




\subsubsection{Negated Conjunction}

Negated conjunctions are resolved with the branching rule:

\begin{center}
\begin{prooftree}
{not line numbering,
single branches}
[\enot(\metaA{}\eand\metaB{})
	[\enot\metaA{}]
	[\enot\metaB{}]
]
\end{prooftree}
\end{center}

Since $\I$ satisfies $\B$, we know that $\V{\I}(\enot(\metaA\eand\metaB))=1$.
By the semantic clauses for negation, $\V{\I}(\metaA\eand\metaB)=0$, and so either $\V{\I}(\metaA)=0$ or $\V{\I}(\metaB)=0$ by the semantic clause for conjunction.
Thus either $\V{\I}(\enot \metaA)=1$ or $\V{\I}(\enot \metaB)=1$.
It follows that at least one branch below the resolution of $\enot(\metaA \eand \metaB)$ in $\B$ is satisfied by $\I$.
Even though we can't say which branch is satisfied by $\I$ we may conclude that $X'$ has a satisfiable branch. 

Stepping back, if $\I$ satisfies a branch $\B$ above the resolution of a negated conjunction in $\B$, then it also satisfies at least one branch below that resolution.





\subsubsection{Disjunction}

Disjunctions are resolved according to the rule:

\begin{center}
\begin{prooftree}
{not line numbering}
[\metaA{}\eor\metaB{}
	[\metaA{}]
	[\metaB{}]
]
\end{prooftree}
\end{center}

Since $\I$ satisfies $\B$, we know that $\V{\I}(\metaA \eor \metaB)=1$, and so either $\V{\I}(\metaA)=1$ or $\V{\I}(\metaB)=1$ by the semantic clause for disjunction.
Thus $\I$ satisfies at least one branch below the resolution of $\metaA \eor \metaB$ in $\B$, and so $X'$ is guaranteed to have a satisfiable branch as desired.

Hopefully the pattern is starting to become clear.
For each of the rules considered so far, we have shown that their resolution on a satisfiable branch leads to at least one satisfiable branch.
% Six resolution rules remain to be considered.
For contrast, consider the alternative disjunction rule mentioned above:

\begin{center}
\begin{prooftree}
{not line numbering}
[\metaA{}\eor\metaB{}
	[\metaA{}
	[\metaB{}, grouped]]
]
\end{prooftree}
\end{center}

If we attempted to go through the same reasoning we've been going through, we'd fail.
Although we know that $\V{\I}(\metaA\eor\metaB)=1$ given our assumptions, and that either $\V{\I}(\metaA)=1$ or $\V{\I}(\metaB)=1$ by the semantics for disjunction, there is no guarantee that both $\V{\I}(\metaA)=1$ and $\V{\I}(\metaB)=1$.
Accordingly, we cannot be sure that the alternative resolution rule for disjunction will always generate a satisfiable branch.
Instead, this alternative rule could lead to inappropriate tree closure. \label{soundprooffailure}
It is this kind of inappropriate tree closure that our official resolution rules avoid, where the proofs above and below show that this is the case.





\subsubsection{Negated Disjunction}

Here is the rule for negated disjunctions:

\begin{center}
\begin{prooftree}
{not line numbering,
single branches}
[\enot(\metaA{}\eor\metaB{})
	[\enot\metaA{}
	[\enot\metaB{}, grouped
	]
	]
]
\end{prooftree}
\end{center}

Since $\V{\I}(\enot(\metaA{}\eor\metaB{}))=1$, we know by the semantic clauses that $\V{\I}(\metaA\eor\metaB)=0$, and so both $\V{\I}(\metaA)=0$ and $\V{\I}(\metaB)=0$.
Thus $\V{\I}(\enot \metaA)=1$ and $\V{\I}(\enot \metaB)=1$, and so $\I$ satisfies a branch below the resolution of $\enot (\metaA \eor \metaB)$ in $\B$.
We may conclude that $X'$ is guaranteed to have a satisfiable branch, and so the resolution rule for negated disjunctions preserves satisfiability.





\subsubsection{Conditional}

The rule for the material conditional is:

\begin{center}
\begin{prooftree}
{not line numbering,
single branches}
[\metaA{}\eif\metaB{}
	[\enot\metaA{}]
	[\metaB{}]
]
\end{prooftree}
\end{center}

Since $\V{\I}(\metaA \eif \metaB)=1$, either $\V{\I}(\metaA)=0$ or $\V{\I}(\metaB)=1$ by the semantic clause for the material conditional, and so either $\V{\I}(\enot \metaA)=1$ or $\V{\I}(\metaB)=1$.
Thus $\I$ satisfies at least one branch below the resolution of $\metaA \eif \metaB$ in $\B$, and so $X'$ is guaranteed to have a satisfiable branch.




\subsubsection{Negated Conditional}

The rule for negated conditionals is:

\begin{center}
\begin{prooftree}
{not line numbering,
single branches}
[\enot(\metaA{}\eif\metaB{})
	[\metaA{}
	[\enot\metaB{}, grouped
	]
	]
]
\end{prooftree}
\end{center}

Since $\V{\I}(\enot(\metaA{}\eif\metaB{}))=1$, we know by the semantics that $\V{\I}(\metaA\eif\metaB)=0$, and so $\V{\I}(\metaA)=1$ and $\V{\I}(\metaB)=0$.
Thus $\V{\I}(\metaA)=1$ and $\V{\I}(\enot \metaB)=1$, and so $\I$ satisfies a branch below the resolution of $\enot (\metaA \eif \metaB)$ in $\B$.
As before, $X'$ is guaranteed to have a satisfiable branch.




\subsubsection{Biconditional}

Here is the biconditional rule:

\begin{center}
\begin{prooftree}
{not line numbering,
single branches}
[\metaA{}\eiff\metaB{}
	[\metaA{}
		[\metaB{}, grouped]
	]
	[\enot\metaA{}
		[\enot\metaB{}, grouped]
	]
]
\end{prooftree}
\end{center}


Since $\V{\I}(\metaA \eiff \metaB)=1$, we know be the semantic clause for the biconditional that either $\V{\I}(\metaA)=1$ and $\V{\I}(\metaB)=1$, or $\V{\I}(\metaA)=0$ and $\V{\I}(\metaB)=0$.
In the latter case, it follows that $\V{\I}(\enot \metaA)=1$ and $\V{\I}(\enot \metaB)=1$.
Thus $\I$ satisfies at least one branch below the resolution of $\metaA \eiff \metaB$ in $\B$, and so $X'$ is guaranteed to have a satisfiable branch.





\subsubsection{Negated Biconditional}

The rule for negated biconditionals is:

\begin{center}
\begin{prooftree}
{not line numbering,
single branches}
[\enot(\metaA{}\eiff\metaB{})
	[\metaA{}
		[\enot\metaB{}, grouped]
	]
	[\enot\metaA{}
		[\metaB{}, grouped]
	]
]
\end{prooftree}
\end{center}

Since $\V{\I}(\enot (\metaA \eiff \metaB))=1$, we know be the semantic clauses that either $\V{\I}(\metaA)=1$ and $\V{\I}(\metaB)=0$, or $\V{\I}(\metaA)=0$ and $\V{\I}(\metaB)=1$.
Accordingly, either $\V{\I}(\metaA)=1$ and $\V{\I}(\enot \metaB)=1$, or $\V{\I}(\enot \metaA)=1$ and $\V{\I}(\metaB)=1$.
Thus $\I$ satisfies at least one branch below the resolution of $\enot (\metaA \eiff \metaB)$ in $\B$, and so $X'$ is guaranteed to have a satisfiable branch.





\subsubsection{Double negation}

Here is our final resolution rule for double negation:

\begin{center}
\begin{prooftree}
{not line numbering, single branches}
[\enot\enot\metaA{}
	[\metaA{}]
]
\end{prooftree}
\end{center}

Since $\V{\I}(\enot \enot \metaA{})=1$, we know by the semantics that $\V{\I}(\enot \metaA)=0$, and so $\V{\I}(\metaA)=1$.
Thus $\I$ satisfies a branch below the resolution of $\enot \enot \metaA$ in $\B$, and so $X'$ has a satisfiable branch.


\subsubsection{Conclusion}

Having established that $X'$ has a satisfiable branch in each of the cases above, we may conclude more generally that any tree $X'$ which is the result of resolving a sentence in $\B$ has a satisfiable branch $\B'$.
Of course, this does not hold for any $X'$ but rather depends on our initial assumption that $X$ is an SL tree with a satisfiable branch $\B$.
By discharging that initial assumption, we may conclude what we aimed to show: if $X$ is an SL tree with a satisfiable branch $\B$, then any tree $X'$ which is the result of resolving a sentence in $\B$ has a satisfiable branch $\B'$.  
Although drawing out this final conclusion by discharging the initial assumption is often left implicit, these details make the mechanics of the proof easy to see.





\subsection{Satisfiable Roots}

By drawing on the previous lemmas, we will now turn to prove something stronger than soundness, where soundness will follow easily as a result.

\label{SatisfiableRootLemma}
\factoidbox{
\textbf{Lemma 3:} Every SL tree with a satisfiable root has a satisfiable branch.
}

Assume $\Gamma$ is satisfiable, i.e., $\Gamma \nmodels \bot$.
The proof will proceed by induction on the length of trees, showing that every tree with root $\Gamma$ has a satisfiable branch.
To begin with, we will show that any tree with root $\Gamma$ of length $0$ has a satisfiable branch.




\subsubsection{Base Case}

Let $X$ be any tree of length $0$ with root $\Gamma$.
Given the assumption that $\Gamma \nmodels \bot$, we know that there is an interpretation $\I$ that satisfies $\Gamma$.
Since the root is the only branch in the tree, it follows that $\I$ satisfies the root, and so $X$ has a satisfiable branch. 





\subsubsection{Induction Hypothesis}

We want to prove that every tree with root $\Gamma$ has a satisfiable branch.
To do so, we will \textit{assume for weak induction} that every tree of length $n$ with root $\Gamma$ has a satisfiable branch. 
Put otherwise, for every tree $X$ with root $\Gamma$, if $\length(X) = n$, then $X$ has a satisfiable branch $\B$. 
Note that in writing induction proofs for yourself, it is important to make the induction hypothesis just as clear, flagging it by writing `assume for induction' or similar.



\subsubsection{Induction Step}

Given our induction hypothesis, we are ready to complete the induction step, showing that every tree $X'$ with root $\Gamma$ and $\length(X')=n+1$ has a satisfiable branch.
To do so, we will assume that $X'$ is an arbitrary tree with root $\Gamma$ and $\length(X')=n+1$.
By showing that $X'$ has a satisfiable branch, we may generalize on $X'$ at the end of the proof.

Let $X$ be any tree with root $\Gamma$ where $X'$ is the result of resolving a sentence $\metaA$ in a branch $\B$ of $X$.
Put otherwise, $X$ can be obtained by taking any branch of $X'$ and undoing the last resolution rule to be applied in that branch. 
Since $\length(X')=n+1$, we know that $\length(X)=n$, and so \textit{by hypothesis} $X$ has a satisfiable branch $\B^\star$. 
Note that in writing induction proofs, it is important to make use of the induction hypothesis perfectly clear.

Since either $\B^\star=\B$ or not, there are two cases to consider.
We will consider these in order, where you should make the consideration of different cases just as clear in your proofs.

\textit{Case 1:} 
  Assume $\B^\star=\B$.
  Since $\B^\star$ is satisfiable and $X'$ is the result of resolving $\metaA$ in $\B\ (=\B^\star)$, we know by \textbf{Lemma 2} that $X'$ has a satisfiable branch $\B'$. 

\textit{Case 2:} 
  Assume $\B^\star\neq\B$.
  Thus $X'$ is the result of developing $X$ by resolving a sentence in a different branch than $\B$.
  Since $\B^\star$ is satisfiable and occurs in $X$, we know that $\B^\star$ also occurs in $X'$, and so $X'$ has a satisfiable branch $\B'\ (=\B^\star)$.
  
Since the cases above are exhaustive, covering all possibilities, and $X'$ has been shown to have a satisfiable branch in both of cases, we may conclude that $X'$ has a satisfiable branch independent of any condition. 
Given that $X'$ was \textit{any} tree with root $\Gamma$ and $\length(X')=n+1$, we may conclude that every tree with root $\Gamma$ and $\length(X')=n+1$ has a satisfiable branch. 



\subsubsection{Conclusion}

Having completed the induction step, we may conclude by \textit{weak induction} that every tree with root $\Gamma$ of any length has a satisfiable branch $\B$. 


\section{Proving Soundness}

Given the lemmas above, we are now in a position to establish the soundness of SL tree proofs.
Recall that soundness is the claim that any time a tree closes, the root must be unsatisfiable: if $\Gamma \proves \bot$, then $\Gamma \models \bot$.
This is equivalent to the contrapositive claim that any time the root \emph{is} satisfiable, the tree \emph{won't} close: if $\Gamma \nmodels \bot$, then $\Gamma \nproves \bot$.
Note that this is intended to hold for any $\Gamma$. 
To prove a universal conditional claim of this form, we will assume that the antecedent holds for an arbitrary $\Gamma$, using the lemmas above to conclude that there is no closed tree with root $\Gamma$, i.e., $\Gamma \nproves \bot$. 
To do so, stating `Assume $\Gamma \nmodels \bot$' will be enough.

% To prove this claim, we will assume that the root $\Gamma$ is satisfiable, drawing on \textbf{Lemma 3} to show that every tree which results from applying a resolution rule has a satisfiable branch.
% In continuing to resolve sentences, we know that the existence of a satisfiable branch is always preserved.
% Thus every tree with root $\Gamma$ has a satisfiable branch which, by \textbf{Lemma 2} is open, and so the tree is not closed.
% As a result, there is no tree with root $\Gamma$ that is closed, and so $\Gamma \nproves \bot$ as desired. 

% Suppose that we have some satisfiable set of sentences $\Gamma$ in the root of a tree $X$ in which no resolution rules have been applied.
% Thus $X$ has a satisfiable branch $\B$, i.e., the root.
% By \textbf{Lemma 3}, every tree $X'$ which is the result of resolving a sentence in $\B$ has a satisfiable branch $\B'$. 
% % Once we establish that claim, it's only a short step to demonstrate that this branch doesn't close.
% % , and so there is an open tree with root $\Gamma$
% By \textbf{Lemma 2}, $\B'$ is open, and so the tree $X'$ is open.
% Since continuing to resolve sentences will only lead to further open trees, there is no way to arrive at a closed tree, completing the proof.

% This is the broad outline of our soundness proof.
% It remains to tighten up the details.
% To do so, we will make use of mathematical induction on the length of SL trees.





\subsection{The Proof}

Assume $\Gamma \nmodels \bot$.
Thus $\Gamma$ is satisfiable. 
Let $X$ be an SL tree with root $\Gamma$.
By \textit{Lemma 3}, $X$ has a satisfiable branch $\B$.
By \textit{Lemma 1}, $\B$ is open, and so $X$ is not closed. 
Generalizing on $X$, we may conclude that there is no closed SL tree with root $\Gamma$, and so $\Gamma \nproves \bot$.
Although we do not know that $\Gamma \nproves \bot$ for any $\Gamma$, we do know that $\Gamma \nproves \bot$ \textit{if} $\Gamma \nmodels \bot$ since this was are starting assumption. 
Discharging our assumption we may conclude in perfect generality that: if $\Gamma \nmodels \bot$, then $\Gamma \nproves \bot$.
It follows by contraposition that if $\Gamma \proves \bot$, then $\Gamma \nmodels \bot$.











\label{sec.sl.soundnessproof.end}
\subsection{Taking Stock}

Recall that \textbf{Lemma 2} showed that the resolution rules in the SL tree system are such that if you begin with a tree with a satisfiable branch, applying the rule will always result in a tree that also has a satisfiable branch.
Since these rules are the only ways that one can develop a tree, \textit{Lemma 3} showed by induction on the length of trees that there is no way for a tree with a satisfiable root to develop into a tree without any satisfiable branches.

Branches are closed just in case they contain a sentence and its negation.
Since \textbf{Lemma 1} showed that a satisfiable branch cannot include a sentence and its negation, no satisfiable branch is closed.
Assuming that we start with a satisfiable root, the rules will never result in a tree in which every branch is closed.
Put simply, satisfiable roots never grow into closed trees.
Equivalently, tree closure guarantees the unsatisfiability of the root.
Thus we may conclude that the SL tree proof system is sound, and so can be relied upon to establish the unsatisfiability of a set of sentences $\Gamma$: whenever a tree closes, we know its root is unsatisfiable.

Recall that $\Gamma \models \metaA$ just in case $\Gamma, \enot \metaA \models \bot$, i.e, $\Gamma \cup \set{\enot \metaA}$ is unsatisfiable.
Accordingly, if we want to check whether $\Gamma \models \metaA$, we may use the SL tree proof system to check whether $\Gamma, \enot \metaA \proves \bot$, and if so, conclude by soundness that $\Gamma, \enot \metaA \models \bot$, and so $\Gamma \models \metaA$.

Putting the pieces together, suppose that we have some argument in English that we would like to evaluate for validity.
If we can regiment that argument in SL and show that the premises together with the negation of the conclusion develop into a closed tree, we may conclude that the premises entail the conclusion, and so the argument is valid.

Although soundness tells us that \textit{if} a tree closes, then its root is unsatisfiable, it does not tell us the converse.
Give what we have shown, it remains possible that there could be some unsatisfiable root which does not develop into a closed tree.
Even though we know we can rely on our SL tree proof system when it works, how do we know it will produce a closed tree for any unsatisfiable set?
This question can be restated as: how do we know that the SL tree proof system is \textit{complete}.
The following section will attend to this question, demonstrating that the SL tree proof system is indeed complete, and so can be deployed in confidence, determining the satisfiability of any set of SL sentences whatsoever.



\section{Completeness}

Soundness is the first of two important metalogical theorems considered in this chapter.
The second is completeness.
One can think of soundness as a guarantee against a system proving \emph{too much}; in proving soundness, we were assuring ourselves that a tree would close \emph{only if} the root was unsatisfiable.
Completeness, as the name suggests, concerns whether our system proves \emph{enough}.
We want our system to be sure to close, if the root is unsatisfiable.

\label{definecomplete}
\factoidbox{
\define{completeness}: Every unsatisfiable root has a closed tree: $\Gamma\models{}\bot\Rightarrow\Gamma\vdash{}\bot$. It follows as a consequence that: if $\Gamma \entails \metaA$, then $\Gamma \proves \metaA$.
}

Consider the unsatisfiable set of sentences $\{\enot Q, P \eand Q\}$ and resulting tree:

\begin{prooftree}
{
}
[\enot Q, name={q}
[P \eand Q, checked, grouped
	[P, just={\eand: !u}
	[Q, grouped, close={:q, !c}
	]
	]
]
]
\end{prooftree}

In proving completeness, we wish to demonstrate that this will always be the case: \emph{whenever} we begin with an unsatisfiable root, the entire tree will eventually close.
Notice that if we had a different conjunction rule that called for a branching development instead of a linear one, completeness would fail.
Suppose we had this rule:

\begin{center}
\begin{prooftree}
{not line numbering,
single branches}
[\metaA{}\eand\metaB{}
	[\metaA{}]
	[\metaB{}]
]
\end{prooftree}
\end{center}

Using this rule, the tree with root $\{\enot Q, P \eand Q\}$ would remain open:




\begin{prooftree}
{
}
[\enot Q, name={q}
[P \eand Q, checked, grouped
	[P, just={alt.\ \eand: !u}, open
	]
	[Q, close={:q, !c}
	]
]
]
\end{prooftree}

The right branch closes, but this tree has a left branch that remains open, even though the root is unsatisfiable.
So if we modified our proof system by using this rule instead of the linear rule for conjunction, we would have a system that fails to be complete.
We wish to prove that, given the actual rules, our system is complete.




\section{Proving completeness}
\label{sec.completenessproof}

Completeness is the claim that any time a set of SL sentences $\Gamma$ is unsatisfiable, there is a closed tree with root $\Gamma$.
This is equivalent to the contrapositive claim that if there is no closed tree with root $\Gamma$, then $\Gamma$ is satisfiable.
Thus to prove completeness, we will take $\Gamma$ to be any set of SL sentences and assume that there is no closed tree with root $\Gamma$, showing on this assumption that $\Gamma$ is satisfiable.

As with soundness, we will provide an \emph{informal} proof of completeness.
In order to stay organized, the proof will be divided into a number of supporting lemmas.
In particular, \textbf{Lemma 6} will provide the first substantial result, showing that for every SL tree, there is a complete SL tree with the same root.
Since roots are trees of length zero, it follows that every root has a complete tree.
Given the starting assumption that there is no closed tree with root $\Gamma$, it follows that there is some complete tree with root $\Gamma$ with an open branch $\B$ where this result is entirely syntactic in nature.

In order to move from syntax to semantics, \textbf{Lemma 7} shows that \emph{every} complete open branch is satisfiable.
In particular, we will provide the following recipe for constructing an interpretation $\I$ that satisfies $\B$:

\begin{enumerate}
  \item If any atomic sentence $\metaA$ is in $\B$, then $\I(\metaA)=1$.
  \item If any negated atomic sentence $\enot\metaB$ is $\B$, then $\I(\metaB)=0$.
  \item Let $\I(\metaC)=1$ for every sentence letter $\metaC$ that does not occur in $\B$.
\end{enumerate}

Since it doesn't matter what truth-value the sentence letters that do not occur in $\B$ have, we make the arbitrary choice of setting them all equal to $1$ in (3). 
Rather, it is (1) and (2) which count.
Given this recipe, $\I$ satisfies every literal in $\B$.
It remains to show that $\I$ also satisfies every other sentence in $\B$, where this constitutes the core of the completeness proof.
%% TODO: CONTINUE
Here we will exploit the recursive definition of the sentences in SL.
Since there is a fixed number of ways that sentences can be created from simpler sentences, we need only check that for every sentence form, if $\I$ satisfies a branch downstream of a sentence with that form in a completed branch, then $\I$ also satisfies that sentence.
Whereas the soundness proof reasoned from the top of a branch down, the completeness proof will reason from the bottom up.

This provides the broad outline of the proof.
As with the proof soundness, it will help to establish some preliminary lemmas in order to streamline the proof of completeness.
This will make it easier to keep track of what is going on by splitting the proof into smaller, more manageable pieces.
We will attend to these details in the following section.


Accordingly, we may conclude that the branch $\B$ is satisfiable.
Since every branch in a tree includes the root of that tree, we may conclude that $\Gamma$ is satisfiable as desired. 

We can then construct an interpretation $\I$ based on $\B$, as follows: 


\subsection{Lemma 4}

Consider an SL tree $X$ with root $\Gamma$ where $\metaA$ is resolvable in some branch of $X$. 
However, $\metaA$ may be resolvable in more than one branch of $X$. 
In order to produce an SL tree $X'$ which results from resolving $\metaA$ in every branch in which it is resolvable in $X$, the following lemma shows that every SL tree whatsoever has at most a finite number of branches.

\label{FiniteBranchLemma}
\factoidbox{
\textbf{Lemma 4:} Every SL tree $X$ has a finite number of branches. 
}

\textit{Proof:}
We will proceed by induction on the length of an SL tree.

\subsubsection{Base Case}

Assume $X$ is an SL tree where $\length(X)=0$. 
Thus $X$ has one branch consisting of the root, and so the number of branches in $X$ is finite. 




\subsubsection{Induction Hypothesis}

Assume that $X$ has a finite number of branches for any SL tree $X$ of $\length(X)=n$. 


\subsubsection{Induction Step}

Let $X'$ be an SL tree with $\length(X')=n+1$, and $X$ be any SL tree where $X'$ is the result of resolving some SL sentence $\metaA$ in a branch $\B$ of $X$.
It follows that $\length(X)=n$.
By hypothesis, $X$ has a finite number of branches.
Since $X'$ has at most one more branch than $X$, we may conclude that $X'$ has a finite number of branches. 
Generalizing on $X'$, it follows that $X'$ has a finite number of branches for every SL tree $X'$ of $\length(X')=n+1$.




\subsubsection{Conclusion}

It follows by weak induction that every SL tree $X$ has a finite number of branches. 
Given any SL tree $X$ with root $\Gamma$ where $\metaA$ is resolvable in some branch of $X$, it follows by \textbf{Lemma 4} that $\metaA$ is resolvable in a finite number of branches of $X$.
Accordingly, we may order those branches, resolving $\metaA$ in each branch in turn, referring to the result as $X'$. 
Since the root remains unchanged throughout this process, we may conclude that there is an SL tree $X'$ with root $\Gamma$ in which $\metaA$ has been resolved in every branch of $X'$.



\subsection{Lemma 5}


Consider the following recursive definition of the \define{resolution} $\res(\metaA)$ of an SL sentence $\metaA$:

\begin{enumerate}
  \item $\res(\metaA)=0$ if $\metaA$ is a literal.
  \item For any SL sentences $\metaA$ and $\metaB$:
    \begin{itemize}
      \item $\res(\enot\enot\metaA)=\res(\metaA)+1$.
      \item $\res(\metaA \eand \metaB)=\res(\metaA)+\res(\metaB)+1$.
      \item $\res(\enot(\metaA \eand \metaB))=\res(\enot\metaA)+\res(\enot\metaB)+1$.
      \item $\res(\metaA \eor \metaB)=\res(\metaA)+\res(\metaB)+1$.
      \item $\res(\enot(\metaA \eor \metaB))=\res(\enot\metaA)+\res(\enot\metaB)+1$.
      \item $\res(\metaA \eif \metaB)=\res(\enot\metaA)+\res(\metaB)+1$.
      \item $\res(\enot(\metaA \eif \metaB))=\res(\metaA)+\res(\enot\metaB)+1$.
      \item $\res(\metaA \eiff \metaB)=\res(\metaA)+\res(\metaB) + \res(\enot\metaA)+\res(\enot\metaB)+1$.
      \item $\res(\enot(\metaA \eiff \metaB))=\res(\metaA) + \res(\enot\metaB) + \res(\enot\metaA)+\res(\metaB)+1$.
    \end{itemize}
\end{enumerate}

Intuitively, $\res(\metaA)$ is an upper bound on the number of times that $\metaA$ and its descendants could be resolved in an SL tree. 
% We then define $[X]$ to be the smallest set of sentences to satisfy: 
%
% \begin{enumerate}
%   \item $\varphi\in[X]$ if $X$ is a root and $\varphi$ occurs in $X$.
%   \item For any SL tree $Y$ which results from resolving $\varphi\in[X]$ on every open branch in $X$ in which $\varphi$ occurs:
%     \begin{itemize}
%       \item[$(\neg)$] $[Y]=([X]/\set{\varphi})\cup\set{\psi}$ if $\varphi=\neg\neg\psi$;
%       \item[$(+)$] $[Y]=([X]/\set{\varphi})\cup\set{\psi,\chi}$ if $\varphi\in\set{\psi\wedge\chi,\psi\vee\chi}$;  
%       \item[$(-)$] $[Y]=([X]/\set{\varphi})\cup\set{\neg\psi,\neg\chi}$ if $\varphi\in\set{\neg(\psi\wedge\chi),\neg(\psi\vee\chi)}$;
%       \item[$(\supset)$] $[Y]=([X]/\set{\varphi})\cup\set{\neg\psi,\chi}$ if $\varphi=\psi\supset\chi$; 
%       \item[$(\not\supset)$] $[Y]=([X]/\set{\varphi})\cup\set{\psi,\neg\chi}$ if $\varphi=\neg(\psi\supset\chi)$; 
%       \item[$(\equiv)$] $[Y]=([X]/\set{\varphi})\cup\set{\psi,\chi, \neg\psi,\neg\chi}$ if $\varphi\in\set{\psi\equiv\chi,\neg(\psi\equiv\chi)}$.
%     \end{itemize}
% \end{enumerate}
%
% Letting $\mathbb{L}$ be the set of SL literals, we define $X_U=[X]/\mathbb{L}$ to be the set of SL sentences in $X$ that can be resolved. 
Letting $[X]$ be the set of resolvable SL sentences that occur in any branch of $X$, we may define the \define{resolution} of an SL tree $X$ as follows: 
  $$\res(X)=\sum\limits_{\metaA\in [X]}\res(\metaA).$$
Given this definition, we may prove the following preliminary results:

\label{ResolutionTreeLemma}
\factoidbox{
\textbf{Lemma 5:} For any SL tree $X$ with root $\Gamma$ and $\metaA\in[X]$, there is an SL tree $Y$ with root $\Gamma$ where $\res(Y)<\res(X)$.
}

\textit{Proof:}
Let $X$ be an SL tree with root $\Gamma$ where $\metaA\in[X]$.
By \textbf{Lemma 4}, every SL tree has at most finitely many branches, and so finitely many branches in which $\metaA$ is resolvable. 
Thus there is an SL tree $Y$ with root $\Gamma$ that results from resolving $\metaA$ in every branch in $X$ in which $\metaA$ is resolvable.
Since $\metaA$ is resolved in every branch in $Y$, it follows that $\metaA\notin[Y]$.

Given that $\metaA\notin[Y]$ and $\metaA\in[X]$, it remains to consider the children of $\metaA$ which are the only sentences that could belong to $[Y]$ without also belonging to $[X]$, thereby increasing the resolution of $Y$ without increasing the resolution of $X$.
However, even if the children of $\metaA$ belong to $[Y]$ without already occurring in $[X]$, we may show that $\res(Y)<\res(X)$: 

\textit{Case 1:}
Assume $\metaA=\enot\enot\metaB$ where $\metaB\in[Y]$ and $\metaB\notin[X]$.
Since $\res(\metaB)<\res(\metaA)$ and we already know that $\metaA\notin[Y]$ and $\metaA\in[X]$, we may conclude that $\res(Y)<\res(X)$.

\textit{Case 2:}
Assume $\metaA=\metaB\eand\metaC$ where $\metaB,\metaC\in[Y]$ and $\metaB,\metaC\notin[X]$.
Since $\metaA\notin[Y]$ and $\metaA\in[X]$ where $\res(\metaB)+\res(\metaC)<\res(\metaA)$, we may conclude that $\res(Y)<\res(X)$.

\textit{Case 3:}
Assume $\metaA=\enot(\metaB\eand\metaC)$ where $\enot\metaB,\enot\metaC\in[Y]$ and $\enot\metaB,\enot\metaC\notin[X]$.
Since $\metaA\notin[Y]$ and $\metaA\in[X]$ where $\res(\enot\metaB)+\res(\enot\metaC)<\res(\metaA)$, we may conclude that $\res(Y)<\res(X)$.

\textit{Case 4:}
Assume $\metaA=\metaB\eor\metaC$ where $\metaB,\metaC\in[Y]$ and $\metaB,\metaC\notin[X]$.
Since $\metaA\notin[Y]$ and $\metaA\in[X]$ where $\res(\metaB)+\res(\metaC)<\res(\metaA)$, we may conclude that $\res(Y)<\res(X)$.

\textit{Case 5:}
Assume $\metaA=\enot(\metaB\eor\metaC)$ where $\enot\metaB,\enot\metaC\in[Y]$ and $\enot\metaB,\enot\metaC\notin[X]$.
Since $\metaA\notin[Y]$ and $\metaA\in[X]$ where $\res(\enot\metaB)+\res(\enot\metaC)<\res(\metaA)$, we may conclude that $\res(Y)<\res(X)$.

\textit{Case 6:}
Assume $\metaA=\metaB\eif\metaC$ where $\enot\metaB,\metaC\in[Y]$ and $\enot\metaB,\metaC\notin[X]$.
Since $\metaA\notin[Y]$ and $\metaA\in[X]$ where $\res(\enot\metaB)+\res(\metaC)<\res(\metaA)$, we may conclude that $\res(Y)<\res(X)$.

\textit{Case 7:}
Assume $\metaA=\enot(\metaB\eif\metaC)$ where $\metaB,\enot\metaC\in[Y]$ and $\metaB,\enot\metaC\notin[X]$.
Since $\metaA\notin[Y]$ and $\metaA\in[X]$ where $\res(\metaB)+\res(\enot\metaC)<\res(\metaA)$, we may conclude that $\res(Y)<\res(X)$.

\textit{Case 8:}
Assume $\metaA=\metaB\eiff\metaC$ where $\metaB,\metaC,\enot\metaB,\enot\metaC\in[Y]$ and $\metaB,\metaC,\enot\metaB,\enot\metaC\notin[X]$.
Since $\metaA\notin[Y]$ and $\metaA\in[X]$ where $\res(\metaB)+\res(\metaC)+\res(\enot\metaB)+\res(\enot\metaC)<\res(\metaA)$, we may conclude that $\res(Y)<\res(X)$.

\textit{Case 9:}
Assume $\metaA=\enot(\metaB\eiff\metaC)$ where $\metaB,\enot\metaC,\enot\metaB,\metaC\in[Y]$ and $\metaB,\enot\metaC,\enot\metaB,\metaC\notin[X]$.
Since $\metaA\notin[Y]$ and $\metaA\in[X]$ where $\res(\metaB)+\res(\enot\metaC)+\res(\enot\metaB)+\res(\metaC)<\res(\metaA)$, we may conclude that $\res(Y)<\res(X)$.




\subsection{Lemma 6}

\label{CompleteTreeLemma}
\factoidbox{
\textbf{Lemma 6:} For every SL tree $X$ with root $\Gamma$, there is a complete SL tree $X'$ with root $\Gamma$.
}

\textit{Proof:}
Drawing on \textbf{Lemma 5}, the proof goes by induction on the resolution of SL trees.

\subsubsection{Base Case}

Assume that $X$ is an SL tree with root $\Gamma$ where $\res(X)=0$.
It follows that every SL sentence $\metaA$ in $X$ is a literal, and so no sentence in any branch of $X$ is resolvable. 
Thus $X$ is complete, and so there is a complete SL tree $X'\ (=X)$ with root $\Gamma$.


\subsubsection{Induction Hypothesis}

Assume for weak induction that for every SL tree $X$ with root $\Gamma$ and $\res(X)=n$, there is a complete SL tree $X'$ with root $\Gamma$.



\subsubsection{Induction Step}

Let $X$ be an SL tree with root $\Gamma$ where $\res(X)=n+1$.
Since $\res(X)>0$, there is some resolvable SL sentence $\metaA$ in an open branch of $X$, and so $\metaA \in [X]$. 
By \textbf{Lemma 5}, there is some SL tree $Y$ with root $\Gamma$ where $\res(Y)<\res(X)$.
By hypothesis, there is a complete SL tree $Y'$ with root $\Gamma$.




\subsubsection{Conclusion}

We may conclude by weak induction that for every SL tree with root $\Gamma$, there is a complete SL tree $X'$ with root $\Gamma$. 
Since every root is a tree, every root has a complete SL tree.

In order to get a sense of why this result is important, suppose we were to assume that a root $\Gamma$ does not have a closed tree, i.e., $\Gamma \nproves \bot$.
Given \textbf{Lemma 6}, we know that $\Gamma$ has a complete tree $X$, and so $X$ must be a complete open tree with a complete open branch $\B$.
In the following lemma, we will show that complete open branches are satisfiable, where it follows as an immediate result that $\Gamma$ is satisfiable $\Gamma \nmodels \bot$. 
Accordingly, \textbf{Lemma 6} and \textbf{Lemma 7} make up the core of the proof that the SL tree proof system is complete.




\subsection{Lemma 7}

% We begin by defining another important property that a branch of an SL tree may have.
% A branch $\B$ of an SL tree is \define{hintikka} just in case it satisfies the following conditions:
%
% \begin{enumerate}
%   \item If $\neg\neg\metaA$ occurs in $\B$, then $\metaA$ occurs in $\B$.
%   \item If $\metaA\eand\metaB$ occurs in $\B$, then $\metaA$ and $\metaB$ occur in $\B$.
%   \item If $\enot(\metaA\eand\metaB)$ occurs in $\B$, then $\enot\metaA$ or $\enot\metaB$ occurs in $\B$.
%   \item If $\metaA\eor\metaB$ occurs in $\B$, then $\metaA$ occur or $\metaB$ occurs in $\B$.
%   \item If $\enot(\metaA\eor\metaB)$ occurs in $\B$, then $\enot\metaA$ and $\enot\metaB$ occur in $\B$.
%   \item If $\metaA\eif\metaB$ occurs in $\B$, then $\enot\metaA$ or $\metaB$ occurs in $\B$.
%   \item If $\enot(\metaA\eif\metaB)$ occurs in $\B$, then $\metaA$ and $\enot\metaB$ occur in $\B$.
%   \item If $\metaA\eiff\metaB$ occurs in $\B$, then $\metaA$ and $\metaB$ occur in $\B$ or $\enot\metaA$ and $\enot\metaB$ occur in $\B$.
%   \item If $\enot(\metaA\eiff\metaB)$ occurs in $\B$, then $\metaA$ and $\enot\metaB$ occur in $\B$ or $\metaA$ and $\enot\metaB$ occur in $\B$.
% \end{enumerate}
%
% % Letting $[\B]$ be the set of all SL sentences that occur in a branch $\B$, we define $\B$ to be \define{hintikka} just in case $[\B]$ is Hintikka.
% In order to establish the following lemma, we will show that every complete open branch of an SL tree is Hintikka, and so satisfiable as a result.
%


\label{HintikkaLemma}
\factoidbox{
\textbf{Lemma 7:} Every complete open branch in an SL tree is satisfiable.
}

\textit{Proof:}
Let $\B$ be a complete open branch in an SL tree. 
Accordingly, every sentence in $\B$ has been resolved in $\B$. 
% As a result, $\B$ is Hintikka.
% For instance, if $\enot\enot\metaA$ occurs in $\B$, the resolution rule for $\enot\enot\metaA$ must have been applied since $\B$ is complete and open, and so $\metaA$ occurs in $\B$. 
% Similarly, if $\metaA\eor\metaB$ occurs in $\B$, the resolution rule for $\metaA\eor\metaB$ must have been applied, and so either $\metaA$ or $\metaB$ occur in $\B$. 
% The other cases are similar, and so $\B$ is Hintikka.
In order to show that $\B$ is satisfiable, let $\I$ be an SL interpretation where $\I(\metaA)=1$ just in case $\metaA$ is a sentence letter that occurs in $\B$.
% Given any sentence letter $\metaA$ where $\enot\metaA$ occurs in $\B$, we know that $\metaA$ cannot occur in $\B$ since $\B$ is open, and so $\I(\metaA)=0$ by the definition of $\I$.
% Thus $\I(\enot\metaA)=1$ for every sentence letter $\metaA$ that does not occur in $\B$. 
We will now show that $\I$ satisfies $\B$ by induction on the resolution of sentences in $\B$.



\subsubsection{Base Case}

Assume $\res(\metaA)=0$ for an arbitrary sentence $\metaA$ that occurs in $\B$. 
It follows that $\metaA$ is a literal.
If $\metaA$ is a sentence letter, then $\I(\metaA)=1$ as observed above, and so $\V{\I}(\metaA)=1$.
If $\metaA=\enot\metaB$, we know that $\I(\metaB)=0$, and so $\V{\I}(\enot\metaB)=\V{\I}(\metaA)=1$.
Thus $\I$ satisfies the literals in $\B$.


\subsubsection{Induction Hypothesis}

Assume that for every SL sentence $\metaA$ that occurs in $\B$, if $\res(\metaA)\leq n$, then $\V{\I}(\metaA)=1$. 


\subsubsection{Induction Step}

Let $\metaA$ be an SL sentence that occurs in $\B$ where $\res(\metaA)=n+1$.
Since $\res(\metaA)>0$, there is an application rule for $\metaA$. 
Given that $\B$ is complete and open, $\metaA$ has been resolved in $\B$. 
There are nine cases to consider, corresponding to the nine resolution rules.


\textit{Case 1:} 
If $\metaA=\neg\neg\metaB$, then $\metaB$ occurs in $\B$.
Since $\res(\metaB)\leq n$, we know by hypothesis that $\V{\I}(\metaB)=1$, and so $\V{\I}(\metaA)=\V{\I}(\enot\enot\metaB)=1$ by the semantic clause for negation. 

\textit{Case 2:} 
If $\metaA=\metaB\eand\metaC$, then $\metaB$ and $\metaC$ occur in $\B$.
Since $\res(\metaA) = \res(\metaB) + \res(\metaC) + 1$, we know that $\res(\metaB)\leq n$ and $\res(\metaC)\leq n$, and so $\V{\I}(\metaB)=1$ and $\V{\I}(\metaC)=1$ by hypothesis.  
Thus $\V{\I}(\metaA)=\V{\I}(\metaB\eand\metaC)=1$ by the semantic clause for conjunction.

\textit{Case 3:} 
If $\metaA=\enot(\metaB\eand\metaC)$, then $\enot\metaB$ or $\enot\metaC$ occur in $\B$.
Since $\res(\metaA) = \res(\enot\metaB) + \res(\enot\metaC) + 1$, we know that $\res(\enot\metaB)\leq n$ and $\res(\enot\metaC)\leq n$, and so $\V{\I}(\enot\metaB)=1$ or $\V{\I}(\enot\metaC)=1$ by hypothesis.  
By the semantics for negation, $\V{\I}(\metaB)=0$ or $\V{\I}(\metaC)=0$, and so either way $\V{\I}(\metaB\eand\metaC)=0$ by the semantics for conjunction.
Thus $\V{\I}(\metaA)=\V{\I}(\enot(\metaB\eand\metaC))=1$.

\textit{Case 4:} 
If $\metaA=\metaB\eor\metaC$, then $\metaB$ or $\metaC$ occur in $\B$.
Since $\res(\metaA) = \res(\metaB) + \res(\metaC) + 1$, we know that $\res(\metaB)\leq n$ and $\res(\metaC)\leq n$, and so $\V{\I}(\metaB)=1$ or $\V{\I}(\metaC)=1$ by hypothesis.  
Either way, $\V{\I}(\metaA)=\V{\I}(\metaB\eor\metaC)=1$ by the semantics for disjunction.

\textit{Case 5:} 
If $\metaA=\enot(\metaB\eor\metaC)$, then $\enot\metaB$ and $\enot\metaC$ occur in $\B$.
Since $\res(\metaA) = \res(\enot\metaB) + \res(\enot\metaC) + 1$, we know that $\res(\enot\metaB)\leq n$ and $\res(\enot\metaC)\leq n$, and so $\V{\I}(\enot\metaB)=1$ and $\V{\I}(\enot\metaC)=1$ by hypothesis.  
By the semantics for negation, $\V{\I}(\metaB)=0$ and $\V{\I}(\metaC)=0$, and so $\V{\I}(\metaB\eor\metaC)=0$ by the semantics for disjunction.
Thus $\V{\I}(\metaA)=\V{\I}(\enot(\metaB\eor\metaC))=1$.

\textit{Case 6:} 
If $\metaA=\metaB\eif\metaC$, then $\enot\metaB$ or $\metaC$ occur in $\B$.
Since $\res(\metaA) = \res(\enot\metaB) + \res(\metaC) + 1$, we know that $\res(\enot\metaB)\leq n$ and $\res(\metaC)\leq n$, and so $\V{\I}(\enot\metaB)=1$ or $\V{\I}(\metaC)=1$ by hypothesis.  
By the semantics for negation, $\V{\I}(\metaB)=0$ or $\V{\I}(\enot\metaC)=1$.
Either way, $\V{\I}(\metaA)=\V{\I}(\metaB\eif\metaC)=1$ by the semantics for the conditional.

\textit{Case 7:} 
If $\metaA=\enot(\metaB\eif\metaC)$, then $\metaB$ and $\enot\metaC$ occur in $\B$.
Since $\res(\metaA) = \res(\metaB) + \res(\enot\metaC) + 1$, we know that $\res(\metaB)\leq n$ and $\res(\enot\metaC)\leq n$, and so $\V{\I}(\metaB)=1$ and $\V{\I}(\enot\metaC)=1$ by hypothesis.  
By the semantics for negation, $\V{\I}(\metaB)=1$ and $\V{\I}(\metaC)=0$, and so $\V{\I}(\metaB\eif\metaC)=0$ by the semantics for the conditional.
Thus $\V{\I}(\metaA)=\V{\I}(\enot(\metaB\eif\metaC))=1$.

\textit{Case 8:} 
If $\metaA=\metaB\eiff\metaC$, either $\metaB$ and $\metaC$ occur in $\B$ or $\enot\metaB$ and $\enot\metaC$ occur in $\B$.
Since $\res(\metaA) = \res(\metaB) + \res(\metaC) + \res(\enot\metaB) + \res(\enot\metaC) + 1$, we know that $\res(\metaB)\leq n$, $\res(\metaC)\leq n$, $\res(\enot\metaB)\leq n$, and $\res(\enot\metaC)\leq n$.
Thus $\V{\I}(\metaB)=\V{\I}(\metaC)=1$ or $\V{\I}(\enot\metaB)=\V{\I}(\enot\metaC)=1$ by hypothesis.  
In the latter case, $\V{\I}(\metaB)=\V{\I}(\metaC)=0$ by the semantics for negation, and so either way $\V{\I}(\metaA)=\V{\I}(\metaB\eiff\metaC)=1$ by the semantics for the biconditional.

\textit{Case 9:} 
If $\metaA=\enot(\metaB\eiff\metaC)$, then $\metaB$ and $\enot\metaC$ occur in $\B$ or $\metaA$ and $\enot\metaC$ occur in $\B$.
Since $\res(\metaA) = \res(\metaB) + \res(\enot\metaC) + \res(\enot\metaB) + \res(\metaC) + 1$, we know that $\res(\metaB)\leq n$, $\res(\enot\metaC)\leq n$, $\res(\enot\metaB)\leq n$, and $\res(\metaC)\leq n$.
Thus $\V{\I}(\metaB)=\V{\I}(\enot\metaC)=1$ or $\V{\I}(\enot\metaB)=\V{\I}(\metaC)=1$ by hypothesis.  
Either way, $\V{\I}(\metaB)\neq\V{\I}(\metaC)$ by the semantics for negation, and so $\V{\I}(\metaB\eiff\metaC)=0$ by the semantics for the biconditional.
Thus $\V{\I}(\metaA)=\V{\I}(\enot(\metaB\eiff\metaC))=1$.





\subsection{Conjunction}
\label{conjunctionsound}
Suppose our completed, open branch contains a conjunction of the form \metaA{}\eand\metaB{}. Since it is a \emph{completed} branch, this means that the conjunction resolution rule must have been applied to this conjunction. (Remember, a branch isn't completed until every complex formula has a check mark next to it.) So, given the conjunction rule,

\begin{center}
\begin{prooftree}
{not line numbering}
[\metaA{}\eand\metaB{}
	[\metaA{}
	[\metaB{}, grouped
	]
	]
]
\end{prooftree}
\end{center}

we know that the branch must also contain \metaA{} and \metaB{}. If we assume that $\mathcal{I}$ satisfies both these sentences, then we know from the definition of the truth of a conjunction in SL that $\mathcal{I}$ satisfies (\metaA{}\eand\metaB{}) too. In other words, there's no way to satisfy the simpler sentences that come after this resolution rule, without also satisfying the conjunction.



\subsection{Negated conjunction}

Suppose a negated conjunction appears in the open branch. Since the branch is complete, you know that the negated conjunction rule has been applied:

\begin{center}
\begin{prooftree}
{not line numbering,
single branches}
[\enot(\metaA{}\eand\metaB{})
	[\enot\metaA{}]
	[\enot\metaB{}]
]
\end{prooftree}
\end{center}

We are assuming that the negated conjunction is in an open branch. This is consistent with either one of the branches below closing, but they cannot both close. If they did, the negated conjunction would not be in an open branch. So we know that at least one branch is open. So either $\mathcal{I}(\metaA{})=0$ (if the left branch is our open branch) or $\mathcal{I}(\metaB{})=0$ (if the right branch is the open branch). Since at least one of these sentences is assigned 0, their conjunction must also be assigned 0, which means the negated conjunction we're considering is assigned 1. So once again, if the material in at least one branch below the resolution rule is satisfied, then the negated conjunction is satisfied too.

\subsection{Disjunction}

Disjunctions are very similar to negated conjunctions. Since the tree is complete, any disjunction $\metaA{}\eor\metaB{}$ has a branch below it containing \metaA{}, and one containing \metaB{}. Whichever of these disjuncts is in the open branch, $\mathcal{I}$ satisfies that disjunct, and so satisfies the disjunction too.

\subsection{Negated disjunction}

Negated disjunctions are similar to conjunctions. If a negated disjunction is in an open branch, then the negation of each disjunct is also in that branch. So, suppose that $\mathcal{I}$ assigns 0 to each disjunct. Then it also assigns 0 to their disjunction. So once again, if the material below the negated disjunction is satisfied, then so is the negated disjunction itself.

\subsection{Conditional}

If a conditional is in a completed open branch, then it has been resolved by this branching rule:

\begin{center}
\begin{prooftree}
{not line numbering,
single branches}
[\metaA{}\eif\metaB{}
	[\enot\metaA{}]
	[\metaB{}]
]
\end{prooftree}
\end{center}

If the left development is the open branch, then we suppose that  $\mathcal{I}(\metaA{})=0$, which means that $\mathcal{I}(\metaA{}\eif\metaB{})=1$. If this right development is the open branch, then we suppose that $\mathcal{I}(\metaB{})=1$, which \emph{also} means that $\mathcal{I}(\metaA{}\eif\metaB{})=1$. So if the material below in at least one branch is satisfied, then the conditional is satisfied too.

\subsection{Negated conditional}

If a negated conditional \enot(\metaA{}\eif\metaB{}) is in the open branch, then so too are \metaA{} and \enot\metaB{}. So $\mathcal{I}(\metaA{})=1$ and $\mathcal{I}(\metaB{})=0$. So $\mathcal{I}$ falsifies the conditional, satisfying the negated conditional.

There are three more kinds of sentences that exist in SL.

\subsection{Biconditional}

Suppose a biconditional is in an open branch. If the branch is completed, then this rule has been performed:

\begin{center}
\begin{prooftree}
{not line numbering,
single branches}
[\metaA{}\eiff\metaB{}
	[\metaA{}
		[\metaB{}, grouped]
	]
	[\enot\metaA{}
		[\enot\metaB{}, grouped]
	]
]
\end{prooftree}
\end{center}

One of these developments is the open branch. If it's the left branch, then, supposing that $\mathcal{I}$ assigns 1 to both \metaA{} and \metaB{}, $\mathcal{I}$ must also assign 1 to the biconditional $\metaA{}\eiff\metaB{}$. If it's the right branch, then, supposing that $\mathcal{I}$ assigns 0 to both \metaA{} and \metaB{}, this also means that $\mathcal{I}$ must also assign 1 to the biconditional $\metaA{}\eiff\metaB{}$. So whichever branch is satisfied by $\mathcal{I}$, the biconditional is also satisfied.

\subsection{Negated biconditional}

Exactly the same reasoning as above applies to negated biconditionals, except this time, the branches each assign \emph{opposite} truth values to \metaA{} and \metaB{}. So for our interpretation to satisfy either branch, it must falsify the biconditional, thus satisfying the negated biconditional.

\subsection{Double negation}

Finally, suppose there is a double-negated sentence in our completed open branch. Then this rule has been performed:

\begin{center}
\begin{prooftree}
{not line numbering, single branches}
[\enot\enot\metaA{}
	[\metaA{}]
]
\end{prooftree}
\end{center}

If $\mathcal{I}(\metaA{})=1$, then, given the definition of truth in SL, $\mathcal{I}(\enot\metaA{})=0$, and $\mathcal{I}(\enot\enot\metaA{})=1$. So once again, if our interpretation satisfies what comes below, then it satisfies the double-negation above.

\subsection{Summarizing the completeness proof}

What we've just shown is that, for any sentence of SL, if it has one of the nine structures just canvassed --- if it's a conjunction, a negated conjunction, a disjunction, etc. --- then, if it is in a completed open branch where the sentences below it are satisfied by $\mathcal{I}$, then it too is satisfied by $\mathcal{I}$. Given the way that $\mathcal{I}$ was selected, we know that $\mathcal{I}$ must satisfy every atomic sentence, and every negated atomic sentence, in the open branch. And since the nine structures considered are the only ways to develop more complex sentences, this implies that \emph{every} SL sentence in the open branch is satisfied by $\mathcal{I}$. This includes the root. Since interpretation $\mathcal{I}$ satisfies the root, this of course means that the root is satisfiable. That is to say, if a completed branch remains open, this guarantees that the root is satisfiable. Equivalently, if the root is unsatisfiable, a completed tree is guaranteed to close. Completeness is proven.

\section{Testing Alternate Rules}

We can use the reasoning involved in the soundness and completeness proofs above to consider various alternative tree rules. We saw one example of this on p.\ \pageref{unsoundrule} above, when we observed that an alternate, linear rule for disjunctions would result in an unsound tree system. Here again was the rule we considered:

\begin{center}
\begin{prooftree}
{not line numbering,
single branches}
[\metaA{}\eor\metaB{}
	[\metaA{}
	[\metaB{}, grouped]]
]
\end{prooftree}
\end{center}

Note, however, that if we think through the \emph{completeness} reasoning, we'll find that the completeness proof would still hold. If we assume that some interpretation $\mathcal{I}$ satisfies both \metaA{} and \metaB{}, we can be assured that it also satisfies the disjunction $\metaA{}\eor\metaB{}$. So changing the disjunction rule to this linear one would \emph{not} interfere with the completeness of our tree system. Our trees would still close any time they began with unsatisfiable roots. But as we saw on p.\ \pageref{soundprooffailure}, the soundness proof would fail, which is why a system with this rule could start in a satisfiable root, and result in a closed tree.

When either the soundness or the completeness proof fails, you know you are working with an inappropriate rule. To conclusively demonstrate this, you can provide a counterexample to the failed metalogical theorem. A counterexample to soundness would be a tree with a satisfiable root that closes. A counterexample to completeness would be a completed tree, with an unsatisfiable root, that remains open. The rule above violates soundness, so using that rule we can construct a tree with a satisfiable root, that closes. Constructing the right counterexample takes a bit of thought. The rule puts both disjuncts into a single branch below, and we want it to close, despite having a satisfiable root. So adding the negation of just one of the disjuncts to the root will close the tree, without making the root unsatisfiable:

\begin{center}
\begin{prooftree}
{not line numbering,
single branches}
[A \eor B, checked
[\enot A, grouped
	[A
	[B, grouped, close]]
]
]
\end{prooftree}
\end{center}

This tree is a counterexample to soundness, using the hypothetical rule mentioned above. Note that a counterexample is a tree that uses SL sentences, not the Greek letters \metaA{} and \metaB{} that we use in the statements of the rules.

Let's work through one more example. Suppose we changed the conjunction rule to this one:
	\begin{center}
	\begin{prooftree}
	{not line numbering}
	[\metaA{}\eand\metaB{}
		[\metaA{}
		[\metaB{}, grouped
		]
		]
		[\metaB{}]
	]
\end{prooftree}
\end{center}
Would our system still be sound? To answer this, we assume that the conjunction $\metaA{}\eand\metaB{}$ is satisfiable, and ask whether this guarantees that at least one branch below is also satisfiable. It does. (In fact, both branches are guaranteed to be satisfiable.) So the system will still be sound.

Would the system still be complete? To answer this, we ask whether each branch is such that, if we assume that an interpretation satisfies the developments below, it is guaranteed to satisfy the conjunction above. Begin with the left branch. If some interpretation satisfies both \metaA{} and \metaB{}, then it will certainly satisfy the conjunction $\metaA{}\eand\metaB{}$. So that branch looks fine. (Indeed, that branch is exactly the same as the linear development of our actual conjunction rule, so this reasoning is the same as that on p.\ \pageref{conjunctionsound}.)

But what of the right branch? Assume that some interpretation satisfies \metaB{}; does that guarantee that it satisfies $\metaA{}\eand\metaB{}$? Certainly not. So if the right branch is our open branch, a completed tree using this rule may remain open, even if its root is unsatisfiable. Completeness will be violated. Let's construct a counterexample of that form. We want a tree that includes a conjunction, whose root is unsatisfiable, but whose right branch remains open. Notice that the right branch `ignores' the first conjunct; this is a clue that a good way to construct a counterexample will be to locate the unsatisfiability within that first conjunct. Suppose, for example, that we let \metaA{} itself stand for a contradiction. If so, any conjunction with \metaA{} as a conjunct will be unsatisfiable. But if \metaB{} is not a contradiction, then a tree with root $\metaA{}\eand\metaB{}$ will remain open. Let's develop a tree with this root: $(P \eiff \enot P) \eand Q$. Note that this sentence has a contradictory first conjunct, and a contingent, atomic second conjunct.

\begin{center}
\begin{prooftree}
	{not line numbering}
	[(P \eiff \enot P) \eand Q, checked
		[(P \eiff \enot P), checked
		[Q, grouped
			[P
			[\enot P, grouped, close
			]
			]
			[\enot P
			[\enot\enot P, grouped, close
			]
			]
		]
		]
		[Q, open]
	]
\end{prooftree}
\end{center}

The left branch closes after we perform the unchanged biconditional rule on the contradictory first conjunct, but the right branch remains open. Since this is a completed tree with an open branch and an unsatisfiable root, it is a counterexample to completeness.

Remember that we are considering a modification to the tree system that uses a different conjunction rule. In this example I used $P \eiff \enot P$ as my \metaA{}, which let me use the unchanged biconditional rule. But if I had used a contradiction that used a conjunction, like $P \eand \enot P$, I would have had to have used the revised rule within the left branch too.

It is also possible to construct counterexamples using simpler sentences if you add to the root. Instead of introducing a contradictory conjunct, we could have simply added to the root, in a way that makes the root unsatisfiable, but leaves the right branch open. Suppose for instance we put both $P \eand Q$ and $\enot P$ in the root. Then we'd have an unsatisfiable root, but the right branch would remain open:
	\begin{center}
	\begin{prooftree}
	{not line numbering}
	[P\eand Q, checked
	[\enot P, grouped
		[P
		[Q, grouped, close
		]
		]
		[Q, open]
	]
	]
\end{prooftree}
\end{center}

This too is a counterexample to completeness, given the rule in question. So we see here two different kinds of strategies for generating counterexamples to completeness.

\iffalse

\practiceproblems

\solutions
\problempart
\label{pr.SL.soundness-resolutions}
Following are possible modifications to our SL tree system. For each, imagine a system that is like the system laid out in this chapter, except for the indicated change. Would the modified tree system be sound? If so, explain how the proof given in this chapter would extend to a system with this rule; if not, give a tree that is a counterexample to the soundness of the modified system.
\begin{earg}
\item Change the rule for conjunctions to this rule:
	\factoidbox{
	\begin{center}
	\begin{prooftree}
	{not line numbering}
	[\metaA{}\eand\metaB{}
		[\metaA{}]
		[\metaB{}]
	]
\end{prooftree}
\end{center}
}

\item Change the rule for conjunctions to this rule:
	\factoidbox{
	\begin{center}
	\begin{prooftree}
	{not line numbering}
	[\metaA{}\eand\metaB{}
		[\metaA{}]
	]
\end{prooftree}
\end{center}
}

\item Change the rule for conjunctions to this rule:
	\factoidbox{
	\begin{center}
	\begin{prooftree}
	{not line numbering}
	[\metaA{}\eand\metaB{}
		[\metaA{}
		[\enot\metaB{}, grouped
		]
		]
	]
\end{prooftree}
\end{center}
}

\item Change the rule for disjunctions to this rule:
	\factoidbox{
	\begin{center}
	\begin{prooftree}
	{not line numbering}
	[\metaA{}\eor\metaB{}
		[\metaA{}
		[\metaB{}, grouped
		]
		]
	]
\end{prooftree}
\end{center}
}

\item Change the rule for disjunctions to this rule:
	\factoidbox{
	\begin{center}
	\begin{prooftree}
	{not line numbering}
	[\metaA{}\eor\metaB{}
		[\metaA{}]
		[\metaB{}]
		[\metaA{} \eand \metaB{}]
	]
\end{prooftree}
\end{center}
}

\item Change the rule for conditionals to this rule:
	\factoidbox{
	\begin{center}
	\begin{prooftree}
	{not line numbering}
	[\metaA{}\eif\metaB{}
		[\enot\metaA{}]
		[\metaB{}\eor\metaA{}]
	]
\end{prooftree}
\end{center}
}

\item Change the rule for conditionals to this rule:
	\factoidbox{
	\begin{center}
	\begin{prooftree}
	{not line numbering}
	[\metaA{}\eif\metaB{}
		[\enot\metaA{}\eor\metaB{}]
	]
\end{prooftree}
\end{center}
}

\item Change the rule for biconditionals to this rule:
	\factoidbox{
	\begin{center}
	\begin{prooftree}
	{not line numbering}
	[\metaA{}\eiff\metaB{}
		[\metaA{}
		[\metaB{}, grouped
		]
		]
	]
\end{prooftree}
\end{center}
}

\item Change the rule for disjunctions to this rule:
	\factoidbox{
	\begin{center}
	\begin{prooftree}
	{not line numbering}
	[\metaA{}\eor\metaB{}
		[\metaA{}]
		[\metaB{}]
		[\metaC{}]
	]
\end{prooftree}
\end{center}
(This would mean that one can put whatever SL sentence one likes in the rightmost branch.)
}

\end{earg}

\problempart
\label{pr.SL.completenessresolutions}
For each of the rule modifications given in Part \ref{pr.SL.soundness-resolutions}, would the modified tree system be complete? If so, explain how the proof given in this chapter would extend to a system with this rule; if not, give a tree that is a counterexample to the completeness of the modified system.

\fi 
