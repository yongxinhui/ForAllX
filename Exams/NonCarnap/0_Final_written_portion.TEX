 \documentclass[12pt]{article}
 
\usepackage{0_Fitch_for_probs} 
 
 \usepackage{geometry}
\geometry{verbose,tmargin=.8in,bmargin=.6in,lmargin=1in,rmargin=1in}



\usepackage{multicol}
 


\usepackage{amsmath} %for align* environment and gather*
\usepackage{xref}
 %    \textheight     10.0truein
 \usepackage{graphics}
 \usepackage{pstricks}
% \usepackage{pst-tree}
% \usepackage{pst-node,pst-tree}
 \usepackage{makeidx}
 

 
 %\usepackage{forallx-ubc-Hunt} %calls local modified style file, but could lead to conflicts. i ought to make my own `common.sty' file for the problem sets. maybe using the same `common' file as the lecture notes? i really ought to just have a single consistent set of style files for the book, problem sets, and lecture notes! 
 
 %\def\therefore{\ensuremath{\ldotp\dot{}\,\ldotp}}
% disjunction
\def\eor{\ensuremath{\vee}}
% conjunction: 
% {\,^{_{_{_{_{\mbox{\footnotesize\textbullet}}}}}}} gives the dot
\def\eand{\ensuremath{\,\&\,}}
% conditional: \rightarrow gives the right arrow
\def\eif{\ensuremath{\supset}}
% biconditional: \leftrightarrow gives the left and right arrow
\def\eiff{\ensuremath{\equiv}}
% negation: {\sim} gives the swung dash 
%\def\enot{\ensuremath{\neg}}
%\def\enot{\ensuremath{\sim}} %note that \sim is defined as a relation, which leads to spacing issues. adding a \! leads to more spacing issues (piled up double negations).  

\def\enot{\ensuremath{{\sim}}} %redefining as {\sim} treats the tilde as a unary operator, rather than a relation, solving a lot of the spacing issues. 

\let\oldsim\sim %renames any \sim commands as \oldsim. 
\renewcommand{\sim}{{\oldsim}} %redefines \sim as unary operator version of \sim, in case there are any straggling \sim commands in the wild

% metalanguage variables: change greek to A and B if you prefer
\def\metaA{\ensuremath{\varPhi}}
\def\metaB{\ensuremath{\varPsi}}
\def\metaC{\ensuremath{\varOmega}}
\def\metaD{\ensuremath{\varDelta}}
\def\metaSetX{\ensuremath{\mathcal{X}}}
\def\metaSetY{\ensuremath{\mathcal{Y}}}
\def\metaSetZ{\ensuremath{\mathcal{Z}}}

%Calgary script and metav commands: 
\newcommand*{\script}[1]{\ensuremath{\mathscr{#1}}} %previously had \mathcal here; getting an issue 
\newcommand*{\metav}[1]{\ensuremath{\mathcal{#1}}}

\def\proves{\ensuremath{\vdash}}
\def\entails{\ensuremath{\vDash}}
\def\nproves{\ensuremath{\nvdash}}
\def\nentails{\ensuremath{\nvDash}}

\newcommand{\qt}[2]{(#1 #2) \,}
%\newcommand{\substitute}[2]{[#2/#1]}

%JH: idea to use ceiling notation since this looks like a partial bracket, mirroring a partial substitution instance
%\newcommand{\substitutesome}[2]{\mbox{$\lceil#2/#1\rceil$}}

%\newcommand*{\script}[1]{\ensuremath{\mathscr{#1}}}

% \pagestyle{empty}
 
 % Tree stuff
 
% \iffalse %pre using 0_Fitch_for_probs.sty
 
  \usepackage{prooftrees} %i copied over prooftrees file from Ichikawa source files, which I think is pre-2019 version
  
 %Note that I probably ought to just update my prooftrees package, since I downloaded the zip file and can just paste over the older version! (but who knows what else this could change...)
%JRH: adding in definition of line no override, local option included in 2019 revision. since my prooftrees package is not up to date! 
% see code here: https://tex.stackexchange.com/questions/415976/manually-set-line-numbers-if-prooftrees-sty
% see p. 24 of prooftrees manual for directions on using this. works w/ {}, e.g. line no override={n+1}

%the command `vdotsline' lets you put anything in number column, without a period appearing afterwards. so it's like `line no override' without \linenumberstyle
%e.g. for vertical dots vertically aligned, use: vdotsline={\\[-0.55em] \vdots}

\forestset{
  line no override/.style={
    before drawing tree={
      for name/.process={Ow}{proof tree proof line no}{line no ##1}{
        content=\linenumberstyle{#1},
        typeset node,
      },
    },
  },
  no line no/.style={
    before drawing tree={
      for name/.process={Ow}{proof tree proof line no}{line no ##1}{
        content=,
        typeset node,
      },
    },
  },
  vdotsline/.style={
    before drawing tree={
      for name/.process={Ow}{proof tree proof line no}{line no ##1}{
        content=#1,
        typeset node,
      },
    },
  },
  default preamble={
	single branches,
	close with=\ensuremath{\times},
	just sep=1.75em,
	line no sep=1.75em
	}
}

%\fi 


\begin{document}

%\input macs
%\input fitch
\newcommand{\detritus}[1]{}


\thispagestyle{empty}




%\bigskip %\bigskip


\begin{center}
\large Final (Written Portion) for 24.241 \\[1ex] 
\normalsize 100 `points' (will scale to 50\% of Final grade, i.e. 13.5 grade points)
%\large Solutions: Please Never Share or Upload to Internets or I'll have to make many new problems in the future and that would be a BIG BUMMER \\[3ex] 
% \textbf{Answer FOUR questions total: 1 and 2, 3 Xor 4, 5 Xor 6 (exclusive or's!)}
\end{center}



To ease Symbolization on \textit{Carnap}, below is the symbolization key for F1.1--F1.6: 
\begin{multicols}{2}
\begin{itemize}
\item Domain of Discourse: people\\  (your peeps, my peeps, all the peeps)
\item $\mathbf{u}$: A name for yourself, i.e. `\textbf{u}rself' 
\item $\mathbf{j}$: Little \textbf{j}ohnny, our fictional hero  
\item $Sx$: $x$ is a \textbf{S}cholar 
\item $Axy$: $x$ \textbf{A}ccepts y
\item $Pxy$: $x$ has the \textbf{P}ower to heal $y$
\end{itemize}
\end{multicols}

\medskip

\begin{enumerate}
%could put the rule on the back to give people scrap paper 
\item 
Consider a system $\mathit{SND}^*$ just like  $\mathit{SND}$ except that we add the rule ``negated disjunction intro.'' (\enot \eor I): $\{ \enot \metav{P}, \enot \metav{Q} \} \vdash \enot (\metav{P} \eor \metav{Q})$ (\textit{see back of page}). Does this rule preserve soundness? If so, extend our inductive proof by adding a case (showing that the new line is righteous); If not, provide a concrete counterexample to $\mathit{SND}^*$ soundness. [14pts] \\

For problems 2 and 3, let $\Gamma$ and $\Delta$ be possibly infinite sets of QL-sentences:

\item If $\Gamma \cup \{ \enot \metav{P} \}$ is unsatisfiable and $\Delta \cup \{ \enot \metav{Q} \}$ is unsatisfiable, prove or provide a counterexample to the following: $\{ \Gamma \cup\Delta \} \vdash_{\mathit{QND}} (\metav{P} \eand \metav{Q})$. [10pts]

\item If $\Gamma \vdash_{\mathit{QND}} \metav{P}$ and $\Delta \cup\{\metav{H}\}$ is unsatisfiable, prove or provide a counterexample to the following: $\{ \Gamma \cup \Delta \} \vDash \big( (\metav{P} \eor \metav{R}) \eand \enot \metav{H} \big) $ [10pts] \\

%\item If $\Gamma \vdash_{\mathit{QND}} \metav{P}$ and $\Delta \cup\{\metav{H}\}$ is unsatisfiable, prove or provide a counterexample to the following: $(\Gamma \cup \Delta) \vDash (\metav{P} \eor \metav{R}) $ (note that I really do mean ``$\metav{H}$" and then ``$\metav{R}$.") [10pts] \\

\item Assume that $\Gamma^{\ast}$ is a maximally-SND-consistent set. Prove that the following membership condition holds: $\enot \metav{P} \eif \enot \metav{Q} \in \Gamma^{\ast}$ if and only if either $\metav{P}\in \Gamma^{\ast}$ or $\metav{Q}\notin \Gamma^{\ast}$. 

In your proof, you may use case (a) of our membership lemma (book's 6.4.11a) but no other cases! Note that you will need to provide two non-trivial schematic natural deductions (probably written on paper). [24pts] \\


\item Using the Soundness and Completeness theorems (or the Consistency Lemma) for system $\mathit{QND}$, prove that Quantifer Logic (without identity) is compact. [14pts] \\

\item Prove \textbf{ONE} of the following two claims [14 points]:
\begin{enumerate}
\item Let $\Gamma$ be a set of QL-sentences. Prove that if $\Gamma$ is satisfiable, then $\Gamma$ is QND-consistent (i.e. syntactically consistent in derivation system QND). 

\item If $c$ does not occur in a QND-Consistent set $\Gamma_k \cup \{ \qt{\exists}{x} \metav{Q} \}$, then $\Gamma_k \cup \{\qt{\exists}{x} \metav{Q}, \metav{Q}\substitute{x}{c} \}$ is QND-consistent. \\ (NB: \textit{The Logic Book} mentions a lemma to prove this claim that is IRRELEVANT (Lemma 11.1.10). Instead, you must proceed purely syntactically.) \\

% (NB: \textit{The Logic Book} mentions a lemma to prove this claim that is insufficient without a proof of \#6a! And even then, it's not enough b/c at this point in the construction, we don't know that $\Gamma_k \cup \{ \qt{\exists}{x} \metav{Q} \}$ is quantificationally consistent, i.e. satisfiable. So do NOT appeal to Lemma 11.1.10 in your proof of \#6b. Here, you must proceed purely syntactically. 

%If \metav{c} does not occur in a QND-C set $\Gamma_k \cup \{ \qt{\exists}{\metav{x}} \metav{Q} \}$, then $\Gamma_k \cup \{\qt{\exists}{\metav{x}} \metav{Q}, \metav{Q}\substitute{\metav{x}}{\metav{c}} \}$ is QND-consistent
%you may assume that we have proven QND is sound. 
\end{enumerate}
%%Josh: add in a diagram and tell them they can use my line numbers. Or draw on the board during the exam! 

\item Briefly describe the significance of both (i) complete open branches and (ii) maximally-SND-consistent sets in the completeness proofs of STD and SND, respectively. \\ How are these ideas connected or similar? (i.e. what functional role(s) do they play in the completeness proofs of these derivation systems?)  [14pts]

%Imagine that we add the following rule, , to system SND. Add a case to our inductive proof of the Soundness of SND

%%idea: two mandatory problems, then two sets where they choose one! 



\newpage



\makebox[\textwidth]{\textbf{REMINDERS for when (you think) you're done}:}

\item[] If time remains, check your work for silly mistakes!!!

\item[] DON'T LEAVE ANY QUESTION BLANK!!!! Plz WRITE SOMETHING, so that you can be awarded partial credit. 

\item[] Make sure that you have answered EACH part of EACH question

\item[] Make sure you actually \textbf{clicked `Submit'} on Carnap for EACH problem!

\item[] Make sure you've written \textbf{YOUR NAME} on the first page of looseleaf \\ \makebox[\textwidth]{(and username if your first name is `Daniel')}

%Make sure you've written \textbf{YOUR NAME} on any looseleaf \\ \makebox[\textwidth]{(and username if your first name is `Daniel')}

\item[] Make sure you gave the requisite natural deductions in SND for Question \#4 (membership lemma).  (for partial credit: describe the kind of deductions you \textit{would} need.)

\begin{center}
\textbf{For \#1}: Syntactic Definition of new rule ``\textbf{Negated Disjunction Intro}.'' (\enot \eor I): 
\end{center}
\begin{equation*}
\begin{nd}
\have[h]{h}{\enot\metav{P}}
\ellipsesline
\have[j]{j}{\enot\metav{Q}}
\have[k+1]{a}[\ ]{\enot(\metav{P} \eor \metav{Q})}\by{\enot \eor I}{h, j}
\end{nd}
\end{equation*}


%\newpage


\iffalse

\item (i) Translate the following argument into the language of sentential logic. (ii) Check its validity using a tree, and state your conclusion. If the argument is invalid, use the tree to find a truth value assignment that makes its premises true and conclusion false.

\begin{quote}
If logic monkeys are hirsute, then logic monkeys are orgulous. And if space dogs are splenetic, then space dogs are bilious. So both if logic monkeys are hirsute then space dogs are bilious, and if space dogs are splenetic then logic monkeys are orgulous. 
\end{quote}

Symbolization Key: H = logic monkeys are hirsute; O = logic monkeys are orgulous; S = space dogs are splenetic; B = space dogs are bilious

\fi 






























\end{enumerate}


\end{document}